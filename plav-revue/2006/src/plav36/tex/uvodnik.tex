\section{Hanka Svobodová}

\noindent
(1953), rodačka z valašského Slavičína, po většinu své profesionální dráhy působila v literárních časopisech – v 80. letech v Sovětské literatuře, poté, rovněž na půdě Lidového nakladatelství, v porevoluční literární revue Dokořán a nakonec, v první polovině 90. let, ve slavné Světové literatuře. V tomto časopisu setrvala až do jeho smutného konce. Když „ekonomická cenzura“ začala vyřazovat ze stránek knih i periodik mnohé kvalitní překladové texty jako čtenářsky a nakladatelsky málo atraktivní a ujal se jich bubeníčkovský literární kabaret, Hanka se i přes zdravotní handicap stala jednou z~nejvěrnějších návštěvnic těchto večerů. Přesto, že prostě „na bubeníčky“ přijít je pro ni stále těžší. O~to raději ji mezi sebou vidíme jak dnes, tak při kterékoli další podobné příležitosti.

\podpis{Jana Mertinová, Jiří Josek, Milan Dvořák}
\podpis{Libor Dvořák, Jiří Honzík, Hana Kofránková}


