\section{Arkadij a Boris Strugačtí: \\ Ďábel mezi lidmi}

\noindent
\textit{(kapitola z románu)}

\medskip

\noindent
14

\noindent
Náš drak špatně skončil, protože jím zmítaly nejtemnější a nejhanbatější orgány jeho těla.

Ten den jsem s~Alisou byl u~starého přítele, jehož jsem svého času dostal z~těžkého infarktu. Bylo to ve vile vojenského městečka asi čtyřicet kilometrů od Tašlinska. Poslali pro nás auto. Mráz a slunce. Čtyři manželské páry. Novoroční přípitek šampaňským. Pak noční procházka. Sladký a příjemný spánek. Ráno lehká snídaně. A~zase mráz a slunce. Báječný výlet na běžkách, sváteční oběd a po něm lenošivý spánek. Karty se sklenkou koňaku. V~deset večer jsme skončili, lehce povečeřeli a vypravili se do hajan. Následovalo časné jitro, opět s~mrazem a sluncem. Poslední objetí na rozloučenou: ,,Ale stejně se vídáme málo, kamaráde!“ Pak čtyřicet kilometrů zpáteční cesty po uježděném sněhu. Zastav támhle u~toho domu, příteli. Díky. A~pětirublovka do hnědé dlaně se zažranými stopami paliv a maziv. Tenhle Medvídek Pú – ten si panečku umí žít!

V~tu chvíli jsem ještě netušil, jaké svízele mě už za chvíli čekají. Jak to svého času vyjádřil Sam Weller: ,,Kdybyste věděl, sire, kdo je tady poblíž, asi byste zpíval jinak – se spokojeným šklebem prohlásil jestřáb, když zaslechl, jak si za rohem vyzpěvuje červenka\ldots“

Zrovna jsem se oblékal do normálních šatů, když v~mé pracovně zadrnčel telefon. Volal mi službukonající lékař. Sotva jsem se ozval, strašlivým hlasem zavřeštěl:

,,To je Alexej Andrejevič?! Zaplaťpánbu, konečně! Šéf po vás pátrá jako blázen\ldots“

,,Šťastný a veselý!“ pronesl jsem přísně a přímo fyzicky cítil, jak se mi srdce propadá do žaludku.

,,No ano, jistě\ldots, šťastnej a veselej\ldots Ale teď hned vás potřebujeme tady v~nemocnici, Alexeji Andrejiči. Máme tady takovou nepříjemnost\ldots“

,,Počkej. Jak máme? Myslíš u~vás na chirurgii?“

,,Ale kdepak. Naopak u~vás na interně. Šéf se po vás shání už od včerejška.“

Soustředil jsem se a v~duchu probral všechny nejhorší eventuality. Ne, to všechno nebylo ono. I~kdyby polovina mých babiček a infarktářů umřela během hodiny, šéf by určitě nemarnil sváteční den na shánění vedoucího lékaře interny. Takový on není. Ale co by to tedy mohlo být? Odpověď už mi ševelil teď o~poznání tlumenější hlas ve sluchátku:

,,Berankin včera natáh kopyta!“

,,Umřel?“

,,Přesně tak – umřel,“ vydechl kolega ve službě.

,,Na zánět nervů?“ otázal jsem se zpitoměle celý bez sebe. Vzápětí jsem se ale vzpamatoval a vyhrkl: ,,Dobře. Hned jedu.“

Vše bylo najednou přímo oslnivě jasné. Berankin ty bačkory natáhl na mém území, takže i kdyby příčinou smrti bylo kousnutí kobry, vylizovat si to budu právě já; už jsem se viděl, jak tonu v~moři hlášení, vysvětlujících dopisů, obžalobných stížností ze strany příbuzenstva i zlověstných oficiálních dotazů od různých instancí. Protože v~tomhle případě jde o~Berankina, a ne o~nějakého bezejmenného penzistu z~Pugačovky\ldots Se srdcem v~kalhotách a s~olovem v~nohou jsem tedy zamířil do nemocnice a cestou skoro nahlas spílal jak zesnulému Berankinovi, tak zcela zdravému primáři a koneckonců i svému až příliš hravému osudu.

 \looseness-1
Službukonající lékař už na mě čekal. Sdělil mi, že šéf sedí ve své pracovně, je v~troskách a čeká jen a jen na mě, abychom spolu probrali pár otázek. No dobře, řekl jsem, na to je čas. Co a jak se tu vlastně odehrálo? Kolega ve službě mě zdvořile upozornil, že on nastupoval až včera večer, tedy ve chvíli, kdy se vše podstatné už stalo a v~nemocnici zavládl chaos, o~nějž se postarali všelijací kompetentní uředníci, předvádějící především různá stádia alkoholické euforie.

No dobře, řekl jsem, ale já bych si stejně potřeboval nějak dát dohromady, co a v~jakém sledu se tu odehrávalo. Jistě chápeš, kamaráde, že dřív, než stanu před šéfem a začnu mu něco hlásit, tak musím o~věci mít nějaké ponětí\ldots Prostě kdyby se po mně sháněl, tak řekni, že jsem u~sebe v~ordinaci. Odvlekl jsem se do domáckého prostředí své interny. Berankin je Berankin, táhlo mi hlavou neodbytně, protože v~duchu už jsem viděl, jak se nade mnou vznáší hrozivý stín prokurátora.

Nařídil jsem uklizečce, aby někde sehnala vrchní sestru. Když vstoupila, úplně jsem zapomněl pozdravit a rovnou jsem se zeptal:

,,Jaká byla diagnóza?“

,,Akutní srdeční nedostatečnost.“

,,A kdo má chorobopis?“

,,Šéf.“

Tak. Podepřel jsem si tvář pěstí a nařídil jí:

,,A teď mi vykládejte, jak to všechno proběhlo.“

Znejistěla. Včera samozřejmě taky slavila, stejně jako já. Takže to máme další vroubek. Vtom už se ale do mé ordinace vevalil veškerý zdravotní personál oddělení, který byl zrovna v~práci, a k~tomu ještě tři babky sanitářky, které se včera rovněž staly svědkyněmi. No a tahle trojice se dnes přišla podívat, jak to celé dopadne. Přesně takhle jsem to pochopil a dělal jsem, že jsem nezaslechl poznámku naší nejstarší sanitářky Elvíry, která s~neomalenou upřímností oznámila: ,,Musely jsme se přece vo takovou radost s~vostatníma podělit\ldots“ A~pak už jsem se všechno dověděl.

Včera po obědě za Berankinem dorazila jeho choť. Ta samoz\-řejmě nehodlala čekat dole s~plebsem ve frontě na pláště a vtrhla na oddělení rovnou, v~tulením kožichu a vyšívaných měkkých vysokých botách. Svého mučedníka obdarovala četnými laskominkami, které opravdu stály za řeč: ,,Máš tam kaviárek, nějaký pitíčko, sýr a taky něco pečenýho\ldots“ Kaviárek skutečně přinesla, ,,něco pečenýho“ reprezentoval kus uzeného a vzácná ryba balyk, kdežto k~pitíčku dodala koňak v~našich zeměpisných šířkách opravdu nevídaný. Sama manželka se dostavila opilá. (,,Trochu v~náladičce,“ pojmenovala to noblesní sanitářka Simočka; ,,Na šrot,“ popsala tentýž stav hrubiánka Galina z~chirurgie; ,,Nalitá až po vobočí,“ popsala to nevzrušeně teta Elvíra.) Manželka se ostatně ve špitále dlouho nezdržela. Nejspíš si spolu dali novoročního panáka, ona zase odplula a svého pána a velitele nám tu nechala, ať si vzácný koňáček dumlá sám.

Nějakou chvíli byl na oddělení klid a mír, jenže pak se najednou dveře prominentního pokoje rozlétly a na prahu stanul Berankin – v~rozepnutém přepychovém županu, ve strakatém ručně pleteném svetru a v~protirevmatických teplých podvlékačkách; zkrátka a dobře vystavil na odiv veškeré terapeutické prostředky. Pacienti a jejich návštěvy, poklidně sedící ve stínu ubohých nemocničních palem, zaskočeně strnuli. Berankin je všechny přehlédl hrozivým pohledem a promluvil. Chřtán měl opravdu děsivý, to se mu tedy musí nechat. Když se rozeřval, drnčela okenní skla, zvonilo nádobí a moje nebohé babičky na pokojích schovávaly hlavy pod přikrývku. Veškerá jeho podobná vstoupení byla obžalobná a výhružná. Stejně tomu bylo i v~tomto posledním případě.

Jak vyplynulo ze svědeckých výpovědí, repetoár byl obvyklý, proložený v~jeho případě běžným, jinak ale naprosto nepředvídatelným přeskakováním od tématu k~tématu. On nedovolí těm zdejším rasům a koňskejm řezníkům, aby na něm konali svý hnusný pokusy a na jejich základě pak psali nějaký učený pojednání. Protože on moc dobře ví, že nemocnici za úplatky zaplavili všelijaký darmožrouti, který se tu za státní peníze jen tak válej, aby se vyhnuli poctivý práci, a eště ke všemu věčně couraj na hajzl kolem jeho dveří. On už se ale podívá na zoubek všem těm, co v~nemocnici okrádaj lid a krměj ho všelijakýma pomejema\ldots

Když se ho snažili upokojit návštěvníci, pohrozil jim, že je nechá shnít v~kriminále. Pak se ho pokusila zastrkat zpátky na pokoj sanitářka, které oznámil, že časy, kdy se poctivejm lidejm zacpávaly huby, už jsou dávno pryč. Pak přiběhl zoufalý službukonající lékař; tomu Berankin pro změnu nařídil, aby do týdne odtáhl do Izraele. Nakonec ho kolem nedozírného pasu jemně objala teta Elívra a začala ho laskavě přemlouvat, aby si radši zase lehnul. Vzepřel se a jal se blábolivě vyřvávat, co je zač teta Elvíra a co byli zač její rodiče\ldots

Takhle to tedy všechno vzniklo. Sotva se Berankin pustil do líčení sexuálních zvráceností, jichž se dopouštěli Elvířini dávno zesnulí otec a matka, najednou zmlkl. Uprostřed slova. Jako když vypnete rádio. Simočka, která se poděšeně ukrývala za zády službukonajícího lékaře, to všechno viděla na vlastní oči. Berankin umlkl, ksicht mu zmodral, on ještě škytl, temně zalkal a svalil se ke straně. Ani ho nestačili zachytit. Dopadl na zem, zaškubal nohama, vytřeštil oči a strnul. To bylo všechno.

,,Co jste udělali?“ zeptal jsem se tupě.

Udělali všechno, co měli a co se v~podobných případech předpdokládá a doporučuje. Defibrilaci. Intubaci. Jenže nic nezabíralo. Mrtvola zůstala mrtvolou. Veškeré snahy měly stejný účinek, jako když pícháte injekce do protézy. Berankin chcípnul. A~dobře mu tak. Však on to Vološin řekl správně a od srdce. Vyšel ze dveří, podíval se na zem a prohlásil – samozřejmě tak, aby to všichni slyšeli: ,,Psovi psí smrt\ldots“

Srdce se ve mně zastavilo.

,,Počkat, počkat,“ zarazil jsem ochotné vypravěče dřevěnějícím jazykem. ,,Kdo že vyšel?“

,,No Vološin\ldots Toho přece znáte – ten jednovokej\ldots“

,,A odkud vyšel?“

,,No z~vedlejšího pokoje! Copak jste zapomněl, Alexeji Andrejeviči? Tam leží jeho žena Ljusjenka\ldots“

Zatímco jsem se z~téhle ohromující zprávy vzpamatovával, ukázalo se, že Kim je mimořádně starostlivý manžel. Ženu navštěvuje skoro denně a pokaždé ji přinese nějaké lahůdky. Dokonce si ušil vlastní nemocniční plášť a jak říkaly naše ženské, určitě si ho sám pere, škrobí a žehlí. Často prý do nemocnice chodí s~dcerkou. Taková hubeňoučká, pihovatá, ale je vidět, jak se o~ni staraj – vlásky má vždycky učesaný, v~nich má mašli a šatečky má taky moc pěkný\ldots Včera taky přišel s~holčičkou. Vona je to chudinka, pane doktore, neslyší a nemluví, ale rozumí teda všemu, zvlášť když ji vosloví táta – to ví všechno hned. Když přišli včera, pokračovala teta Elvíra, tak Ljusjence na pokoji uspořádali učiněnou hostinu. Hned jsem jim donesla horkou vodu. Tasjenka všem nalévala a roznášela čaj. A~ty preclíky, pane doktore, to teda bylo pošušňáníčko, eště teplý, ty určitě pek sám\ldots

,,Počkej, teto Elvíru,“ zarazil jsem proud její výmluvnosti. ,,Preclíkama mě nekrm\ldots Byl už Vološin na chodbě, když Berankin upadl?“

,,No to ne!“ prohlásila rezolutně Elvíra. ,,Už jsem přece řikala, že Kim vyšel do chodby až pak\ldots“

,,Jasně,“ přikývla Galina z~chirurgie. ,,Jakmile žuchnul, tak jsem běžela pomoct Georgiji Ruvimovičovi, no a Vološin zrovna otevřel dveře, eště mi s~nima vrazil do zadku\ldots“

Všichni jako by v~tu chvíli něco začuli a s~vyčkávavou zvědavostí se na mě zadívali. Já se ale jen zeptal:

,,A co bylo dál?“

Pak, po všech těch nezbytných, ale zbytečných procedurách odvlekli mrtvolu do márnice, jenže navečer se do nemocnice nahrnula parta darmožroutů z~milice nebo ze státní bezpečnosti, vzhledem ke svátečnímu dni většinou taky dost nacamraných, a s~nima jeho stará\ldots teda vdova. Ta už vůbec nevěděla, která bije, a jen se pořád dožadovala, aby tělo jejího muže třeba na nudličky rozřezali, ale aby se zjistilo, kdo se tohoto teroristickýho aktu dopustil. Pak zaťali drápy do nebohého službukonajícího lékaře –  jestli náhodou soudruha Berankina neudeřil nebo do něj nestrčil, když ten v~polemickém zápalu možná užil nějakého toho ostřejšího slůvka. Pak vyslýchali ostatní personál. Jeden se dokonce dopotácel až do kotelny, kde byl dnes ráno nalezen, jak spí v~objetí s~naším ožralým topičem\ldots

Poslouchal jsem – a zároveň neposlouchal. Takže je v~tom opravdu zase Kim, uvažoval jsem horečně. Teď už nemůžu strkat hlavu do písku a dál se ukolébávat jalovými konstrukcemi o~fatální shodě okolností. Taková shoda je možná jednou, dokonce i dvakrát\ldots, ale smiluj se, Bože, a naznač, jaká je vůle Tvá? Jistě, to se včas dovíme. Jak to kdysi dávno zpívával nositel jediné medaile Za odvahu strejda Kosťa: ,,přenést\ldots“

Najednou jsem v~prázdném rohu za ramenem sanitářky Simočky zaznamenal jakýsi pohyb. Zadíval jsem se tím směrem pozorněji. Zjevila se tam promodralá Berankinova tlama s~vyceněnými zlatými zuby a očima obrácenýma sloup, chvilku tam trčela, pak se odporně zašklebila a zmizela. Otřel jsem si z~čela pot. Cres\-cen\-do, napadlo mě najednou. Takhle se tomu říká mezi muzikanty. Cres\-cen\-do. Nejdřív přišel okresní výbor strany. Pak psí jatka. Pak Potřebův záchvat. A~teď Berankin. Tentokrát ovšem ne prostě Berankin, ale zabitý Berankin.

Vyprovodil jsem sestřičky i sanitářky ze své ordinace ven a sám se vydal k~šéfovi. Z~kalných hlubin neproniknutelného tajemství na známou každodenní teploučkou mělčinu. Vody je tu sotva do polou lýtek. Zřetelně tu vidíte každého trilobitíka, hemžícího se v~písku pod nohama. Šéf není žádný Kim Vološin, ten je průzračný jako sklíčko – drobný kariérista, patolízal a zbabělec, jehož vytouženým snem je teplé místečko lektora v~krajském ústavu národního zdraví. Takže hlavně žádné mimořádné události. V~jeho nemocnici nic takového nikdy nebylo, není a nebude. Dokud je tu šéfem on, nemůže být o~mimořádných událostech řeči – nanejvýš o~nezodpověných dobrodruzích a nerozumných hrdopýšcích, na něž on včas a hlavně předem upozorní tam, kde se o~podobné věci zajímají. Tím spíš, když jde o~takovou osobnost jako byl zesnulý Berankin. To je skvrna na pověsti nemocnice. Vyšetřování bude přísné a důrazné. Za tohle nás prokuratura po hlavičce hladit nebude. Sláva mu, on je pašák! Hromadil se ve mně vztek. Veselý vztek, řekl bych. Mrtvoly po koutech už se mi nezjevovaly. Však taky chybělo jen opravdu málo. A~už je to tady – ten kruciální výrok: ,,A vy si laskavě uvědomte, Alexeji Andrejeviči, že kaštany z~ohně za vás nikdo tahat nebude. Až si nás budou volat na koberec, tak půjdete vy a ne já. Protože já mezitím můžu docela klidně onemocnět.“

Drze jsem se protáhl, zívl, vstal a odešel – a zanechal ho tak v~příjemném zmatku: Nezbláznil jsem se náhodou, nebylo by náhodou nejlepší nechat mi nasadit svěrací kazajku a s~příslušným komentářem mě odeslat na příslušná místa?

Sešel jsem dolů na prosekturu. Mojsej Naumovič seděl na ,,stole smutku“ (jak on sám říkal tomuto zlověstnému monstru z~umělého mramoru) a kouřil. Krátce na mě pohlédl a monotónním hlasem zahuhlal:

,,Ten chlap byl zdráv jako buk. Chcete si přečíst pitevní zprávu?“

Zakroutil jsem hlavou:

,,Ne. Jen ať si v~tom čtou na neurologii. Ostatně jakou jste stanovil diagnózu?“

Nejdřív neřekl nic a pak ze ,,stolu smutku“ hekavě slezl.

,,Diagnóza\ldots,“ zavrčel jen. ,,Ta je normální. Úplně stejná, jako ta, co s~ní přišel doktor na oddělení. Akutní srdeční nedostatečnost. Co jiného taky konstatovat, když se někomu zničehonic zastaví srdce. Může se to stát nebo ne?“

Automaticky jsem přikývl, že může. Mojsej Naumovič se najednou rozesmál.

,,Představte si, Alexeji Andrejeviči, že ta dáma\ldots myslím ta vdovička\ldots trvala na okamžité pitvě, a to přímo před jejíma očima. Aby vrazi v~bílých pláštích nemohli nic ututlat. Tak jsem se pokusil jí to rozmluvit a dostat ji ven – ale kdepak! Zastali se jí ti\ldots vyšetřovatelé\ldots nebo co to bylo zač. S~rudými řidičáky. S~těmi jsem se tedy dohadovat nechtěl. Tak dobře, říkám si, když chcete\ldots \\ Ale pak si stěžujte sami na sebe!“ A~znovu se rozesmál, tentokrát docela nepříjemně, což na něj rozhodně nevypadalo, tak nějak zlomyslně. ,,Sotva jsem začal, byli na hromadě. Všechno mi to tu pozvraceli a vypadli. A~vdovu odtáhli s~sebou\ldots Slabí lidé, jak by jistě podotkl soudruh Koba neboli Stalin.“

,,Já pro vás ale mám novinky, Mojseji Naumoviči,“ řekl jsem.

,,Tak to je zlé,“ pokýval hlavou. ,,Vykládejte, Alexeji Andrejeviči.“

A~já vykládal. On poslouchal. Jeho tvář postupně kameněla. Když jsem skončil, asi minutu jsme oba mlčeli. Pak se on skřípavě zeptal:

,,Jak že to ten Vološin řekl? Psovi psí smrt\ldots To vypadá přesně podle Freuda, protože mně to ukazuje na Pugačovku\ldots“

Najednou začal spěchat, Natáhl si svůj olezlý kožíšek, jen tak halabala si kolem krku omotal šálu a nepříliš srozumitelně zadrmolil:

,,Jenže ono mu to zase nevyšlo! Mozek semletý, vnitřnosti rozdrásané\ldots S~tím zřejmě nepočítal. No nic, musíme čekat na další případ\ldots“

Smysl téhle zvláštní sentence mi přiblížil až za pár dnů.

\podpis{přeložil Libor Dvořák}