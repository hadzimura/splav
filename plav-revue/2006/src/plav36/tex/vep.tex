\section{Charles Ludlam: \\ Záhadná Irma Vep}

\noindent
\textit{(fragment divadelní hry)}

\medskip

\noindent
JEDNÁNÍ I.

\medskip

\noindent
\textit{Scéna 1.}

\medskip

\noindent
\textit{Knihovna na Mandacrestu, sídle rodiny Hillcrestů poblíž Hamp\-steadského vřesoviště v době mezi dvěma válkami. Je to velká místnost s francouzskými dveřmi vedoucími do zahrady. Je zde stůl a židle, krb s krbovou římsou, nad níž visí portrét Irmy Vep. Z obou stran krbu jsou dvě křesla. Jsou zde suvenýry z cest: africké masky a japonský malovaný paraván. V policích jsou knihy vázané v safiánu a vpravo a vlevo dveře. Zvedne se opona a ze zahrady přichází francouzskými okny Nicodemus a nese koš. Levou nohu má invalidní a na levé botě velký dřevěný podpatek. Jane aranžuje květiny ve váze.}

\bigskip

\noindent
JANE: Podívej se, co děláš! Celé to tady zamokříš a zablátíš!

\noindent
NICODEMUS: Trocha božího deštíčku neuškodí, děvče.

\noindent
JANE: To je ďábelský déšť!

\smallskip

\noindent
(\textit{Zablýská a zahřmí.})

\smallskip

\noindent
NICODEMUS: A~to bys chtěla, aby to dlouhé sucho nikdy neskončilo! Buď ráda, že prší.

\noindent
JANE: Nedupej tak hlasitě tou svou dřevěnou nohou nebo vzbudíš lady Enid.

\noindent
NICODEMUS: Já myslím, že bude ráda poslouchat mé dupání, až jí řeknu, jak jsem k té dřevěné noze přišel. Že mě tak zohavil vlk, když jsem mu ze spárů vyrval lorda Edgara.

\noindent
JANE: To už je dávno. A~lady Enid o~tom nic neví.

\noindent
NICODEMUS: Však se to ode mne brzy dozví.

\noindent
JANE: Nicodeme, nebudeš přece strašit novou nevěstu lorda Edgara svými historkami s vlkem.

\noindent
NICODEMUS: Čím dřív se to ode mne dozví, tím líp.

\noindent
JANE: Ty si tím svým jazykem jednou vykopeš hrob. Některé věci je lepší neříkat.

\noindent
NICODEMUS: Žijeme snad ve svobodné zemi, nebo ne?

\noindent
JANE: Pšššš!

\noindent
NICODEMUS: Nebo ne?

\noindent
JANE: Jestli tě uslyší lord Edgard, sám uvidíš, jakou svobodu tu máme. Vyhodí tě.

\noindent
NICODEMUS: Po takové svobodě netoužím. Radši budu tedy držet jazyk za zuby.

\noindent
JANE: Musíme za lordem Edgarem stát. Mám strach, že nás teď bude hodně potřebovat.

\noindent
NICODEMUS: A~proč teď víc než jindy? Řek bych, že je z toho nejhoršího venku. Konečně se smířil s tím, že je slečna Irma v hrobě.

\noindent
JANE: Tohle neříkej. Nedovedu si ji v hrobě ani představit. Však víš, jak strašně nesnášela tmu!

\noindent
NICODEMUS: Smířil se s tím a ty se s tím taky musíš smířit. Život pro něho znovu začíná. Truchlil o~mnoho déle, než se sluší a patří, a konečně si přivedl na Mandacrest novou lady Hillcrestovou.

\noindent
JANE: No právě! V tom je ten háček! Já si totiž nemyslím, že lady Enid bude na Mandacrestu dobrou paní.

\noindent
NICODEMUS: A~proč ne?

\noindent
JANE: Je tak\ldots tak\ldots obyčejná. Nikdy nebude mít tu úroveň, co měla lady Irma.

\noindent
NICODEMUS: Děvče, do toho mu my dva mluvit nemůžeme.

\noindent
JANE: A~tahle cuchta mi má poroučet?

\noindent
NICODEMUS: No tak! Takhle o~lady Enid nemluv!

\noindent
JANE: Lady Irma měla přirozenou autoritu a dokonalé způsoby.

\noindent
NICODEMUS: Muž od ženy potřebuje i něco víc než jenom způsoby.

\noindent
JANE: Opravdový muž potřebuje ženu s dobrým původem a charismatem.

\noindent
NICODEMUS: Jestli je to charizmato to, co si myslím, měla by sis dávat pozor na jazyk. Tady jsou vajíčka a mléko. Ta naše želva v poslední době snáší jako divá.

\noindent
JANE: A~kde je smetana?

\noindent
NICODEMUS: Tu jsem slízal.

\noindent
JANE: Ty jsi nenapravitelný.

\noindent
NICODEMUS: V čem?

\noindent
JANE: A~co asi řeknu lordu Edgarovi, až bude chtít smetanu do čaje, no co?

\noindent
NICODEMUS: Řekni si mu, co chceš.

\smallskip

\noindent
(\textit{Blesk a hrom.})

\smallskip

\noindent
JANE: (\textit{Vyjekne}) Jejda!

\noindent
NICODEMUS: No tak! Neplaš se! Nicodemus tě ochrání.

\smallskip

\noindent
\textit{Snaží se ji obejmout.}

\smallskip

\noindent
JANE: (\textit{Vymkne se mu}) Ty ruce pryč! Smrdíš jako prase.

\noindent
NICODEMUS: Kdybys musela spávat v chlívku, taky budeš smrdět.

\noindent
JANE: Nechoď ke mně!

\noindent
NICODEMUS: Přijde doba, kdy se na mě budeš s láskou usmívat, Jane.

\noindent
JANE: Ne dřív, než v pekle nasněží a čerti se budou klouzat.

\noindent
NICODEMUS: Kdybych se umyl, vzal si parádní oblek, špricnul na sebe kolínskou a napomádoval si vlasy, hned bys na mě koukala jinak.

\noindent
JANE: Nedělej si žádné naděje. Jsi služebně pode mnou a pode mnou zůstaneš.

\noindent
NICODEMUS: Přijde den, kdy se ti pode mnou bude moc líbit.

\noindent
JANE: Co si to dovoluješ takhle se mnou mluvit? Já mám školy!

\noindent
NICODEMUS: A~co ses naučila?

\noindent
JANE: Četla jsem bibli, Robinsona Crusoe a Malého lorda Fontleroye, od začátku až do konce.

\noindent
NICODEMUS: A~já zase čet příručku Jak chovat prase, od početí až po zabíjačku.

\noindent
JANE: Pche!

\noindent
NICODEMUS: Nemusíš nade mnou ohrnovat nos, slečinko. Jsme stejné krve ty i já.

\noindent
JANE: Nevytahuj se! Radši zalez do svého chlívku, než řeknu něco, čeho pak budu litovat.

\noindent
NICODEMUS: Nepůjdu, dokud mi nedáš pusu.

\noindent
JANE: Tůdle!

\noindent
NICODEMUS: (\textit{Honí ji po místnosti}) Dej mi pusinku a já ti ukážu svýho velkýho lorda.

\noindent
JANE: Nech si ty lascivnosti, sprosťáku.

\noindent
NICODEMUS: Jaký lacinosti?

\smallskip

\noindent
(\textit{Hrom, nahoře se ozvou kroky})

\smallskip

\noindent
JANE: No vidíš! Probudils lady Enid. Utíkej, než tě načapá v domě.

\noindent
NICODEMUS: Copak ona spala? Vždyť je večer!

\noindent
JANE: Ona tak žije. Ve dne spí a celou noc je vzhůru.

\noindent
NICODEMUS: To jsou ty městský móresy. Lord Edgar mi říkal, že byla herečka.

\noindent
JANE: (\textit{Šokovaná}) Herečka? Ale fuj!

\noindent
NICODEMUS: Jen si to představ! Opravdická herečka na Mandacrestu.

\noindent
JANE: Je to úpadek! Ale musíme se s tím smířit, když je teď z ní paní domu?

\noindent
NICODEMUS: Ty mi nerozumíš. Chci říct, že si lord Edgar uměl vybrat.

\noindent
JANE: Vy chlapi jste všichni stejní. Jak snadno se necháte napálit. (\textit{Jsou slyšet kroky}) Už sem jde. Uteč!

\noindent
NICODEMUS: Já si ji chci prohlídnout.

\noindent
JANE: Nic na ní není. A~určitě by nechtěla, abys ji okukoval. Jen si jdi do svého chlíva.

\noindent
NICODEMUS: Tam mám kolikrát lepší společnost než v tomhle domě.

\smallskip

\noindent
\textit{Odejde.}

\smallskip

\noindent
LADY ENID: (\textit{Za scénou}) S kým to tu mluvíš?

\noindent
JANE: To byl jenom Nicodemus. Přinesl vejce.

\noindent
LADY ENID: Už je pryč?

\noindent
JANE: Ano, lady Enid.

\noindent
LADY ENID: Už zapadlo slunce?

\noindent
JANE: Venku leje, milostivá, takže se to nepozná.

\noindent
LADY ENID: Zatáhni závěs a zatop v krbu. Já jdu dolů.

\noindent
JANE: Kruci. Já si neodpočinu.

\smallskip

\noindent
\textit{Zatáhne závěs přes francouzské okno. Prohlédne se v zrcadle, ovíjí se kapesníčkem, srovná si vlasy a límeček.}

\noindent
\textit{Vystoupí lady Enid.}

\smallskip

\noindent
LADY ENID: Je tu teplo a útulno. Děkuji, Jane

\noindent
JANE: Chcete čaj?

\noindent
LADY ENID: Když budeš tak hodná.

\noindent
JANE: (\textit{Chladně}) Je to moje práce.

\noindent
LADY ENID: Lord Edgar je doma?

\noindent
JANE: Vstal za rozbřesku a odjel.

\noindent
LADY ENID: Kam?

\noindent
JANE: Na projížďku. Dělá to tak každý den. (\textit{Za scénou píská konvice}) Už píská konvice.

\smallskip

\noindent
\textit{Odejde.}

\noindent
\textit{Lady Enid se rozhlíží po pokoji a prohlíží si obraz i knihy. Vyhlédne francouzským oknem do zahrady. Pak upoutá její pozornost obraz nad krbem. Stojí před ním a dlouho na něj zírá.}

\smallskip

\noindent
JANE: (\textit{Vrací se s tácem}) Co do toho chcete?

\noindent
LADY ENID: Prosím?

\noindent
JANE: Do čaje.

\noindent
LADY ENID: Nic.

\noindent
JANE: (\textit{Nevěřícně}) Ani smetanu nebo cukr?

\noindent
LADY ENID: Jsem na dietě, víš? Kvůli divadlu.

\noindent
JANE: Ale to už máte za sebou.

\noindent
LADY ENID: (\textit{S povzdechem}) Asi ano. Je to ale síla zvyku. Zřejmě si už do konce života nedám čokoládu ani pivo. (\textit{Ukáže na portrét}) Kdo je ta dáma?

\noindent
JANE: To je lady Hillcresová\ldots Tedy minulá lady Hillcrestová.

\noindent 
LADY ENID: Byla krásná, že?

\noindent
JANE: Na světě se jí žádná jiná nevyrovná. Ježíš, pardon.

\noindent
LADY ENID: To je v pořádku, Jane. Měla jsi ji moc ráda, viď?

\noindent
JANE: (\textit{Podává jí šálek čaje}) Byla jako mé druhé já.

\noindent
LADY ENID: Aha. (\textit{Usrkne čaje a zarazí se}) Děláš silný čaj.

\noindent
JANE: Když dělám čaj, tak dělám čaj, když dělám vodu, tak dělám vodu.

\noindent
LADY ENID: Ale ne najednou v jedné konvici, předpokládám.

\noindent
JANE: (\textit{Chvíli jí nedojde, že to bylo míněno jako vtip}) Aha!

\noindent
LADY ENID: Ty mě nemáš ráda, Jane, viď že ne.

\noindent
JANE: Nic proti vám nemám.

\noindent
LADY ENID: To doufám. Ještě to tak! Ještě abys proti mně něco měla. To by bylo hrozné soužití.

\noindent
JANE: To jo. Nemám proti vám nic.

\noindent
LADY ENID: Ale ráda mě nemáš.

\noindent
JANE: Neznám vás. Musím si na vás nejdřív zvyknout.

\noindent
LADY ENID: Úplně mě zamrazilo. Jako by mi kočka přeběhla přes hrob.

\noindent
JANE: Nesedíte v průvanu?

\noindent
LADY ENID: Trochu tu táhne. Mohla bys zavřít to francouzské okno?

\noindent
JANE: Nicodemus za sebou zase nezavřel! Kdybych mu to říkala tisíckrát, tak je to málo. Pán už se vrací.

\noindent
LADY ENID: (\textit{Spěšně}) Kde je? (\textit{Jde k oknu}) Ano. (\textit{Schová se za závěs}) Jdi stranou, ať nás nevidí.

\noindent
JANE: Co to vleče? Má plnou náruč vřesu a něco táhne za sebou.

\noindent
LADY ENID: Co táhne za sebou?

\noindent
JANE: Nějaké velké zvíře. Proboha, určitě zastřelil toho vlka.

\noindent
LADY ENID: (\textit{Nervózně}) Jakého vlka?

\noindent
JANE: Toho, co nám rdousí jehňátka. Aspoň se konečně vyspíme, když nám tu nebude celou noc výt.

\noindent
LADY ENID: On zabil vlka?

\noindent
JANE: Ano, a jeho mrtvolu vleče s sebou.

\noindent
LADY ENID: Je mrtvý? Skutečně mrtvý?

\noindent
JANE: Mrtvější už být nemůže.

\noindent
LADY ENID: A~kudy ho vleče?

\noindent
JANE: Tou cestičkou kolem tůjí.

\noindent
LADY ENID: Tudy chodí, ale půjde přes můstek?

\noindent
JANE: Sama jsem zvědavá. Blíží se tam. Ne, uhnul, a bere to kolem zimolezu.

\noindent
LADY ENID: Takže se z toho ještě nedostal.

\noindent
JANE: Nemůžete se mu divit, že nejde přes mostek. Vzhledem k tomu, co se tam stalo.

\noindent
LADY ENID: Jak vidím, na Mandacrestu se na mrtvé dlouho nezapomíná.

\noindent
JANE: Bať. Ale spíš ti mrtví nezapomínají na nás. Nechtějí se s námi rozloučit. Nechtějí nás tu nechat o~samotě. (\textit{Rychle se vzpamatuje}) Pán bude mít hlad. (\textit{Jde k dveřím}) Jak chcete biftek, madam?

\noindent
LADY ENID: Hodně udělaný!

\noindent
JANE: Žádnou krev?

\noindent
LADY ENID: Ne.

\noindent
JANE: Tak vidíte. Další rozdíl. Lady Irma ho měla ráda hodně krvavý.

\smallskip
\noindent
\textit{Odejde.}

\smallskip

\noindent
LADY ENID: (\textit{Otočí se a podívá se na portrét}) Nekoukej se tak na mě! Já ti ho nevzala. Dříve nebo později tě stejně musel někdo vystřídat a náhodou jsem to byla já. Vím, jak se musíš cítit, když máš to domácí štěstí přímo před nosem. Ale s tím už, holka, nic nenaděláš. Život jde dál.

\smallskip

\noindent
\textit{Vystoupí lord Edgar s plnou náručí vřesu a vleče za sebou mrtvého vlka.}

\smallskip

\noindent
LORD EDGAR: Tam je ale slota!

\noindent
LADY ENID: (\textit{Běží k Edgarovi a dá mu pusu na ústa}) Miláčku Edgare, tak už jsi zpátky.

\noindent
LORD EDGAR: Prosím tě, Enid, ne tady před\ldots

\noindent
LADY ENID: Před kým? Nikdo tu není. (\textit{Zarazí se}) Nebo ty myslíš ji? (\textit{Ukáže na obraz})

\noindent
LORD EDGAR: Je mi to trochu trapné. Líbat se před ní.

\noindent
LADY ENID: Fakt je, že se na to zvlášť dobře netváří.

\noindent
LORD EDGAR: Ale, Enid. Vždyť je mrtvá.

\noindent
LADY ENID: Možná právě proto.

\noindent
LORD EDGAR: Nebudeme o~ní mluvit.

\noindent
LADY ENID: Tak dobře.

\noindent
LORD EDGAR: Cítíš se u~nás dobře?

\noindent
LADY ENID: Ano, docela dobře. Jane mě sice nemá ráda, ale já si ji získám.

\noindent
LORD EDGAR: Doufám, že se ti bude u~nás líbit.

\noindent
LADY ENID: Určitě bude. Ach, Edgare, Edgare!

\noindent
LORD EDGAR: Ach, Enid, Enid!

\noindent
LADY ENID: Ach, Edgárku, Edgárečku!

\noindent
LORD EDGAR: Ach, Enidko, Enidečko.

\noindent
LADY ENID: (\textit{Nejistě}) Edgare.

\noindent
LORD EDGAR: (\textit{Mírně káravě}) Enid.

\noindent
LADY ENID: (\textit{S úlevou}) Edgare.

\noindent
LORD EDGAR: (\textit{Blahosklonně}) Enid.

\noindent
LADY ENID: (\textit{Tulí se k němu}) Edgare, Edgare, Edgare.

\noindent
LORD EDGAR: (\textit{Chlácholivě}) Enid, Enid, Enid.

\noindent
LADY ENID: (\textit{Vášnivě}) Edgare!

\noindent
LORD EDGAR: (\textit{Vzrušeně}) Enid!

\noindent
LADY ENID: (\textit{Ještě vášnivěji}) Edgare!

\noindent
LORD EDGAR: (\textit{Ještě více vzrušeně}) Enid!

\noindent
LADY ENID: (\textit{Extaticky}) Edgare!

\noindent
LORD EDGAR: (\textit{Extaticky}) Enid!

\noindent
LADY ENID: (\textit{Orgasticky}) Edgare!

\noindent
LORD EDGAR: (\textit{Orgasticky}) Enid!

\noindent
LADY ENID: (\textit{Unaveně}) Edgare!

\noindent
LORD EDGAR: (\textit{Zemdleně}) Enid!

\noindent
LADY ENID: Edgare?

\noindent
LORD EDGAR: Enid?

\noindent
LADY ENID: Sundej ten obraz.

\noindent
LORD EDGAR: To nemůžu.

\noindent
LADY ENID: Proč ne?

\noindent
LORD EDGAR: Prostě nemůžu.

\noindent
LADY ENID: Je už tři roky mrtvá.

\noindent
LORD EDGAR: Já vím, ale\ldots

\noindent
LADY ENID: Začneme znovu. Zapomeneme na minulost.

\noindent
LORD EDGAR: Jak rád bych, Enid! Věř mi. Moc rád!

\noindent
LADY ENID: Nikdy nebudeme mít klid, jestli se na nás bude pořád koukat.

\noindent
LORD EDGAR: V tom máš bohužel pravdu.

\noindent
LADY ENID: Tak sbalme všechny její věci do nějaké truhlice a tu někam uložme, kde ji čas od času budeš moct navštěvovat jako v nějaké svatyni. Ale doma ne.

\noindent
LORD EDGAR: Máš samozřejmě naprostou pravdu. Ale\ldots

\noindent
LADY ENID: Ale co?

\noindent
LORD EDGAR: Musel jsem jí slíbit, že nechám stále hořet věčný plamínek před jejím portrétem.

\noindent
LADY ENID: To je hloupost.

\noindent
LORD EDGAR: Ale musel jsem to slíbit.

\noindent
LADY ENID: Sfoukni to.

\noindent
LORD EDGAR: Nemůžu zrušit slib.

\noindent
LADY ENID: Myslela jsem, že teď patříš mně. Že my dva patříme k sobě.

\noindent
LORD EDGAR: To ano, ale tohle se stalo, ještě než jsme se potkali.

\noindent
LADY ENID: Co je pro tebe důležitější? Moje láska, nebo ten slib?

\noindent
LORD EDGAR: Enid, miláčku. Takhle to nestav.

\noindent
LADY ENID: Rozhodni se. Buď a nebo.

\noindent
LORD EDGAR: Prosím. Tohle ode mne nechtěj.

\noindent
LADY ENID: Miluješ mě?

\noindent
LORD EDGAR: Copak o~tom můžeš pochybovat?

\noindent
LADY ENID: Takže ses rozhodl. Sfoukni to.

\noindent
LORD EDGAR: Myslíš, že můžu? (\textit{Sfoukne světýlko})

\noindent
LADY ENID: Vidíš? Nic se nestalo.

\noindent
LORD EDGAR: A~my hloupí si mysleli, že se něco stane!

\smallskip

\noindent
(\textit{Smějí se})

\smallskip

\noindent
LADY ENID: A, miláčku, pokud jde o~to tahání mrtvých zvířat do salónu, s tím budeš muset taky přestat.

\noindent
LORD EDGAR: Ty ses rozhodla mě dočista zreformovat, viď?

\noindent
LADY ENID: Tak trochu ano.

\noindent
LORD EDGAR: Zavolám Nicodema, aby to uklidil. A~ty se zatím převleč na večeři.

\noindent
LADY ENID: Božínku, já mám hlad!

\noindent
LORD EDGAR: Ať jsi brzy zpátky.

\noindent
LADY ENID: Slibuji, že budu.

\smallskip

\noindent
\textit{Odejde.}

\smallskip

\noindent
LORD EDGAR: (\textit{Jde k obrazu}) Promiň mi to, Irmo, prosím. Prosím tě, promiň!

\smallskip

\noindent
\textit{Vystoupí Nicodemus.}

\smallskip

\noindent
NICODEMUS: Kde máme novou paní?

\noindent
LORD EDGAR: Převlíká se. Znáš ženské. Trvá jim to věčnost.

\noindent
NICODEMUS: Takže jste tu bestii konečně dostal, pane Edgar.

\noindent
LORD EDGAR: Ano. Zabil jsem ji. Už tu nebude řádit.

\noindent
NICODEMUS: A~co to zvíře uvnitř. I~to už má řádění dost?

\noindent
LORD EDGAR: Zatím je v klidu a míru. Nic lepšího čekat nemůžeme, že?

\noindent
NICODEMUS: Jste, pane Edgar, muž činu.

\noindent
LORD EDGAR: Vyvrhni ho a vnitřnosti spal.

\noindent
NICODEMUS: Kůži chcete nechat?

\noindent
LORD EDGAR: Ne, spal ji taky, do posledního chlupu.

\noindent
NICODEMUS: A~co popel? Co s popelem?

\noindent
LORD EDGAR: Rozházej ho po vřesovišti.

\noindent
NICODEMUS: Aby vítr vyl jako vlk?

\noindent
LORD EDGAR: Ne, máš pravdu. Hoď ho do náhonu.

\noindent
NICODEMUS: Za ní?

\noindent
LORD EDGAR: Ano, za ní. A, Nicodeme.

\noindent
NICODEMUS: Pane?

\noindent
LORD EDGAR: Sundej ten obraz a\ldots

\noindent 
NICODEMUS: A~co, pane?

\noindent
NICODEMUS: Spal ho s tím vlkem.

\smallskip

\noindent
\textit{Odejde.}

\noindent
\textit{Nicodemus jde k římse a pokouší se sundat obraz ze stěny.}
 
\noindent
\textit{Vystoupí Jane.}

\smallskip

\noindent
JANE: Co to děláš?

\noindent
NICODEMUS: Pán mi poručil sundat ten obraz.

\noindent
JANE: To nesmíš! Nesmíš sundat lady Irmu!

\noindent
NICODEMUS: Smím a taky že to udělám. Je to pánův rozkaz.

\noindent
JANE: Zadrž! Zadrž! Nesahej na ten obraz! Jejda. Věčné světýlko zhaslo. Bože, co s námi bude!

\noindent
NICODEMUS: Já za to nemůžu. Bylo zhasnutý, už když jsem přišel. Lord Edgar ho musel sfouknout.

\noindent
JANE: (\textit{Ukazuje na vlka}) A~co je tohle?

\noindent
NICODEMUS: Jseš slepá? To je vlk. Pán ho zabil.

\noindent
JANE: Chvála Bohu. Je to možné?

\noindent
NICODEMUS: Máme co oslavovat.

\noindent
JANE: (\textit{Opatrně se k mrtvole přibližuje}) Není žádný důvod k oslavě, Nicodeme Underwoode. Ten vlk, kterého zabil, není ten pravý vlk.

\bigskip

\noindent
\textit{Scéna 2.}

\medskip

\noindent
\textit{Scéna zůstává stejná. Je pozdě večer, celý dům už spí. Jane prohrabává poslední uhlíky v krbu. Tiše v županu vystoupí lady Enid. Stojí za Jane, která je k ní zády, a pozoruje ji. Jane si náhle uvědomí její přítomnost a vyděšeně vyjekne. To naopak vyděsí lady Enid, která rovněž vyjekne.}

\medskip

\noindent
LADY ENID: Nechtěla jsem vás polekat.

\noindent
JANE: Já vás taky ne, ale vy jste přišla strašně potichu. To mi nedělejte.

\noindent
LADY ENID: Promiň, Jane. Ty už tu žiješ dlouho, že? Kolik let jsi říkala? Šestnáct?

\noindent
JANE: Osmnáct. Přišla jsem, když se paní vdávala, jako její komorná. A~když umřela, pán si mě tu nechal jako hospodyni. Já ho znám už z dětství. Vyrůstala jsem kousek odtud na faře.

\noindent
LADY ENID: Aha.

\smallskip

\noindent
\textit{Dlouhá odmlka.}

\smallskip

\noindent
JANE: Všecko se tu strašně změnilo.

\noindent
LADY ENID: Zažilas tu hodně úprav?

\noindent
JANE: Ano, a ještě víc utrpení.

\noindent
LADY ENID: Hillcrestovi jsou asi dost starý rod, viď?

\noindent
JANE: To teda jo. Sahají až… já nevím kam, ale jsou tu už spoustu století.

\noindent
LADY ENID: Ale lord Edgar mi říkal, že je jedináček.

\noindent
JANE: Ano, poslední pozoruhodný květ na starém solidním rodovém kmeni.

\noindent
LADY ENID: Byl vždycky tak náruživý lovec? Už jako dítě?

\noindent
JANE: Kdepak. Začal s tím až po smrti milostivé paní. Ale to by byla dlouhá historie. Tím vás nebudu unavovat.

\noindent
LADY ENID: Jane, vyprávěj. Všechno, co se týká lorda Edgara, mě moc a moc zajímá.

\noindent
JANE: A~kde je pán?

\noindent
LADY ENID: Už tvrdě spí. Jane, byla bys tak hodná a něco mi o~té rodině řekla? Já teď stejně neusnu a tak ráda bych tu s tebou hodinku poseděla a poklábosila.

\noindent
JANE: Když si to přejete. Jenom si přinesu šití a pak tu s vámi posedím, jak dlouho budete chtít. Slyšíte ten vítr? To je ale ďábelská noc. Nechcete na zahřátí trochu punče?

\noindent
LADY ENID: Jenom jestli si dáš taky.

\noindent
JANE: Ano, já ráda punč a punč rád mě.

\smallskip

\noindent
\textit{Jde ke stolu, vezme si šití a nalije z kovové mísy, kterou hřála na uhlících v krbu, dva šálky punče. Jeden přinese lady Enid a usadí se proti ní na židli. Venku kvílí meluzína.}

\smallskip

\noindent
LADY ENID: Ten vítr kvílí.

\noindent
JANE: To není vítr. To je vlčí vytí.

\noindent
LADY ENID: To tady máte tolik vlků?

\noindent
JANE: Jen jednoho. Viktora.

\noindent
LADY ENID: Viktora?

\noindent
JANE: Odchytili ho a ochočili ještě jako mládě. Ale jeho srdce zůstalo divoké. Madam Irma si ho chovala jako domácího mazlíčka.

\noindent
LADY ENID: Jako pejska?

\noindent
JANE: Byl větší než pes. Tak velký, že mu chlapec mohl rajtovat na zádech. I~když se to Viktorovi nijak nelíbilo, to vám povím. Ale kvůli paničce to snášel. Poslouchal jenom ji. Nejšťastnější byl, když jí ležel u~nohou a z tlamy mu visel ten jeho obrovský rudý jazyk. Co byla s outěžkem, nikdy se od ní na krok nevzdálil, takže když to na paní přišlo, lord Edgard ho radši zavřel v kůlně. A~když sténala při porodu, Viktor vyl jak pominutý.

\noindent
LADY ENID: Lord Edgar mi říkal, že měl syna, ale ten ještě jako malý zemřel.

\noindent
JANE: Byla to strašná tragédie, ale vám stydne punč. Dopijte to, a já vám přinesu repete.

\noindent
LADY ENID: (\textit{Dorazí obsah šálku a podá ho Jane}) Zabily ho neštovice, že?

\noindent
JANE: Neštovice? Kdo vám to povídal?

\noindent
LADY ENID: Nikdo, jen mě to napadlo.

\noindent
JANE: Jestli vám lord Edgar řek, že to byly neštovice, pak to byly neštovice.

\noindent
LADY ENID: Ne, nic mi neřekl. To byla jen taková úvaha.

\noindent
JANE: Nedivím se, že vám nic neřek. O~tomhle se moc špatně mluví. Tady máte ten punč.

\noindent
LADY ENID: Děkuji.

\noindent
JANE: A~tenhle je pro mě.

\noindent
LADY ENID: Ráda bych věděla, jak to doopravdy bylo. Mohla bys mi to říct?

\noindent
JANE: (\textit{Alkohol jí rozvazuje jazyk}) Jednoho slunečného zimního dne si šli Viktor a chlapec hrát ven na vřesoviště v čerstvě napadaném sněhu. Vlk se ale vrátil sám. Čekali jsme. Křičeli do ochraptění. A~večer jsme ho našli v náhonu. Mrtvého. Měl prokousnuté hrdlo.

\noindent
LADY ENID: To je hrůza!

\noindent
JANE: Lord Edgar chtěl Viktora zabít, ale lady Irma ho bránila. Tvrdila, že to Viktor neudělal.

\noindent
LADY ENID: Třeba ne.

\noindent
JANE: Chlapec měl prokousnuté hrdlo. Jak jinak mohl zemřít? Strašně se kvůli tomu pohádali. Pán říkal, že paní má vlka radši, než měla své vlastní dítě. Ale já si spíš myslím, že se děsila té dvojí ztráty. Kdyby přišla i o~Viktora, co by jí zbylo? A~tak když pán přišel Viktora zastřelit, paní vlka vyhnala ven do vřesoviště, házela po něm kameny a křičela: „Uteč, Viky, uteč!“ Myslím, že to ubohé zvíře vůbec netušilo, co se stalo, protože dodnes se vrací a hledá lady Irmu.

\noindent
LADY ENID: Chudák Viktor. Chudáček chlapec. Chudinka Irma.

\noindent
JANE: Ubohý lord Edgar.

\noindent
LADY ENID: Ano, jistě, ubohý lord Edgar!

\noindent
JANE: Ale to nejpodivnější jsem vám ještě neřekla.

\noindent
LADY ENID: A~co?

\noindent
JANE: Sníh je jako mapa. Vydala jsem se po stopách. Viktorovy stopy odbočily stranou. Chlapce zabil vlk, po kterém zůstaly ve sněhu lidské stopy.

\noindent
LADY ENID: Ne! Chceš říct, že ten chlapec byl zavražděn?

\noindent
JANE: Chci říct, že to spáchal vlkodlak.

\noindent
LADY ENID: Co to je?

\noindent
JANE: Člověk, který na sebe v noci bere podobu vlka.

\noindent
LADY ENID: Ale to jsou pověry!

\noindent
JANE: Pověry. Nevysvětlitelné úkazy, se kterými si věda neví rady. Samozřejmě že všechno ukazovalo na Viktora. Chlapec lezl na skalku, upadl a rozbil si koleno. Nechal to milující zvíře, aby mu olízalo ránu. Zvíře okusilo krve. Divočina v něm se ozvala. Otočilo se proti chlapci a zaťalo své tesáky do jeho něžného hrdla. Dokonale logické vysvětlení. Ale co ty stopy ve sněhu? Nebylo by pro vlkodlaka dobré, aby vinu svalil na skutečného vlka?

\noindent
LADY ENID: Ukázalas to ostatním? Myslím ty stopy.

\noindent
JANE: Nebrali mě vážně. Tvrdili, že ty stopy patří mně. Že jsem je udělala já. Nechala jsem je přitom, aby mě neposlali do blázince. Málokdo věří v nadpřirozené jevy, když i ty přirozené jsou pro mnoho lidí záhadou.

\noindent
LADY ENID: Máš pravdu. Ale co ty stopy?

\noindent
JANE: Kéž bych je mohla předložit jako důkaz. Ale kdepak loňské sněhy jsou? A~pak, vlkodlakovo největší alibi spočívá v tom, že lidi v jeho existenci nevěří. Ale já už radši půjdu spát. Ozývá se moje revma.

\noindent
LADY ENID: Nech světlo svítit, Jane. Já si ještě chvíli budu číst.

\noindent
JANE: Tahle kniha vás může zajímat. Je to pojednání lorda Edgara o~starověkém Egyptě.

\noindent
LADY ENID: Děkuji.

\noindent
JANE: Moc dlouho neponocujte. K snídani budou uzenky a ledvinky, o~ty byste neměla přijít.

\noindent
LADY ENID: Jane, jak se ten chlapec vlastně jmenoval?

\noindent
JANE: To nevíte? Přece taky Viktor. Dobrou noc, lady Enid.

\smallskip

\noindent
\textit{Odejde.}

\smallskip

\noindent
LADY ENID: (\textit{Sedí zády k francouzskému oknu a čte si. Přes průsvitný závěs se za oknem ozářeným občasným bleskem objeví stín člověka. Kostnatá ruka sáhne po klice. Zaklepe prsty na sklo.}) Co to bylo? Sen nebo skutečnost? Bože můj, co to bylo? (\textit{Náhle se jedna tabulka francouzského okna roztříští, dovnitř se vsune ruka a otevře západku. Dveře se otevřou a do místnosti se vkrade vyzáblá postava. Paprsek světla ozáří bledou tvář, která se upírá na lady Enid}) Kdo jste? Co tu chcete? (\textit{Hodiny odbijí jednu hodinu v noci. Vetřelec zasyčí}) Co chcete? Proboha, co po mně chcete?

\smallskip

\noindent
\textit{Lady Enid se snaží doběhnout ke dveřím, ale vetřelec jí chytne za dlouhé kadeře, a jak si je otáčí kolem prstu, přitahuje lady Enid ke krbové římse. Lady Enid vyjme z vázy růži a vrazí její trny do vetřelcova oka. Vetřelec zasténá a pustí ji. Lady Enid běží přes místnost, vetřelec za ní. Ona ho bodne nůžkami z Janina šití. Vetřelec zavrávorá a otevřeným francouzským oknem vypadne do zahrady. Lady Enid se vrátí ke krbu a snaží se uklidnit.  S ulehčením si oddychne, ale vtom vetřelec vstoupí opět dovnitř. Sevře jí zezadu, zakryje jí ústa dlaní a vleče ji ke dveřím. Dveře zamkne, pak s ní přejde na druhou stranu ke druhým dveřím. Následuje několik výkřiků, když se vetřelec s vyceněnými tesáky zakousne do jejího hrdla a hlasitě chlemtá. Lady Enid zaječí. Z domu se ozvou kroky.}

\smallskip

\noindent
LORD EDGAR: (\textit{Za scénou}) Jane, slyšelas ten výkřik?

\noindent
JANE: (\textit{Za scénou}) Slyšela, milostpane. Odkud to šlo?

\noindent
LORD EDGAR: (\textit{Za scénou}) Pánbůh ví. Znělo to tak blízko a přece tak vzdáleně! Jen jsem to zaslechl, vyskočil jsem z postele a oblékl se.

\noindent
JANE: (\textit{Za scénou}) Všude je teď klid.

\noindent
LORD EDGAR: (\textit{Za scénou}) Ano. Ale jestli se mi to nezdálo, někdo tu křičel.

\noindent
JANE: (\textit{Za scénou}) To by se nám musel zdát stejný sen.

\noindent
LORD EDGAR: (\textit{Za scénou}) Kde je lady Enid?

\noindent

\noindent
JANE: (\textit{Za scénou}) Ona není s vámi?

\smallskip

\noindent
\textit{Lady Enid opět hlasitě zaječí.}

\smallskip

\noindent
LORD EDGAR: (\textit{Za scénou}) Zase jsem to slyšel. Prohledej dům, Jane! Odkud to šlo? Nevíš? (\textit{Lady Enid zaječí znovu}) Můj ty bože! Zase jsem to slyšel! (\textit{Pokouší se zvenku otevřít pravé dveře, ale ty jsou zamčené}) Enid! Enid! Jsi tam? Proboha, promluv! Nebo ty dveře vyrazím! (\textit{Buší do dveří}) Jane, dones páčidlo.

\noindent
JANE: (\textit{Za scénou}) A~kde je?

\noindent
LORD EDGAR: (\textit{Za scénou}) Ve sklepě. Pospěš si! Honem! Utíkej, utíkej! Ach, Enid, Enid!

\noindent
JANE: (\textit{Za scénou}) Už ho nesu.

\smallskip

\noindent
\textit{Edgar násilím otevře dveře a vpadne do místnosti.}

\smallskip

\noindent
NICODEMUS: (\textit{Za scénou}) Lady Enid! Lady Enid! Proboha ne! Lady Enid! (\textit{Vystoupí Nicodemus a v náručí nese bezvládnou lady Enid, dlouhé vlasy jí zakrývají obličej. Na noční košili má pár krůpějí krve}) Pomozte! Pomoc! (\textit{Vyjde s tělem pravými dveřmi ven}) Kam ji asi zatraceně nesu? Na onen svět?

\noindent
LORD EDGAR: (\textit{Jde za ním}) Co je? Co se stalo? Kdo ti to udělal?

\noindent
NICODEMUS: (\textit{Vrací se}) Kdo, nebo co? Ve vřesovišti jsem něco zahlíd.

\noindent
LORD EDGAR: Co něco?

\noindent
NICODEMUS: Vypadalo to jako psí lebka. Se psím tělem. Přes pohřební svíci zíraly mrtvolné oči, až se mi duše strachem zachvěla. (\textit{Náhle se za oknem objeví právě taková hrůzná tvář. Vydá ohlušující ryk, který přejde v řehot a hlava buší do okenní tabulky. Nicodemus zasípá zastřeným hlasem}) Tam! Tamhle je to zas!

% \smallskip
% 
% \noindent
% \textit{Hlava vyloudí chechtot, který je elektronicky nahalen.}
% 
% \smallskip
% 
% \noindent
% LORD EDGAR: Pánbůh nám pomáhej!
% 
% \noindent
% NICODEMUS: Ať už je to cokoli, já to dostanu! Jdu za tím!
% 
% \noindent
% LORD EDGAR: Ne, ne! Nechoď!
% 
% \noindent
% NICODEMUS: Musím!
% 
% \noindent
% LORD EDGAR: Vezmi si pušku! Blázne!
% 
% \noindent
% NICODEMUS: Ať se se mnou cokoli stane, tu příšeru musím následovat.
% 
% \smallskip
% 
% \noindent
% \textit{Odejde.}
% 
% \smallskip
% 
% \noindent
% LORD EDGAR: Počkej na mě! (\textit{Sundá ze zdi pušku})
%  
% \noindent
% NICODEMUS: (\textit{Za scénou}) Vidím to! Vidím! Kráčí to po zdi kolem vistárií.
% 
% \noindent
% LORD EDGAR: Noc je temná. Měsíc nevyšel. (\textit{Zvenka se ozývají zvířecí zvuky, hluk zápasu a děsivé výkřiky. Dveře se rozletí a dovnitř dopadne lidská noha, která dříve patřila Nicodemovi}) U~frasa! (\textit{Vyběhne ven a je slyšet zvenku}) Kudy? Kudy mám jít?
% 
% \noindent
% NICODEMUS: (\textit{Za scénou}) Tady jsem. Pomoc! Pomoc!
% 
% \smallskip
% 
% \noindent
% \textit{Za scénou se ozvou výstřely.}
% 
% \smallskip
% 
% \noindent
% JANE: (\textit{Vykoukne pravými dveřmi}) Nestřílel tu někdo?
% 
% \noindent
% LADY ENID: (\textit{Za scénou}) Jane! Jane!
% 
% \noindent
% JANE: Tady jsem, madam.
% 
% \noindent
% LADY ENID: Pojď sem. Potřebuju tě. Bojím se být sama.
% 
% \noindent
% JANE: Už jdu. Jen si vezmu svíčku, abych si posvítila na to, co vás bolí. Kletba Druidů se naplňuje. Keltská kletba Druidů.
% 
% \smallskip
% 
% \noindent
% \textit{Odejde.}
% 
% \noindent
% \textit{Jsou slyšet kroky a zvuk vlečeného břemene.}
% 
% \smallskip
% 
% \noindent
% NICODEMUS: (\textit{Vystoupí středem}) Viděl jsem to. Dotkl jsem se toho. Bojoval jsem s tím. Bylo to chladné a lepkavé jako mrtvola. Nic lidského to ale nebylo.
% 
% \noindent
% LORD EDGAR: (\textit{Vstoupí pod rukou Nicodemovou}) Proč by taky bylo? To nebyl člověk! Říkals přece, že to byl pes!
% 
% \noindent
% NICODEMUS: Chvíli to vypadalo jako vlk, chvíli jako žena. Utrhlo mi to nohu a zakouslo se do ní.
% 
% \noindent
% LORD EDGAR: U~ďasa. To snad ne!
% 
% \noindent
% NICODEMUS: Kdyby nebyla dřevěná, určitě by ji to sežralo.
% 
% \noindent
% LORD EDGAR: Ne!
% 
% \noindent
% NICODEMUS: Ano! Ano! Byl to duch! Ojídač mrtvol. Celou dobu to tak odporně srkalo. Jako by mi to vysávalo morek z kostí. Doteď cítím jeho přítomnost. Je blízko. Svatební postel. Dětské lůžko. Smrtelné lože. Přichází ona, bledý vampýr, v očích vichřice, a moře rudne krví pod jejími netopýřími křídly. Nastavuji ústa jejímu polibku. Očima mě sráží k zemi. Cítím, jak do mě noří svůj zelený tesák víly.
% 
% \smallskip
% 
% \noindent
% \textit{Za scénou se ozve vytí.}

% \smallskip
% 
% \noindent
% LORD EDGAR: Co to bylo?
% 
% \noindent
% NICODEMUS: Pouze vlk.
% 
% \noindent
% LORD EDGAR: Ne. To byl Viktor. Viktor se vrátil, aby mě strašil. (\textit{Vyráží pryč}) Dej mi svůj revolver. Tentokrát ho určitě dostanu. (\textit{U dveří se objeví kožešina}) Podívej. Tamhle je! Tentokrát mi neuteče.
% 
% \noindent
% NICODEMUS: (\textit{Mazlí se se svou dřevěnou nohou}) Ne! Pane! Nechoďte ven! Všechna pomoc je marná!
% 
% \smallskip
% 
% \noindent
% \textit{Do meluzíny se vloudí drásavý nářek.}
% 
% \smallskip
% 
% \noindent
% LORD EDGAR: Pusť mou nohu, skřete zatracený! Tvou duši pošlu do všech pekel!
% 
% \smallskip
% 
% \noindent
% \textit{Odejde.}
% 
% \smallskip
% 
% \noindent
% NICODEMUS: Ne! Milostpane! Je to Irma. Irma Vep! Duch ženy, který vydechuje lidský popel a osamělý kvílí do deštivé noci.
% 
% \smallskip
% 
% \noindent 
% \textit{Jsou slyšet výstřely, rychlé kroky a další vytí v dálce.}
% 
% \smallskip
% 
% \noindent 
% JANE: (\textit{Vtrhne dovnitř}) Co je tu za křik? Vzbudil bys mrtvého.
% 
% \noindent 
% NICODEMUS: Pán je zase v~ráži. Loví.
% 
% \noindent 
% JANE: Zase vlka?
% 
% \noindent 
% NICODEMUS: Tentokrát je to prý určitě Viktor.
% 
% \noindent 
% JANE: Viktor?
% 
% \noindent 
% NICODEMUS: Aspoň to tvrdí.
% 
% \noindent 
% JANE: Tak tu jen tak nekoukej a jdi za ním. Běž mu pomoct. Buď aspoň k něčemu dobrý!
% 
% \noindent 
% NICODEMUS: Ne. To po mně nechtěj. Venku na vřesovišti je cosi tak strašného, až člověku tuhne krev v žilách.
% 
% \noindent 
% JANE: Strašpytle! Jestli mu nepomůžeš ty, půjdu mu pomoct sama!
% 
% \noindent 
% NICODEMUS: No dobře, ženská. Tak počkej, než si přišroubuju nohu. (\textit{Jde ven, hlasitě si nohu přišroubuje a vrací se zpátky})
% 
% \noindent 
% JANE: Jestli ti to neukouslo i něco víc než jenom nohu. Jako bys přestal být chlap. (\textit{Sundá ze stěny pušku})
% 
% \noindent  
% NICODEMUS: Dej sem ten kvér, slečinko. To není nic pro ženskou! (\textit{Přetahují se o~pušku})
% 
% \noindent  
% JANE: Pusť! Pusť! Jdi mi z cesty! Lord Edgar mě potřebuje.
% 
% \smallskip
% 
% \noindent 
% \textit{Z pušky vyjde rána a kulka trefí obraz. Ten začne krvácet.}
% 
% \smallskip
% 
% \noindent
% NICODEMUS: Koukni se, cos udělala. Trefila jsi lady Irmu. A~obraz krvácí! (\textit{Vyrve jí pušku z rukou a odejde za lordem Edgarem}) Lorde Edgare!
% 
% \noindent 
% JANE: (\textit{Volá za ním}) Vem to kolem náhonu a vzhůru k močálu. Na druhou stranu, Nicodeme, na druhou stranu! Zkratkou přes cedrový háj! Pospíchej! Rychleji!
% 
% \smallskip
% 
% \noindent
% \textit{Vystoupí lady Enid.}
% 
% \smallskip
% 
% \noindent
% LADY ENID: Kde je lord Edgar?
% 
% \noindent
% JANE: Prohledává močál. Zdálo se mu, že zahlédl Viktora.
% 
% \noindent
% LADY ENID: Vlka, nebo chlapce?
% 
% \noindent
% JANE: Oba.
% 
% \smallskip
% 
% \noindent	
% \textit{Zatmít.}

\bigskip






\podpis{přeložil Jiří Josek}