\section{Mark Twain: Dobrodružství \\ Huckleberryho Finna}

\noindent
\textit{(kapitola z románu)}

\medskip

\noindent
5

\noindent
Musel jsem zavřít dveře. Pak jsem se otočil – a on byl tady. Vždycky jsem se ho bál, protože mě hrozně mlátil. Myslel jsem, že se ho bojím i teď; ale v tu ránu mi došlo, že jsem se splet. Teda v~prvním leknutí mi trochu vyrazil dech – vůbec jsem ho tu nečekal; ale pak mi najednou došlo, že s tím strachem to nebude tak žhavý.

Tátovi bylo k~padesátce a taky na to vypadal. Měl dlouhý slepený mastný vlasy, visely mu přes obličej jako šlahouny psího vína a pod nima se blejskaly oči. Ty vlasy měl úplně černý, ani trochu prošedivělý; a dlouhý rozježený fousy zrovna tak. Jeho tvář, nebo to, co z ní prosvítalo, byla úplně bezbarvá; bílá, ale ne bílá jako u~jinejch lidí; taková hnusně bílá, až člověku běhal mráz po zádech; nechutně bílá jako žabí břicho nebo leklá ryba. Táta měl na sobě starý hadry – jinak se to nazvat nedalo. Seděl s nohou rozvalenou přes koleno; z rozdrbaný boty mu čouhaly dva prsty, kterejma občas zahejbal. Na zemi ležel jeho klobouk – černej kastrol s promáčknutým dýnkem, taková divná hučka.

Stál jsem a vejral na něj; on seděl a vejral zas na mě a při tom se trochu houpal na židli. Položil jsem svíčku. Všim jsem si, že okno je vytažený; takže sem vylez přes kůlnu. Prohlíd si mě shora zdola, a pak povídá:

„Seš ňákej vypiglovanej. Asi myslíš, že je z tebe velkej pán, \textit{viď že jo}?“

„Možná jo, možná ne,“ povídám.

„Bacha na hubu, jo?“ povídá on. „Stal se z tebe moc velkej frajer, co sem byl pryč. No, já ti srazím hřebínek, než to s tebou skoncuju. Prej seš taky študovanej – umíš číst a psát. Chceš se kasat na vlastního tátu, kerej to neumí? Ale \textit{já} ti to vyženu z hlavy, uvidíš! Kdo ti řek, že se máš montovat do takovejch pitomostí, co? Kdo ti to dovolil?“

„Vdova. Vdova mi to řekla.“

„Jo vdova? A~kdo řek vdově, že může strkat rypák do cizích věcí?“

„To jí nikdo neřek.“

„No, já jí ukážu, jesli se může do všeho montovat. Tak hele: zakazuju ti chodit do školy, rozumíš?  Já jim ukážu, jesli budou kluka učit, aby ohrnoval nos nad vlastním tátou a dělal ze sebe něco \textit{lepčího}! A~běda jak tě nachytám, že couráš kolem školy, slyšíš? Tvoje máma číst neuměla a neuměla ani psát, až do svý smrti! Nikdo z naší rodiny se to \textit{až do smrti} nenaučil! \textit{Já} taky ne! A~ty se tu budeš takhle nafukovat? Já ti to zatrhnu, slyšíš? Tak schválně, předveď mi, jak čteš.“

Vzal jsem knížku a začal o~generálu Washingtonovi a o~těch válkách. Když jsem čet asi půl minuty, táta se rozmách a vyrazil mi knihu z ruky, až přelítla celej pokoj.
 
„No jo. Číst umíš. Moc sem tomu nevěřil, když si mi to řek. Tak koukni: s frajeřinkama je utrum. Já ti to zatrhnu. Pohlídám si tě, ty rozumbrado; a jesli tě načapám někde kolem školy, spráskám tě jak psa! Natošup tě postavím do latě! Povedenej synáček!“

Vzal do ruky malý modrožlutý obrázek, na kterém byly krávy s pasáčkem, a povídá: „Co je to?“

„To jsem dostal za to, že se dobře učím.“

Táta obrázek roztrhal a povídá: „Ode mě dostaneš něco lepčího. Přetáhnu tě bejkovcem.“

Ještě chvíli seděl, něco si mumlal pod fousy a pak povídá: „Stal se z~tebe navoněnej seladon, co? Máš postel s~peřinama, zrcadlo a na zemi kobereček  – a tvůj vlastní táta musel spát v koželužně s~prasatama. Povedenej synáček! Ale já z tebe ty frajeřiny namouduši vymlátím, než to s~tebou skoncuju. A~jako by nestačilo, že se tak nafukuješ, říkaj, že seš taky bohatej. Tak co? – Jak je to? “
 
„Lžou – tak je to.“

„Hele, dej si bacha, jak se mnou mluvíš; už mi pomalu začíná docházet trpělivost, tak si na mě neotvírej hubu. Sem ve městě už dva dny, a celou dobu slyším ze všech stran, že seš v balíku. Slyšel sem to i kolem řeky, dost daleko po proudu. Proto sem tady. Zejtra mi ty peníze přineseš – skasíruju tě.“

„Já žádný peníze nemám.“

„Lžeš. Máš je u~soudce Thatchera. Takže je přineseš a já tě skasíruju.“

„Říkám, že žádný peníze nemám. Zeptej se soudce; ten ti to dosvědčí.“

„Jasně že se ho zeptám; musí je vysolit, a když ne, tak budu vědět proč. Řekni, kolik máš u~sebe? Hned tě skasíruju.“

„Nemám víc než dolar, a ten potřebuju na\ldots“

„To je fuk, nač ho potřebuješ; koukej ho navalit.“

Táta vzal můj dolar, kousnul do něj, jestli je pravej, a pak řek, že si skočí do města pro flašku whisky; že prej od rána suší hubu. Vylez ven na střechu kůlny, ale znova nakouk dovnitř a vynadal mi, že jsem samá frajeřina a dělám ze sebe něco lepčího; a když už jsem myslel, že je opravdu pryč, vrátil se ještě jednou, strčil hlavu do okna a řek, ať si dám bacha s~ tou školou, protože si mě pohlídá a spráská mě jak psa, jestli tam nepřestanu chodit.

Druhý den byl opilej, šel k soudci Thatcherovi, vyhrožoval mu a domáhal se těch peněz; jenže mu to nebylo nic platný, a tak začal sprostě nadávat, že když to nejde po dobrým, pude to k soudu.

Soudce a vdova podali žádost, abych byl otci úředně odebrán a svěřen do pěstounské péče někomu z nich; ale byl tam novej soudce, co se teprv nedávno přistěhoval a o~tátovi nic nevěděl, a ten řek, že soudy se do takovejch věcí nemají plést a rozbíjet rodinu, pokud se to dá vyřešit jinak; on že prej by nechtěl odloučit dítě od otce. A~tak soudce Thatcher s~vdovou museli tu žádost stáhnout.

Tátova radost neznala mezí. Řek, že mě ztříská bejkovcem, až budu samý jelito, jestli mu neseženu nějaký prachy. Půjčil jsem si od soudce Thatchera tři dolary, táta si je vzal a zlinkoval se; pak se naparoval, sprostě nadával, hulákal a tropil výtržnosti; skoro až do půlnoci dělal rambajz po celým městě, než ho sebrali a hned ráno odsoudili zase na tejden do basy. Táta ale řek, že \textit{on} je spokojenej; teď je pánem svýho syna a ten mu dopomůže k blahobytu.

Když ho pustili, ten novej soudce prohlásil, že z něj udělá člověka. Vzal ho k sobě domů, hezky a čistě ho oblíknul, pozval ho na snídani, oběd a večeři s~celou rodinou, prostě jako by se znali odjakživa. Po večeři mu domlouval, že alkohol je metla lidstva a tak podobně, až se táta rozplakal a řek, že byl až doteďka hlupák a svou hloupostí si zničil život; ale chce obrátit list a začít znova, aby se za něj nikdo nemusel stydět, a jen doufá, že soudce mu pomůže a nebude jím opovrhovat. Soudce pravil, že by ho za ta slova nejradši objal; pak se taky rozplakal a jeho žena spustila další vodopád; táta řek, že mu až doteďka nikdo nerozuměl, a soudce řek, že tomu věří. Táta řek, že když je člověk na dně, potřebuje špetku soucitu, a soudce řek, že je to tak; a zase svorně zaslzeli. A~když bylo načase jít spát, táta vstal, napřáh k nim ruku a povídá:
„Heleďte, pánové a dámy, zde je má ruka; chopte se jí a stiskněte ji. Až doteďka patřila vožralýmu praseti; ale teď už ne; odteďka je to ruka chlapa, kerej začal novej život a radši chcípne, než aby se vrátil k tomu starýmu. Někam to zapište  – a pamatujte si, že tydle slova sem řek já. Moje ruka je teďka čistá; nebojte se ji stisknout.“

A~tak mu potřásali rukou, jeden po druhým, jak na ně přišla řada, a všichni plakali. Soudcova žena mu ji dokonce políbila. Táta stvrdil svůj slib i písemně – místo podpisu udělal křížek. Soudce se nechal slyšet, že je to nejslavnostnější okamžik v~dějinách, nebo něco na ten způsob. Pak tátu uložili do parádního pokoje, co měli pro hosty, ale během noci ho přepadla ukrutná žízeň, tak vylez na stříšku verandy, sjel po sloupku a vyhandloval svůj novej kabát za džbán kořalky. Pak zase zalez dovnitř a nasával jako za starejch časů. Před svítáním se vyštrachal z~okna, nalitej jak dělo, skutálel se ze stříšky, nadvakrát si zlomil levou ruku a skoro zmrznul, než ho tam už za světla někdo našel. A~když se přišli podívat do toho svýho pokoje pro hosty, museli nejdřív důkladně ohledat terén, aby se vyhnuli všemožnejm nástrahám.

Soudce se dost vztekal. Řek, že ten, kdo by chtěl tátu napravit, by na něj podle všeho musel jít s flintou, jiná možnost že ho nenapadá. 

\podpis{přeložila Jana Mertinová}