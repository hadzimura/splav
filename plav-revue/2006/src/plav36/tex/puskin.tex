 \section{Alexandr Sergejevič Puškin:\\19. října 1825} 

\begin{verse}
Les odhazuje uschlých listů nach,\\
mrazem je pole postříbřeno lehce.\\
Den nahlédne sem, jak když se mu nechce,\\
a za kopci se skryje v~tmoucích tmách.\\
Hoř, krbe, v~cele poustevníka mnicha.\\
Teď teprve, mé víno, ocením,\\
jak opojená hruď zas žárem dýchá,\\
jak na chvilku mě zbavíš trápení.

\medskip

Jsem smutný, že tu nemám přítele,\\
s~nímž samotu bych zapil, pak ho hostil,\\
ruku mu tiskl s~vřelou srdečností\\
a přál mu štěstí, zdraví, hodně let.\\
Já piju sám. Má fantazie živá\\
nadarmo volá druhy z~daleka,\\
krok ani hlas se kolem neozývá\\
a duše milou zprávu nečeká.

\medskip

Já piju sám -- a kdesi přátelé\\
na březích Něvy na mě vzpomínají.\\
Chybím však jenom já v~tom slastném kraji?\\
Kdo další schází v~kruhu veselém?\\
Kdo před vzpomínkou na bujaré časy\\
dal přednost hrám, jež skýtá společnost?\\
Kdo na zavolání se přestal hlásit,\\
kdo nepřišel, kdo bratrů má už dost?

\medskip

Náš kučeravý pěvec nepřišel\\
a na kytaru nehrál sladkohlasou.\\
Spí v~Itálii, ukryt jižní krásou\\
a dláto přátelské mu nevpíše\\
na ruský hrob pár slůvek v~rodné řeči,\\
aby si o~něm, věrném druhu svém,\\
moh jiný pěvec ze severu přečíst,\\
až v~bludných toulkách navštíví tu zem.

\medskip

Jestlipak v~kroužku přátel sedíš teď,\\
ty cizích obloh lačný milovníku,\\
či brázdíš tropy kolem obratníků\\
a moří na půlnoci věčný led?\\
Jen šťastně pluj! Už z~lycejního prahu\\
jsi na loď přešel samý žert a smích\\
a nalezl tam životní svou dráhu,\\
ty milé dítě moří bouřlivých.

\medskip

Tys na toulkách si doved uchovat,\\
co v~dávných letech mívali jsme rádi.\\
Šum lycea a nevinné hry mládí\\
sis nechával i vprostřed bouří zdát.\\
Tys podával nám ruku z~konce světa,\\
měl v~mladé duši naše podoby\\
a říkal nám, že odloučení léta\\
soud osudu nám nejspíš způsobí.

\medskip

Je krásný ten náš svazek, přátelé,\\
a jako duše věčný, jednostejný,\\
tak bezstarostný, pevný, neochvějný.\\
Vždyť sdružily ho múzy veselé.\\
Ať jakkoli si s~námi osud hraje\\
a planým štěstím zkouší hlavu mást,\\
jsme stejní -- svět jsou pro nás cizí kraje\\
a Carské selo naše pravá vlast.

\medskip

Já střídal kraje jeden za druhým,\\
hnán hromem, chycen do osudu sítí.\\
Přátelství nové začalo mi svítit\\
a já mu znaven složil hlavu v~klín.\\
Pak zmaten řek jsem, jak mě život zkrušil,\\
a s~důvěrou, již ve mně nezlomil,\\
jsem druhům jiným otevřel svou duši.\\
Bratrství jejich ale zhořklo mi.

\medskip

Jenže i v~téhle dáli setmělé,\\
kde vládne chlad a meluzína skučí,\\
já přece tři z~vás sevřel do náručí,\\
mé duše znejmilejší přátelé!\\
Tys, Puščine, byl první z~drahých hostí\\
a rozjasnil jsi tmu mou docela.\\
Zařídils básníkovi v~nemilosti\\
místo dne vyhnanství Den lycea.

\medskip

Byls, Gorčakove, dítě štěsteny,\\
chladný lesk její přízně dál ti sluší.\\
Uchoval sis však čest a volnou duši\\
a pro přátele jsi se nezměnil.\\
Nás odloučením záhy osud poctil,\\
mne od tebe hned odnes do dáli,\\
a přece jsme jak shodou okolností\\
se na venkovské cestě objali.

\medskip

Když ten hněv zlého osudu mě stih\\
a odvál prudkou bouří do daleka,\\
pro všechny cizí, na  tebe jsem čekal,\\
ty luzný pěvče panen permeských.\\
Tys přišel, synu líné inspirace,\\
můj Dělvigu, a probudil tvůj hlas\\
žár srdce, jenž už v~hrudi mé se ztrácel.\\
Pak osudu jsem blahořečil zas.

\medskip

Už od dětství duch písní hořel v~nás\\
a do čarovných záchvěvů nás halil,\\
už od dětství dvě múzy přilétaly.\\
Náš osud krášlil jejich sladký hlas.\\
Mě však už tehdy potlesk uchvacoval.\\
Tys múzám pěl a citům nebránil.\\
Já k~talentu se nezřízeně choval,\\
tys génia si pěstil v~ústraní.

\medskip

Kdo slouží múzám, ať se oprostí\\
od věcí malých, vidí jen ty vznosné!\\
Mládí nás ale šálí neúprosně\\
a opájí nás třpytem marnosti.\\
Když procitnem, je pozdě zpět sny mámit\\
a ohlížet se na života běh\ldots\\
Viď, Vilgelme, že tak to bylo s~námi,\\
můj bratře v~múzách, bratře v~osudech?

\medskip

Je čas, je čas! Pár ran se zacelí.\\
Nic, zač by stálo trpět, v~světě není.\\
Ukryjme život v~stínu osamění.\\
Máš zpoždění, já čekám, příteli.\\
Jen přijď, ať city v~srdci utrápeném\\
rozezní řeč jak zvon svým úderem!\\
Dny bouřlivé a Kavkaz připomenem,\\
Schillera, lásku, slávu proberem.

\bigskip

Jen veselte se, přátelé! Ten čas\\
se blíží, kdy se básník zase přidá.\\
Teď pamatujte, co vám předpovídá\\
-- rok uplyne a budu s~vámi zas.\\
Mé vroucí přání splní se už brzy,\\
rok uplyne a uhlídáme se.\\
Těch zvolání, těch opravdových slzí,\\
těch číší, co se k~nebi povznese!

\medskip

Tu první plnou, ať je v~hlavě šum,\\
a na náš svazek pijme ji až do dna!\\
Pak, jásající múzo naše rodná,\\
nám požehnej -- ať žije lyceum!\\
Na moudré strážce ztřeštěného mládí,\\
ať ještě žijí či je kryje zem,\\
děkovnou číši vypijeme rádi.\\
To zlé je pryč, to dobré vzpomenem.

\medskip

A~znovu plnou, plnou nalít všem,\\
a přípitek zas do dna pijme, bratři!\\
Uhodnout ale zkuste, komu patří.\\
Carovi hurá! Tomu připijem!

\medskip

Je člověk, a tak pochyba v~něm dříme,\\
podléhá vášním, klepům, poměrům.\\
Že bez viny nás stíhá, odpustíme.\\
On dobyl Paříž, stvořil lyceum.

\medskip

Jen slavte, dokud smí vás nosit zem.\\
Náš kroužek, běda, řídne každou chvíli.\\
Ten v~dáli je, ten do hrobu se schýlil.\\
My vadnem, osud sčítá za dnem den.\\
tak, aniž víme, skláníme se, chladnem.\\
Odkud jsme vyšli, teď jdem zase tam.\\
Na koho z~nás to v~stáří ale padne,\\
den lycea že slavit bude sám?

\medskip

Ten chudák -- vprostřed nových pokolení\\
bude jak cizí, nevítaný host.\\
Zastře si oči, spatří to, co není,\\
tu naši dávnou družnou společnost.\\
Pak číší vína ať sám sebe hostí,\\
ať, třeba smuten, uctí ten den rád,\\
tak jako já, ač sám a v~nemilosti,\\
jsem ho dnes prožil bez žalů a ztrát.

\podpis{přeložil Milan Dvořák}
\end{verse}