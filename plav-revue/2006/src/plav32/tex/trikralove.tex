\section{Tři králové: teorie}

\noindent
Tři králové nebyli nikdy oficiálně svatořečeni. Podle legendy nalezla jejich ostatky císařovna Helena, přivezla je do Konstantinopole (nynější Istanbul) a darovala je milánskému biskupovi. Když Barbarossa roku 1158 dobyl Milán, převezl ostatky Tří králů  kolínský arcibiskup Reinhard z Dasselu do Kolína nad Rýnem -- učinil tak roku 1164. Ostatky byly uloženy do chrámového pokladu starého kostela sv. Petra. Relikvie nebyly pouze cílem mnoha poutníků, ale byly považovány za státní symboly. Jeden z nejvýznamnějších zlatotepců tehdejší doby, Mikuláš z Verdunu, pro ně zhotovil skříňku z drahocenných materiálů.

Relikviář má podobu baziliky. Nad dvěma schránkami je umístě\-na třetí. V horní části skříňky jsou pod jednoduchými oblouky trůnící apoštolové s modely měst, jež odkazují na biskupství, která založili. Přední štítová strana představuje Krista jako soudce obklopeného anděly. Pod ním se Tři králové blíží k Matce Boží. Na protilehlé straně je Kristův křest jako odkaz na první zjevení Páně, které se původně slavilo ve stejný den jako svátek Tří králů.

Od roku 1948 je relikviář umístěn na mramorovém podstavci za hlavním oltářem kolínského dómu.

Svátek Tří Králů odjakživa doprovázely různé lidové zvyky. Ve střední Evropě je zažité žehnání domů, při němž se na dveře svěcenou křídou píší písmena C+M+B, u~nás v Čechách obvykle K+M+B. Nejsou to ale patrně počáteční písmena jmen ,,Třech králů``: Kašpar, Melichar a Baltazar, jak se lidově traduje, nýbrž zkratka latinského ,,Christus mansionem benedicat``: ,,Kristus žehnej tomuto domu``.

Tříkrálový den znamená v Evropě faktický konec Vánoc.

Ve folklorní tradici začíná nyní masopust -- až do předvelikonočního půstu trvající doba zábav a plesů. V českých zemích měla tříkrálová obchůzka duchovního s ministranty, kostelníkem, učitelem a žáky neměnnou podobu od třicetileté války, kdy v tomto regionu začala rekatolizace. S modlitbou a zpěvem se domy vykuřovaly kadidlem a vykropovaly svěcenou vodou a na dveře či rám se psala zmíněná tři písmena a datum. Později se v některých vsích ujal této povinnosti kantor, který za ni dostával výslužku.

6. lednem konči dvanáctidenní období Vánoc. I~divadelní hra Večer tříkrálový od Williama Shakespeara se jmenuje v originále The Twelveth Night\ldots Na dvanáct dní Vánoc upomíná i anglická koleda:

\medskip

\section{The Twelve Days of Christmas} 

\begin{verse}
\looseness-1
On the first day of Christmas, my true love sent to me \\
A~partridge in a pear tree.

\smallskip

On the second day of Christmas, my true love \\ \hspace*{\fill} sent to me \\
Two turtle doves, \\
And a partridge in a pear tree. 

\smallskip

On the third day of Christmas, my true love sent to me \\
Three French hens, \\
Two turtle doves, \\
And a partridge in a pear tree. 

\smallskip

On the fourth day of Christmas, my true love sent to me \\
Four calling birds, \\
Three French hens, \\
Two turtle doves, \\
And a partridge in a pear tree.

\smallskip

On the fifth day of Christmas, my true love sent to me \\
Five golden rings, \\
Four calling birds, \\
Three French hens, \\
Two turtle doves, \\
And a partridge in a pear tree. 

\smallskip

On the sixth day of Christmas, my true love sent to me \\
Six geese a-laying, \\
Five golden rings, \\
Four calling birds, \\
Three French hens, \\
Two turtle doves, \\
And a partridge in a pear tree.

\smallskip

On the seventh day of Christmas, my true love \\ \hspace*{\fill} sent to me \\
Seven swans a-swimming, \\
Six geese a-laying, \\
Five golden rings, \\
Four calling birds, \\
Three French hens, \\
Two turtle doves, \\
And a partridge in a pear tree.

\smallskip

On the eighth day of Christmas, my true love sent to me \\
Eight maids a-milking, \\
Seven swans a-swimming, \\
Six geese a-laying, \\
Five golden rings, \\
Four calling birds, \\
Three French hens, \\
Two turtle doves, \\
And a partridge in a  pear tree.

\smallskip

On the ninth day of Christmas, my true love sent to me \\
Nine ladies dancing, \\
Eight maids a-milking, \\
Seven swans a-swimming, \\
Six geese a-laying, \\
Five golden rings, \\
Four calling birds, \\
Three French hens, \\
Two turtle doves, \\
And a partridge in a pear tree.

\smallskip

On the tenth day of Christmas, my true love sent to me \\
Ten lords a-leaping, \\
Nine ladies dancing, \\
Eight maids a-milking, \\
Seven swans a-swimming, \\
Six geese a-laying, \\
Five golden rings, \\
Four calling birds, \\
Three French hens, \\
Two turtle doves, \\
And a partridge in a pear tree.

\smallskip

On the eleventh day of Christmas, my true love \\ \hspace*{\fill} sent to me \\
Eleven pipers piping, \\
Ten lords a-leaping, \\
Nine ladies dancing, \\
Eight maids a-milking, \\
Seven swans a-swimming, \\
Six geese a-laying, \\
Five golden rings, \\
Four calling birds, \\
Three French hens, \\
Two turtle doves, \\
And a partridge in a pear tree.

\smallskip

On the twelfth day of Christmas, my true love \\ \hspace*{\fill} sent to me \\
Twelve drummers drumming, \\
Eleven pipers piping, \\
Ten lords a-leaping, \\
Nine ladies dancing, \\
Eight maids a-milking, \\
Seven swans a-swimming, \\
Six geese a-laying, \\
Five golden rings, \\
Four calling birds, \\
Three French hens, \\
Two turtle doves, \\
And a partridge in a pear tree! 
\end{verse}