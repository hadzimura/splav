\section{O. Henry: Dary tří králů}

\medskip

\noindent
Dolar a osmdesát sedm centů. Víc nic. A~z toho šedesát centů v jednocentových mincičkách zachráněných po jedné po dvou po nelítostném smlouvání s hokynářem, zelinářem nebo řezníkem, až tváře hořely nevyřčeným doznáním škudlilství, jež takový způsob nakupování prozrazuje. Třikrát to Della přepočítala. Jeden dolar a osmdesát sedm centů. A~zítra budou vánoce.

Zjevně se nedalo dělat nic jiného než plácnout sebou na ošuntělé kanapátko a skučet. Della to udělala. Čímž potvrdila mravní naučení, že život je směsice vřískání, stýskání a výskání, přičemž stýskání převažuje.

Zatímco se nám paní domu zvolna propracovává od prvního stadia ke druhému, porozhlédněme se po bytečku. Podnájem za osm dolarů týdně. Interiér se nedá popsat než větou holou.

V chodbě v přízemí schránka na dopisy, v níž by ses dopisu nedočkal, a elektrický zvonek, z něhož bys zvonění nevymáčkl. K tomu příslušná tabulka hlásající ,,James Dillington Young``.

Honosné Dillington uprostřed jména vylétlo na stožár v předešlém období blahobytu, kdy jeho majitel pobíral třicet dolarů týdně. Nyní, kdy se příjem scvrkl na dvacet dolarů, písmena ve slově Dillington zašla, rozmazala se a jako by seriózně začala uvažovat o~tom, že se i ona scvrknou ve skromné a nevýbojné D. Nicméně kdykoli přišel James Dillington Young domů a vyšplhal se ke svému bytu, byl oslovován ,,miláčku`` a náramně objímán paní Youngovou, kterou už znáte pod jménem Della. Potud vše v pořádku.

Della se vyplakala a začala si hadříkem a pudrem restaurovat tváře. Stála u~okna a bez zájmu sledovala ošklivou šedivou kočku na ošklivém šedivém plotě na ošklivém šedivém dvorku. Zítra bude Štědrý den a ona má jen dolar osmdesát sedm centů na dárek pro Jima. Měsíce už střádá cent k centíku a tohle je výsledek. S dvaceti dolary týdně si vyskakovat nemůžeš. Výdaje byly vyšší, než čekala. Vždycky jsou vyšší. Dolar a osmdesát sedm centů na dárek pro Jima. Pro jejího Jima. Kolikrát si v šťastných chvilkách představovala, co hezkého mu koupí. Něco hezkého a vzácného a přepychového -- prostě něco aspoň trochu hodného té cti, že mu může patřit.

Mezi dvěma okny pokoje viselo zrcadlo. Možná jste takové už viděli v podnájmu za osm dolarů týdně. Velice útlá a velice bystrá osoba sledující svůj odraz v soustavě podélných šmouh získá poměrně slušnou představu o~svém zevnějšku. Štíhlá Della to umění zvládala.
Náhle se prudce odvrátila od okna a přistoupila k zrcadlu. Oči jí zářivě planuly a z tváře jí během dvaceti vteřin zmizela veškerá barva. Rychlým pohybem si uvolnila vlasy a nechala je splývat rozpuštěné v celé jejich délce.

Ve vlastnictví Youngových byly dvě věci, na které byli svrchu jmenovaní patřičně hrdí. Jednak to byly Jimovy zlaté hodinky, patřící kdysi otci a předtím praotci. Jednak Delliny vlasy. Kdyby za oknem na protilehlé straně světlíku bydlela sama královna ze Sáby, Della by si neodpustila tu a tam nechat své rozpuštěné vlasy po koupeli povlávat z okna ven a degradovat tak veškeré šperky a cennosti Jejího Veličenstva na pouhou veteš. A~kdyby sám král Šalamoun byl tady v baráku domovníkem a suterén měl přecpaný svými poklady, Jimovi by to nedalo, aby pokaždé, když půjde kolem, nevytáhl hodinky, jen aby viděl, jak si ten monarcha rve závistí vous.

Teď se tedy Delliny krásné vlasy hrnuly v proudech kolem ní, čeřily se a blyštěly jak kaskády hnědých vod. Spadaly jí až pod kolena a zahalovaly ji celou jako roucho. Spěšně a nervózně Della své vlasy opět spoutala. Jen na chvilku zaváhala a zarazila se v nehybném postoji, slzička či dvě ukáply na odraný koberec.

A~už měla na sobě svůj starý hnědý kabátek a svůj starý hnědý klobouk. Se zavířením sukní a pořád ještě s jiskrou v oku proplula dveřmi a pelášila po schodech dolů na ulici a dála dál.

Zarazila se až před cedulí: ,,Madam Violetová. Vlasové doplňky všeho druhu``. Vyběhla do prvního patra, udýchaná, jen tak tak se stačila sebrat. Madam, mohutná, tvarohovitě bílá netýkavka, byla na hony vzdálená fialce.

,,Koupíte mé vlasy?`` zeptala se Della.

,,Kupuju vlasy,`` řekla madam. ,,Sundejte si klobouk, ať se můžu podívat.``

A~už se zas rozlily hnědé kaskády.

,,Dvacet dolarů,`` řekla madam a zkušenou rukou si potěžkala tu záplavu.

,,Dejte mi to rychle,`` řekla Della.

Ojoj! Ale následující dvě hodinky se vznášela jak na křídlech motýlích. Radši však zanechme otřepaných metafor. Prostě rabovala krámy v honbě za dárkem pro Jima.

Nakonec ho našla. Jedině pro Jima byl stvořen, pro nikoho jiného. V žádném z obchodů, které prošmejdila skrz naskrz, nenašla nic podobného. Byl to platinový řetízek jednoduchého střídmého vzoru, patřičně proklamující svou cenu pravou svou podstatou, a ne nějakými křiklavými parádičkami. Takto se vždy pozná dobrá věc. A~tato byla skutečně hodná Jimových HODINEK. Ledva řetízek uviděla, věděla, že musí patřit Jimovi. Patřili k sobě. Neokázalost a hodnota -- to charakterizovalo oba dva. Odevzdala dvacet jedna dolarů a s osmdesáti sedmi centy pospíchala domů. S takovýmto řetízkem na hodinkách si může Jim plným právem s chutí ověřovat čas v jakékoli společnosti. Vždyť jakkoli skvostné hodinky má, dosud se na ně musel často dívat pokradmu, protože místo na řetízku visely mu na kusu umolousaného řemínku.

Když se Della vrátila domů, měla co dělat, aby své rozdychtění zchladila trochou rozumu a rozvahy. Vzala si želízka na vlasy, zapálila plynový hořák a jala se napravovat pohromu, kterou jí velkorysost spojená s láskou napáchaly na hlavě. A~to je vždy nesmírně těžký úkol, přátelé -- kolosálně.

Za tři čtvrtě hodiny měla hlavu pokrytou drobnými hustými kudrlinkami, takže se náramně podobala uličníkovi z páté obecné. Prohlédla si svůj obraz v zrcadle pečlivě a kriticky.

,,Jestli mě Jim nezabije hned,`` sdělila si z očí do očí, ,,tak mi poví, že vypadám jak hopsanda z kabaretu. Ale co jsem mohla dělat? Co jsem mohla dělat s dolarem a osmdesáti sedmi centy?``

V sedm hodin už byla káva uvařena a na kraji sporáku vyčkávala rozpálená pánvička na své dvě porce řízků.

Jim se dosud nikdy neopozdil. Della složila řetízek v dlani a posadila se na roh stolu hned u~dveří. A~pak už slyšela jeho kroky na schodech v přízemí a jen maličko pobledla. Ráda pronášela krátké tiché modlitbičky za ty nejprostší věci a nyní zašeptala: ,,Bože, dej, ať se mu ještě líbím.``
Dveře se otevřely, Jim vstoupil a zavřel za sebou. Měl propadlé tváře a na nich vážný výraz. Chudák, bylo mu sotva dvaadvacet, a už má být hlavou rodiny. Potřeboval by nový kabát a chodí bez rukavic.

Stanul ve dveřích a zarazil se jak křepelák, když ucítí křepelku. Oči upíral na Dellu a bylo v nich něco, čemu nerozuměla a co ji děsilo. Nebyl to hněv ani úžas, nesouhlas ani vztek, žádný z pocitů, na které byla připravena. Pouze na ni upřeně zíral a ona nevěděla, co jeho pohled znamená.
Svezla se ze stolu a přistoupila k němu.

,,Jime, miláčku,`` zasténala, ,,nedívej se na mě takhle. Nechala jsem si ustřihnout vlasy a prodala je, protože bych Vánoce nepřežila, kdybych pro tebe neměla dárek. Zase mi narostou, uvidíš! Viď, že ti to nevadí? Mně vlasy rostou strašně rychle. Řekni ,Veselé Vánoce‘, Jime, a buďme šťastní. Ani nevíš, jaký krásný -- jaký nádherný dáreček pro tebe mám.``

,,Ty sis dala ostříhat vlasy?`` namáhavě ze sebe vysoukal Jim; jako by se k této do očí bijící skutečnosti musel dopracovat po značném duševním úsilí.

,,Dala jsem si je ostříhat a prodala jsem je,`` řekla Della. ,,Copak mě už takhle nemáš rád? Jsem to já, i když nemám vlasy.``

Jim se pátravě rozhlížel po místnosti.

,,Takže vlasy už nemáš?`` řekl s výrazem takřka idiotským. ,,Hledat je nemusíš,`` řekla Della. ,,Prodala jsem je. Jsou pryč. Je Štědrý večer, víš? Tak buď na mě hodný, protože jsem se jich vzdala kvůli tobě. Ty vlasy se daly spočítat,`` řekla s náhlou sladkou vroucností, ,,ale mou lásku k tobě nikdo změřit nedokáže. Můžu dát smažit řízky, Jime?``

Rázem se Jim probral z transu. Vzal Dellu do náručí. Na deset sekund všímejme si radši s diskrétní pozorností nějakého nepodstatného objektu v druhém koutě pokoje. Osm dolarů týdně nebo milión ročně, jaký je v tom rozdíl? Matematik či učenec by jistě usuzoval chybně. Tři králové se dostavili s převzácnými dary, ale na tohle by se nezmohli. Kteroužto temnou narážku ještě objasníme.

Jim vyňal balíček z náprsní kapsy kabátu a hodil ho na stůl.

,,Nemysli si hned o~mně bůhvíco,`` řekl Jim. ,,Žádná ondulace, amputace či operace tvých vlasů nemůže způsobit, abych miloval svou holčičku míň než dosud. Ale až rozbalíš tenhle balíček, pochopíš, proč mě to ze začátku tak vzalo.``

Bílé prstíky jaly se hbitě cupovat provázek a papír. Vzápětí radostný výkřik! a pak, běda! rychlý, typicky ženský přechod k bědování a lamentaci, vyžadující bezprostřední zapojení veškerých chlácholivých schopností pána domu.

Neboť zde ležely HŘEBENY -- souprava hřebenů, na temeno i pro týl, které Della už dávno chodila uctívat k výkladní skříni na Broadwayi. Nádherné hřebeny z pravé želvoviny s obrubou posázenou drahokamy -- náramně barevně ladící s těmi překrásnými ztracenými vlasy. Nesmírně drahé hřebeny, jak věděla, po nichž její srdce prahlo a toužilo bez sebemenší naděje, že jí kdy budou patřit. A~nyní jsou její, avšak ty kadeře, které měly být zdobeny touto vysnívanou ozdobou, jsou fuč.

Přece však je přivinula k hrudi a za chvíli byla i schopna vzhlédnout, upřít své uslzené oči na Jima, usmát se a říct: ,,Mně vlasy rostou strašně rychle, Jime.``

A~pak už vyskočila jako kotě, kterému přižehli kožíšek, a halasila: ,,Počkej, počkej, počkej!``

Jim ještě neviděl její krásný dárek. Dychtivě mu ho podávala na otevřené dlani. Matný vzácný kov se blýskal odlesky jejího horoucího nadšení.
,,No, není perfektní, Jime? Sběhala jsem celé město, než jsem ho našla. Teď se musíš koukat, kolik je hodin, aspoň stokrát za den. Půjč mi hodinky. Ať vidím, jak se k nim ten řetízek hodí.``

Ale Jim neposlechl. Svalil se na kanape, založil si ruce za hlavu a usmíval se.

,,Víš co; Dello?`` řekl. ,,Odložme svoje vánoční dárky načas stranou. Jsou až moc krásné, než abychom si je teď mohli užít. Hodinky jsem prodal, abych ti mohl koupit hřebeny: A~teď už dej, prosím tě, smažit ty řízky.``

Tři králové byli, jak víte, mužové moudří -- náramně moudří --, kteří přinesli dárky děťátku v jesličkách. Vynalezli umění vánočního obdarovávaní. A~protože byli moudří a rozumní, i jejich dárky byly moudré a rozumné, v případě duplikace snad i prozřetelně vyměnitelné. A~já se vám tu neobratně vnucuju s nezáživným povídáním o~tom, jak dvě pošetilé děti zcela nerozumně jeden druhému obětovaly to nejcennější, co měly. A~přece v posledním slově všem mudrcům a rozumbradům těchto časů budiž řečeno toto: ze všech těch, kteří obdarovávají, byli tito dva nejmoudřejší. Neboť moudré je dávat a přijímat takové dárky. Vždy a všude. Takoví jsou králové králů.


\podpis{přeložil Jiří Josek}