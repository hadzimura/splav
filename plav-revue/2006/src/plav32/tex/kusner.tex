\section{Alexandr Kušner: Klanění tří králů}

\begin{verse}
V moskevské malé ulici, \\
do které sníh se valil, \\
jak od východu mudrci \\
jsme nad kolébkou stáli. \\
Tvořily mřížky opticky \\
nezářnou aureolu, \\
zatímco láhve s chlebíčky \\
pokryly desku stolu. \\
My mžourali jsme polotmou \\
a dojem přetrvával, \\
že někde v koutě tady jsou \\
i telátko, i kráva. \\
To všechno je jak na plátně \\
od Huga van der Guse \\
-- hospodyně, pruh světla v tmě, \\
mudrci sklánějí se, \\
a vůbec, klid je takový \\
pro tohle okamžení, \\
že ani po Herodovi \\
tu nikde stopy není. \\
Celý ten zlý a krutý svět, \\
co zná jen svoje cíle, \\
jako by hlavu sklonil teď \\
před velebností chvíle. \\
Ulice jak z dob biblických \\
vzhled měla okamžitě \\
a v okně vítal bílý sníh \\
to narozené dítě. \\
To kvůli němu noční tmou \\
se sešli všichni hosté \\
a věděli, že nastanou \\
zázraky všední, prosté.
\end{verse}

\podpis{přeložil Milan Dvořák}