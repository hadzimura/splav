\section{Daniela Fischerová: \\ Strč prst skrz krk!}

\noindent
\textit{Jazyk je nejpohyblivější sval našeho těla, ale kosti se v něm nedohledáme. Díkybohu, není si co zlomit.  Jinak už by nebylo jazyka zdravého, jazyky v sádře a zinkoklihu by nám trčely z úžasem otevřených úst. Neboť není skvělejších výtrysků rozjívené tvořivosti nežli v tvarech zvaných jazykolamy. Netvoří je básníci, ale sám duch řeči. Chrčí, hrčí, šišlá, mišlá, piští, sviští, skřípe, sípe, kničí, syčí v nich esence toho kterého jazyka. Smysl se zcela podřizuje zvuku.  Ta odvěká otázka vší poezie, totiž je-li podstatnější zvuk či obsah, je v nich zodpovězena. Zvuk jako hravé štěně smýká svým pánem -- smyslem -- na druhém konci vodítka. Tím vznikají obrazy surrealistické síly. Představme si ve vší doslovnosti, jak třistatřiatřicet stříbrných křepelek přelétá přes třistatřiatřicet stříbrných střech! Nebo jak tři pohlavně pozměněné čarodějnice živící se prostitucí -- v děsivě stručné angličtině to zní three switched witch-bitches -- hledí na hodinky. Ve Finsku vodní skřet syčí v modrém výtahu a ve Španělsku hltají na poli obilí tři smutní tygři -- ach, k čemu dožene šelmu duch jazyka a hlad! Co je to však proti hladu ubohých dojemných německých hlemýžďů, kteří v~hrůze olizují jiné hlemýždě, i když -- jak praví jazykolam --  hlemýždi hlemýžďům vůbec nechutnají. To jsou tak otřesné výjevy, že babylónská věž se chvěje v základech. Ostatně, co se ducha jazyka týče: Němci tvrdí, že Modré zelí zůstává modrým zelím a  šat nevěsty zůstává šatem nevěsty. Hle, německá spolehlivost, na níž stojí svět.}

\textit{Co však je zvláště poťouchlé, drazí překladatelé a tlumočníci, je fakt, že dostáváte něco, co se vám vlastně pouze vysmívá. Pokud co ze samé podstaty přeložit nelze, je to jazykolam. Jeho květ kvete až za tou tajemnou mezí, kterou nepřekročíme. Přeložme smysl a zvuková hra zmizí. Nechme se vést zvukem -- vznikne nesmysl. Ano, dáváme vám to nejnepřeložitelnější z nejnepřeložitelnějšího. Kdo si troufá, opakujte se mnou: nejnepřeložitelnější z nejnepřeložitelnějšího, nejnepřeložitelnější z nejnepřeložitelnějšího, nejnepřeložitelnější z nejnepřeložitelnějšího. A~jestli někdo zjistí, že jsou jeho ústa na ten úkol příliš tuhá, inu… ať, mu je naolejuje Julie.}

\textit{Ale nenaolejuje-li je, jakáž pomoc!, pak si, milý nešťastníku, strč prst skrz krk.}
