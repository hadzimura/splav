\section{Giacomo Leopardi: \\ Dopis markýze Roberti}

\noindent
\textit{Giacomo Leopardi (1798--1837) byl již coby dítě nucen se svým otcem absolvovat pro něj zajisté nudné salónní dýchánky. Chtěl se nějak pomstít organizátorce těchto sedánek, rodinné přítelkyni Leopardiů, a tak ve svých 12 letech sepsal dopis adresovaný markýze Roberti a podepsaný Befanou.}

\looseness+1
\textit{V Itálii existuje jedinečná tradice, spojená s postavou Befany, jejíž jméno je lidovou pokrouceninou řeckého termínu Epifania (svátek Tří králů).}

\textit{ Befana je stařena, přelétávající v noci z 5. na 6. ledna (čímž uzavírá dvanáctidenní cyklus po Božím narození) na koštěti ze střechy na střechu a z ohromného pytle, který nese přes rameno, podělí všechny děti. Každé dostane punčochu, v níž je vždy trochu popela a uhlí (každé dítě v minulém roce nějaký ten hříšek spáchalo) a bonbóny či jiné sladkostí coby odměna za dobré chování minulé i záloha na to budoucí.}

\textit{Ikonografie Befany je neměnná: tmavá široká sukně, zástěra s množstvím kapes, rozevlátá šála, na hlavě šátek či špičatá hučka, rozšmaťchané bačkory a celek oživen množstvím různobarevných záplat. Je to taková hodná ježibaba. Italům, když chtějí popsat ženu nepřitažlivého vzhledu (takovou, co je třeba ,,ošklivá jak noc``), stačí použít jméno Befany.}

\medskip

\noindent
\textbf{Legenda Befany}

\noindent
Tři králové se ubírali do Betléma, aby se poklonili Ježíškovi. Dorazili k jednomu domku a rozhodli se, že se tu zastaví a zeptají se na cestu. Zabušili na dveře a stařeny, která jim otevřela, se zeptali, zda nezná cestu do Betléma, protože tam se narodil Spasitel. Žena pořádně nepochopila, kam že se to ubírají a nedokázala jim nijak poradit. Králové ji tedy vyzvali, ať se k nim připojí -- což odmítla s tím, že toho má moc na práci. Teprve poté, co tři králové odešli, pochopila stařena, jaké chyby se dopustila, a rozhodla se vydat se spolu s nimi hledat Jezulátko. Pátrala po nich dlouhé hodiny, avšak bez výsledku. Alespoň pak podarovávala každé dítě, které potkala, v naději, že by to mohl být Ježíšek.

\medskip

\noindent
A~tak se to opakuje každým rokem, kdy o~tříkrálovém večeru Befana navštěvuje každý dům, v němž je dítě, aby jej obdarovala.

\medskip

\noindent
\textbf{\ldots paní markýze Roberti}

\noindent
Přemilá paní,

\noindent
Když už jsem na cestě, chtěla jsem navštívit Vás a všechny pány hochy a slečny dívky přítomné na Vašem dýchánku, ale sníh mi naboural jízdní řád, a tak se nemohu déle zdržet. Myslela jsem tedy, že sem alespoň na okamžik skočím a vyčůrám se Vám na zápraží a pak budu pokračovat v mé pouti. Nechávám tu tedy alespoň pár drobností pro tyto synky a dcerky, aby byli hodní. Ale řekněte jim, že pokud o~nich dostanu špatné zprávy, napřesrok jim přinesu trochu ho..a.

Každému jsem chtěla připravit jeho vlastní dárek, například jednomu ten kornout a jinému zas tamten, ale obávala jsem se, že bych se nemusela projevit nestranně a ten, kdo by měl kornout kratší, záviděl by držitelům kournoutů delších. Rozhodla jsem se proto nechat to náhodě a zařiďte to Vy. V přiložené obálce najdete určitý počet lístků se stejným počtem čísel na nich napsaných. Vhoďte všechny do nočníku a dobře je promíchejte vlastníma rukama. Pak nechť každý vytáhne si svůj lístek a uvidí číslo. Poté přiloženým klíčem odemkněte kufr. Nejprve tam najdete něco dobrého a myslím, že přítomní páni a dámy to ocení, protože jsou jen bandou mlsounů. Pak tam najdete všechny kornouty, označené příslušnými čísly. Každý nechť si vezme ten svůj a odejde v pokoji. Kdož nebude spokojen s kornoutem, co na něj připadne, nechť smění jej za kornouty svých druhů. Pokud nějaký kornout zbude, vyzvednu si jej na zpáteční cestě. Napřesrok to pak rozpočítám lépe.

A~Vy pak, přemilá paní, vezte že po celý tento rok máte dobře o~tyto pány a dámy pečovati, nejen kávou, což je samozřejmostí, ale i paštičkami, pudinky, moučníčky, koláči, oplatkami, kornoutky a dalšími dárky. A~nebuďte lakomá, nenechte se prosit, protože kdo chce pořádat dýchánky, musí mít dlaň otevřenou, a když každého večera nabídnete jiný moučníček, ještě více chválena budete a Vaše dýchánky nazývány budou dýchánky moučníčkovými. Mezitím do příště buďte veselí a všichni jděte tam, kam Vás posílám. A~zůstaň\-te tam, dokud se nevrátím -- vy nenažranci, šťouralové, oslové a paraziti, jeden jak druhý.

Befana

\podpis{přeložil Petr Kautský}