RUŠTINA
Шла Саша по шоссе и сосала сушку.
Šla Saša po silnici a cucala preclík. 
Сделан коплак не по-колпаковски,
сделан колокол не по-колоколовски.
Надо колпак переколпаковать, перевыколпаковать.
Надо колокол переколоколовать, перевыколоколовать.
Kukla je udělána nekuklovsky,
Zvon je udělán nezvonovsky.
Kuklu je třeba překuklovat, převykuklovat!
Zvon je třeba přezvonovat, převyzvonovat.
Говорит попугай попугаю: «Я тебя, попугай, попугаю!»
Попугаю в ответ попугай: «Попугай, попугай, попугай!»
Povídá papoušek papouškovi: „Já tě, papoušku, postraším!“
Papouškovi odpovídá papoušek: „Jen postraš, papoušku, jen postraš!“
Всех скороговорок не перескороговоришь,
не перевыскороговоришь.
Všechny jazykolamy nepřejazykolámeš,
nepřevyjazykolámeš.
Расскажи мне про покупки!
Про какие покупки?
Про покупки, про покупки,
про покупочки твои!
Vyprávěj mi o nákupech!
O jakých že nákupech?
O nákupech o nákupech,
O nákupečcích tvých.
Стоит гора посреди двора.
На дворе – трава, на траве – дрова.
Не руби дрова на траве двора!
Stojí hora uprostřed dvora.
Na dvoře je tráva, na trávě je dříví.
Neštípej dříví na trávě dvora.
Ехал Грека через реку.
Видит Грека: в реке рак.
Сунул Грека руку в реку,
Рак за руку Грека – цап!
Jel Greka přes řeku.
Vidí Greka, že v řece je rak.
Strčil Greka ruku do řeky,
Rak Greku za ruku chňap!
Четыре чёрненьких чумазеньких чертёнка
чертили чёрными чернилами чертёж.
Čtyři čerňoučtí umazaňoučtí čertíci
rýsovali černým inkoustem rys.
Шли сорок мышей
Несли сорок грошей.
Две мыши поплоше
Несли по два гроша.
Šlo čtyřicet myší,
neslo čtyřicet grošů.
Dvě myši, co byly slabší,
nesly po dvou groších.
Карл украл у Клары кораллы,
а Клара украла у Карла кларнет.
Karel ukradl Kláře korály,
kdežto Klára ukradla Karlovi klarinet.
Мама мыла Милу мылом. Мила мыло не любила.
Maminka myla Milu mýdlem. Mila mýdlo neměla ráda.


ČÍNŠTINA
Vážení zájemci o text originálu nechť se laskavě obrátí na prasátko, králíčky nebo myšičku.
Prasátko nosí králíčka. Králíček nosí myšičku. Všichni společně bouchají do stromečku. Stromeček spadl, pevně tlačí na myšičku, strom tlačí na myšičku, myšička tlačí na králíčka, králíček tlačí na tlusté prasátko. Prasátko je potlačeno, křičí nepřetržitě. Králíček, myšička, stromeček, stromeček, králíček, myšička.
Za plotem leží vrták. Pod vrtákem je položena cihla. Neví se, zda bude cihla vrtat do vrtáku nebo vrták bude vrtat do cihly.
Trhat znakový papír. Za oknem trhat znakový papír. Je znakový papír, trhat znakový papír. Není znakový papír, nelze volně trhat znakový papír.

Polština


 

 
Ta ramka tu, ta ramka tam. 
 

 


by Jan Brzechwa