\section{Roman Brandstaetter: \\ Poslední Večeře}

\begin{verse}

Během této noci, \\
bez níž bychom nemohli být, \\
neboť z ní povstala míra pro lidské svědomí,\\
apoštolové byli jak pastýři\\
ohnivých luhů. Byla to skutečnost plnější,\\
než jakou si lidé uvykli vidět\\
ve svém životě, v pěstování svojí zahrádky,\\
v mořeplavbě a hudbě, v milování ženy,\\
v hromadění majetku.\\
A i když ještě nevěděli,\\
že poznání nebes je poznáním utrpení,\\
vytušili už, že nic míň nenastane, nežli chvíle\\
přepodstatnění všech hodnot.\\
Báli se té chvíle. Obráceni k Jeho tváři\\
jak k otevřenému údolí,\\
na jehož dně teče řeka proroků,\\
chvěli se neklidem.

\medskip

A mezitím \\
Mladík poklekl \\
a začal jim mýt nohy, \\
jako se umývá znavený práh nebes. \\
Očištěni vodami večera, \\
usedli zahanbeně \\
kolem paschální hodiny, \\
na jejíž odbití čekali \\
od počátku  \\
svých vyznačených dní.

\medskip

Zašelestil třtinový závěs \\
protkaný mušlemi. \\

\medskip

Zůstali sami, hledíce na Mladíka, \\
který omočil v míse \\
kus přesného chleba, \\
vztáhl dlaň \\
a častoval je,  \\
každého zvlášť, \\
tím pro ně nesrozumitelným pokrmem. \\
Rabbi řekl: \\
,,Ten chléb je mým tělem. Jezte ho. \\
V tom chlebě je látka všech světů, \\
její počátek i konec. \\
V tom chlebě jsou samotné a pusté prostory, \\
také prostory lidnaté a šťastné, \\
také unavené a ke mně se vracející\ldots \\
Takové je moje tělo.`` \\
To řekl, podal jim kalich vína \\
a dál vybízel: \\
,,To víno je mojí krví. Pijte ho. \\
V tomto víně je čas \\
soustav slunečních, \\
které byly, jsou a budou, \\
těch, které pominuly, \\
i těch, které povstanou. \\
Taková je má krev. \\
Živte se mým prostorem jak chlebem \\
a pijte můj čas jak víno. \\
Já jsem v nich.``

\medskip

Tak mluvil, a oni vcházejíce \\
v eschatologickou zem, \\
počali požívat ovoce \\
krvavých stromů \\
a zakoušeli v sobě teplo \\
vinic \\
rostoucích \\
na laskavých svazích Boha.

\bigskip


\textbf{Getsemany}

\medskip

Když byla jeho duše smutná až k smrti, \\
vzdálil se co by kamenem dohodil \\
a padl na tvář svoji jako na tvář světla. \\
Apoštolové ohradili se před Ním \\
stěnou snu. O čem mohou snít lidé \\
v lisovně krve a potu? O čem? \\
Možná počítali ve snu \\
chleby padající na Galileu \\
a ryby vzlétající k nebi.

\medskip

Stál co by kamenem dohodil \\
mezi lidmi a Bohem \\
a modlil se: \\
,,Otče, bojím se lidským strachem \\
a jsem smutný lidským smutkem. \\
Prosím Tě, odejmi ode mne \\
tu hodinu kříže, \\
ale ne podle mé, \\
podle Tvojí vůle. \\
Prosí Tě toto mé znavené a slabé tělo, \\
prosí Tě tyto mé okoralé rty, \\
prosí Tě tyto mé opuchlé nohy. \\
Ale i když Tě prosím, \\
nevyslyš modlitbu mého těla, \\
nevyslyš modlitbu mých úst, \\
nevyslyš modlitbu mých rukou, \\
nevyslyš modlitbu mých nohou, \\
nevyslyš modlitbu mého krvavého potu. \\
Neboť nemůže být zapřen nářek žalmů \\
a nemůže být zapřen jazyk proroků \\
a nemůže být zapřeno bytí člověka, \\
bratra všech mých útrap a vraha mojí krve. \\
Proto ať nadejde hodina kříže \\
a ať se stane Tvoje vůle, \\
neboť ona je i mojí vůlí, \\
Bože,  \\
oděný mým třesoucím se a zpoceným tělem.``

\medskip

Tak se modlil, \\
a oni, i když slyšeli jeho slova, \\
nerozuměli ničemu. Nechtěli se vrátit \\
do skutečnosti, leželi bez hnutí \\
na dně snu \\
a předstírali, že ani neslyší \\
šelest anděla, který se vlekl \\
středem jejich těl \\
jak ulicemi vymřelého města. 

\bigskip

\textbf{Stabat Mater}

\medskip

Chodím ulicemi starého Jerusaléma. \\
Bezzubé zdivo \\
stulené do liturgických šálů \\
pronáší modlitby jak starci, \\
kteří nalezli v sobě, u cíle života, \\
hvězdu v mládí ztracenou. \\
Všichni nosíme v sobě hvězdu \\
a všichni ji najdeme v hodině naší smrti. 

\medskip

Jsem na prostranství poprav. \\
Jsem na prostranství Boha. \\
Jsem v samém středu zločinu \\
jako lis na vinici \\
a vidím: 

\medskip

Stála pod křížem židovská Madona, \\
obelisk bolu volající k nebi \\
z pece Treblinky, vlasy rozpuštěné \\
trochu neladně, jak Píseň písní. \\
Ó Matko Boha, Matko stojící \\
snad na témže místě, kde já nyní stojím, \\
zahleděná ve žloutnoucí tělo svého syna, \\
v plod života a svatou svíci \\
na kříž přibitou. Zvolna taje vosk \\
a stále menší plamínek se úží, \\
bledne jak motýl, a skládaje křídla \\
jak ruce k modlitbě, hasne vprostřed temnot, \\
které Bůh jako meč z rány obludy vytrhl \\
ze samého středu slunečného dne.

\medskip

Ó zakryj oči a nehleď na drama, \\
kováři sedmi bolestí, jež Matka \\
zakoušela v srdci, když její syn \\
klesal jak blesk po kmeni kříže \\
do olivového sadu ráje \\
a všem lidem doložil, \\
že Bůh dovede tak jako člověk umírat.

\medskip

Klopím oči. Ještě vidím lem \\
žalmového šatu a fragment sandálu \\
opřený lehce o lhostejný kámen \\
a stále cítím na sobě pohled  \\
hrdličky vědoucí, že člověk \\
je vinen smrtí toho děcka, \\
které v svém lůně panenském a čistém \\
v námaze nosila po devět měsíců \\
jak Písmo svaté.

\medskip

Ó Matko, stojící \\
snad na témže místě, kde já nyní stojím, \\
máš před sebou vrchol Hory lebek \\
a za sebou tmu Starého města, \\
uzounké uličky, které jak šediví soumaři \\
důstojně kráčejí údolím. \\
Nemohu za to, že stále vidím  \\
zářivě modré oči své zemřelé matky \\
v tvých očích, Panno z královského rodu, \\
jak pramen v prameni. \\
Tu ženu stojící u tvé zlaté brány \\
vezmi za ruku jak starou žebračku \\
a osvěž její unavené čelo \\
veršem mé litanie, \\
jak jsem ji pro tebe neobratně utkal, \\
Madono Treblinská, \\
svatá Madono táborů smrti, \\
svatá Madono zaživa trápených, \\
svatá Madono škvařeného těla, \\
svatá Madono plynové komory, \\
svatá Madono rozpálených roštů, \\
svatá Madono lidského popela, \\
modli se za popel \\
mé matky, \\
za její spálený život, \\
za její spálené vlasy, \\
za její spálené oči. \\
Pros, Madono na vrcholu Hory lebek, \\
Madono Treblinská!!

\bigskip

\textbf{Ježíš rozmlouvá s Marií z Magdaly}

\medskip

Nedotýkej se mě, Marie, nedotýkej. \\
Jsem stále ještě nasáklý \\
vůní nardu linoucí se z pohřebního plátna. \\
Ještě jsem se nevrátil k svému Otci. \\
Nech zase klesnout svoji dlaň \\
a nehleď udiveným zrakem \\
na mé čelo zastíněné \\
širokým křídlem klobouku. \\
Vrať se ke svým bratřím \\
a pověz jim, nevěřícím, \\
žes mluvila se zahradníkem, \\
který zasadil do jejich krve \\
strom moudrosti. \\
Nedotýkej se mě, Marie, nedotýkej. \\
Těžko se lze osvobodit od těla.

\end{verse}

\podpis{přeložil František X. Halaš}