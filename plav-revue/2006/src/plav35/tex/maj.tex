\section{Karel Hynek Mácha: Máj}

\begin{verse}
Byl pozdní večer -- první máj -- \\
večerní máj -- byl lásky čas. \\
Hrdliččin zval ku lásce hlas, \\
kde borový zaváněl háj. \\
O lásce šeptal tichý mech; \\
květoucí strom lhal lásky žel, \\
svou lásku slavík růži pěl, \\
růžinu jevil vonný vzdech. \\
Jezero hladké v křovích stinných \\
zvučelo temně tajný bol, \\
břeh je objímal kol a kol; \\
a slunce jasná světů jiných \\
bloudila blankytnými pásky, \\
planoucí tam co slzy lásky.

\bigskip

\textbf{May}

\smallskip

It was late eve -- the first in May --  \\
Eve in May -- it was love’s hour. \\
The turtle-dove’s voice called to love, \\
Where the pine grove wafting lay. \\
Love whispered soft the quiet moss; \\
The blossoming tree lied love’s woe, \\
The nightingale sang love to the rose, \\
The rose’s shown by an odorous sigh. \\
Smooth the lake in shadow’d bushes \\
Darkly sounded secret pain, \\
The shore embraced it round and again; \\
And the bright suns of other worlds \\
Wandered through the azure zones, \\
Burning there like tears of love. 

\smallskip

\podpis{přeložil James Naughton, 2000}

% \bigskip
% 
% \textbf{May}
% 
% \smallskip
% 
% Late evening, on the first of May-- \\ 
% The twilit May--the time of love. \\
% Meltingly called the turtle-dove, \\
% Where rich and sweet pinewoods lay. \\
% Whispered of love the mosses frail, \\
% The flowering tree as sweetly lied, \\
% The rose's fragrant sigh replied \\
% To love-songs of the nightingale. \\
% In shadowy woods the burnished lake \\
% Darkly complained a secret pain, \\
% By circling shores embraced again; \\
% And heaven's clear sun leaned down to take \\
% A road astray in azure deeps, \\
% Like burning tears the lover weeps. 
% 
% \smallskip
% 
% \podpis{přeložila Edith Pargeter}


\bigskip

\textbf{Mai}

\smallskip

Il était tard -- le premier mai -- \\
un soir de mai -- le temps d'aimer. \\
Et la voix de la tourterelle \\
si frêle invitait à l'amour \\
parmi les pins qui embaumaient. \\
D'amour les mousses chuchotaient; \\
l'arbre en fleur semblait un regret, \\
le tendre rossignol chantait \\
à la rose qui soupirait. \\
Le lac dans l'ombre des buissons \\
exhalait une peine obscure \\
dans les bras du rivage sombre; \\
et les clairs soleils d'autres monde \\
erraient par des chemins d'azur; \\
on eut dit des larmes d'amour.

\smallskip

\podpis{přeložil Charles Moisse, 1965}

\bigskip

\textbf{Mai}

\smallskip

Spätabend war’s -- es war der erste Mai -- \\
Ein Abendmai -- es war der Minne Zeit. \\
Die Turteltaube lockt’ zur Seligkeit \\
Im duft’gen Kieferhain, so traut und treu. \\
Von Liebe flüsterten die stillen Moose, \\
Und Liebeswehe log der Blütenbaum, \\
Von Liebe sang die Nachtigall der Rose, \\
Der Abendwind verriet den Rosentraum. \\
Im tiefgeheimen Schmerze seufzte leise \\
Der glatte See, und leise seufzte auch \\
Das kühle Strandgebüsch im Abendhauch, \\
Indes die Strahlensonnen and’rer Kreise \\
Still irrten durch die blaukristall’nen Sphären \\
Und glühten dort wie heiße Liebeszähren.

\smallskip

\podpis{přeložil Alfred Waldau}

% \bigskip
% 
% \textbf{Май}
% 
% \smallskip
% 
% День первый мая – вечер был – \\
% вечерний май – любви хвала, \\
% к ней властно горлица звала, \\
% где бор сосновый шумы вил. \\
% Шептал влюбленно тихий мох, \\
% любовной грустью лгал расцвет, \\
% пел соловей любовь – в ответ \\
% струился роз душистый вздох. \\
% Гладь озера в кустах тенистых \\
% звучала тайной скрытых мук, \\
% ее замкнул прибрежий круг; \\
% а солнц сиянья в мир пречистых \\
% ушли скитаться в голубени, \\
% горя, как слезный перл влюбленный. 
% 
% \smallskip
% 
% \podpi{přeložil Евгений Недзельский, 1937}


\pagebreak

\textbf{Maggio}

\smallskip

Giunto era maggio -- ed era tarda \\
la sera -- e la pineta auliva. \\
Stagion d'amore! Nunzia schiva \\
ne fu la tortora maliarda. \\
Financo il muschio silenzioso \\
n'avea sentor. Gemea d'un moneo \\
sospir d'amor l'arboreo tronco \\
fraa' fior sepolto. I rnelodioso \\
messaggio inviava a donna rosa \\
il piccol menstrel d'amor, \\
le s'erge il sen -- tutt'odorosa \\
spandeva l'ebbro suo tremor. \\
Il lago levigato, bruno \\
d'ombrosi arbusti, con la sponda \\
che l'abbracciava, era tutt'uno \\
nel cupo murmure dell'onda.

\smallskip

\podpis{přeložil Riccardo Selvi}

\bigskip

\textbf{Május}

\smallskip

Már késő est volt -- kora május -- \\
Esti május -- szerelmi kor, \\
Szerelmet mikor gerle szól, \\
Hol fenyves ébred, mély, homályos. \\
Moha súgott szerelmi kínt \\
S hazudott a virágzó fa, \\
Szerelmét csíz vallotta ma, \\
Illattal vallott rózsa szint. \\
Berkekben sötét, kusza hangot, \\
Titkos kínt zúgott síma tó, \\
Ölelte part, oltalmazó; \\
Más, más világokban bolyongott \\
Száz fényes nap, azúr lángszőnyeg, \\
Lángolva, mint szerelmi könnyek.

\smallskip

\podpis{přeložil Efraim Israel, 2000}

% \bigskip
% 
% \textbf{Maj}
% 
% \smallskip
% 
% By\l wieczór -- maj -- mi\l ości czas -- \\
% wieczorny maj; do mi\l owania \\
% synogarlicy g\l os nak\l ania, \\
% gdzie pachnie w mgle iglasty las. \\
% Już o mi\l ości szepce mech; \\
% k\l ami\k{a} jej ból rozkwit\l e drzewa, \\
% s\l owik j\og{a} s\l odko róży śpiewa, \\
% s\l owik j\og{a}ą s\l odko róży śpiewa, \\
% p\l onie nią róży wonny dech. \\
% Jezioro w cieniu \lóz gęstwinnych \\
% brzmi ciemną mową tajnych mąk, \\
% a brzeg je obejmuje w krąg; \\
% tam s\l ońca blade światów innych \\
% b\lądzą przez b\lękit i tak drży \\
% ich światlo jak mi\l ości \l zy.
% 
% \smallskip
% 
% \podpis{přeložil Józef Waczków, 1971}

\bigskip

\textbf{Máj}

\smallskip

Bol pozdný večer -- prvý máj -- \\
večerný máj -- bol lásky čas. \\
Hrdličkin zval ku láske hlas, \\
kde bôrový rozváňal háj. \\
O láske šepkal tichý mach; \\
kvitnúci strom lkal lásky žiaľ, \\
slávik ju ruži vyspieval, \\
a ruža vzdychla vôňou v tmách. \\
Jazero hladké v húšt kroviny \\
zvučalo temne tajný bôľ, \\
breh objímal ho navôkol; \\
a slnká jasné svetov iných \\
blúdili cez blankytné pásky, \\
planúc tam ako slzy lásky.

\smallskip

\podpis{přeložil Ján Kostra, 1963}

\end{verse}