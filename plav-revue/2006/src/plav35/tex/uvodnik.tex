\section{Večer plný duchovna i duchů\ldots}

\noindent
Pro velikonoční ,,bubeníčkovský`` večer jsme vybírali především z textů dvou autorů, Romana Brandstaettera (1906--1987) a Vladimíra Josefa Koudelky (1919--2003). Životní příběh prvního z nich je v kontextu dnešního křesťanského svátku skoro symbolický. Narodil se do polské židovské rodiny a většinu života také prožil v silně katolickém Polsku. Pro jeho duchovní vývoj ovšem mělo rozhodující význam to, že v první půlce 40. let na jistou dobu přesídlil do Palestiny. Právě tady, v zemi svých předků, se definitivně rozhodl konvertovat ke katolické víře. Do Polska se Brandstaetter vrátil v roce 1948. V 60. letech tu vznikly jeho básnické sbírky \textit{Pieśń o moim Chrystusie} (Píseň o mém Kristu) a \textit{Hymny Maryjne} (Mariánské hymny). Právě ty se kolem poloviny 70. let dostaly do rukou českého faráře Františka X. Halaše, který se pustil do jejich překládání.

Dopisy ,,Velkého starého bratra`` Vladimíra Koudelky jeho ,,malé sestřičce``, jinak řádové sestře Růženě, pocházejí z let 1991--2003. Rady, které prostřednictvím těchto listů své sestře v Kristu udílel, jsou velmi upřímné a zvláštně neortodoxní. Ta neortodoxnost je jednak dílem duchovní odvahy, ale také specifické práce s jazykem. Sám Koudelka mluvil o své češtině jako o ,,hrbolaté``. Protože většinu života prožil ve Švýcarsku, nebyla pro něj tak docela přirozeným nástrojem dorozumívání. A snad právě díky tomu jeho formulace stran víry a duchovního života neplynou automaticky a klišovitě. Naopak. Je v nich poznat hledání toho nejpřesnějšího výrazu pro ten který pocit či myšlenku. Dopisy, ze kterých dnes budeme číst, vydalo Karmelitánské nakladatelství hned ve třech jazykových verzích (anglicky, polsky a německy). 

Část dnešního večera ovšem bude rovněž vítací. Budeme vítat jaro! A to rovnou Máchovým \textit{Májem}. Podařilo se nám sehnat jeho překlad do celkem osmi jazyků: tří slovanských (polština, ruština, slovenština), dvou germánských (angličtina, němčina), dvou románských (francouzština, italština) a jednoho ugrofinského (maďarština). To, jak se s nejslavnější českou básní vypořádali naši evropští bratři překladatelé, ovšem nebudeme posuzovat pouze díky mluvčím jednotlivých jazyků, ale také při společné produkci Máchova hřbitovního intermezza: čeká nás společný Sbor duchů pod duchovním vedením Hany Kofránkové\ldots

Doufáme, že se v tom májovém Babylonu dnes úplně neztratíme. Přejeme Vám pěkné svátky velikonoční!

\podpis{Alena Nováková a Standa Rubáš}