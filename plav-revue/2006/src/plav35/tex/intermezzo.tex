\begin{verse}
\textbf{Intermezzo I}

\textit{Půlnoc}

\smallskip

(krajina)

\smallskip

V rozlehlých rovinách spí bledé lůny svit, \\
kolem hor temno je, v jezeru hvězdný kmit, \\
nad jezerem pahorek stojí. \\
Na něm se sloup, s tím kolo zdvíhá, \\
nad tím se bílá lebka míhá, \\
kol kola duchů dav se rojí; \\
hrůzných to postav sbor se stíhá.

\smallskip

\textit{Sbor duchů}

\smallskip

,,V půlnočních ticho je dobách; \\
světýlka bloudí po hrobách, \\
a jejich modrá mrtvá zář \\
svítí v dnes pohřbeného tvář, \\
jenž na stráži -- co druzí spí -- \\
o vlastní křížek opřený \\
poslední z pohřbených zde dlí. \\
V zenitu stojí šedý mrak \\
a na něm měsíc složený \\
v ztrhaný mrtvý strážce zrak, \\
i v pootevřené huby \\
přeskřípené svítí zuby.``

\smallskip

\textit{Jeden hlas}

\smallskip

,,Teď pravý čas! -- připravte stán -- \\
neb zítra strašný lesů pán \\
mezi nás bude uveden.`` 

\smallskip

\textit{Sbor duchů} \\
(sundávaje lebku)

\smallskip

,,Z mrtvého kraje vystup ven, \\
nabudiž život -- přijmi hlas, \\
buď mezi námi -- vítej nám. \\
Dlouho jsi tady bydlil sám, \\
jiný tvé místo zajme zas.``

\smallskip

\textit{Lebka} \\
(mezi nimi kolem se točíc)

\smallskip

,,Jaké to oudů toužení, \\
chtí opět býti jedno jen. \\
Jaké to strašné hemžení, \\
můj nový sen. -- Můj nový sen! --``

\smallskip

\textit{Jeden hlas}

\smallskip

,,Připraven jestiť jeho stán. \\
Až zítra půlnoc nastane, \\
vichr nás opět přivane. \\
Pak mu buď slavný pohřeb dán.``

\smallskip

\textit{Sbor duchů}

\smallskip

,,Připraven jestiť jeho stán. \\
Až zítra půlnoc nastane, \\
vichr nás opět přivane. \\
Pak mu buď slavný pohřeb dán.``

\pagebreak

\textit{Jeden hlas}

\smallskip

,,Rozlehlým polem leť můj hlas; \\
pohřeb v půlnoční bude čas! \\
Co k pohřbu dá, každý mi zjev!``

\smallskip

\textit{Čekan s kolem}

\smallskip

,,Mrtvému rakví budu já.``

\smallskip

\textit{Žáby z bažiny}

\smallskip

,,My odbudem pohřební zpěv.``

\smallskip

\textit{Vichr po jezeru}

\smallskip

,,Pohřební hudbu vichr má.``

\smallskip

\textit{Měsíc v zenitu}

\smallskip

,,Já bílý příkrov tomu dám.``

\smallskip

\textit{Mlha po horách}

\smallskip

,,Já truchloroušky obstarám.``

\smallskip

\textit{Noc}

\smallskip

,,Já černá roucha doručím.``

\smallskip

\textit{Hory v kolo krajiny}

\smallskip

,,Roucha i roušky dejte nám.``

\smallskip

\textit{Padající rosa}

\smallskip

,,A já vám slzy zapůjčím.``

\smallskip

\textit{Suchopar}

\smallskip

,,Pak já rozduji vonný dým.``

\smallskip

\textit{Zapadající mračno}

\smallskip

,,Já rakev deštěm pokropím.``

\smallskip

\textit{Padající květ}

\smallskip

,,Já k tomu věnce uviji.``

\smallskip

\textit{Lehké větry}

\smallskip

,,My na rakev je donesem.``

\smallskip

\textit{Svatojánské mušky}

\smallskip

,,My drobné svíce ponesem.``

\smallskip

\textit{Bouře z hluboka}

\smallskip

,,Já zvonů dutý vzbudím hlas.``

\smallskip

\textit{Krtek pod zemí}

\smallskip

,,Já zatím hrob mu vyryji.``

\smallskip

\textit{Čas}

\smallskip

,,Náhrobkem já ho přikryji.``

\smallskip

\textit{Přes měsíc letící hejno nočního ptactva}

\smallskip

,,My na pohřební přijdem kvas.``

\pagebreak

\textit{Jeden hlas}

\smallskip

,,Slavný mu pohřeb připraven. \\
Ubledlý měsíc umírá, \\
Jitřena brány otvírá, \\
již je den, již je den!``

\smallskip

\textit{Sbor duchů}

\smallskip

,,Již je den, již je den!``

\smallskip

(zmizí) 

\bigskip

\textbf{Intermezzo I}

\smallskip

\textit{Midnight}

\smallskip

(The landscape)

\smallskip

In the widespread plains pale moonlight sleeps, \\
Darkness around the hills, in the lake stars’ glimmer, \\
Above the lake a hillock stands. \\
On it a pillar, with it rises a wheel, \\
Above it glimpsed is a white skull, \\
All about a crowd of spirits swarms; \\
A band of horrid shapes in self-pursuit.

\smallskip

\textit{The band of spirits}

\smallskip

``‘Tis silent at the midnight hour; \\
Will-o’-wisps roam o’er the graves, \\
Shining with their dead blue glow \\
Upon the one buried here today, \\
Who on guard -- while the others sleep - \\
Here upon his own cross leaning, \\
The latest burial, now remains. \\
At the zenith a grey cloud stands, \\
Resting now upon it the moon \\
Shines upon the guard’s stiff \\
Gaze of death and in the gritted \\
Teeth of his half parted mouth.”

\smallskip

\textit{One voice}

\smallskip

``Now’s the time! -- prepare the resting place -- \\
For tomorrow the dread forests’ lord \\
Will be brought to be amongst us.”

\smallskip

\textit{The band of spirits} \\
(taking down the skull) 

\smallskip

``Step out from the land of death, \\
Take on life -- receive a voice, \\
Be amongst us -- welcome here. \\
Long you had to dwell alone, \\
Now another takes your place.”

\smallskip

\textit{The skull} \\
(spinning about amongst them) 

\smallskip

``What longing is here of limbs, \\
Which only wish to be united. \\
What dreadful swarming here, \\
My dream. -- ‘Tis my new dream!”

\pagebreak

\textit{One voice}

\smallskip

``Ready is the resting place. \\
When tomorrow midnight comes, \\
The gust will blow us here again. \\
Then let him have solemn burial.”

\smallskip

\textit{The band of spirits}

\smallskip

``Ready is his resting place. \\
When tomorrow midnight comes, \\
The gust will blow us here again. \\
Then let him have solemn burial.”

\smallskip

\textit{One voice}

\smallskip

``Through the wide-spread field fly my voice: \\
The burial is at the midnight hour! \\
What each will bring, now tell me!”

\smallskip

\textit{The scaffold with the wheel}

\smallskip

``The dead man’s coffin I shall be.”

\smallskip

\textit{The frogs from the marsh}

\smallskip

``We shall sing the burial song.”

\smallskip

\textit{The gust over the lake}

\smallskip

``The wind shall have the burial music.”

\smallskip

\textit{The moon at the zenith}

\smallskip

``I shall add the white shroud.”

\smallskip

\textit{The mist over the hills}

``I shall give the mourning veils.”

\smallskip

\textit{Night}

\smallskip

``I shall bring black robes.”

\smallskip

\textit{The hills about the landscape}

\smallskip

``Give to us these robes and veils.”

\smallskip

\textit{The falling dew}

\smallskip

``And I shall lend you tears.”

\smallskip

\textit{The heath}

\smallskip

``I blow a fragrant vapour.”

\smallskip

\textit{Settling raincloud}

\smallskip

``I sprinkle the coffin with rain.”

\smallskip

\textit{Falling blossom}

\smallskip

``I weave it garlands.”

\smallskip

\textit{Light breeze}

``We bear them to the coffin.”

\smallskip

\textit{Fire-flies}

\smallskip

``We bring tiny candles.”

\pagebreak

\textit{The storm from the depths}

\smallskip

``I waken the hollow voice of bells.”

\smallskip

\textit{A mole from underground}

\smallskip

``I meantime dig his grave.”

\smallskip

\textit{Time}

\smallskip

``I cover it with a tomb.”

\smallskip

\textit{A flock of nocturnal birds flying across the moon}

\smallskip

``We shall come to his funeral feast.”

\smallskip

\textit{One voice}

\smallskip

``His solemn burial is ready. \\
The pale moon dies, \\
Morning’s star unlocks the gates, \\
‘Tis day! ‘tis day!”

\smallskip

\textit{The band of spirits}

\smallskip

``‘Tis day; ‘tis day!”

\smallskip

(They vanish)

\smallskip

\podpis{přeložil James Naughton, 2000}

\bigskip

\textbf{Erstes Intermezzo}

\smallskip

\textit{Mitternacht}

\smallskip

(Landschaft)

\smallskip

An weiter Eb’ne träumt des Mondes bleicher Glanz, \\
Rings um die Berge Nacht, im See der Sterne Tanz, \\
Ein Hügel ragt am Seegestad empor, \\
Darauf ein Pfahl, an dem ein Rad sich schwingt, \\
Und auf dem Rad ein weißer Schädel blinkt, \\
Rings um das Richtrad schwärmt ein Geisterchor, \\
Der grausig einen Totenreigen schlingt.

\smallskip

\textit{Geisterchor}

\smallskip

Um Mitternacht ist Alles still und stumm; \\
Irrlichter wandern auf den Gräbern um \\
Und schau’n mit ihrem blauen Totenlicht \\
Dem heut’ Begrab’nen in das Angesicht. \\
Er darf als Letztbegrabener nicht ruhen, \\
Er hält die Wacht, an’s eig’ne Kreuz gelehnt, \\
Indes die Andern schlafen in den Truhen. \\
Ein Graugewölk sich im Zenite dehnt, \\
Worauf die Mondeskugel ruhet und \\
Des Kirchhofwärters Blick, zerrissen, tobt, \\
Die weißen Zähne im halboff’nen Mund \\
Beglänzt mit ihrem matten Strahlenrot.

\smallskip

\textit{Eine Stimme}

\smallskip

Jetzt ist es Zeit! -- So ordnet das Gepränge! \\
Der nächste Morgen führt in uns’re Reih’n \\
Den schauerlichen Herrn der Wälder ein.

\pagebreak

\textit{Geisterchor} \\
(Den Schädel vom Rade herabnehmend)

\smallskip

So tritt heraus denn aus der Totenmenge, \\
Nimm Leben an, empfange Stimme, Schatz, \\
Bleib’ da bei uns, sollst uns willkommen sein, \\
Denn lang genug schon warst du hier allein, \\
Ein And’rer kommt nun bald auf deinen Platz. 

\smallskip

\textit{Der Schädel} \\
(In der Mitte des Reigens umherwirbelnd)

\smallskip

Wie streckt all mein Gebein so seltsam sich, \\
Eins will’s wohl wieder sein -- ich fass’ es kaum, \\
Wie wimmelt es doch hier so schauerlich -- \\
Mein neuer Traum! -- Mein neuer Traum! --

\smallskip

\textit{Eine Stimme}

\smallskip

Schon steht bereit sein neues Nachtquartier. \\
Und wenn die nächste Mitternacht beginnt, \\
Und wieder uns zusammenträgt der Wind, \\
Hernach begraben wir ihn prunkvoll hier.

\smallskip

\textit{Geisterchor}

\smallskip

Schon steht bereit sein neues Nachtquartier. \\
Und wenn die nächste Mitternacht beginnt, \\
Und wieder uns zusammenträgt der Wind, \\
Hernach begraben wir ihn prunkvoll hier.

\smallskip

\textit{Eine Stimme}

\smallskip

Mein Aufruf fliege durch die weiten Auen; \\
Die Mitternacht soll das Begräbnis schauen! \\
Sag’ jeder an, was er zum Fest will bringen!

\smallskip

\textit{Der Pfahl mit dem Rade}

\smallskip

Ich liefere dem toten Mann die Truh’.

\smallskip

\textit{Die Frösche im Sumpfe}

\smallskip

Wir wollen ihm die Totenlieder singen.

\smallskip

\textit{Der Wind über dem See}

\smallskip

Die Grabmusik bestellt der Wind dazu.

\smallskip

\textit{Der Mond im Zenit}

\smallskip

Ich will das weiße Leichentuch ihm spenden.

\smallskip

\textit{Der Nebel auf den Bergen}

\smallskip

Ich werde dann die Trauerflöre senden.

\smallskip

\textit{Die Nacht}

\smallskip

Ich händige euch schwarze Kleider ein.

\smallskip

\textit{Die Berge rings in der Landschaft}

\smallskip

So wollet uns die Trauerschleier senden.

\smallskip

\textit{Der fallende Tau}

\smallskip

Und ich, ich will euch meine Tränen leih’n.

\pagebreak

\textit{Der Anger}

\smallskip

Ich komme dann mit Weihrauchduft entgegen.

\smallskip

\textit{Die sinkende Wolke}

\smallskip

Und besprenge dann den Sarg als Regen.

\smallskip

\textit{Die fallende Blüte}

\smallskip

Von mir soll er die Trauerkränze haben.

\smallskip

\textit{Die leichten Lüftchen}

\smallskip

Auf seinen Sarg hin wollen wir sie legen.

\smallskip

\textit{Johanniskäfer}

\smallskip

Wir wollen dann die hellen Kerzen tragen.

\smallskip

\textit{Donner aus der Tiefe}

\smallskip

Ich will den dunpfen Klang der Glocken wecken.

\smallskip

\textit{Der Maulwurf unter der Erde}

\smallskip

Indessen werde ich das Grab ihm graben.

\smallskip

\textit{Die Zeit}

\smallskip

Mit grünem Rasen will ich es bedecken.

\smallskip

\textit{Nachtvögelgeschwärme} \\
(über den Mond hinfliegend)

\smallskip

Wir lassen uns zum Leichenschmaus ansagen.

\smallskip

\textit{Eine Stimme}

\smallskip

Bestellt ist Leichenfahrt und Festgelag! \\
Erbleichend stirbt der Mond im Dämmerflore, \\
Der Morgenstern erschließt die goldnen Tore -- \\
Schon graut der Tag! -- Schon graut der Tag!

\smallskip

\textit{Geisterchor}

\smallskip

Schon graut der Tag! -- Schon graut der Tag! 

\smallskip

(Er verschwindet)

\smallskip

\podpis{přeložil Alfred Waldau}

\bigskip

\textbf{I. Intermezzo}

\smallskip

\textit{ÉJFÉL.}

\smallskip

(Táj.)

\smallskip

A széles fennsíkon szunnyad a fényes hold, \\
A tóra csillagárt borít az égi bolt. \\
Kis domb: vesztőhely néz a partra. \\
Fönt roppant rúd, rúdon korong, \\
Rajt' koponya, és körbe' csont. \\
Kerék köré lelkek csapatja, \\
Dermesztő rémek raja ront. 

\smallskip

\textit{Szellemek kara.}

\smallskip

\frqq Az éj-szakon az úr a csend, \\
Új síron lidérc fénye leng. \\
Oly szederjes e síri fény, \\
Miként a sírontúli lény, \\
Ki alvó népét őrzi itt, \\
S fejfájának támaszkodik: \\
Az új halott nem alhatik. \\
Nem ég az égnek alja, nem, \\
Felleg fedi, fent hold vakít, \\
Holt fényt bámul a vaksi szem, \\
Tátongó fejre fény szakad, \\
Az állkapcsa vicsori, vad. \flqq

\smallskip

\textit{Egy hang.}

\smallskip

\frqq Itt az idő! -- Kap vánkosul \\
Kerékagyat a szörnyű úr, \\
Készítsétek hát nyughelyét! \flqq

\smallskip

\textit{Szellemek kara }\\
(levéve a koponyát).

\smallskip

\frqq Halál honából jöjj elénk, \\
Éledj -- a szót szomjasan idd, \\
A társaságunk befogad. \\
Sokáig laktál itt magad, \\
Jön új halott, felvált ma itt. \flqq

\smallskip

\textit{A koponya} \\
(körükben a keréken forogva).

\smallskip

\frqq A csontjaimnak vágya, jaj, \\
Hogy lenne újra egybe mind! \\
Mily rémes-hemzsegő e raj! \\
Új álmom int! -- Új álmom int! -- \flqq

\smallskip

\textit{Egy hang.}

\smallskip

\frqq A nyughely íme készen áll. \\
Holnap, ha új éjfél jön, új, \\
A vihar minket erre fúj: \\
Holtnak dicső gvászpompa jár. \flqq

\smallskip

\textit{Szellemek kara.}

\smallskip

\frqq A nyughely íme készen áll. \\
Holnap, ha új éjfél jön, új, \\
A vihar minket erre fúj: \\
Holtnak dicső gyászpompa jár. \flqq

\smallskip

\textit{Egy hang}.

\smallskip

\frqq Hangom a sík mezőbe szól: \\
Gyászos kor lesz az éji kor! \\
Mit ad majd, ki-ki mondja meg. \flqq

\smallskip

\textit{A vesztőhely a kerékkel.}

\smallskip

\frqq A koporsója leszek én. \flqq

\smallskip

\textit{A békák a mocsárból.}

\smallskip

\frqq Mi mondunk gyászos éneket. \flqq

\smallskip

\textit{A vihar a tó fölött.}

\smallskip

\frqq Kíséret gondja lesz enyém. \flqq

\smallskip

\textit{A Hold a Zenithen.}

\smallskip

\frqq Száll-ítok fehér szemfedőt. \flqq

\smallskip

\textit{A köd a hegyeken.}

\smallskip

\frqq Nagy árnyék-leplem lepje őt. \flqq

\smallskip

\textit{Az éj.}

\smallskip

\frqq En kölcsönzöm a gyászruhát. \flqq

\smallskip

\textit{A hegyek koszorúja.}

\smallskip

\frqq Adjatok gyászt, testünk-fedőt. \flqq

\smallskip

\textit{A hulló harmat.}

\smallskip

\frqq Ruhátok' könnyem járja át. \flqq

\smallskip

\textit{A föld.}

\smallskip

\frqq A föld lehelli illatát. \flqq

\smallskip

\textit{A fekete felleg.}

\smallskip

\frqq Megöntözöm koporsaját. \flqq

\smallskip

\textit{A hulló virág.}

\smallskip

\frqq En reá koszorút fonok. \flqq

\smallskip

\textit{A könnyű szél.}

\smallskip

\frqq A koporsóra viszem én. \flqq

\smallskip

\textit{A szentjánosbogarak.}

\smallskip

\frqq Mi leszünk majd a gyertyafény. \flqq

\smallskip

\textit{A mélyről örvénylő vihar.}

\smallskip

\frqq En ébresztem harang szavát. \flqq

\smallskip

\textit{A vakond a föld alatt.}

\smallskip

\frqq Sírgödröt fúrni: enyém a gond. \flqq

\smallskip

\textit{Az idő.}

\smallskip

\frqq S az én gondom a síri domb. \flqq

\smallskip

\textit{A hold előtt szálló madársereg.}

\smallskip

\frqq Mi tartjuk a gyászlakomát. \flqq

\smallskip

\textit{Egy hang.}

\smallskip

\frqq A gyászpompa el nem marad. -- \\
Kihúny a sápadt hold, nem ég, \\
Kitárj a kapuj át az ég. \\
Nappal van! Ujra nap! \flqq

\smallskip

\textit{Szellemek kara.}

\smallskip

\frqq Nappal van! Ujra nap! \flqq

\smallskip

(eltűnnek.)

\smallskip

\podpis{přeložil Efraim Israel, 2000}

\end{verse}