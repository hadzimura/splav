\section{Michel Faber: \\ Kvítek karmínový a bílý}

\noindent
\textbf{Část první: V ulicích}

\medskip

\noindent
\textit{Kapitola první}

\medskip

\noindent
Pozor, kam šlapete. A buďte ve střehu, ještě se vám to bude hodit. Město, do něhož vás vedu, je rozlehlé a spletité a vy jste do něj zavítali poprvé. Možná si představujete, že ho dobře znáte z jiných příběhů, které jste četli, ale ty příběhy vám nadbíhaly, vítaly vás jako přátele, chovaly se k vám, jako byste sem patřili. Věc se má tak, že jste vetřelci z úplně jiného času a místa. 

Když jsem poprvé zachytila váš pohled a vy jste se rozhodli jít se mnou, nejspíš jste si mysleli, že sem prostě dorazíte a uděláte si pohodlí. A teď tu skutečně jste a najednou vás někdo v třeskutém mrazu vede neproniknutelnou tmou, škobrtáte po nerovné ulici a nic nepoznáváte. Rozhlížíte se doleva a doprava, mrkáte v ledovém větru, uvědomujete si, že jste vešli do neznámé ulice, dvou řad neosvětlených domů plných neznámých lidí. 

A přece jste si mne nevybrali naslepo. Byla vzbuzena jistá očekávání. Upejpavost stranou: doufali jste, že uspokojím všechny vaše touhy, které jste ze samé plachosti nedokázali ani pojmenovat, anebo že si se mnou alespoň užijete. Teď váháte, pořád se mě držíte, ale máte nutkání pustit se mne. Když jste se mě poprvé chopili, nenadáli jste se, že budu tak silná, ani jste neočekávali, že vás popadnu tak pevně a rychle. Do tváří vás bodá déšť se sněhem, ostré jehličky tak mrazivé, až pálí, jako žhnoucí oharky ve větru. V uších vás začíná bolet. Ale nechali jste se svést na scestí a teď už je pozdě se obrátit zpátky.

V tuhle hodinu je noc popelavá, černošedá a téměř čitelná, tak jako netknuté stránky spáleného rukopisu. Klopýtáte kupředu do obláčků vlastního vydýchaného vzduchu, pořád za mnou. Dlažební kostky pod vašima nohama jsou mokré a zaneřáděné, vzduch je mrazivý, kyselý a páchne po pomalu se rozkládajícím trusu. Znedaleka se ozývají opilecké hlasy, těch pár zaslechnutých slov ovšem nezní jako pečlivě vybraný úvodní proslov velkého romantického dramatu; spíš si snažně přejete, aby se ty hlasy již nepřiblížily. 

Hlavní postavy tohoto příběhu, s nimiž se chcete důvěrně seznámit, stále zůstávají daleko. Nečekají na vás; nic pro ně neznamenáte. Jestli si myslíte, že kvůli vám vstanou z vyhřátých postelí a půjdou vám těch několik mil naproti, pak se mýlíte. 

Možná vás tedy napadne: proč jsem vás sem přivedla? Proč otálím a neseznámím vás s lidmi, o kterých jste si mysleli, že je brzy uvidíte? Odpověď je jednoduchá: jejich služebnictvo by vás nepustilo dovnitř. 

Chybí vám ty správné známosti, a proto jsem vás sem zavedla: abyste si je udělali. Osoba, která nic neznamená, vás musí představit člověku, který skoro nic neznamená, a tento člověk dalšímu a tak dále a tak podobně, až nakonec můžete překročit práh a jste takřka jedním z rodiny.
Proto jsem vás přivedla sem do Church Lane ve čtvrti St Giles: našla jsem pro vás toho pravého.

Musím vás ale předem varovat, že vás představuji na společenském dně, posledním z posledních. Od bohatství Bedford Square a Britského muzea jsme možná jen pár set metrů, ale dělí nás od něj New Oxford Street jako řeka příliš široká, než aby šla přeplavat, a vy jste na špatné straně. Ujišťuji vás, že princ velšský si nikdy nepotřásl rukou s obyvateli této ulice, dokonce tu nikomu ani letmo nepokývl hlavou, ba ani pod pláštíkem noci nevyzkoušel zdejší prostitutky. A třebaže v Church Lane žije víc nevěstek než takřka v kterékoli jiné ulici v Londýně, jejich úroveň neodpovídá nárokům urozených pánů. Pro znalce je koneckonců žena mnohem víc než zdechlina, ti vám neprominou, že postele tu jsou špinavé, pokoje bídně vybavené, krby studené a že venku nečekají drožky.

Stručně řečeno, tohle je úplně jiný svět, kde prosperita je bájný sen stejně vzdálený jako hvězdy. Church Lane je z těch ulic, kde i kočky jsou z nedostatku masa vyzáblé a oči mají zapadlé, z těch ulic, kde byste muže, kteří se označují za dělníky, jen těžko přistihli při dělné práci a kde takzvané pradleny jen zřídkakdy perou prádlo. Dobrodějové tu nemohou prokázat žádné dobro, místní je odsud vyženou, ať si jdou po svých se zoufalstvím v srdci a s lejnem na botách. Vzorová ubytovna pro prokazatelně chudé, již tu před dvaceti lety otevřeli za velkých filantropických fanfár, dávno upadla do rukou osob špatné pověsti a hrozivě zchátrala. Z ostatních, věkovitějších domů čiší podzemní chlad, přestože jsou dvě až tři podlaží vysoké -- jako kdyby je vykopali z hluboké jámy, jako by šlo o rozkládající se archeologické pozůstatky ztracené civilizace. Staletí staré budovy se opírají o berle z železného potrubí; na rozpraskaná a zvetšelá místa jim majitelé přikládají léčivé obklady ze štuku, ovinují je smyčkami prádelních šňůr, flikují je hnijícím dřevem. Střechy, to je jedna velká bláznivá změť, horní okna popraskaná a začernalá jako cihlové zdivo a obloha nad nimi vypadá pevnější než vzduch, klenutý strop jako skleněná střecha továrny či železniční stanice -- kdysi dávno byla jasná a průhledná, dnes zatažená špínou.

Jenže vy jste dorazili za deset minut tři uprostřed mrazivé listopadové noci, a tak nemáte chuť kochat se výhledem. Ze všeho nejdřív musíte vymyslet, jak uniknout chladu a tmě, abyste se dostali pod střechu a k tomu, co jste si slibovali už od prvního doteku se mnou: informacím intimní povahy.

Krom slabého světla plynových lamp pouličního osvětlení na vzdálených rozích ulice nevidíte v Church Lane jediné světlo, ale to proto, že vaše oči přivykly výmluvnějším známkám bdění, než se lze nadít od chabé záře dvou svíček za ukoptěnými okenními tabulkami. Přicházíte ze světa, kde lze tmu vymést jediným zmáčknutím vypínače, ale život dokáže rozdat karty i jinak. Může se stát, že nebudete mít ruku plnou trumfů.
Pojďte se mnou do místnosti, kde se svítí. Dovolte mi, abych vás zatáhla dovnitř zadním vchodem, nechte se vést stísněnou chodbou, která smrdí po zvolna pobublávajícím koberci a špinavém prádle. Dovolte mi, abych vás zachránila před zimou. Vyznám se tu.

Na těch schodech dejte pozor, kam šlapete; některé jsou shnilé. Vím které; věřte mi. Došli jste tak daleko, proč tedy nezajít o kousek dál?
 Trpělivost je ctnost a bude bohatě odměněna.

Přirozeně -- já jsem se o tom nezmínila? -- vás teď opustím. Ano, bohužel je tomu tak. Ale zanechám vás v dobrých rukou, výborných rukou. Tady, v té světničce v podkroví, navážete první důvěrnější známost.

Je to milá dušinka; bude se vám líbit. A jestli ne, příliš na tom nesejde: jakmile vás navede na správnou cestu, můžete ji bez dalších cavyků opustit. Za pět let, co se protlouká světem, se nedostala ani na dohled dámám a pánů, mezi nimiž se budete později pohybovat; pracuje, žije a dozajista umře v Church Lane, napevno uvázaná k téhle barabizně.

Tak jako mnoho prostých žen, obzvláště prostitutek, se jmenuje Caroline a zastihnete ji, jak dřepí nad velkou keramickou mísou po okraj naplněnou vlažným roztokem vody, kamence a síranu zinečnatého. Improvizovanou štětkou, pozůstávajícím z dřevěné lžíce omotané starým obvazem, se teď pokouší otrávit, vysát či jinak zničit to, co jí dovnitř vpravil muž, s nímž jste se těsně minuli. Caroline štětku znovu a znovu namáčí a voda je čím dál špinavější -- což je pro ni jasné znamení, že sémě toho chlapa krouží uvnitř mísy, a ne v ní. 

Jak se utírá lemem dlouhé košile, povšimne se, že světlo z obou svíček slábne: jedna z nich už není víc než blikotající pahýl. Zapálí novou?
Nu, to záleží na tom, kolik je hodin, a Caroline nemá hodiny -- ostatně jako valná většina lidí v Church Lane. Jen pár lidí ví, jaký rok se píše, nebo dokonce to, že mělo uplynout osmnáct a půl století od té doby, co jednoho židovského buřiče odvlekli na popraviště za rušení veřejného pořádku. V téhle ulici nechodí lidé spát v nějakou určenou hodinu, ale když se zpijí ginem, nebo když jsou tak vyčerpaní, že už nemají sílu dál někomu ubližovat. V téhle ulici se lidé budí, když opium v cukrových vodičkách jejich dětí přestane ty chudinky držet na uzdě. V téhle ulici se slabší povahy odplazí do postele hned po západu slunce, leží vzhůru a poslouchají krysy. Do této doléhají církevní zvony a státní fanfáry jen velmi, velmi slabě. 
Caroline jako hodiny poslouží zatažená obloha a její světélkující obsah. Slova „tři hodiny ráno“ jsou pro ni možná bezobsažná, ale zato dokonale rozumí postavení měsíce oproti domům přes ulici. Chvíli se u okna snaží prohlédnout skrz tabulky potažené zmrzlou špínou, pak zakroutí kličkou a jemným tlakem okno otevře. Hlasité prasknutí ji na chvíli vyděsí, jestli snad nerozbila sklo, ale to se jen láme led. Jeho střepinky se sypou na ulici pod ní.

Tentýž vítr, který zatvrdil led, teď zaútočí na Carolinino polonahé tělo, celý dychtivý udělat z lesklé kapičky potu na husí kůži jejího ňadra blyštivou jinovatku. Caroline vezme roztřepený límec volné košilky do ruky, zatne ji v pěst a přitáhne si ji těsně až ke krku. Cítí, jak jí tvrdne bradavka přitisknutá na předloktí. 

Venku vládne takřka naprostá tma, neboť nejbližší kandelábr je šest domů od ní. Dláždění Church Lane již není jako pocukrované, v plískanici se sněhový poprašek rozpustil, takže dole zůstaly hroudy a chuchvalce břečky, které v nažloutlém světle plynových lamp připomínají monstrózní výrony semene.

Svět tam venku vám připadá opuštěný, když se zatajeným dechem stojíte za ní. Ale Caroline ví, že další holky jako ona jsou ještě vzhůru, stejně jako metaři, strážní a zloději, i blízká lékárna má otevřeno, kdyby snad někdo potřeboval laudanum. Na ulicích jsou pořád opilci, kteří buď usnuli s nedozpívanou písničkou na rtech, nebo právě umírají zimou, a ano, je dokonce možné, že kolem obchází nějaký chlípník a hledá lacinou holku.
Caroline zvažuje, že se obleče, vezme si šálu a zkusí štěstí v přilehlých ulicích. Má hluboko do kapsy, protože většinu dne prospala a pak si nechala ujít svolného klienta -- nepozdával se jí jeho kukuč, připadal jí jako syfilitik. Teď lituje, že ho nechala jít. Už měla dávno vědět, že nemá cenu čekat na ideálního chlapa.

Na druhou stranu, kdyby měla znovu vyrazit, musela by zapálit další dvě svíčky, své poslední. Musí vzít v potaz i nevlídné počasí: při těch prostocvicích v posteli se člověku pěkně rozehřeje a pak vyjde ven a celý zase prochladne; kdysi dávno jí jeden medik při natahování kalhot řekl, že takhle se dá chytnout zápal plic. Caroline chová k zápalu plic zdravou úctu, i když si ho plete s cholerou a má za to, že při kloktání ginu s bromidem by měla dobrou šanci na přežití.

Jacka Rozparovače se bát nemusí, na něj je ještě čtrnáct let brzy a do té doby, než přijde na scénu, zemře z více či méně přirozených příčin. A on se do St Giles stejně nebude obtěžovat. Jak jsem vás už upozorňovala, představuji vás na společenském dně. 

Obzvláště jedovatý poryv větru přinutí Caroline zabouchnout okno a zase se neprodyšně uzavřít v krabicovitém pokojíku, který není její a jejž si ani přesně vzato nepronajímá. A protože nechce být líná coura, ze všech sil se snaží představit si, jak venku kráčí s tajuplným výrazem, zkouší vyvolat představu potenciálního kunčofta, který vystoupí ze tmy a řekne jí, že jí to sluší. Nepřipadá jí to pravdě\-podobné. 

Caroline si promne tvář hrstí svých vlasů, vlasů tak hustých a černých, že se jimi s obdivem probírají i ti největší hrubiáni. Jsou velmi jemné, dýchá z nich teplo a na tvářích a očních víčcích ji příjemně hladí. Když ale ruce odtáhne, zjistí, že jedna svíčka utonula v loužičce tuku, zatímco ta druhá se stále snaží udržet planoucí hlavičku nad ní. Den skončil, musí si to přiznat, víc si toho dnes nevydělá.

V rohu jinak prázdné místnosti se prohýbá zkrabacená a zpola rozpáraná postel připomínající obvázanou končetinu, kterou kdosi nemoudře a nevybíravě začal vytírat špinavou podlahu. Konečně přišel čas využít tuto postel ke spánku. Caroline velmi opatrně vklouzne pod deku, tak aby podpatky neprotrhla oslizlé prostěradlo. Boty si sundá později, až jí bude tepleji a přestane ji být protivné pomyšlení na to, jak bude muset rozepnout ty dlouhé řady knoflíků. 

Knot zbývající svíčky utone dřív, než se na něj stačí naklonit a sfouknout ho. Caroline spočine hlavou na polštáři páchnoucím po alkoholu a mnoha a mnoha čelech. 

Už se nemusíte schovávat. Udělejte si pohodlí, v místnosti je koneckonců naprostá tma a tak to také zůstane až do východu slunce. Kdyby se vám chtělo, mohli byste dokonce zariskovat a lehnout Caroline po bok, protože spí, jako kdyby ji do vody hodil, a nevnímala by vás -- jen byste se jí nesměli dotknout.

No tak, nic vám nebrání. Teď už spí. Zvedněte přikrývku a polehoučku vklouzněte za ní. Nevadí, jestli jste žena: v téhle době spolu ženy spí zcela běžně. A jestli jste muž, tak na tom sejde ještě méně: už tu před vámi byly stovky jiných.

\podpis{přeložil Viktor Janiš}