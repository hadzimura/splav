\section{George Eliotová: Middlemarch}

\noindent
(\textit{úryvek})

\medskip

\noindent
Svatba se slečnou Brookovou se ho rozhodně týkala víc než kohokoli z těch, kdo ji už stačili odsoudit, a já osobně zatím prožívám s větší účastí jeho úspěch než zklamání milého sira Jamese. Pana Casaubona totiž blížící se datum svatby nijak netěšilo a pomyšlení na procházku růžovým manželským sadem, v němž bude podle všech zkušeností cesta lemovaná květy, ho dlouhodobě nelákalo víc než podzemní chodby, v nichž byl zvyklý si svítit na cestu loučí. Sám si nepřiznal -- a tím spíš to nemohl ani naznačit někomu jinému --, že ačkoli získal líbeznou a velkodušnou dívku, nepřineslo mu to kupodivu žádnou radost (i tu totiž řadil k věcem, které je třeba hledat takříkajíc s lupou v ruce). Mnohokrát si sice procházel pasáže z klasiků, kde se o radosti zamilovaných mluví, jenže takovým procházením se člověk unaví a pak mu zbývá jen málo energie k tomu, aby ta moudra osobně uváděl do praxe.   

\looseness+8
Chudák pan Casaubon si představoval, že si po dlouhých létech staromládeneckého bádání teď vybere svůj podíl radovánek i s úroky a že pokladnice jeho citů spolehlivě uhradí všechny závazky, které na sebe její majitel vzal; ať jsme totiž seriózní nebo lehkomyslní, všichni se ve svých úvahách zaplétáme do metafor a děláme na jejich základě osudné životní kroky. Panu Casaubonovi teď hrozila melancholie, způsobená paradoxně právě vědomím, že mu štěstí mimořádně přeje. Z vnějších příčin si totiž nijak nedokázal vysvětlit jistou citovou prázdnotu, která se dostavovala právě ve chvílích, kdy by měl podle všeho očekávání prožívat největší štěstí, právě když vyměňoval šeď lowické knihovny, na niž byl zvyklý, za návštěvy na tiptonském statku. Bylo to únavné a on byl nucen svou únavu skrývat a prožívat ji naprosto osaměle, stejně jako zoufalství, jež mu občas hrozilo, když se plahočil bažinou svého spisování a cíl se zdál pořád stejně daleko. Jeho osamění bylo tou nejhorší samotou ze všech, protože se vyhýbalo soucitu. Pochopitelně si přál, aby Dorothee připadal právě tak šťastný, jak by měl její úspěšný nápadník podle obecného mínění být, a coby spisovatel spoléhal na její mladistvou důvěru a obdiv. Rád v ní vzbouzel nový zájem a ochotu naslouchat a dodával si tak odvahy: v hovorech s ní líčil všechno, čeho dosáhl a chce dosáhnout, se sebedůvěrou pedagoga, která je odrazem žákovy důvěry, a na chvíli unikal před oním chladným pomyslným publikem, které ho v hodinách namáhavé, otupující práce tížilo mlžnými výpary podsvětních stínů.
Po tom pimprlovém divadle světových dějin, z něhož převážně sestávalo vzdělání tehdejších dívek, se totiž Dorothee vyprávění pana Casaubona o jeho velké knize zdálo být plné nových náhledů a tenhle zážitek objevování, překvapení z toho, že stoikové a alexandrijci se při bližším seznámení jeví jako lidé s myšlenkami ne až tak nepodobnými těm jejím, zatím tlumily její obvyklou touhu po nějaké teorii, která by dokázala propojit její život i zásady s onou úžasnou minulostí a dosáhnout toho, aby i ty nejvzdálenější prameny vědění nějak ovlivňovaly její činy. Takové důkladnější učení teprve přijde -- pan Casaubon jí všechno vysvětlí: na další zasvěcování do myšlenek se těšila stejně jako na manželství a její mlhavé představy o obojím se navzájem prolínaly. Pokud by si někdo myslel, že by Dorothea snad stála o podíl na snoubencově učenosti jen pro učenost samu, byl by na velkém omylu. Ačkoli o ní freshittští a tiptonští tvrdili, že je chytrá, v kruzích, jejichž přesnější slovník definuje chytrost jako pouhou schopnost nabývat a používat vědomostí bez ohledu na charakter, by to nepovažovali za výstižný popis. Veškerá její touha po poznání se vlévala do proudu vřelé lidské účasti, kterým se obvykle nechávala unášet ve svých myšlenkách a duševních hnutích. Nechtěla se znalostmi zdobit -- vystavovat je navenek, nepropojené s  nervy a krví, které uváděly v život její činy, a kdyby napsala knihu, udělala by to jako svatá Tereza jedině na příkaz vyšší instance, schopné zapůsobit na její svědomí. Přece ale toužila po něčem, co by její život naplnilo racionálním a zároveň vroucným snažením -- a jelikož doby zjevených rad i duchovních vůdců minuly a jelikož modlitba jen umocňovala touhu, ale nedávala žádný návod k naplnění, jakou lampou si měla svítit na cestu, když ne věděním? Olej do ní mají jistě jedině učenci -- a kdo by mohl být učenější než pan Casaubon?

Dorotheino radostné a vděčné očekávání tedy v těch krátkých týdnech námluv nedostalo žádnou trhlinu, a ačkoli její nápadník občas pociťoval určitou prázdnotu, nemohl to ani v nejmenším přičítat odlivu jejího láskyplného zájmu.       
     
Roční období bylo natolik příznivé, že bylo možné podniknout svatební cestu až do dalekého Říma, na čemž panu Casaubonovi velmi záleželo, protože si chtěl prohlédnout jakési rukopisy uchovávané ve Vatikánu. 

,,Stejně lituji, že s námi nejede vaše sestra,`` řekl jednou ráno, nedlouho poté, co se potvrdilo, že Celie jet nechce a že Dorothea po její společnosti netouží. ,,Budete muset být dlouhé hodiny sama, Dorotheo, protože okolnosti mě nutí ten čas, který strávíme v Římě, maximálně využít, a kdybyste měla nějakou společnost, připadal bych si svobodnější.``

Slova ,,připadal bych si svobodnější`` se Dorothey nepříjemně dotkla. Poprvé jí při hovoru s panem Casaubonem zrudly tváře podrážděním.
,,Asi jste mi špatně rozuměl,`` řekla, ,,jestli si myslíte, že nedokážu pochopit cenu vašeho času -- a že se ochotně nevzdám všeho, co by vám bránilo ho co nejlíp využít.``

,,To je od vás moc hezké, má drahá Dorotheo,`` odpověděl pan Casaubon, který si vůbec nevšiml, že se jí to dotklo, ,,ale kdybyste měla společnici, svěřil bych vás obě průvodci a mohli bychom tak využít čas k dvojímu účelu.``

\looseness+1
,,Prosím vás, už o tom nemluvte,`` řekla Dorothea trochu povýšeně. Hned se ale zalekla, že to přehnala, obrátila se k němu, dotkla se jeho ruky a už jiným tónem dodala: ,,Nedělejte si o mě starosti, prosím. Můžu si o samotě promyslet spoustu věcí. A paní Tantrippová mi bude jako společnice úplně stačit, postará se, oč bude třeba. Celii bych vedle sebe nesnesla: jen by se s námi trápila.``

Nadešel čas se převléknout. Ten večer se měla konat slavnostní večeře, poslední z večírků, které se na Grange pořádaly, jak bylo před svatbou zvykem, a Dorothea byla ráda, že se může po zazvonění hned vzdálit, jako by se na večer potřebovala připravit pečlivěji než obvykle. Styděla se za svoje podráždění, jehož příčiny nemohla ani sama pro sebe pojmenovat. Třebaže nechtěla být neupřímná, její odpověď se nedotkla toho, co ji doopravdy zranilo. To, co řekl pan Casaubon, bylo úplně rozumné, ale přesto jí po jeho slovech prokmitl hlavou neurčitý pocit, že se k ní chová jako cizí člověk.        

Co je to za divnou sobeckou slabost, říkala si v duchu. Vždyť když budu mít manžela, který mě v tolika věcech převyšuje, musím mít přece na paměti, že mě potřebuje míň než já jeho.

Když se Dorothea ujistila, že pan Casaubon má ve všem pravdu, znovu se uklidnila, a když ve svých stříbrošedých šatech vcházela do salónu, byla přímo ztělesněním vyrovnané důstojnosti -- tmavohnědé vlasy, rozdělené jednoduchou pěšinkou a vzadu stočené do velkého uzlu, dokonale ladily s jejím chováním a výrazem, jimiž se ani v nejmenším nesnažila udělat na někoho dojem. Někdy vypadala Dorothea ve společnosti tak uvolněně, jako by to byla sama svatá Barbora vyklánějící se ze své věže do průzračného vzduchu; jakmile ji ale nějaký podnět zvenčí přiměl k reakci, působily po takových chvílích klidu její emoce a slova o to důrazněji.      

\looseness+1
Ten večer o ní mezi hosty přirozeně padlo mnoho poznámek, protože se u Brookových sešla velká společnost, co do mužské účasti rozmanitější než na kterémkoli z předchozích večírků, jež se na Grange konaly od té doby, co u pana Brooka začaly bydlet jeho neteře, a hovory tedy probíhaly ve dvojicích a trojicích víceméně  nesourodých. Byl tu nově zvolený starosta Middlemarche, jenž byl majitelem malé továrny, jeho švagr, bankéř a filantrop, který byl ve městě natolik výraznou figurou, že ho podle své slovní zásoby někteří nazývali metodistou a jiní pokrytcem, a byli tu také různí příslušníci svobodných povolání. Paní Cadwalladerová se nechala slyšet, že Brooke si začíná předcházet middlemarchské voliče, ale ona má radši sedláky u desátkové večeře, kteří jí skromně připijí na zdraví a nestydí se za nábytek po dědečkovi. Před reformou, která podstatně přispěla k politickému uvědomování obyvatelstva, se totiž v tomhle koutě Anglie vnímaly ostřeji společenské než politické rozdíly a pestrou skladbu hostí z různých vrstev tedy lidé brali jako projev páně Brookovy přílišné uvolněnosti, způsobené nezřízeným cestováním a nadměrnou konzumací myšlenek.       

Jen co Dorothea vyšla z jídelny, naskytla se příležitost k několika emotivním poznámkám.

,,To je ale ženská, ta slečna Brooková! Mimořádně pěkná, přisámbůh!`` pravil pan Standish, starý právník, který se tak dlouho staral o záležitosti statkářské šlechty, až se sám stal statkářem -- a své ,,přisámbůh`` teď pronesl patřičně hluboce, jako by to byl erb pečetící slova význačného muže.

Zdálo se, že se pan Standish obrací svou poznámkou k bankéři Bulstrodeovi, ale ten neměl rád neomalenost a nesnášel, když někdo bral jméno Boží nadarmo, a tak jen pokývl hlavou. Mezitím se těch slov chytil pan Chichely, starý mládenec mezi čtyřicítkou a padesátkou, milovník lovu, který obličejem připomínal velikonoční kraslici, na hlavě měl několik pečlivě naaranžovaných vlasů a z jeho držení těla bylo znát, že si je vědom své důstojnosti. 

,,Ano, ale můj typ to není. Já dávám přednost ženám, co se trochu víc snaží se nám zalíbit. Ženě sluší špetka rafinovanosti -- měla by být trošku koketa. Mužský má rád hozenou rukavici. Čím víc se ženská snaží vás dostat, tím líp.``

,,Na tom něco je,`` poznamenal pan Standish, který byl žoviálně naladěn. ,,A většina jich takových je, přisámbůh. Nejspíš to slouží nějakým vyšším záměrům: Prozřetelnost je tak stvořila, nemám pravdu, Bulstrode?``

,,Já osobně bych koketerii připisoval působení jiných mocností,`` pravil pan Bulstrode. ,,Podle mě je to spíš ďáblovo dílo.``     
     
,,No jistě, každá ženská by měla mít v sobě kus čerta,`` přitakal pan Chichely, jemuž pozorování krásného pohlaví zjevně poněkud zdeformovalo teologické představy. ,,A nejradši mám blondýnky s labutí šíjí, co se tak krásně nesou. Mezi námi, starostova dcera se mi zamlouvá víc než slečna Dorothea i slečna Celie. Kdybych se chtěl ženit, dal bych slečně Vincyové přednost před nimi oběma.``

,,Tak do toho, do toho,`` povzbuzoval ho žertem pan Standish, ,,jak vidíte, dneska jsou v kursu padesátníci.``

Pan Chichely významně zavrtěl hlavou: nehodlal si koledovat o sladké ,,ano`` od ženy, kterou by si sám vybral. 

Slečna Vincyová, jež měla tu čest být jeho ideálem, samozřejmě na večírku chyběla, protože pan Brooke se svým heslem ,,všeho s mírou`` přece jen nechtěl, aby se jeho neteře setkaly s dcerou middlemarchského továrníka jinde než na veřejnosti. Ženská část společnosti nezahrnovala nikoho, vůči komu by mohly mít lady Chettamová nebo paní Cadwalladerová nějaké výhrady. Paní Renfrewová, vdova po plukovníkovi, měla totiž nejen bezvadné vychování, ale navíc vzbuzovala zájem svou chorobou, nad níž si lékaři lámali hlavu a u níž by bylo pro úspěšné vyléčení možná zapotřebí doplnit odbornou erudici ještě špetkou šarlatánství. Lady Chettamová, která připisovala své pozoruhodné zdraví domácímu hořkému pivu a nepřetržité lékařské péči, se s velkým vypětím obrazotvornosti vžila do popisovaných příznaků paní Renfrewové a prožívala s ní neuvěřitelnou neúčinnost všech posilujících léků.            

,,Kam se z těch léků ztrácí všechen ten posilující účinek, má milá?`` podivila se ta vlídná, ač impozantní dáma, a když pozornost paní Refrewové upoutalo něco jiného, zadumaně se obrátila k paní Cadwalladerové. 

,,Posiluje tu nemoc,`` pravila paní rektorová, která byla tak urozená, že se také musela amatérsky zabývat medicínou. ,,Všechno závisí na tom, jakou má kdo přirozenost: někomu se v těle vyrábí tuk, někomu krev a někomu zase žluč -- tak to vidím já. A všecko, co takový člověk sní a vypije, mu slouží jako surovina.``   

,,V tom případě by měla brát léky, které ji oslabí -- chci říct tu nemoc, pokud máte pravdu, má drahá. A to, co říkáte, zní rozumně.``

,,Ovšemže je to rozumné. Vezměte si dva druhy brambor, co rostou ze stejné hlíny. Jedny jsou pořád vodnatější --``

,,Á! Jako chudinka paní Renfrewová -- moje řeč. Vodnatelnost! Ještě nic neotéká -- všechno je to vevnitř. Podle mě by měla brát léky, co z ní tu vodu vytáhnou, nemyslíte?  -- Nebo vysušovací horkovzdušnou lázeň. Existuje spousta odvodňovacích kůr.``

,,Ať zkusí brožury jistého pána,`` poznamenala polohlasem paní Cadwalladerová, protože viděla vcházet pánskou společnost. ,,Ten je suchý až dost.``

,,Kdo, má milá?`` zeptala se lady Chettamová, roztomilá dáma, jež nebyla natolik bystrá, aby zkazila paní Cadwalladerové požitek z vysvětlení.     
,,Přece ženich -- Casaubon. Od zásnub se nám sesychá ještě rychleji: to bude tím žárem vášně.``

,,Ten na tom asi se zdravím není nijak valně,`` prohlásila lady Chettamová ještě tlumeněji. ,,A to jeho bádání je taková suchá, nezáživná práce, přesně jak říkáte.`` 

,,Bodejť, vedle sira Jamese vypadá jako Smrťák, kterému na tu slávu potáhli lebku kůží. Dejte na mě: do roka a do dne ho ta holka bude nenávidět. Teď k němu vzhlíží jako k pánubohu a za nějakou dobu  bude u druhého extrému -- ode zdi ke zdi. Taková jankovitá povaha!``

,,To je strašné! Bude zřejmě tvrdohlavá. Ale povězte mi -- vy o něm víte všechno -- je to hodně špatné? Jaký on vlastně je?`` 

,,Jaký? Je jako špatně předepsaný lék -- nechutný už od pohledu, a když si ho vezmete, zaručeně se vám přítíží.``

,,Nic horšího si nedovedu představit,`` odtušila lady Chettamová, která si takový lék uměla představit natolik živě, že zřejmě získala naprosto přesný obrázek o páně Casaubonových nectnostech. ,,Ale James nedá na slečnu Brookovou dopustit. Říká, že i tak by se v ní jiné ženy mohly zhlížet.``     

,,To říká jen ze zdvořilosti. Celinka se mu líbí víc, na to můžete vzít jed, a ona mu taky přeje. Doufám, že vám se Celinka zamlouvá?``

,,Jistě -- má mnohem radši muškáty a připadá mi taková povolnější, i když nemá tak pěknou postavu. Ale když už jsme mluvily o lécích -- povězte mi něco o tom novém mladém doktorovi, panu Lydgateovi. Prý je úžasně inteligentní: rozhodně na to vypadá -- má takové  klenuté čelo\ldots``   

,,Má dobré vychování. Slyšela jsem, jak si povídali s Humphreym. Mluvit umí.``

,,Ano. Pan Brooke říká, že je příbuzný Lydgateových z Northumberlandu, to je moc dobrá rodina. U obyčejného praktika by to člověk nečekal. Já osobně vidím radši, když má doktor blíž ke služebnictvu; takoví často bývají ještě chytřejší. Ujišťuju vás, že na diagnózy chudáka Hickse jsem se mohla naprosto spolehnout: nepamatuju si, že by se někdy mýlil. Byl dost drsný, takový řezník, ale znal mě dokonale. Byla to pro mě rána, když tak najednou umřel. No ne, podívejte, s jakým zaujetím se slečna Brooková baví s tím panem Lydgatem!``

,,Ta s ním probírá chalupy a špitály,`` poznamenala paní Cadwalladerová, jež měla bystré uši i schopnost domýšlet si souvislosti. ,,Nejspíš to bude nějaký filantrop, takže ho určitě vezme pod křídla Brooke.``

,,Jamesi,`` ozvala se lady Chettamová, když se její syn přiblížil, ,,přiveď mi sem pana Lydgatea a představ nás. Chci si ho vyzkoušet.``

Vzápětí ona milá dáma prohlásila, jak ji těší, že se může s panem Lydgatem seznámit, protože slyšela o jeho úspěších s novou metodou léčení horečky.  

Pan Lydgate jako každý správný lékař uměl vyslechnout s dokonale vážnou tváří sebevětší nesmysl a jeho tmavé, upřeně hledící oči mu dodávaly vzhled zaujatého posluchače. Stěží by se našel dokonalejší protiklad k oplakávanému Hicksovi, zejména pokud jde o jistou nedbalou eleganci Lydgateova oblékání i vyjadřování. Přesto přese všechno si získal velkou důvěru lady Chettamové. Potvrdil jí to, o čem byla přesvědčena, totiž že její tělo reaguje zcela specificky, protože připustil, že každé tělo reaguje specificky, a nepopřel, že to její může reagovat specifičtěji než ostatní. Nebyl zastáncem drastického srážení horečky a pouštění žilou, ale nedoporučoval ani přemíru portského s chininem, a když s něčím vyjadřoval souhlas, říkal ,,to je možné`` s tak uctivým výrazem, že si lady Chettamová utvořila velmi příznivé mínění o jeho lékařských schopnostech.

,,Ten váš chráněnec se mi opravdu zamlouvá,`` prohodila k panu Brookovi těsně před odchodem. 

,,Chráněnec? Prosím vás! Kdo to má být?`` podivil se pan Brooke. 

,,Mladý Lydgate, ten nový doktor. Podle mě výtečně zná svoje řemeslo.``

,,Ách, Lydgate! To není můj chráněnec -- jen jsem kdysi znával jeho strýce, a ten mi o něm napsal. Ale myslím, že to bude znamenitý doktor -- studoval v Paříži, znal se s Broussaisem, má nápady\ldots Chce pozvednout medicínu.``

,,Lydgate má spoustu úplně nových nápadů, co se týká okysličování krve, stravovacího režimu a tak dále,`` pokračoval pan Brooke, když se náležitě rozloučil s lady Chettamovou a vrátil se jako hostitel ke skupince middlemarchských. 

,,Ále, u všech čertů, vám to připadá rozumné? Stavět se proti starým dobrým metodám, které udělaly z Angličanů to, co jsou?`` ozval se pan Standish. 

,,Úroveň lékařských znalostí je u nás velmi nízká,`` podotkl pan Bulstrode, který mluvil polohlasem a vypadal poněkud nezdravě. ,,Za sebe mohu říci, že příchod pana Lydgatea vítám. Doufám, že mu časem budu moci  bez obav svěřit vedení naší nové nemocnice.``  

,,To je všechno moc hezké,`` opáčil pan Standish, který neměl pana Bulstrodea rád. ,,Jestli ho chcete nechat, ať si zkouší ty svoje experimenty na pacientech ve vaší nemocnici a pár lidí milosrdně zabije, prosím. Ale nebudu platit z vlastní peněženky za to, aby na mě někdo dělal pokusy a nakonec mě nechal vykrvácet nebo co. Mám radši krapet prověřenou léčbu.``             
 
,,Lepší vykrvácet finančně, než doopravdy, ehm\ldots co, Standishi?`` kývl pan Brooke na právníka.

,,No když to berete takhle!`` opáčil pan Standish jen s  takovou mírou znechucení nad tímhle neprávnickým slovíčkařením, jakou si člověk může dovolit vůči významnému klientovi.    

,,Já bych byl vděčný za každou kúru, která mě vyléčí, a přitom ze mě neudělá kostlivce jako z chudáka Graingera,`` poznamenal starosta, růžolící pan Vincy, podle nějž by se malíři mohli učit míchat zdravou tělovou barvu a jenž představoval ostrý kontrast k františkánské bledosti pana Bulstrodea. ,,Když je člověk při těle, alespoň s ním nemoc tak snadno nezacloumá, jak někdo prohlásil -- a podle mě moc trefně.``

Pan Lydgate samozřejmě nebyl v doslechu. Odešel z večírku brzy a byl by se tam hrozně nudil, nebýt několika nových lidí, jimž byl představen, a mezi nimi zejména slečny Brookové, jejíž mladistvý půvab, blížící se sňatek s tím zchátralým učencem a zájem o společensky prospěšné věci působily svou neobvyklou kombinací pikantně.   

Má dobré srdce -- to krásné děvče --, ale všecko bere trochu moc vážně, pomyslel si. S takovými ženami je těžká řeč. Všechno by si chtěly odargumentovat, jenže při své nevzdělanosti nedokážou nikdy proniknout k jádru věci, a tak se nakonec řídí jen svým mravním cítěním a udělají to, co jim zrovna připadá nejlepší.

Slečna Brooková zjevně nevyhovovala vkusu pana Lydgatea o nic víc než vkusu pana Chichelyho. V očích posledně jmenovaného, jehož úsudek byl už zralý, představovala vyslovený omyl přírody, speciálně určený k tomu, aby otřásal jeho důvěrou v zákonitosti tohoto světa, zahrnující též přizpůsobování krásných dívek  brunátným starým mládencům. Lydgate ale takové zralosti ještě nedosáhl a nedalo se vyloučit, že v budoucnu zažije něco, co změní jeho názor na nejdůležitější přednosti ženy.

Slečnu Brookovou ale ani jeden z našich dvou pánů pod jejím dívčím jménem už znovu nespatřil. Zanedlouho po onom večírku se stala paní Casaubonovou a vydala se na cestu do Říma.         


\podpis{přeložila Zuzana Šťastná}