\section{Viktoriánská Anglie\ldots}

\noindent
Víte, že George Eliot nebyl muž, ale žena, a navíc skvělá spisovatelka? Román s názvem \textit{Middlemarch}, který napsala na konci 60. let 19. století, představuje jeden z největších textů viktoriánské Anglie. Doslova a do písmene. Jeho český překlad čítá něco kolem jednoho tisíce normostran! Snad i to trochu vysvětluje, proč byl do češtiny poprvé přeložen až teď -- i když je starý už víc než sto třicet let. To román \textit{Kvítek karmínový a bílý} od Michela Fabera je nepoměrně mladší. Vyšel před čtyřmi roky. Přesto však do dnešního večera rovněž patří. I on se totiž odehrává ve viktoriánské Anglii, pouze v trochu jiných reáliích. Ze salónů a budoárů \textit{Middlemarche} nás přenese do vykřičených londýnských čtvrtí. Anglie v dobách královny Viktorie zdaleka nebyla jen světem po krk upnutých dam a džentlmenů s aristokraticky zdviženým obočím\ldots Zkrátka a dobře, dnešní večer nám dá nahlédnout -- hned z několika různých konců -- do jednoho pozoruhodného období anglických dějin.

\podpis{Standa Rubáš}