\section{Fjodor Ivanovič Šaljapin: \\ Maska a duše}

\noindent
\textit{Fragment z knihy vzpomínek. Vyjde koncem roku v nakladatelství H+H. }

\medskip

\noindent
\looseness-1
Žhavě mě vábil sen, že do života ruských lidí, který mi připadal tak pochmurný, bude vneseno nové světlo, že přestanou tolik plakat, tak nesmyslně a tupě trpět, tak temně a hrubě se veselit v opilství. A tak jsem se celou duší připojil ke svým přátelům a spolu s nimi snil, že jednou nespravedlivé zřízení smete revoluce a na jeho místo pro štěstí ruského lidu postaví nový. 
Člověkem, který na mě v tomto směru měl silný a řekl bych rozhodující vliv, byl můj přítel Alexej Maximovič Peškov -- Maxim Gorkij. To on svým vášnivým přesvědčením a příkladem upevnil mou vazbu se socialisty, to jemu a jeho nadšení jsem uvěřil víc než komukoli a čemukoli jinému na světě. 

Vzpomínám si, že jméno Gorkého jsem poprvé uslyšel od svého milého přítele S. V. Rachmaninova. Bylo to v Moskvě. Přijde ke mně jednou do Leonťjevské uličky Serjoža Rachmaninov a přinese knihu. 

,,Přečti si,`` povídá, ,,jaký se nám objevil úžasný spisovatel. Bude asi mladý.``

Mám dojem, že to byla první Gorkého sbírka: \textit{Malva, Makar čudra} a další povídky z úvodního období. Ty příběhy se mi opravdu velmi líbily. Vanulo z nich cosi, co leží blízko mé duši. Musím říci, že až dosud se mi zdá, když čtu Gorkého díla, že města, ulice i lidi, jež popisuje, všechny znám. Všechny jsem je viděl, ale nikdy jsem si nepomyslel, že bude pro mě tak zajímavé znovu si je prohlédnout skrze knihu. Vzpomínám si, že jsem autorovi poslal do Nižního Novgorodu dopis s vyjádřením svého nadšení. Odpověď jsem ale nedostal. Když jsem v roce 1896 zpíval na tamější výstavě, Gorkého jsem ještě neznal. V rove 1901 jsem ale do Nižního přijel znovu. Zpíval jsem ve Veletržním divadle. Jednou jsem při představení \textit{Života za cara} dostal vzkaz, že v divadle je Gorkij a že se chce se mnou seznámit. V další přestávce ke mně přišel člověk s tváří, která mi připadla originální a přitažlivá, i když nepříliš hezká. Pod krásnými dlouhými vlasy, nad trochu směšným nosem a vystupujícími lícními kostmi žhnuly citem hluboké, dobré oči zvláštního jasu, připomínající jas jezera. K tomu kníry a malá bradka. Napůl se usmál, podal mi ruku, pevně stiskl mou a mně rodnou povolžskou výslovností se zdůrazněným ,,o`` řekl: ,,Slyšel jsem, že jste taky náš bratříček Isaakij (jeden z nás).``

,,Tak nějak,`` odpověděl jsem.

A tak jsme od prvního stisku ruky zřejmě pocítili jeden ke druhému sympatii -- nebo přinejmenším já. 

Začali jsme se často setkávat. Někdy on přišel za mnou do divadla, i přes den, a šli jsme spolu do Kunavina na pelmeně, naše oblíbené severské jídlo, jindy jsem já šel k němu, do jeho skromného bytu, vždy přeplněného návštěvníky. Bývali tu všemožní lidé, i zádumčiví, i veselí, i soustředěně ustaraní, i prostě lhostejní, ale z větší části byli všichni mladí a přicházeli, jak se mi zdálo, napít se chladné vody z toho krásného pramene, jaký jsme v A. M. Peškovovi-Gorkém spatřovali.

Prostota, dobrota, nenucenost toho zdánlivě bezstarostného mladého muže, jeho srdečná láska k jeho dětičkám, tehdy malým, a zvláštní laskavost Povolžana k manželce, okouzlující Jekatěrině Pavlovně Peškovové, to vše si mě tak získalo a tak mě to uchvátilo, že mi připadalo, že jsem konečně nalezl ten krb, u kterého je možno pozapomenout, co je to nenávist, naučit se lásce a začít žít jakýmsi zvláštním, jedině radostným a ideálním životem člověka. U toho krbu jsem už docela uvěřil, že pokud na světě jsou opravdu dobří, upřímní lidé, z duše milující svůj lid, tak je to Gorkij a jemu podobní, kteří jako on vidí tolik utrpení, strádání a všemožných otisků žalného lidského žití. Tím těžší pro mě bylo a tím víc mě mrzelo, když jsem viděl, jak Gorkého sebrali četníci, odvedli ho do vězení a poslali do vyhnanství na sever. \footnote{Maxim Gorkij žil v Nižním Novgorodu pod přísným policejním dozorem. Šaljapin toho byl svědkem. ,,Na sever`` však spisovatel do vyhnanství poslán nebyl.} V tu chvíli jsem začal pozitivně věřit, že lidé, kteří si říkají socialisté, tvoří kvintesenci lidského rodu, a má duše začala žít spolu s nimi. 

Dost často jsem pak v jarních a letních měsících přijížděl na Capri, kde (mimochodem v pronajatém, ne ve vlastním domě) Gorkij občas žil. Atmosféra v jeho domě byla revoluční. Já se ale musím přiznat, že mě zaujímaly a poutaly pouze humanitární poryvy těch idejemi rozrušených lidí. Když jsem ovšem činil nepříliš časté pokusy načerpat nějaké vědomosti ze socialistických knížek, měl jsem hned od první stránky pocit nevýslovné nudy a dokonce, přiznám zahanbeně, odpudivosti. Ale však také doopravdy -- k čemu já jsem potřeboval vědět, kolik se z pudu železa udělá koleček do hodin, kolik za jejich výrobu dostane první podvodník, kolik druhý, kolik třetí a co zůstane podvedenému dělníkovi? Hned jsem chápal, že někdo je ošizen a někdo šidí. ,,Upravovat vztahy`` mezi jedním a druhým se mi upřímně řečeno nechtělo. A tak jsem socialistickou učeností pohrdl\ldots Ale je to škoda! Kdybych se o socialismu víc poučil, kdybych věděl, že za socialistické revoluce musím o všechno až do posledního chlupu přijít, zachránil bych možná nejednu stovku tisíc rublů, kdybych si ruské revoluční peníze včas převedl do zahraničí a proměnil je v buržoazní valuty.

\bigskip

\noindent
\textbf{Pod bolševiky }

\noindent
Prozatímní vláda je svržena. Ministři jsou zatčeni. Do dobytého hlavního města triumfálně vjíždí Vladimír Iljič Lenin. O lidech, kteří se přes noc stali vládci Ruska, jsem měl velmi chabé ponětí. Mezi jiným jsem nevěděl, co je zač ten Lenin. Vůbec se mi zdá, že historické ,,postavy`` se vytvářejí buď ve chvíli, kdy je vezou na popraviště, nebo ve chvíli, kdy posílají na popraviště jiné. V té době se ještě střílelo jednotlivě a tajně, takže Leninova génia jsem já, politický nevzdělanec, dosud příliš nezpozoroval. O takovém Trockém jsem věděl víc. Ten chodil po divadlech a hned z galerie, hned z lóže hrozil pěstmi a přezíravým tónem říkal obecenstvu: ,,Na ulicích teče krev lidu a vy se, vy necitelní buržousti, chováte tak nízce, že posloucháte nicotné ubohosti, které vám prozpěvují neschopní komedianti\ldots`` O Leninovi jsem však byl naprosto nepoučen, a proto jsem ho na Finské nádraží vítat nejel, i když ho vítal Gorkij, který měl myslím v té době k bolševikům nepřátelský vztah. 

První trest boží, který mě -- zřejmě za onen čin -- stihl -- bylo, že mi jacísi mladí muži zrekvírovali automobil. Opravdu, nač potřebuje ruský občan auto, když ho nepoužil k tomu, aby jako věrný poddaný jel vítat vůdce světového proletariátu? Usoudil jsem, že můj automobil potřebuje ,,lid``, a velmi lehko jsem se utěšil. V tyhle první dny panství nových lidí si hlavní město ještě jasně neuvědomovalo, čím bude v praxi pro Rusko bolševický režim. A je tu první strašlivý otřes. Námořníci v nemocnici bestiálně zavraždili ,,nepřátele lidu``, pacienty Kokoškina a Šingareva, zatčené ministry Prozatímní vlády, nejlepší představitele liberální inteligence. \footnote{Andrej Ivanovič Šingarev (1869--1918), lékař, Fjodor Fjodorovič Kokoškin (1871--1918), soukromý docent, členové ústředního výboru strany kadetů.}

Vzpomínám se, jak mě po téhle vraždě otřesený Gorkij vyzval, abych s ním šel na ministerstvo spravedlnosti domáhat se propuštění dalších zatčených členů Prozatímní vlády. Přišli jsme do jakéhosi prvního patra velkého domu, kdesi na Koňušenné, myslím blízko Něvy. Přijal nás tam člověk v brýlích a s mocnou hřívou vlasů. Byl to ministr spravedlnosti Štejnberg.\footnote{I. Z. Štejnberg, levicový eser, od 9. (22.) prosince 1917 lidový komisař spravedlnosti. V dubnu 1918 z Rady lidových komisařů vystoupil.} V započatém rozhovoru jsem zaujímal skromnou pozici stafáže. Mluvil jen Gorkij. Rozčilen a bledý prohlašoval, že takové chování k lidem je odporné. ,,Trvám na tom, aby byli členové Prozatímní vlády okamžitě propuštěni, jinak se s nimi stane to, co se stalo se Šingarevem a Kokoškinem. Je to ostuda pro revoluci.`` Štejnberg se ke Gorkého slovům postavil s velkou sympatií a slíbil, že co nejdříve udělá všechno, co může. Kromě nás se myslím na úřady s podobnými naléhavými žádostmi obracely i jiné osoby, které vedly politický Červený kříž. Za nějakou dobu byli ministři propuštěni. V úloze zastánce nevině zatýkaných Gorkij v té době vystupoval velmi často, řekl bych dokonce, že to bylo hlavním smyslem jeho života v prvním období bolševismu. Setkával jsem se s ním často a pozoroval u něho velmi mnoho něhy k té třídě, které hrozil zánik. Z laskavosti srdce zatčené nejen osvobozoval, ale dával také peníze, aby tomu či onomu člověku pomohl zachránit se před tehdy zuřivě řádící, nevzdělanou a hrubou silou a uprchnout do zahraničí.    

Gorkij své pocity neskrýval a bolševickou demagogii otevřeně odsuzoval. Vzpomínám si na jeho řeč v Michajlovském divadle. \footnote{Šaljapin má na mysli vystoupení Maxima Gorkého na zasedání Svazu umělců v Michajlovském divadle 12. března 1917.} Revoluce, prohlásil, není bezduché řádění, ale ušlechtilá síla, soustředěná v rukou pracujícího lidu. Je to vítězství práce, stimulu, který uvádí svět do pohybu. Jak se tyto vznešené úvahy rozmělnily řečmi, které se ozývaly v témž Michajlovském divadle, na náměstích a v ulicích a krvelačnými výzvami k ničení a drancování! Velmi brzy jsem pocítil, jak zklamaně Gorkij pohlíží na rozvíjející se události i na nastupující nové činitele revoluce. 

Znovu, ne poprvé a ne naposled, musím říci, že je mi velmi málo pochopitelná také jedna podivná ruská skutečnost. Někdo řekne, že ten a ten je darebák, a hned se toho všichni chytnou. Kaž\-dý ochotně opakuje ,,darebák``, lehce jako laciný bonbon převaluje to slovo v ústech. Tak na tom v té době byl Gorkij. Hluboce trpěl a odevzdával svou duši -- odvážím se to tak říci -- obětem revoluce, a nějací velkododavatelé morálky šířili pověsti, že prý nemyslí na nic jiného než jak si doplnit umělecké sbírky, na které údajně vynakládá obrovské peníze. Jiní se vyjadřovali ještě lépe: Gorkij prý využívá utrpení a neštěstí oloupených aristokratů a bohatých lidí a za pakatel od nich skupuje drahocenná umělecká díla. Gorkij se opravdu nadšeně věnoval sběratelství, ale co to bylo za sběratelství: jednou sbíral staré ručnice nebo jakési čínské knoflíky, jindy španělské hřebínky a vůbec všemožné nesmysly. Pro něho to byly ,,výtvory lidského ducha``. U čaje nám třeba ukázal takový úžasný knoflík a říkal: ,,Tohle udělal člověk! Jakých výšin může dosáhnout lidský duch! Vytvořil takovýhle knoflík, zdánlivě nepotřebný. Chápete, jak je třeba si člověka vážit, jak je třeba milovat lidskou osobnost?``

My, jeho posluchači, jsme skrze obyčejný, ale čínskou řezbou zdobený knoflík jasně spatřili, že člověk je krásný výtvor Boží\ldots

Poněkud jinak ale pohlíželi na člověka lidé, kteří drželi v rukou moc. Navíc ti si zapínali a rozpínali, přišívali a odstřihovali jiné ,,knoflíky``. 
Revoluce probíhala plnou parou. 

\medskip

\noindent
Zažil jsem nesmírně skličující a pobouřené ohromení, když jsem krátce nato v Monte Carlu dostal od přítele malíře Serova hromadu novinových výstřižků o své ,,monarchistické demonstraci``. V \textit{Ruském slově}, které redigoval můj přítel Doroševič, jsem spatřil báječně vyvedenou kresbu, na níž jsem byl zobrazen u nápovědovy budky s vysoko zdviženýma rukama a doširoka otevřenými ústy. Pod obrázkem bylo napsáno: ,,Monarchistická demonstrace v Mariinském divadle vedená Šaljapinem.`` Když se tohle píše v novinách, myslel jsem si, co se asi předává od úst k ústům! Proto jsem se vůbec nepodivil smutné poznámce připsané Serovem: ,,Je to ale neštěstí, když už i ty lezeš po čtyřech. Měl by ses stydět.``

Napsal jsem mu, že nemá věřit hloupým pomluvám, a ten vzkaz jsem mu vytkl. Jenže zpráva o tom, jak jsem zradil lid, se mezitím donesla i do departmentu Alpes Maritimes. Jednoho dne jsem se vracel z Nice do Monte Carla, seděl jsem v kupé a rozmlouval s přítelem, když tu náhle nastoupili do vozu nějací mladí lidé, frekventanti, studenti, snad i nějací mladí úředníci, a začali na mou adresu pronášet všemožné urážky:
 ,,Lokaj, darebák, zrádce!``

Zabouchl jsem dveře kupé. Ti mladí nalepili na okno papír, na němž bylo velkými písmeny napsáno: ,,Slouho!``

Když o tom vyprávím svým ruským přátelům a ptám se, proč mě ti lidé uráželi, dosud mi odpovídají: ,,Protože na vás byli hrdí a měli vás rádi.`` 
Podivná, jakási uslintaná láska!

Byli to samozřejmě mladí lidé, svůj neurvalý čin si dovolili z důvodů naprosté ignorance a pochybné výchovy. Jak jsem si ale měl vyložit chování jiných, skutečně kulturních osob, které tisíce ctí a váží si jich jako učitelů života. 

Rok před touhle příhodou jsem zpíval právě v Monte Carlu. Do šatny za mnou přiběhl rozrušený člověk a s nefalšovanou upřímností mi řekl, že je mým zpěvem i hereckým výkonem hluboce dojat, že pouhým tímto večerem je jeho život naplněn. Snad bych si nadšených slov a pochval ani nijak zvlášť nevšiml, kdyby neuvedl své jméno -- Plechanov.

O tom člověku jsem samozřejmě slyšel. Byl to jeden z nejváženějších a nejvzdělanějších vůdců ruských sociálních demokratů a přitom nadaný publicista. A když mi řekl: ,,Tolik bych si s vámi chtěl posedět a vypít šálek čaje,`` s upřímným potěšením jsem odpověděl: ,,Samozřejmě, přijďte za mnou do Hôtel de Paris. Budu moc šťastný.`` 

,,A nedovolil byste mi přijít s manželkou?``

,,Jistěže, jistěže, s manželkou, budu moc rád.`` 

Plechanovovi ke mně přišli, popili čaj a rozmlouvali se mnou. Plechanov mi podobně jako Gogol pravil: ,,Jen mít víc takových lidí, to by se ve Vinnici dařilo\ldots``

Když odcházel, požádal mě o fotografii. Byla radost ho poslouchat a bylo příjemné vědět, že ho zajímá můj snímek. napsal jsem mu na něj: ,,Se srdečnými city.``

A pak, několik dní poté, co mi mladí lidé plivali do tváře urážky, jsem přišel domů, nalezl tam obálku, která mi byla adresována z Mentonu, a v ní fotografii, na níž jsem si přečetl dva nápisy, jeden svůj starý -- ,,Se srdečnými city`` -- a druhý čerstvý od Plechanova: ,,Vrací se vzhledem k nepotřebnosti.`` A ve stejné době mi v Petěrburgu známý ruský literát \footnote{Je míněn A. V. Amfitěatrov (1862--1938).} napsal dopis plný výtek a výčitek. Prý jsem poškodil jméno ruského kulturního člověka. Později jsem se dozvěděl, že pohoršený autor dal těmto svým intimním citům zármutku hektografickou podobu. Kopie svého mně adresovaného dopisu rozeslal po redakcích všech novin v hlavních městech. Ať se potomci pravoslavných dozvědí, jak ušlechtilé pocity choval.

Musím se otevřeně přiznat, že tahle štvanice dolehla na mou duši jako těžký balvan. Snažil jsem se pochopit podivnost toho neuvěřitelného vztahu ke mně a začal jsem se sám sebe ptát, zda jsem snad opravdu nespáchal nějaký strašný zločin. Není nakonec samotné mé účinkování v imperátorském divadle zradou lidu? Velmi mě zaměstnávala otázka, jak na ten incident pohlíží Gorkij. 

Ten byl v té době na Capri a mlčel. Po straně jsem se doslechl, že mnozí, kteří za ním na Capri přijeli, si nenechali ujít příležitost mnohoznačně mrknout zaostřeným očkem směrem ke mně. Když jsem ukončil sezonu, napsal jsem Gorkému, že bych k němu chtěl přijet, ale než to udělám, rád bych věděl, zda se i on nenakazil všeobecnou psychózou. Odpověděl mi, že je skutečně rozčilen z pověstí, které mu lidé hučí do uší. Proto mě prosí, abych mu napsal, co se doopravdy stalo. Napsal jsem mu to. Gorkij odpověděl žádostí, abych okamžitě přijel. 

Proti svému zvyku očekávat hosty doma nebo na přístavním molu mi tentokrát přijel naproti až k parníku. Ten citlivý přítel pochopil a vycítil, jaké trápení jsem v té době prožíval. Byl jsem tím jeho ušlechtilým gestem tak dojat, že jsem se radostným rozrušením rozplakal. Alexej Maximovič mě uklidnil, když mi ještě jednou dal na srozuměnou, že ví, zač stojí malicherná lidská špína.

\medskip

\noindent
Během téhle knihy jsem mnohokrát mluvil o Alexeji Maximoviči Peškovovi (Gorkém) jako o blízkém příteli. Na přátelství toho vynikajícího spisovatele a stejně vynikajícího člověka jsem byl celý život hrdý. Teď je to přátelství zkaleno a já mám pocit, že kdybych tuto pro mě smutnou okolnost zamlčel, rovnalo by se to utajování pravdy. Nesluší se nosit na klopě čestný řád, když právo na jeho nošení začalo být pochybné. Proto v této knize, která je mým zúčtováním, považuji za nutné věnovat několik stránek svým vztahům s Gorkým. 

Už jsem vyprávěl, jak prostě, rychle a pevně se naše vzájemné přátelství v Nižním Novgorodě na počátku tohoto století utvořilo. I když jsme se seznámili relativně pozdě -- oba jsme v té době už nabyli proslulosti --, Gorkij mi vždycky připadal jako přítel z dětství, tak mladý a bezprostřední vztah jsme jeden k druhému pociťovali. A~opravdu jsme také svá časná mladická léta prožili jakoby spolu, bok po boku, i když jsme o existenci toho druhého neměli ani ponětí. Z chudého a temného života předměstí -- on nižegorodského, já kazaňského -- jsme se oba stejnými cestami vydali za zápasem a slávou. A byl den, kdy jsme zároveň, ve stejnou hodinu, oba zaklepali u dveří Kazaňského operního divadla a zároveň jsme skládali zkoušku do sboru. Gorkého přijali, mě odmítli. Nejednou jsme se tomu později spolu zasmáli. Potom jsme se ještě vícekrát stali sousedy v životě, pro nás oba stejně skličujícím a obtížném. Já jsem stál v~,,řetězu`` na volžském přístavním molu a z ruky do ruky přehazoval melouny, on tamtéž jako přístavní dělník nejspíš tahal nějaké pytle z parníku na břeh. Já jsem byl u ševce a Gorkij, pravděpodobně někde blízko, u nějakého pekaře\ldots

Láska k člověku vlastně nepotřebuje zdůvodňování. Máme rádi, protože máme rádi. Ale moje srdečná celoživotní láska ke Gorkému byl nejen instinktivní. Ten člověk měl všechny vlastnosti, které mě na lidech vždy přitahovaly. Stejně jako opovrhuji netalentovanou ješitností, skláním se upřímně před nadáním pravým a nefalšovaným. Gorkij mě uváděl v nadšení svým vynikajícím literárním talentem. Vše, co napsal o ruském životě, je mi tak známé, blízké a drahé, jako bych při každé skutečnosti, o níž vypráví, byl sám osobně přítomen.

Na lidech si vážím znalostí. Gorkij toho věděl tolik! Viděl jsem ho ve společnosti vědců, filosofů, historiků, malířů, inženýrů, zoologů a nevím koho dalšího. A ti kompetentní lidé vždy, když s ním rozmlouvali o svém speciálním námětu, jako by v něm nalézali spolužáka. Gorkij zná věci velké i malé se stejnou úplností a solidností. Kdyby mě například napadlo se ho zeptat, jak žije hýl, dokázal by mi o něm Alexej Maximovič povědět takové podrobnosti, že kdyby se dali dohromady všichni hýlové za tisíce let, nemohli by to o sobě vědět.

Dobro je krása a krása je dobro. V Gorkém se to slévalo. Nedokázal jsem se bez nadšení dívat, jak se mu v očích leskly slzy, když slyšel krásnou píseň nebo se kochal opravdově uměleckým dílem malíře. 

Vzpomínám si, jak vznešeně chápal poslání inteligenta. Jednou na večírku u kteréhosi moskevského spisovatele v domku ve dvoře na Arbatu se literáti v přestávkách mezi zpívání \textit{Poutníka} za doprovodu guslí, popíjením vodky a jejím zajídáním pustili do sporu, co vlastně znamená inteligent. Přítomní spisovatelé a inteligenti se vyslovovali různě. Jedni říkali, že je to člověk se zvláštními intelektuálními vlastnostmi, druzí, že je to člověk výjimečně duševně stavěný, a tak dál, a tak dál. Svou definici inteligenta přednesl také Gorkij a já jsem si ji zapamatoval: ,,Je to člověk, který je v každém okamžiku života připraven postavit se s odhalenou hrudí všem do čela na obranu pravdy a nelitovat ani vlastního života.``

Za přesnost slov neručím, ale smysl předávám věrně. Věřil jsem v Gorkého upřímnost a cítil, že to není pustá fráze. Nejednou jsem ho viděl s odhalenou hrudí v čele všech\ldots

Pamatuji si ho nemocného, bledého, silně kašlajícího pod eskortou četníků na moskevském nádraží. Posílali ho tehdy do vyhnanství kamsi na sever. My, jeho přátelé, jsme ho vyprovázeli do Serpuchova. Tam nemocnému umožnili odpočinout si a vyspat se v posteli. V malém hotýlku pod dohledem týchž četníků jsme uspořádali veselý večer na rozloučenou, veselý proto, že fyzické utrpení Gorkého nijak nevyvádělo z míry, stejně jako ho nevyváděli z míry četníci a vyhnanství. Žila víra ve věc, pro kterou trpěl, a to nám všem skýtalo povznesený pocit, zkalený lítostí nad jeho nemocí. Tak bezstarostně a vesele se smál prudkým proměnám života a tak malý význam jsme přikládali skutečnosti fyzického uvěznění našeho přítele, protože jsme věděli, kolik má v sobě vnitřní svobody. 

Vzpomínám si, jak byl vzrušený a bledý v den 9. ledna 1905, kdy se prostí ruští lidé, vedení Gaponem, vydali k Zimnímu paláci, aby na kolenou prosili cara o svobodu, a v odpověď na prostomyslnou poníženou prosbu dostali od vlády olověné kulky do prsou: ,,Nevinné lidi zabíjejí, lotři!``
A i když jsem téhož večera zpíval ve Šlechtickém shromáždění, měl jsem tehdy v sobě touž pravdu s Gorkým. 

Je pochopitelné, s jakou radostí jsem od Gorkého slyšel mně adresovaná slova: Ať mi o tobě řeknou cokoli špatného, Fjodore, nikdy neuvěřím. Ani ty nevěř, když ti řeknou něco špatného o mně.`` 

A také si pamatuji: ,,Kdyby se naše cesty někdy jakkoli rozešly, budu tě mít rád. Ani toho tvého Susanina nepřestanu mít rád.``

A skutečně jsem mnohokrát v životě pocítil Gorkého lásku, jeho oddanost ke mně a jeho důvěru. Pevně držel své slovo.

Když jsem během bolševické revoluce pociťoval výčitky svědomí, že bych měl opustit rodnou zemi a trápil jsem se vzniklou životní i pracovní situací, zeptal jsem se jednou Gorkého, co si myslí, že bych měl udělat. Jeho láskyplný cit mi odpověděl: ,,No, kamaráde, myslím, že bys teď měl vodsuď odjet.`` Vodsuď to znamenalo z Ruska. 

Odjel jsem až dost dlouho po jeho radě, ale odjel jsem. Prožil jsem už řádný čas v zahraničí, když jsem jednou dostal od Gorkého dopis s návrhem, abych se vrátil do Sovětského svazu. \footnote{Šaljapin má nejspíš na mysli list z 15. listopadu 1928, ve kterém Maxim Gorkij psal: ,,Moc ti tě chtějí poslechnout v Moskvě. Říkali mi to Stalin, Vorošilov a další. Vrátili by ti i ‚skálu‘ na Krymu a nějaké další cennosti.``} Se vzpomínkou, jak těžké pro mě bylo tam žít a pracovat a s nepochopením, proč se názor Alexeje Maximoviče změnil, jsem mu odpověděl, že do Ruska by se mi teď jet nechtělo, a otevřeně jsem vysvětlil důvody. Psal jsem o tom Gorkému na Capri. \footnote{Maxim Gorkij tehdy nežil na Capri, ale v Sorrentu.} Alexej Maximovič ovšem už mezitím stačil Rusko navštívit a zřejmě tam shledal novou možnost, jak bych v zemi mohl žít a pracovat. Já ale, to se musím přiznat, jsem v tu možnost neuvěřil. Otázka mého návratu do Ruska tak dočasně zůstala viset ve vzduchu. Gorkij se k ní nevracel. Později ale, když jsem se ocitl v Římě (zpíval jsem tam v představeních), jsem se s ním setkal osobně. Stále ještě přátelsky mi tehdy řekl, že je \textit{nutné}, abych se vydal do vlasti. Znovu a rozhodněji jsem to odmítl a řekl, že tam jet nechci. Nechci, protože nevěřím v možnost tam žít a pracovat tak, jak já život a práci chápu. A ne že bych se bál některého konkrétního z vládců a vůdců, bojím se, abych tak řekl, celého uspořádání vztahů. Bojím se ,,aparátu``. I ty nejlepší záměry některého z vůdců vůči mně mohou zůstat nenaplněny. Jednoho krásného dne může nějaké zasedání nebo kolegium zničit vše, co mi bylo slíbeno. Budu třeba chtít jet do zahraničí, ale nechají mě doma, zadrží mě a ani bohovi nikam nepustí. A pak si hledejte, kdo to spískal. Jeden řekne, že on o tom nerozhoduje, druhý řekne: ,,Vyšel nový dekret.`` A ten, kdo vám to slíbil a komu jste uvěřili, rozhodí rukama a prohlásí: ,,Můj milý, tohle je přece revoluce, požár. Jak můžete vznášet nějaké nároky na mě?`` 

Alexej Maximovič, to je pravda, jezdí tam a zpět. Jenže on je přece účinkující osobou revoluce, je vůdce, a já, já nejsem komunista, menševik ani socialista revolucionář, monarchista ani kadet. A když potom člověk na otázku, kdo jsi? odpoví takhle, tak mu pěkně řeknou: ,,No právě proto, že nejsi to ani ono, ale čert ví co, seď, ty parchante, na Presně.``

Jenže já se svou zbojnickou povahou jsem tuze rád volný a žádné rozkazy, carské ani komisařské, nesnáším.

Pocítil jsem, že se Alexeji Maximovičovi moje odpověď příliš nelíbila, a když jsem potom, donucen bezohledným postojem sovětského zřízení k mým zákonným právům dokonce i v zahraničí, učinil ze svého rozhodnutí nevracet se do Ruska všechny logické závěry a ,,dovolil si`` ta svá práva hájit, rozevřela se v našem přátelství hluboká trhlina. Netajím se a rozrušen říkám, že mezi nemnohými ztrátami a několika rozchody posledních let je pro mě ztráta Gorkého jedna z nejtěžších a nejbolestnějších. 

Myslím si, že vnímavý a chytrý Gorkij by dokázal, kdyby chtěl, pochopit mé pohnutky v této otázce méně zaujatě. Já ze své strany nedokážu vyslovit předpoklad, že by tenhle člověk mohl jednat pod vlivem nízkých pohnutek a domnívám se, že všechno, co se v poslední době s mým milým přítelem dělo, má nějaké vysvětlení, neznámé mně ani jiným, ale odpovídající jeho osobnosti a povaze. 

Co se tedy stalo? Zjišťuji, že se stalo to, že jsme náhle začali rozdílně chápat a hodnotit dění v Rusku. myslím si, že v umění stejně jako v životě dvě pravdy být nemohou. Pravda je jenom jedna. Kdo ji má, to se neopovažuji rozhodnout. Možná já, možná Alexej Maximovič. Na společné, na pravdě dřívějších let, se v každém případě už neshodujeme.

Vzpomínám si například, s jakým příjemným rozechvěním jsem jednou poslouchal, jak se Alexej Maximovič nadšeně vyjadřoval o I. D. Sytinovi. \footnote{Ivan Dimitrijevič Sytin (1851--1934) -- vydavatel a knihkupec} ,,Tohle je člověk!`` pronášel se zářícíma očima. ,,Jen si to představte, obyčejný mužik, ale tak chytrý, tak dobře to má v hlavě srovnané, tak energický a tak daleko to dotáhl!``

Opravdu -- čím začal a kam to dotáhl. Vždyť všichni tihle ruští mužici, Alexejevové, Mamontovové, Sapošnikovové, Sobašnikovové, Treťjakovové, Morozovové, Ščukinové, všichni byli ve hře národa a země takovými trumfy, no a teď jsou to kulaci, škůdcovský živel, který je třeba nemilosrdně vymýtit! Já se ale nedokážu vzdát nadšení jejich talenty a kulturními zásluhami. Tak hrozně mě mrzí, když teď vím, že jsou považováni za nepřátele lidu, které je třeba bít, a když se dozvídám, že tuhle myšlenku sdílí i můj první přítel Gorkij.

Nadále si myslím a cítím, že svoboda člověka v životě i v práci je vrcholným dobrem. Štěstí se lidem nemá vnucovat silou. Nevíme, kdo jaké štěstí potřebuje. Mám nadále rád svobodu, kterou jsme kdysi oddaně milovali s Alexejem Maximovičem oba.

\podpis{přeložil Milan Dvořák}