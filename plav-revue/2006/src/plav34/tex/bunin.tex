\section{Ivan Bunin: Gorkij}

\noindent
\textit{Část úvahy z roku 1936, napsané po Gorkého smrti.}

\medskip

\noindent
Počátek onoho podivného přátelství, které mě s Gorkým spojovalo -- podivného proto, že jsme po téměř dvě desetiletí byli považováni za přátele, ale ve skutečnosti jsme jimi nebyli -- tak tedy onen počátek spadá do roku 1899. Konec pak do roku 1917. Tehdy se stalo, že muž, k nepřátelství s nímž jsem za těch skoro dvacet let neměl jediný osobní důvod, se najednou stal mým odpůrcem, který ve mně dlouho vyvolával děs a rozhořčení. Časem tyhle pocity vyhořely a on pro mě jakoby přestal existovat. Jenže teď se stalo něco zcela nečekaného:

\medskip

\noindent
,,L´écrivain Maxime Gorki set décédé\ldots Alexis Péschkoff\ldots connu en littérature sous le nom Gorki, était né en 1868 a Nijni-Novgorod d´une famille de cosaques\ldots``

\noindent
,,Zemřel spisovatel Maxim Gorkij\ldots Alexej Peškov, známý v literatuře pod jménem Gorkij, se narodil\ldots roku 1868 v Nižním Novgorodu v kozácké rodině\ldots``

\medskip

\noindent
\looseness+1
Prosím, ještě jedna legenda s ním spojená. Nejdřív byl bosák, a teď je dokonce kozák\ldots I když je to s podivem, o lecčem v Gorkého životě nemá dodnes nikdo ponětí. Kdo zná jeho životopis opravdu hodnověrně? Proč bolševici, kteří ho prohlásili za největšího z géniů a vydávají jeho nepřeberné spisky v milionových nákladech, doposud veřejnosti neposkytli jeho životopis? Osud tohoto člověka je vskutku pohádkový. Už tolik let světové slávy, naprosto bezpříkladné co do nezaslouženosti a založené na přímo neslýchaně šťastné shodě okolností nejen politických, ale i leckterých jiných -- například na faktu, že publikum o jeho životě neví prakticky nic. Jistě, talent to je, ale až doposud se nenašel nikdo, kdo by poctivě a odvážně promluvil o tom, jakého druhu je talent, který stvořil kupříkladu takovou věc jako je Píseň o sokolovi -- tedy píseň o tom, že vskutku neznámo proč ,,vysoko do hor vplazila se užovka a ulehla tam`` a pak tam za ní přiletěl jakýsi neuvěřitelně hrdý sokol. Všichni jen opakují: ,,bosák, který vzlétl ze dna lidového moře\ldots`` Jenže nikdo už nezná dosti důležité údaje, které si přečteme například v Brockhausově slovníku: ,,Gorkij-Peškov Alexej Maximovič. Narodil se v roce 1869 v typicky buržoazním prostředí: Otec byl přednostou kanceláře velké paroplavební společnosti. Matka pocházela z rodiny bohatého barvíře\ldots`` O zbytku toho životopisu nikdo nemá řádnou povědomost a všichni vycházejí jen z jeho autobiografie, na níž je podezřelý už její styl: ,,Číst a psát jsem se učil podle dědečkova žaltáře a později, už jako kuchtík na parníku, od kuchaře Smurého, člověka podivuhodně silného, drsného -- a zároveň jemného\ldots`` Zač stojí už jen tenhle věčně cukrkandlový gorkovský obrázek! ,,Smuryj mě, který jakýkoli potištěný papír až do této chvíle zuřivě nenáviděl, naučil přímo zběsilé lásce ke čtení a já začal málem k zešílení hltat Někrasova, časopis Niva, Uspenského, Dumase\ldots Od kuchtíků jsem se dostal mezi zahradníky a v té době polykal klasiky a literaturu prostonárodní. V patnácti jsem si usmyslel, že budu studovat, a odjel jsem do Kazaně -- s bláhovou jistotou, že vědění je každému zájemci poskytováno zadarmo. Ukázalo se ovšem, že tak tomu není, díky čemuž jsem nastoupil zaměstnání v preclíkárně. Během téhle práce jsem  navázal mezi studenty pár známostí\ldots A v devatenácti jsem do sebe vsadil kulku, a když jsem si odstonal, kolik se v takovémto případě sluší a patří, zase jsem ožil, abych se věnoval obchodu jablky\ldots V odpovídajícím věku jsem pak byl povolán k řádnému odsloužení vojenské povinnosti, ale když se zjistilo, že děravé neberou, stal jsem se poslíčkem u advokáta Lanina; jenže pak jsem si rychle uvědomil, že mezi inteligencí se necítím právě nejlíp, a vydal jsem se na toulky po jihu Ruska\ldots``

\looseness+1
Ve dvaadevadesátém Gorkij v novinách Kavkaz otiskl svou první povídku Makar Čudra, která začíná banalitami vskutku mimořádnými: ,,Vítr po stepi roznášel zádumčivou melodii vln, nabíhajících na břeh\ldots Mlha podzimní noci se polekaně zachvívala a stejně polekaně uskakovala před záblesky ohně, nad nímž se tyčila hřmotná postava starého cikána Makara Čudry. Ten pololežel v majestátní póze svobodného člověka, metodicky pobafával ze své obrovité dýmky, vypouštěl z nosu i úst hustá klubka kouře a říkal: ‚Ví takovej otrok, co je to širej svět? Ví, co jsou to stepní dálavy? Potěší jeho srdce mluva mořský vlny? Kdepak! Protože je otrok, bráško!‘`` Tři roky nato spatřil světlo světa i slavný Čelkaš. To už se ale o Gorkém mezi inteligencí hodně mluvilo, a kdekdo doslova hltal jeho Makara Čudru i další dílka ze spisovatelova pera -- povídky jako Jemeljan Piljaj nebo Děda Archip a Ljoňka\ldots Gorkij se mezitím proslavil i satirickými prózami, například povídkou O milovníku pravdy čížkovi a o datlovi, který lhal; znám byl i jako fejetonista, psal fejetony (do Samarské gazety), které podepisoval kupříkladu pseudonymem Jehudiil Chlamys. Pak ale spatřil světlo světa Čelkaš\ldots

\looseness+1
Právě v této době ke mně dolehly první konkrétní zvěsti o něm; v Poltavě, kam jsem občas zajížděl, se najednou rozkřiklo: ,,U Kobeljaků se usadil mladý spisovatel Gorkij. Figura vskutku malebná. Takový statný chlapák ve volné peleríně, klobouku se širokánskou krempou a málem dvacetikilovou sukovicí v ruce\ldots`` Opravdu jsme se s Gorkým seznámili na jaře devětadevadesátého. Přijedu do Jalty, vykračuji si to po nábřeží a co vidím: Proti mně jde s někým Čechov a kryje si tvář novinami -- snad před prudkým sluncem, snad před tím, co kráčí vedle něj, cosi hučí basem a jeho ruce neustále vzlétají z peleríny k obloze. Zdravím se s Čechovem a on mi povídá: ,,Seznamte se, tohle je Gorkij.`` Tak se představuji, dívám se na něj a přesvědčuji se, že v Poltavě mi ho popsali zčásti správně -- měl pelerínu, širák i sukovici. Pod pelerínou jsem zahlédl žlutou hedvábnou halenu, kterou si přepásal dlouhou hedvábnou šňůrou krémové barvy a která byla také při spodním lemu i kolem stojatého límečku vyšívaná různobarevnými hedvábíčky. Rozhodně to ale nebyl chlapák a rozhodně nebyl statný, jen vysoký a mírně přihrblý ryšavý mladík s nazelenalýma těkavýma očima, pihovatým kachním nosem se širokým chřípím a žlutým knírem, který si, mírně pokašlávaje, stále uhlazoval prsty -- vždycky si do té špetky jemně plivne a zase knír vyžehlí. Vykročili jsme dál, on si zapálil, zhluboka zatáhl a pak už jen dál basovitě hučel a mával rukama. Jakmile dokouřil papirosu, nechal do jejího náustku skanout sliny, aby oharek uhasily, pak ho teprve odhodil a mluvil dál; na Čechova jen občas letmo pohlédl a snažil se zjistit, jaký jeho vyprávění zanechává dojem. Hovořil nahlas, jakoby z celého srdce, horoucně a neustále v metaforách, neustále v jakýchsi udatovských provoláních, chtěně obhroublých a primitivních. Bylo to nekonečně dlouhé a neuvěřitelně nudné vyprávění o jakýchsi volžských boháčích kupeckého i mužického původu, nudné především pro jednotvárný sklon k nadsázce -- všichni ti boháči se posluchači jevili jako bohatýři z ruských bylin -- a kromě toho i k nepřiměřenosti obrazů a patosu. Čechov téměř neposlouchal. Jenže Gorkij pořád mluvil a mluvil\ldots

Málem od téhož dne mezi námi vzniklo cosi jako přátelské sblížení, z jeho strany snad až poněkud sentimentální, jako by mnou byl ostýchavě nadšen:

,,Vy jste přece poslední spisovatel šlechtického původu, z oné kultury, která světu dala Puškina a Tolstého!``

Jakmile si Čechov chytil drožku a odjel k sobě do Autky, Gorkij mě pozval k sobě do Vinogradné ulice, kde si pronajal pokoj; tam mi s nemotorným, šťastným, komicky přihlouplým úsměvem, v němž směšně krčil nos, ukázal fotografii své ženy s tlustým okatým dítětem v náručí, pak kousek nebesky blankytného hedvábí a se stále stejnými grimasami řekl:

,,To jsem jí koupil na blůzku, víte\ldots Té ženě\ldots Přivezu jí to jako dárek\ldots``

\looseness-1
Teď to ovšem byl člověk zcela jiný než na nábřeží v Čechovově společnosti -- milý, legračně se pitvořící, až sebemrskačsky skromný a hovořící nikoli basem a s hrdinskou hrubostí, ale užívající spíš jakéhosi neustále se omlouvajícího, hraně srdečného povolžského repetění s důrazem na každé ,,o``. V tomto i onom případě hrál -- se stejným potěšením a nepřetržitě; později jsem se dověděl, že by dokázal vést tyhle monology od rána do večera -- vždy stejně obratně, dokonale zvládaje tu či onu roli, a v působivých místech, kdy se snažil být co nejpřesvědčivější, snadno ze svých nazelenalých očí uměl vymáčknout i slzy. Zároveň jsem odhalil i některé další jeho povahové rysy, které jsem pak ve stále stejně podobě sledoval dlouhá a dlouhá léta. První rys byl ten, že ve společnosti byl úplně jiný, než když jsme spolu zůstali sami dva, popřípadě když byl úplně sám -- ve společnosti nejčastěji temně hučel, tváří v tvář publiku, které jím bylo nadšeno, bledl samolibou ctižádostí, neustále hovořil o něčem hodně drsném, důležitém a vznešeném, své obdivovatele a obdivovatelky rád poučoval a hovořil s nimi jednou příkře a přezíravě, jindy zase suše a nabádavě; když jsme spolu zůstávali sami dva a nebo když se ocitl mezi lidmi, které měl rád, byl najednou milý, tak nějak naivně rozdychtěný, skromný a snad až zbytečně ostýchavý. Dalším jeho důležitým povahovým rysem bylo jeho zbožnění literatury a kultury, o nichž mluvil až k zbláznění rád. To, co mi později opakoval bezpočtukrát, začal říkat už tehdy v Jaltě:

,,Pochopte -- vy jste skutečný spisovatel především proto, že kulturu a dědictví vznešených uměleckých hodnot ruské literatury máte přímo v krvi. A našinec, tvořící pro nového čtenáře, se musí této kultuře neustále učit, ctít ji z celého srdce -- jedině v takovém případě do nás jako do spisovatelů taky něco bude!``

Nepochybně v tom byla ta jeho věčná hra, v onom sebepokoření, které je nadutější než pýcha. Ale bylo to i upřímné -- copak by se jinak něco takového dalo tvrdit tolik let, občas ještě ke všemu se slzou v oku?

\podpis{přeložil Libor Dvořák}