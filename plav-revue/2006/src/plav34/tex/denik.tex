\section{Ivan Bunin: Proklaté dny}

\noindent
\textit{Fragmenty z jedné knihy. Kniha jako celek vyjde v nakladatelství Argo pod názvem ,,Proklaté dny``, i když by se mohlo zdát jinak.}

\medskip

\noindent
\textbf{,,Deník``}

\medskip

\noindent
\textbf{3. října}

\noindent
Včera ve tři s Koljou do Osinových dvorů. Skorodnoje zdáli vypadá jako takový rezavě hnědý medvěd. Jeli jsme přes Remerský les. Doubky byly všechny bronzové. Lesem se táhne kouzelný rybník, na jednom místě do jeho zrcadla stéká dokonale zlatý odraz jakéhosi sklánějícího se stromu. Strážní budka, psík vedle ní je tak vzteklý, že se mu naježila všechna srst -- a to nás zná; z budky vyběhl ten děda, rozverně veselý, těkavý a zmatený děda, kterého jsem viděl ve Skorodném. Fjodor Mitrofanyč samozřejmě lhal, když řekl, že Vaňkovi zabavili pušky; byl tu poprask, protože střílel na rybníce panské kachny, zrovna vedle téhle budky, ale nezabavili. ,,Kde ses tu vzal?`` -- ,,Hnedlinko to bude\ldots`` S nezřízenou radostí a zároveň tajemně: ,,Du si koupit prasátko, vod Boris Borisyče\ldots Boris Borisyč vodpovídá\ldots`` (místo ,,povídá`` -- a velice často to říká nevhod -- ačkoli\ldots). Byli jsme v Polském (vesnička). Dva naprosto modré rybníky -- za námi, když jsme vyjížděli do kopce, podvečerní slunce. Ohromila mě malebnost a odloučenost Logofetovy usedlosti. Kolem sad a u domu stromy -- ten hlavní je ryšavě bronzový, bronzový. Naše někdejší rodové sídlo. Zmocnila se mě myšlenka to koupit. Skla v oknech domu hoří jako stříbrná slída, jako zář/ivé/ hvězdy, už zdáli. V Osinových Dvorech jsme potkali dva mužiky: jeden byl zrzek, s bramborovitým nosem, takový oslnivě laskavý profesor, ale mě uchvátil ten druhý: IV. /?/ stol., Boris Godunov, ta masivnost nosu, rtů, tlustého chřípí, profil skoro strašlivě hrubý, černé hrubé vlasy se pod čapkou mísí se stříbrem. Někdejší lidé byli asi opravdu onačejší. Jaká nicotnost a malichernost se zračí v rysech dnešní omladiny! Ti mužici říkali, že o novém zřízení toho moc nevědí. A odkud taky? Celý život viděli jen Osinové Dvory. A tak se \textit{nemohou} zaj/ímat/ o něco jiného, o svůj stát. Jak by mohla fungovat lidovláda, když tu není znalost vlastního státu, smysl pro tento stát -- pro ruskou zemi, a nejen pro vlastní děsjatinu!

Šest hodin večer. Zrovna byl venku. Jak je tak krásně. Už zrovna tak na podzimní plášť. Po rukou přebíhá příjemný chládek. Jaké štěstí je dýchat tenhle sladký chladivý vítr, který volně a klidně už kolik dní vane od jihu, kráčet po suché hlíně, dívat se na sad, na strom, ještě oděný v hnědém listí, které rudne snad ve slunce západu (i když ten západ je skoro bezbarvý), snad vlastními barvami. Celá alej je zasypána rudnoucím, suchým, svraštělým listím, které čímsi sladce voní. Jak nový je pohled skrz průhledný sad, za nímž je vidět nepatrně zelenavý vzduch nad dolinou, a západ celý sad plní narůžovělým svitem. Téměř vše je holé, skoro všechny javory na náspu a alej atd., jen jabloně jsou /v/ zlatavě přibronzovělém drobném mrtvém listoví.

Vláda je ,,tvrdá, rozhodla se potlačit pogromy``. Jak směšné! To chce přemlouvat? Ne, tohle \textit{ona} nedokáže! ,,Oni ani ti ministři nejsou o moc lepší než my ostatní!`` Včera v poledne rozmluva s vojákem Alexejem -- zběsile proti Kornilovovi, za všechno může vrchnost, ,,nám bolševikům, proletariátu, nikdo nevěnuje pozornost, a třeba takový Němci\ldots`` Jsou to kojenci, položivočišná chátra!

Náladu mám teď dobrou, napsal dvě nové básně /nečit./.

Šel procházkou do Kolontajevky, ráno poslal knihu Bělevskému (poslední díl svázané Nivy).

Nikdo není hmotařštější než náš lid. Ten vykácí všechny sady. Dokonce i když jedí a pijí, chuť je nezajímá -- hlavní je se nažrat. Ženské vaří vyloženě s podrážděním. A jak vlastně nesnášejí jakoukoli vládu, jakékoli donucení! Jen to zkuste a zaveďte povinné vzdělání! Těm je třeba vládnout s revolverem u spánku. A jak využijí každé živelné pohromy, když se moc vymkne z rukou -- teď by třeba zabíjeli doktory (cholerové vzpoury), i když samozřejmě nejsou takoví idioti, aby úplně uvěřili, že lékaři otrávili studny. Je to zlý lid! Podílet se na veřejném životě, na řízení státu nemohou a po celé dějiny ani \textit{nechtějí}.

Procházka do Kolontajevky byla nádherná: jaká temně modrá zeleň modřínů, které ještě nezežloutly! (Jsou ještě takové, i když většina už zasypala všechny pěšinky svými vlásky.) Šli jsme cestičkou -- před námi břízy, jejich kmeny, pak tenké trubky modřínů, šerá tma a skrz tohle modravě kamenné nebe (slunce bylo za námi, táhlo na čtvrtou). Böcklin postřehl cosi nového, cosi zázračného. A jak se ty tenké trubky, poseté ostrými suky, lehce pohybují. Tak plavně, velmi plavně a vlastně do všech stran!

 U Bachtějarovic stohu už je skrz holý sad vidět chrám. Jak nový pohled, sotva skončilo léto!

Inteligence lid neznala. Mládež se učila spíš Erfurtský program!

\pagebreak

\noindent
\textbf{Půlnoc 17.--30. dubna} 

\noindent
Náladu měl ráno hroznou -- byl v Crédit Lyonnaise, v trezoru pro šekovou knížku. Ze vzteku jsem si ze sejfu vyzvedl všechno -- pracovní verze, dopisy, rodinné stříbro atd. -- nechal jsem tam jen dluhopis v hodnotě 500 r. -- tuším válečný dluhopis. Propukl tam skandál -- Globovi se ztratil briliantový prsten v ceně nějakých deseti tisíc. Buran, který mu ho zřejmě ukradl -- nějaké bolševické panstvo -- strašně řval a mlel pořád jedno a to samé. Kdyby tu vopravdu byl, tak by tu přece byl.

Moskvu vyzdobují. Nepopsatelný pocit -- jaký cynismus, jaký (\ldots) výsměch do tváře toho stáda -- ruského lidu! A právě ten\-hle lid, ti divoši, prasata špinavá, líná krvežíznivá a dnes celým světem opovrhovaná budou slavit  \textit{internacionalistický} svátek. A co teprve to chámské, neslýchaně nejapné a podlé kácení Skobeleva! Sochu stáhli a povalili tváří na náklaďák\ldots Zrovna v den, kdy přišla zpráva, že Turci dobyli Kars! Zítra, v den, kdy byl zrazen Kristus, se konají oslavy zrádců Ruska!

Večer u Avilovovů. Chvála Bohu prší! Kéž by to tak bylo i zítra, pokud možno průtrž mračen! Člověk opravdu čeká už jen na zázrak -- tak strašlivě ho duše bolí! Kéž by je bouře zabila a potopa zaplavila! Říká se, že rozhořčených kácením pomníků je opravdu hodně. (Jenže copak oni si s tím lámou hlavu, s nějakým naším rozhořčením?!) ,,To nařídili Němci (nebo Turci)!`` Náš domovní výbor je zbabělý -- shání látku na rudé vlajky, všichni se bojí, že když nesplní příkaz a nebudou ,,slavit``, tak trestu neuniknou. A takhle je to po celé Moskvě. Budiž proklet den mého zrození v této proklaté zemi!

A Ajchenvald -- a vlastně nejen on -- zcela vážně přemítá o události tak nicotné, jako že Andrej Bělyj a Blok, ten ,,něžný rytíř krásné dámy``, se stali bolševiky! No Bože, a koho to zajímá, co se stalo nebo nestalo ze dvou zkurvysynů, ze dvou arciblbů! 

\medskip

\noindent
\textbf{Poledne 1. máje} 

\noindent
Od rána šedivo a chladno. Teď už vykukuje a začíná hřát slunce. Obludně čeledínské detaily Dybenkova chování (právě začal jeho soud). Idiotsky rétorický kvil Andreje Bělého v Žizni na adresu prvního máje. Tyhle noviny jsou vůbec jen snůška banalit a nestydatosti. A přitom do nich přispívají skoro všechny naše celebrity. Mizerové!

\bigskip

\noindent
\textbf{,,Proklaté Dny``}

\medskip

\noindent
\textbf{6. února} 	

\noindent
Noviny píší o počínajícím tažení Němců proti nám. Všichni si říkají: ,,Kéž by!``

Šli jsme na Lubjanku. Místy byly ,,mítingy``. Ryšavec v kabátě se stojacím persiánovým límečkem, ryšavé kučeravé obočí. S čerstvě vyholenou napudrovanou tváří a s ústy plnými zlatých korunek, jednotvárně, jako by to předčítal, mluví o nespravedlnosti starého režimu. Vztekle mu oponuje pán s nosem pršáčkem a s vykulenýma očima. Do toho všeho se horlivě a většinou  nevhod pletou ženské, přerušují spor (zcela zásadní, jak se ráčil vyjádřit ryšavec) nepodstatnými detaily a chvatnými historkami z vlastního soukromí, jež mají dokázat, že se tu odehrává čertvíco. Několik vojáků, kteří zřejmě ničemu nerozumějí, ale jako obvykle o čemsi (přesněji řečeno o všem) pochybují a podezíravě pokyvují hlavami.

Přistoupil i nějaký sedláček, děda s bledými naducanými tváře\-mi a šedou špičatou bradou, již, sotva přišel, zvědavě strčil do davu, zabořil ji mezi rukávy dvou vytrvale mlčících a o to pozorněji naslouchajících pánů: začal také poslouchat, ale zřejmě také ničemu nerozuměl a nikomu a nic nevěřil. Pak přistoupil vysoký modrooký dělník a dva další vojáci se slunečnicovými semínky v hrsti. Oba vojáci jsou krátkonozí, žvýkají semínka a nedůvěřivě a chmurně se rozhlížejí. Ve tváři dělníkově neustále hraje krutě veselý úsměšek a pohrdání; stojí stranou davu a tváří se, že se tu zastavil jen tak na okamžik, jen tak pro zábavu a říká si: Však já vím, že tu všichni melou jen hlouposti.

Jedna dáma si překotně stěžuje, že teď nemá co do úst, že mívala soukromou školu, ale že musela všechny žačky rozpustit, protože pro ně nemá ani kousek jídla:

,,Komu ti bolševici přinesli co dobrého? Nikomu, všem je hůř, především nám obyčejným lidem!``

Naivním hláskem ji přerušila jakási zmalovaná čubička a začala vykláda, že Němci tu budou každou chvíli a že všichni draze zaplatí za to, co tu provedli.

,,Než přijdou Němci, tak vás všechny podřežeme,`` poznamenal chladně dělník a šel pryč.

Vojáci přikývli -- ,,to je rozumná řeč!`` -- a taky se vzdálili.

O tomtéž se mluvilo i ve vedlejším houfu, kde se dohadoval nějaký praporčík s jiným dělníkem. Praporčík se snažil promlouvat co možná nejmírněji a uplatňovat logiku. Téměř podlézal, ale dělník na něj stejně křičel:

,,Takový jako vy by měli spíš mlčet! A nešířit mezi lidem propagandu!``

K. říkal, že včera u nich zase byl R. Seděl tam asi čtyři hodiny a celou tu dobu nepřítomně zíral do knížky, která se povalovala na stole a která pojednávala o magnetických vlnách, potom pil čaj a snědl všechen chleba, který rodina dostala na příděl. Takový mírný, tichý, rozhodně žádný halama -- a  najednou si přijde, neomaleně se rozsedne, sní všechen chleba a je mu úplně jedno, co na to hostitelé. Lidé vlčí věru rychle!
Blok se otevřeně přidal k bolševikům. Otiskl článek, který nadchl Kogana (P.S). Ještě jsem to nečetl, ale celkem jsem tušil co tam bude a převyprávěl to Erenburgovi -- a ono se ukázalo, že jsem to vytušil téměř přesně. Ta písnička není chytrá, Blok je zkrátka hlupák.

\medskip

\noindent
\textit{Z Gorkého Nové žizni:} ,,Ode dneška musí být i sebenaivnějšímu prosťáčkovi jasné, že pokud jde o politiku lidových komisařů, nelze hovořit ani o elementární poctivosti, natožpak o nějaké statečnosti a revolučních ideálech. Máme co do činění se spolkem dobrodruhů, kteří se ve jménu vlastních zájmů, ve jménu prodloužení agonie svého hynoucího samoděržaví o několik dalších týdnů odhodlali k té nejzavrženíhodnější zradě zájmů vlasti i revoluce, zájmů ruského proletariátu, v jehož jménu dnes řádí na uvolněném trůně Romanovců.``

\medskip

\noindent
\textit{Z Vlasti Naroda:} ,,S ohledem na neustále se množící a každou noc se opakující případy fyzického násilí, páchaného při výsleších v sovětu dělnických poslanců, žádáme Radu lidových komisařů, aby podobným výtržnickým praktikám a akcím zabránila\ldots``

Je to stížnost z Borovičů.

\medskip

\noindent
\textit{Z Ruského slova:} ,,Tambovší mužici ze vsi Pokrovskoje sestavili následující protokol: 30. ledna jsme coby pospolitost soudili dva drav\-ce, naše občany Nikitu Alexandroviče Bulkina a Adriana Alexandroviče Kudinova. Podle všeobecné dohody naší pospolitosti byli souzeni a taky okamžitě zabiti.`` Vzápětí ona ,,pospolitost`` sestavila i svérázný zákoník trestů za zločiny:

,,Když někdo někoho udeří, poškozený musí útočníka udeřit desetkrát.

Když někdo někomu úderem způsobí poranění či zlomeninu, bude zbaven života.

Když někdo dopustí krádeže nebo přijme kradenou věc, bude zbaven života.

Když se někdo dopustí žhářství a bude přistižen při činu, bude zbaven života.``

Krátce nato byli při činu přistiženi dva zloději. ,,Pospolitost`` je okamžitě ,,postavila před soud`` a odsoudila k trestu smrti. Nejdřív zabili prvního: rozrazili mu hlavu přezmenem, vrazili mu vidle do boku, mrtvolu vysvlékli donaha a vyhodili na silnici. Pak se pustili do druhého\ldots

Takové věci dnes čtete každý den.

Na Petrovce sekají mniši led na vozovce. Kolemjdoucí zlomyslně jásají:

,,Vida! Tak už vás hnali! Teď budete muset dělat, co se vám řekne!``

Ve dvoře jednoho domu na Povarské štípe voják v kožené budně dříví. Kolem jde mužik, zastaví se, dlouho zírá, pak pokýve hlavou a trpce řekne:

,,Zatracepená práce safraportská! Zatracepenej dezeltýre jeden špatná! S Ruskem je konec.``

\medskip

\noindent
\textbf{7. února} 	

\noindent
\textit{Úvodník z Vlasti Naroda:} ,,Nastává strašlivá hodina -- Rusko a Revoluce hynou. Vše na obranu Revoluce, která ještě docela nedávno jásavě zářila do celého světa!``

Že se boha nebojíte! Kdy zářila?

\medskip

\noindent
\textit{Z Ruského slova}: ,,Zabit byl bývalý náčelník štábu generál Januškevič. Zatkli ho v Černigově a na pokyn místního revolučního tribunálu ho eskortovali do Petropavlovské pevnosti v Petrohradu. Eskortu tvořili dva rudoarmejci. Když vlak přijížděl do stanice Oredež, jeden z nich generála čtyřmi výstřely zabil.``

Sníh je ještě zimně třpytný, ale nebe již skrz oblačný svítivý opar září jarním jasem.

Na Strastné vylepují plakát na benefici Javorské. Tlustá, neomalená a hubatá  růžově zrzavá ženská vybafla:

,,Hele, co si to tu vylepujou! Kdo to po nich bude mejt? Vida je, buržousty, do dívadla by chodili! To by se jim mělo zakázat. My do žádnýho dívadla nechodíme. A furt strašej Němcema -- prej že přídou, a voni vurt nejdou a nejdou!``

Po Tverské jde dáma se skřipcem na nose, ve vojenské beranici, v ryšavém plyšovém žaketu, v potrhané sukni a naprosto příšerných kaloších.

Mnoho dam, kursistek a důstojníků postává po nárožích a cosi prodávají.

Do tramvaje vkročil mladý důstojník, zrudl a řekl, že ,,velmi lituje, ale bohužel nemůže zaplatit za lístek``.

V podvečer. Na Rudém náměstí oslepuje nízké slunce a dozrcadlova uježděný sníh. Trochu mrzne. Zašli jsme do Kremlu. Po nebi pluje měsíc a růžová oblaka. Je ticho, všude obrovské sněhové závěje. Před dělostřeleckým skladem poskřipuje válenkami voják v kožichu, s tváří jakoby vytesanou ze dřeva. Jak zbytečná se teď zdá tato stráž!

Vyšli jsme z Kremlu -- kolem běží nějací kluci a s nadšeně nepřirozenou intonací vykřikují:

,,Mogilev dobyla německá vojska!`` 

\medskip

\noindent
\textbf{8. února} 	

\noindent
Andrej (sluha bratra Julije) je stále opovážlivější, až z toho jde hrůza. Slouží u něj už dvacet let a ve vztahu k nám byl vždy vytrvale prostý, milý, uvážlivý, zdvořilý a srdečný. Ale teď jako by přišel o rozum. Slouží stále ještě pečlivě, ale je vidět, kolik ho to stojí úsilí, nemůže se na nás ani podívat, rozmluvám s námi se vyhýbá, uvnitř je celý vztekle rozechvělý, a když už někdy nevydrží mlčet, vyhrkne nějaký záhadný nesmysl.

Když jsme dnes ráno byli u Julije, N. N. jako vždy mluvil o tom, že je vše ztraceno a že Rusko se hroutí do propasti. Andreji, který zrovna servíroval čaj, se najednou roztřásly ruce a tvář se zalila žárem:

,,Ano, ano, hroutí se! Ale kdo za to může, no kdo? Buržoazie! Však ještě uvidíte, jak ji budou řezat, to teda uvidíte! Pak si vzpomeňte na toho vašeho generála Alexejeva!``

Julij se zeptal:

,,Tak nám ale, Andreji, vysvětlete, proč tolik nenávidíte zrovna Alexejeva?``

Andrej se na nás ani nepodíval a zašeptal:

,,Já vám nemám co vysvětlovat. To byste měli pochopit sami\ldots``

,,Ale ještě před týdnem jste přece stál skálopevně při něm. Tak co se stalo?``

,,Co se stalo? Jen ještě počkejte -- a pochopíte\ldots``

Přijel kritik Derman -- ten uprchl ze Simferopolu. ,,Říkal, že tam panuje ,,nepopsatelná hrůza``, vojáci a dělníci ,,se doslova brodí po kolena v krvi``. Jakéhosi starého plukovníka zaživa upekli v topeništi lokomotivy.

\medskip

\noindent
\textbf{2. března}  

\noindent
,,Necuda a opilec Rasputin, zlý génius Ruska.`` Jistě, to byl podařený ptáček. Ale co vy, co dnes věčně trčíte ve všech těch Medvědech a Toulavých psech?

Nová literární nízkost, ze které už se snad níž ani klesnout nedá: V odporné putyce otevřeli jakousi Muzikální tabatěrku -- sedí tam spekulanti, podvodníci a pouliční holky a pořádají tam pirožky po sto rublech za kus, pijí chanžu z čajníků a básníci a beletristé (Aljoška Tolstoj, Brjusov atakdále) jim čtou svá i cizí díla, přičemž vybírají ty nejvulgárnější. Brjusov tam prý recitoval Gabrieliádu a vše, co je vytečkováno, pronášel doslova. Aljoška měl tu drzost navrhnout něco takového i mně -- a prý dáme ti veliký honorář.

,,Pryč z Moskvy!`` Ale nechce se. Přes den je přitom až podivuhodně ohavná. Počasí je vlhké, všechno je mokré a blátivé, na chodnících i na vozovkách jámy, vymletý led, o té chátře raději ani nemluvit. Večer a v noci zase všude pusto a prázdno, nebe se za občasnými pouličními lampami zhasle a chmurně černá. Jenže najednou tichá ulička, docela tmavá, jdete a najednou vidíte otevřená vrata a za nimi, v hloubi dvora, nádhernou siluetu starého domu, měkce tmavnoucího na pozadí nočního nebe, které je tu docela jiné než nad ulicí, a před domem stoletý strom, černá krajka jeho obrovité, košaté koruny\ldots

Četl novu Treňovovu povídku (Nádeníci). Hnus. Cosi -- jak je dnes obvyklé -- naprosto vylhaného, vykládajícího o těch nejděsivějších věcech, ale přitom vůbec ne strašného, protože autor nebere věc vážně, jen unavuje svou přehnanou ,,vnímavostí``, ,,lidovostí`` jazyka a vůbec celým způsobem líčení, až by si člověk nejraději odplivl. A nikdo to nevidí, necítí, nechápe -- naopak, všichni jsou nadšeni: ,,Jak šťavnaté, jak barvité!``
 ,,Sjezd sovětů``. Leninův projev. Ach, jaký to je živočich!

Četl o mrtvolách, stojících na mořském dně -- jsou to zabití, utopení důstojníci. A oni si přijdou s Muzikální tabatěrkou.

\pagebreak

\noindent
 \textbf{3. března}  

\noindent
Němci obsadili Nikolajev a Oděsu. Moskvu by prý měli dobýt sedmnáctého, ale já tomu nevěří a neustále se vypravuji na jih.

Majakovskému se na gymnáziu říkalo Idiot Polythemovič.

\bigskip

\noindent
\textbf{,,Oděsa, rok 1919``}

\medskip

\noindent
Píši při páchnoucí kuchyňské lampičce, pálím poslední zbytky petroleje. Jak to bolí, jak to uráží. Milí mí přátelé z Capri, všichni vy Lunačarští a Gorcí, vy opatrovníci ruské kultury a umění, co dřív hořeli svatým hněvem při každém varování nějaké té Nové Žizni ze strany ,,carských opričniků`` -- co byste asi se mnou udělali teď, kdybyste mě přistihli nad tímto zločinným sepisováním u smradlavého kahance nebo u toho, jak budu tyto zápisky provinile zastrkovat někam do štěrbiny pod okenním parapetem?

Ten domovník měl pravdu (\textit{Moskva, podzim roku 17.}).
,,No dovolte?! Naší povinností vždy bylo a je přivést zemi k ústavodárnému shromáždění!``

Domovník, co seděl u vrat a toto vášnivé zvolání zaslechl -- svářící se zrovna rychle procházeli kolem něj -- jen trpce pokýval hlavou:

,,Jenže kam jsme to ve skutečnosti přivedli, vy zkurvenci!``

,,Nejdřív menševici, potom náklaďáky, a pak bolševici a obrněnce\ldots``

Náklaďák -- jak strašlivým symbolem se pro nás všechny stal, kolik z toho náklaďáku uvízlo v našich nejtíživějších a nejděsivějších vzpomínkách! Revoluce hned od prvního dne nerozborně splynula s tímto řvoucím a páchnoucím živočichem, přecpaným z počátku hysterkami a oplzlou vojáckou dezertérskou verbeží, a pak už vyloženými kriminálníky.

Veškerou hrubost soudobé kultury a její ,,sociální patos`` ztě\-les\-ňuje právě náklaďák.

Mluví, křičí a s pěnou u úst se zalyká, oči se skrz nakřivo nasazený skřipec zdají obzvlášť zběsilé. Kravata se mu vzadu vysoukala vysoko na špinavý papírový límeček, vestu má nemožně zaneřáděnou, ramena otahaného sáčka jsou poseta lupy, mastné řídké vlasy jsou rozcuchané a zježené\ldots A to mě chtějí přesvědčovat, že tahle zmije je posedlá ,,plamennou, obětavou láskou`` k člověku, ,,žízní po kráse, dobru a spravedlnosti``! 

A jeho posluchači?

Dezertér, který celý den zbůhdarma postává se slunečnicovými semínky v hrsti a celý den ta semínka mechanicky požírá. Vojenský plášť volně přehozený přes ramena, brigadýrku posunutou do týla. Je ramenatý, s krátkýma nohama. Je poklidně neomalený, žere a čas od času pokládá otázky -- on nemluví, on se neustále jenom ptá; a ani jedné odpovědi přitom nevěří, ve všem hledá jen podvodné tlachy. Odpor k němu, k jeho tlustým stehnům v tlustém zimním khaki, k telecím řasám i k mléku z rozžvýkaných semínek na živočišně pravěkých rtech bolí až fyzicky.

Tatiščevova Ruská historie:
,,Bratr proti bratru, synové proti otcům, otroci proti pánům -- jeden druhého lační usmrtit jen ze ziskuchtivosti, snaží se jeden druhého zbavit nevěda, co praví moudrý duch: \textit{Pídě se po cizím, jednoho dne nad svým vlastním spláčeš\ldots}``

A kolik hlupáčků je přitom přesvědčeno, že v ruských dějinách došlo k velkému ,,posunu``, k čemusi údajně zcela novému, doposud nevídanému!

Všechno neštěstí (a to strašné) spočívá v tom, že o ,,ruské historii`` zatím nikdo neměl ani to nejmenší skutečné ponětí!

\podpis{přeložil Libor Dvořák}