\section{Pranýř překladatelů}

\noindent
\textit{A na závěr něco překladatelských poklesků, jak byly uloveny uživateli serveru www.okoun.cz a www.nyx.cz.}

\medskip

\noindent
\textit{Richard} Novinky.cz o postřeleném americkém školákovi: Zajatému spolužákovi se podařilo utéct a mladík se zbraní se schoval v koupelně. ,,Ohrožovaný důstojník SWAT použil 'smrtící sílu'``, oznámila doslova místní policie.

\medskip

\noindent
\textit{Barclay} Chci se zeptat, jestli bych mohl vlastnit i originální kopie. Titulky k filmu \textit{Ray}.

\medskip

\noindent
\textit{merissa} ,,Maltská whiskey`` se (m.j.) vyskytla v románu \textit{Rituály}. Přeložila Veronika Havlíková.

\medskip

\noindent
\textit{Ajgor} Já se pochlubím sledováním DVD \textit{Bratrstvo neohrožených} (s titulky). Nic proti překladu, ale v titulcích se zásadně objevují dvě jména vedle sebe ve stylu ,,vem sebou Popey and Shifty``, ,,jde sem Winters and Nix`` no a v devátém díle vojáci ,,nasedli na Autobahn\ldots``.

\medskip

\noindent
\textit{Obey} Pardon, že nenavazuji na zdejší diskuzi, ale musím si ulevit. Na pranýř bych si dovolil posadit a do kola vmotat jistou J. Koblížkovou, která je podepsaná coby překladatelka románu \textit{Panský dům} od J. Clavella, který vydalo nakladatelství Mustang v polovině 90. let. Tedy něco tak příšerného jsem ještě neviděl. Jenom pár perliček: americký výraz ,,billion`` je evidentně přeložen jako\ldots bilion. Překladatelka zde předvádí nejen neznalost angličtiny, ale i neprostou absenci zdravého rozumu. Jinak by ji muselo trknout, že hodnota soukromé společnosti v 60. letech asi nebude půl bilionu. Navíc když o pár stránek dál se o 20 milionech píše jako o neuvěřitelné částce, která odvrátí krach jisté společnosti. Slovíčko ,,sorry`` je podle překladatelky asi již součástí českého jazyka, protože se ho neunavuje překládat.
A tak to jde dál a dál. Ani si netroufám odhadnout, co bylo v angl. originále na místě, které autorka přeložila jako ,,sekerník`` (zřejmě tam původně byl nějaký výraz ze slangové business english, který překladatelka neměla ve slovníku.) A ošklivě mučit by zasloužil i korektor, na každé stránce jsem zatím našel překlep nebo hrubku :(

\medskip

\noindent
\textit{trish } Dabing, to jsou chuťovky, obzvlášť u nás. Ve filmu \textit{Táta v sukni} se v jedné scéně Mrs. Doubtfire podivuje, že oblíbenou knížkou malé dcerky je take ,,Stuartovna Malá`` a drží v ruce knížku o neodolatelném malém bílém myšákovi (Stuart Little).

\medskip

\noindent
\textit{Case} Ach jo\ldots zase tu mám jednoho 'trilobita'\ldots po VELMI dlouhé době jsem víceméně ze studijních důvodů otevřel Neffův překlad \textit{Neuromancera}, knížky, kterou znám v originále prakticky nazpaměť\ldots docela často se divím\ldots Hlavní hrdina má pocit, že ho někdo sleduje a chce ho zabít, a proto zaběhne do jakéhosi podniku a na recepční křikne ,,Strčte si bezpečáky někam.`` V originále bohužel ,,Get your security up here.`` Volá svému zákazníkovi, pro kterého má zboží (přičemž z předchozího textu je jasné, že jde o jakousi lahev v chladničce), a sděluje mu, že ,,Mám pro tebe ty písničky, cos chtěl.`` V originále samozřejmě ,,I got the music you wanted``. Barman, který se ohradí proti chování jistých goril v jeho podniku, sděluje hlavnímu hrdinovi, že ,,Tihle to mají znát líp.`` Aneb ,,These should know better.`` Já vždycky věděl, že Neffovi se sice IMHO Neuromancer dost povedl jazykově, ale překlad jako takový že byl horší, ale přece jen se občas i tak dost divím\ldots a to je strana asi 20, teprve\ldots

\medskip

\noindent
\textit{Happyman} V rámci prepínania televízie som v piatok večer tuším na TV Joj natrafil na prekladateľský skvost, čo sa tu už spomínal (ak si dobre pamätám): \textit{Hard to Kill} so Stevenom Seagalom. Nebol som pripravený na dve skvelé hlášky hneď po sebe: 

\noindent
,,Kde jsou policajti, když je potřebuješ?`` 

\noindent
,,Sedí na zadku a cpou se ořechovým těstem.``

\noindent
O pár desiatok sekúnd na to ako upozornenie pre ozbrojených lupičov v obchode: ,,Máte jeden náboj vlevo.``

\medskip

\noindent
\textit{Ajgor} Můj oblíbený film \textit{National Treasure}. Včera jsem viděl originální verzi (s titulkama). Sice chápu že titulky a dabing mají rozdílná pravidla, ale pokud se zásadně liší, asi někde bude chyba. Příklad: dabing: ,,Máte cennou sbírku -- ale jeden knoflík vám chybí``, titulky: ,,Máte cenou sbírku, ta vám musela dát práci``.

\medskip

\noindent
\textit{Happyman} Seriál \textit{Červený trpaslík}, systémová hláška ,,Error finding server\ldots`` preložená ako ,,Chyby hledající server\ldots``. Milé.

\medskip

\noindent
\textit{Y. T.} Seriál na Primě, překlad Radek Svoboda: 

\noindent
,,Něco hrozného jsem o něm zjistila!``

\noindent
,,Je ženatý?``

\noindent
,,Je terminální?``

\medskip

\noindent
\textit{Barclay} Seriál Simpsonovi. ,,Střelili mě do zad při představení Boba Nadějáka.``

\medskip

\noindent
\textit{fuxoft} Dostal jsem na kontrolu překlad reklamy Nestlé k filmu Cars (jehož hlavní hrdina se jmenuje Lightning McQueen). První věta zněla: ,,To je blesková královna!``

\medskip

\noindent
\textit{FIN} Film \textit{Lord of War}, titulky: 

\noindent
,,\ldots with uzi submachine guns\ldots``  

\noindent
 ,,\ldots můj kontakt mě seznámil s ponorkovými zbraněmi uzi\ldots``

\medskip

\noindent
\textit{VEETUHS} Seriál \textit{Star Trek} (debata o čaji). 

\noindent
,,It's flavoured with passion fruit.``

\noindent
,,Je ochucen smyslným ovocem.``


\podpis{vyplavil Radim Kučera}