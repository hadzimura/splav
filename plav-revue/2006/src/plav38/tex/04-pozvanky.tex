\section{Prosincová uctění památky \\ Mileny Hübschmnannové}

\noindent
Občanské sdružení ROMEA, jež se zaměřuje na mediální, vzdělávací a kulturní činnost, představuje projekt, který vznikl v reakci na nečekanou a tragickou smrt přední světové romistky, orientalistky Mileny Hübschmnannové a potřeby zachovat a rozvíjet její odkaz. V této souvislosti se v průběhu prosince uskuteční následující kulturní akce související přímo s osobou Mileny Hübschmannové či navazující na její dlouholetou aktivitu v oblasti záchrany, uchování a rozvoje romské kultury.

\bigskip
 
\noindent
Slavnostní předávání cen \textbf{1. ročníku Literární soutěže Mileny Hübschmannové}, jejímž cílem je podpořit tvorbu romsky píšících autorů, se uskuteční \textbf{7. prosince}. Současně budou vyznamenáni někteří romští spisovatelé za celoživotní přínos rozvoji romského jazyka.
Další  částí slavnostního večera bude křest knihy Milena Hübschmannová ve vzpomínkách, na jejímž vzniku se podílela její rodina a přátele z celého světa. Akce se uskuteční od 18 hodin ve Vinárně Casa Lustiana (Křižíkova 448/115, Praha 8 - Karlín).

\bigskip
 
\noindent
\textbf{8. prosince 2006 - 14. ledna 2007: \\ Výstava Milena Hübschmannová život a dílo}

\noindent
Výstava o životě a díle Mileny Hübschmannové si klade za cíl představit veřejnosti tuto ženu prostřednictvím fotografií a ukázek z jejího díla. Vernisáž proběhne 8. prosince od 18 hodin v Galerii U Zlatého kohouta (Michalská 3, Praha 1).

\bigskip
 
\noindent
\textbf{14. prosince 2006 - 30. ledna 2007: \\ Výstava Milena Hübschmannová -- Můžeme se domluvit}

\noindent
Výstava přibližuje osobu Mileny Hübschmannové, její názory a postoje formou fotografií a díla. Tato výstava zakončí cyklus výstav a přednášek konaných při příležitosti 100 let Židovského muzea v Praze. Vernisáž proběhne od 18 hodin v Židovském kulturním a vzdělávacím centru (Meislova.15, Praha 1).

\bigskip
 
\noindent
\textbf{18. prosinec: \\ Přednáška Romský a židovský holocaust očima 3. generace}

\noindent
Úvodní slovo přednese PhDr. Jana Horváthová. Diskuse s odborníky a zástupci menšin, jejichž rodin se holocaust přímo dotkl, o vnímání holocaustu dnešní společností a jeho odkazu pro generace příští. Přednáška proběhne od 18 hodin v Židovském kulturním a vzdělávacím centru (Meislova.15, Praha 1).

\bigskip
 
\noindent
Doc. Milena Hübschmannová tragicky zahynula 8. září 2005 při automobilové nehodě v Jižní Africe. Zásluh Mileny Hübschmannové lze pochopitelně vyjmenovat mnoho. Tato žena se od roku 1953 až do konce svého života věnovala právě studiu romštiny, snaze o její kodifikaci, znovuobrození a rozvoj. Stála u zrodu mnoha publikací, z nichž celé řady je autorkou. Společně s dalšími kolegy zpracovala učebnici romského jazyka a romsko-český a česko-romský slovník. Sbírala romskou lidovou slovesnost. Z její iniciativy vznikl na Filozofické fakultě Univerzity Karlovy studijní obor romistika. Výraznou měrou napomohla rovněž ke vzniku samostatné romské či romsky psané literatury a sice motivací a neustálou podporou těch Romů, kteří se v soukromí pokoušeli o psanou tvorbu. Právě na snahy o podporu vzniku romské literatury se ROMEA, o. s. rozhodlo navázat založením výše zmíněné literární ceny.

\bigskip
 
\noindent
Finanční prostředky na realizaci projektu poskytlo Ministerstvo školství, mládeže a tělovýchovy ČR.

\bigskip
 
\noindent
O.s ROMEA, Žitná 49, Praha 1, tel.: 257 329 667, romea@romea.cz, www.romea.cz

