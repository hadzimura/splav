\section{Setkání s Dámou}

\noindent
Moje první setkání bylo na přednášce na Evangelické Akademii v Praze, kde nám přednášela o historii  Romů o tom jak začala její práce, která se změnila obrovsky zájem o Romy a vše spojené kolem nich. 
 
O svých  letitých návštěvách na Slovenku v osadách mezi Romy. Co jí zajímalo a co vše sepisovala a jaký má cenný materiál. Znala hodnotu romského jazyka. Ve svých zapsaných řádcích probouzela k životu hodnotu nepsaného  projevu romského světa se vším všudy. Na poprvé  jsem si říkala, proč je tak nádherné a pěkné slyšet z jejich úst to co říkala, o Romech a kolik toho zná, překvapovalo mne její obrovské zaujetí.
Říkala jsem si to bude romni, to není možný, jak nás zná a ví o nás to pravé, to dobré.

Když začala mluvit romsky s tou obrovskou noblesou, její osobitost vyzařovala, ještě více moje sympatie měla. Vážím si o to víc lidí, kteří znají můj rodný jazyk a můžeme spolu mluvit po mém. Ona mne naučila a nejen mne, že romština patří do tohoto světa a je nutné jej zaznamenávat,  jak pro Romy samotné tak pro všechny lidi. Dávat nahlédnout všem kdo chtějí být obohaceni.
 
Poznání něčeho, někoho, není jen to, že něco vím.

Její zájem, vyburcovala v nás ctižádost, hrdost a nutnost projevu romském jazyce a projevovaná úcta k nám Romům od této ženy se nám dostalo plnými doušky. Proto i my už  za jejího života jsme ji přijaly za svou  a bude mít nadále místo v našich srdcích.

\podpis{Emilie Horáčková}


\section{Pláč}

\begin{verse}
Každé slovo má dvě mince, chvíli ruby, chvíli líce. \\
Pláč má také dva významy, \\
pláčeš radostí, pláčeš smutkem i bolestí.

\medskip

Když narodí se dítě, pláčeš, \\
\hspace{\fill}slzy stékají po tvářích hbitě. \\
Pláčeš radostí – vidíš to malé stvoření, to  krásné dítě.

\medskip

Když smutek máš ,,plač`` uleví se ti.  \\
Pláčeš bolest, kterou v sobě máš! \\
Někdy jsou slzy vidět, někdy je skrýváš. \\
Jdeš stranou do ticha, kde tě nikdo nevidí. \\
Kde nic neslyšíš -- jen svůj bolestivý pláč.

\medskip

Když vítáš člověka, někdy úsměv i slzy \\
\hspace{\fill}naráz vstanou v jednu tvář. \\
Když loučíš se  s osobou milovanou, pláčeš a říkáš \\
,,nabrskou shledanou``!

\medskip

Pláč -- slzy, které se řinou, jak malý vodopád. \\
Kdo je vidí -- nezávidí, jsou to slzy slané! \\
Ba – Ne! \\
Pláčem bolest ustane, pláčem radost projevíš. \\
Však my to známe, Ty to také víš!

\medskip

Co dělají s námi slzy, když je zříš. \\
Co slovem neřekneš -- slzami, pláčem odpovíš. \\
Slzy vytrysknou -- Ty je nezastavíš. \\ 
To cit a láska, v srdci -- řídí, ten slaný, malý vodopád.

\medskip

Na každý pád, i když slzy otřeš, z očí je to znát! \\
Že plakal jsi pro něco, co máš rád. \\
Že plakala jsi, protože tě něco bolí. \\
Pláč jiného člověka? \\
Nás dojímá k slzám též.

\medskip

Hudbu romskou, tu když si poslechneš. \\
Tvé srdce zasáhne. \\
Srdcem projevíš svůj pláč. \\
Jen samé slzy, bolest a jen pláč\ldots

\medskip

Každý pláč má dvě mince, chvíli ruby, chvíli líce.
\end{verse}

 \podpis{Emilie Horáčková}

\section{Změní se}

\begin{verse}
Snad každý ví, proč nejsme vítání. \\
Snad každý z nás už poznal ,,co je to mít tmavou tvář``! \\
Každodenní útrapy, obavy a zklamání!

\medskip

Nejen proto, co jsem zač? Také proč a zač? \\
V každé chvíli, každý čas, pociťujeme, \\
\hspace{\fill}že máme tmavou tvář. \\
Tmavé tělo, ruce, černé vlasy, oči. \\
To nehodí se\ldots

\medskip

Snad už lidé pochopili, že nežijí tu jenom bílí! \\
Snad se to změní. \\
Snad lidé najdou v nás i ocenění. \\
Ne jen bolesti, ponížení. \\
To pociťujeme za to, že máme tmavou tvář, že jsme jiní.

\medskip

Snad už o nás dost víte, v lecčem se ale mýlíte. \\
Snad změnili jste někteří i mínění. \\
Snad konečně jste řekli dost!

\medskip

Rom také má duši a v očích bolest zář. \\
Snad už  skončil ten běs. \\
Snad\ldots

\medskip

Společně a sobě rovni, každý má právo žít, na úrovni. \\
Bez hanby, honitby a ponížení.

\medskip

Co svět,  světem stojí vše se mění.
\end{verse}

 \podpis{Emilie Horáčková}

\section{Mangan Tumenge}

\begin{verse}
Me Tumenge vinčinav, \\
te aven bare raja, \\
kaj o Roam peskri paťiv te rahen. \\
Kaj o manuša te dikhen, \\
hoj o Roma hin feder, sar jon peske duminen. 

\medskip

Kanav kaj le Romen o čiripen o pharipen \\
\hspace{\fill}imar jekhvar te omukel!

\medskip

Me Tumenge Romale le Devlester mangav. \\
Lačhipen andro romano dživipen, \\ 
\hspace{\fill}but love tumen ta  avel. \\
Te ačhen bachtale, barikene andro tumáro romipen. 

\medskip

Jak o Delv te sikavel amence odrom kijo lačhipen, \\ 
O jakha amare te phundravel.

\medskip

Sako amen dureder kamela -- bo láčhes, \\
\hspace{\fill}pačivales amence dživaha. \\
Sar kampol, sar manuš. \\
Bari bach amenca džala.

\medskip

,,Devla šuntu avri mire lava``.
\end{verse}

 \podpis{Emilie Horáčková}

\section{Romale} 

\begin{verse}
Ľikeren tumen, ča nekvar dživas. \\
Ta marušen pre peste, \\
den tumen lačho lav he pačiv.

\medskip

Dikľom sar manuš  vaš minuta nane. \\
Odžal a imar pale na avel.

\medskip

Sar dživas kampol savoro te phenel manušeske. \\
Po sar imar nane,  \\
na pheneha so hin tut pro jilo.

\medskip

Romale phari doba dživas, he dživenas amare nipi. \\
Musaj jekh avres teľikerel, te hazdel upre. \\
Savore sam Roma!

\medskip

Oda nič , jak kales hin bvuter, avres nane nič. \\
Hej on hin Romasar savore.

\medskip

Po sar manuš nane, na pheneha so imar kamehas. \\
Roma avaha furt savore. \\
Te tut but love avela, sar raj – sar grofos phireha, \\
furt ča Rom aveha.

\medskip

Predal aver džene melalo -- ca Rom Kalo! \\
Romale ľikeren pes savore -- jekhetane. \\
Jekhe laveha, jekhe dromeha. \\
Sar varekana- amare phure Roma.
 \end{verse}

 \podpis{Emilie Horáčková}

\section{Zvolání} 

\begin{verse}
Procházím se krajinou, \\ 
krásou nesmírnou, všemi barvami malovanou. \\
Sílu k sobě volám ,,dej mi sílu Bože i Ty přírodo``! \\
Ať odolávám, ať jsem skálou pevnou! \\
Ne, květinkou ve větru zmítanou.

\medskip

Když smutná jsem a bolet mám velikou, \\
já stále v myšlenkách jsem s tebou, Bože.

\medskip

Když pomoc hledám, já vím, že nejsem sama, \\
Tebe mám, k tobě se obracím, \\
klekám a ruce spojené o dlaně zvedám.

\medskip

Člověk cestou životem ztrácí. \\
Ty, Bože -- při mně jsi dál. \\
V tebe víru mám,Ty Bože,jsi mi velkou oporou, \\
velkou sílu mi dáváš.

\medskip

Děkuji Ti, že s námi zůstáváš, že naše prosby slyšíš, \\
cestu ukazuješ, naše prohřešky odpouštíš.

\medskip

Tvoji pomoc potřebujeme, proto stále buď při nás. \\
,,Slyšíš mě``.
\end{verse}

 \podpis{Emilie Horáčková}

\section{Vzpomínka}

\begin{verse}
Vstala jsem  a okno otvírám, \\
vše ještě spí a všude zamčeno. \\
Prší, stále prší. \\
Pohlednu na stromy, jak v tom dešti stojí potichu, \\
přijímají kapky deště ke svému životu.

\medskip

Nikdo není venku, je ticho klid. \\
Jen kapky deště  na střechy a okapy \\
\hspace{\fill}do taktu bubnují ,,bum, bum``. \\
Přemýšlím a  hledám odpovědi na jisté otázky, \\
jak smutno je bez lidí a bez lásky. \\
Ticho, klid, stále ticho.

\medskip

Dešti, jak je tě třeba , tam kde roste chleba. \\
Déšť, připomíná mi někdy pláč, slzy, \\ 
\hspace{\fill}které tečou po tvářích, \\
tou nelidskostí, co si lidé samy vytváří! \\
Proč?

\medskip

Proč, když někoho máš rád, bojíš se, \\
\hspace{\fill}co všechno se může stát. \\
I přesto odcházejí. \\
Odešel člověk, koho ráda mám, \\
já bolest v srdci mám.

\medskip

Stále stojím u okna a dívám se ven -- prší. \\
Nebyl to sen!

\medskip

Už tu není a já s úctou vzpomínám.
\end{verse}

 \podpis{Emilie Horáčková}

\section{Kamav pačiv / Chci  úctu}

\begin{verse}
Soske me  romni som! \\
Proč já jsem cikánka! \\
Soske me kadaj uľilľom? \\
Proč jsem se tady narodila ? \\
Dživav romanes, maškar miro nípos. \\ 
Žiji po romsku, mezi svým lidem. \\
Vaš kada man aver džene na kamen! \\
Proto mne jiní lidé nemají rádi! \\
Ča man het katar peste traden. \\
Jen mne od sebe odhánějí. \\
Phenen mange: ,,Dža tuke ke tire, kadaj than nane!`` \\
Říkají mi : ,,Jdi si ke svým, tady místo není!`` \\
CIGANKO! \\
CIKÁNKO! \\
Tebi man avelas sovnakune bala, o jakha sar něbos\ldots \\
Kdybych já měla zlaté vlasy, oči jako nebesa\ldots \\
Nikhaj bi man na tradenas. \\
Nikam by mne nevyháněli. \\
Niko bi mange na phenelas ciganko! \\
Nikdo by mi neříkal cikánko! \\
Tebi avavas aver rado bi man dikhenas! \\
Kdybych byla jiná měli by mne rádi! \\
Sar pre ma dikhena o jilo lenge asala. \\
Jak se na mne podívají srdce jejich se budou smát.

\medskip

Me kanav oda šukariben! \\
Já chci tu krásu! \\
Kanav oda vaš savoro\ldots \\
Chci to vzdám se všeho\ldots \\
Vičinav avri le Devles, le benges. \\
Vyzývám Boha, ďábla. \\
Aven, keren so me kanav! \\
Přijďte,plňte moje přání!

\medskip

Kanav te avel aver te džal upre. \\
Chci být jiná nabýt na cennosti. \\
Na kanav kajso dživipen, kaj man ča lač keren! \\
Nechci takový život, když mne jen haní! \\
Na pačisaľol pes mange dživipen pre sera! \\
Ne líbí se mi život na pokraji! \\
Kanav maškar Tumende. \\
Chci být mezi Vámi. \\
Andre mire kale jakha ča edukh, e dar, o sklamišagos. \\
V mých očích je jen bolest, strach a zklamání. \\
Miri kaľi cipa he o bala, te avela aver -- \\
\hspace{\fill}feder pes mange dživela! \\
Moje černá kůže a vlasy, když bude jiná – \\
\hspace{\fill}líp se mi bude žít! \\
Vičinavas, šunde man avri – jek aviľas. \\
Vyzývala, vyslyšeli mne -- jeden přišel.

\medskip

Miro šero diľiňardo has, kaj kamavas aver te avel. \\
Má mysl byla zatemněná chtivostí, že chci být jiná. \\
O beng terdžiľas paš mande, kerdža so me kanavas. \\
Ďábel si stoupil ke mně, vše změnil co jsem chtěla.

\medskip

Na somas oda imar me, has oda aver džuvľi, šukar, \\
\hspace{\fill}parňa cipaha he šarge balence. \\
To už jsem nebyla já, byla to jiná žena, krásná, \\
\hspace{\fill}s bílou kůží se žlutými vlasy. \\
Bije ladž,o vudera has lake phundrade -- \\
\hspace{\fill}šaj džalas peske všadzi\ldots \\
Bez ostudy, dveře měla otevřené – \\
\hspace{\fill}a  mohla jít kam chtěla\ldots 

\medskip

Miro nipis na prindžardža man . \\
Moji lidé nepoznali mne. \\
Miri dajori obgeľa man -- rodelas peskra čha. \\
Moje maminka obešla mne -- hledala svoji dceru. \\
Sar mulo te terdžol paštute -- \\ 
\hspace{\fill}kajci me pašlende terdžuvavas\ldots \\
Jako duch co stojí vedle tebe -- \\
\hspace{\fill}tolik jsem pro ně stála\ldots

\medskip

Ole džene so kelende kamavas te perel, \\
\hspace{\fill}la parňa cipaha he savoreha soman has? \\
A ti lidé kam patřit jsem chtěla, \\
\hspace{\fill}tou bílou kůží a tím vším co jsem měla? \\
Te chudel man kamenas a ča telel. \\
Dotýkat se jen chtěli a jenom brát. \\
Oda me so kanavas techudel! \\
To já byla co chtěla dostávat! \\
Ča jekh nalačno has. \\
Jen jednu chybu to mělo. \\
O jílo ačhiľas romano. \\
Srdce zůstalo romské. \\
Kada sávoro has predal mande previsardo dživipen! \\
To všechno bylo pro mne převrácení smyslu \\
\hspace{\fill}mého života! \\
Vaš kada diňom  o romano dživipen? \\
Pro toto jsem se vzdala romského života? \\
Vaš kaja kuč som aver! \\
Za tuto cenu byla jsem jiná!

\medskip

Imar kavka na kamavas te dhičhol, \\
\hspace{\fill}na has oda lačhi čara. \\
Tak dál vyhlížet  jsem nechtěla, \\
\hspace{\fill}nebyla to dobrá výměna. \\
Bajinavas, rovavas, le Devles mangavas. \\
Litovala, plakala, Pána Boha prosila.

\medskip

,,So som Devla tekerel,kaj me pale Romni te avel?`` \\
,,Co mám Pane Bože udělat, ať jsem zase Romka?`` \\
O Del šundža mire lava, avľa ke mande. \\
Pán Bůh slyšel moje slova a přišel ke mně. \\
Phendža mange: \\
,,Ač ajsi savi sal, na kampol tuke aver šukariben. \\
 Me kerdžom tut the tut hin kuč pre kaja luma.`` \\
Řekl mi: \\
 ,,Zůstaň taková jaká jsi není ti třeba jiné krásy. \\
  Já stvořil tě i ty máš cenu na tomto světě.`` \\
Geľa het a me somas furt koja gadži, \\
\hspace{\fill}na achľuvavas o leske. \\
Odešel a já byla stále stejná nezměněná, \\
\hspace{\fill}nerozuměla jsem tomu. \\
Daravas so avela dureder! \\
Bála jsem se co bude dál! \\
Miri dar, miro pharipen man omukľa, \\
\hspace{\fill}sar andro kher pes okham sikadža. \\
Můj strach mé břemeno mne opustilo, \\
\hspace{\fill}když do místnosti se slunko vlilo. \\
Sigo ušťilom, o gendalos rodavas -- kaj man te dikhav! \\
Rychle jsem vstala, zrcadlo hledala -- ať se přesvědčím! \\
Pekľom andre vika baripnastar. \\
Vykřikla jsem radostí. \\
,,Oda me pale imar!`` \\
,,Jsem to zase já!`` \\
O jilo mange pale pro than has. \\
Srdce zase bylo na svém místě.

\medskip

Pro Del dumindžom so phendža mange andro suno. \\
Na Pána Boha vzpomněla co řekl mi v mém snu. \\
Sakone manušes -- ta he man hin kuč. \\
Každý člověk -- tedy i já mám svou cenu.

\medskip

,,Me dživava miro dživipen dureder, o leha so man hin,\\
 \hspace{\fill}he savi som. \\
,,Já budu žít svůj život dál, s tím co mám a jaká jsem. \\
Na kampol mange te živel avreskro dživipen. \\
Ne potřebuji žít jiný život. \\
Me dživava miro romipen.`` \\
Já budu žít svůj romský život``.

\medskip

Sako pesko, sakoneskro dživipen mol. \\
Každý svůj, každého život má cenu.

% Drom
% Aves pro svetos phares, užardo sal- barikane o manuša.
% Hordinen tut pro vasta, giľaven giĺora.
% Kim mek na phires sa hin lačhes.
% Sar imar terdžos pre peskre pindre – imar pesko than rodes.
% O berša džan, bo imar peskro drom phires.
% 
% He phučes.
% Ko mange phenela, kaj miro drom?
% Savo miro drom avela?
% Phirava kaj, šaj me dodžava?
% Sar me dživava?Romanes?
%  Sar me kanav?
% 
% Lokes bije dar, bije lač,
% Na kanav te avel nikaske pro pharipen.
% Kana lokneš pes dživelas?
% Akor kana verdana amen has?
% 
% Pal jekh drom  savore džahas.
% E jagori labarahas, giľora giľavahas.
% Phari voďi šundžolas,
% o phure denas duma pal pesko dživipen.
% A men ča šunahas.
% Te sovel pes thovahas, dikhahas andre jagori sar labolas.
% E jagori labolas, e rat zagaruvlas savoro.
% 
% Phučahas?
% Tajsa pro drom arakhaha kas?
% So man le dromeha užarel?
% Le dromenca džahas, 
% loke jilenca.
% Sidžarahas le khameske pašeder.
% 
% Imar oda na džas!
% Sa amendar ľile!
% O phure mule!
% O terne bisteren!
% Soda te džal le khameske anglal.
% 
% 
% Akana dživas pre jekh than.
% e jagori labaras ča varekana.
% O giľora giľavas ča khere- polokes!
% Dživas na pal peskro?
%  Pal gadžeskro?
% Mištes?
% 
% O purano – omukľam,
% amri pačiv bisterdžam!
% Kaj amari pačiv, 
% kaj amari ladž Romani?
% 
% Rozgeľam pes katar peste.
% Sako rodel pesko drom, than.
% Sako phučel korkoro.
% So man tajsa užarel?
% Ko man palal lela?
% Ko man marela?
% Pal savi sera bara prema čhivena?
% 
% Ko man pačiv na dela?
\end{verse}

 \podpis{Emilie Horáčková}


\section{Moje nejstarší vzpomínka}

\noindent
Můj příběh Vás zavede na východní Slovensko v okrese Spišská nová Ves, do vesnice Doubrava. Kde kdysi bydlela  moje babička z matčiny strany, dědečka jsem nepoznala v té době už nežil. Moje babička měla 9 dětí, vychovala je skoro sama všechny a když ovdověla, zůstala do konce svého života sama i když byly ještě mladá. Když děti vyrostly, tak se rozešly do světa – jak za prací i následovaly své nové rodiny. Většina jich odjela do Prahy, zde i jinde pracovali jako dělníci, dost jich v té době pomáhalo na Slapské přehradě. Získali byty a babička se přestěhovala k nám, u nás bydlela.

Na Slovensko jsme jezdili málo, babička byla tady a děti měla v Praze.

Chci Vám vyprávět o tom, jak jsem poprvé jela na Slovensko s babičkou. Jela jsem do její vesnice, kde měla svůj baráček. Strašně ráda na to vzpomínám, ale i smutno mi bylo z toho co jsem viděla.Ten baráček byl celý ze dřeva, byla to taková  jedna větší místnost, která stála u lesa,  dál od vesnických chalup, na každou stranu kam  jsem se podívala, byl pohled na lesy a na pole, ale sluníčko měla přímo, takže ráno bylo krásné vstávání.

Když jsem si toto prohlédla, šla jsem se podívat -- co dělá babička?

Byla za baráčkem a plakala. Mě v té době bylo necelých šest roků, ještě jsem nechodila do školy, neuvědomovala jsem si spoustu věcí, ale tato věc mi zůstala v paměti. Zůstala jsem stát a koukala, jak pláče a přitom opravuje, narovnává desku, chce zatlouct hřebík kamenem? Bylo to pro mne chaotické, nechápala jsem, proč to opravuje, tu zříceninu?

Proč pláče nad takovou troskou? Mnoho let zde nebydlela a baráček se rozpadal a také různí pobertové to rozebírali na dřevo. Moc mne ale mrzelo, že babička pláče. Spoustu věcí mi vyprávěla ze svého dětství, záhy po dovršení tří let, zemřeli její rodiče, které nepoznala. Babička také říkala, že matka její byla židovka. Ona sama byla vychována u sedláka. Možná proto byla tak šikovná i chytrá. Mé vzpomínky jsou zaryté v mé paměti, člověk takové krásné věci i milovanou osobu nemůže zapomenout. V té chvíli jsem koukala, jak a co dělá, ale potom jsem jí šla pomoci. Držela jsem na druhé straně prkno, aby to mohla zatlouct -- přibít! Babička vykřikla ,,pozor``! Já jsem stačila odskočit. Laťka spadla jinam.

Měla jsem babičku moc ráda, chovala jsem k ní úctu, byla tak krásná. Líbily se mi její krásné sukně, nosila jich spoustu -- i pět najednou, na to ještě velikou zástěru a na hlavě nebo na ramennou šátek. Byla i velmi chytrá, sice neuměla číst a psát, ale rozuměla všemu.

Procházely jsme se i po vesnici, velice mne překvapilo, kolik lidí zná babičku volali na ní ,,Ondova, poďtě tu``. Každý ji poznával i za ta léta, která tu nebyla. Kolik lidí jí znalo a hlásilo se k ní. Pamatuji si, že babička jim dávala šátky, oni jí dávali špek, škvarky a to co oni sami chtěli dát. Já dostala slovenské koláče s tvarohem i s makovou náplní. Moc jsem si pochutnala. Nejvíce mne také udivila ta zvířata, která tam všude byla, venku v zahradách pobíhala. Bála jsem se jich a i smrděla. Z Prahy jsem to neznala a ani neměla možnost vidět. Tak jsem radostně zjišťovala co všechno babička znala a jak se tam měla a jak nám bylo krásně. V ten poslední den jsem s babičkou ráno odešly na hřbitov. Babička mi ukázala, kde leží její rodiče, manžel i syn. Upravila hroby, zalila kytky, chvíli postála sama a pak jsme odjely zpět domů do Prahy. V dnešní době ten baráček dávno nestojí. Babička se dožila krásného věku 93 let. Už také nežije, už není, ale vzpomínky mi na ní zůstaly krásné i bolavé, protože tu není se mnou.

 \podpis{Emilie Horáčková}
