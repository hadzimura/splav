\section{Milena Hübschmannová}

\noindent
Velmi mě zarmoutila zpráva o smrti naší největší postavy v novodobé historii romského národa. Milena Hübschmannová bezesporu patřila a bude ještě dlouho patřit mezi naše veliké osobnosti, zejména týkající se romštiny. Milenu jsem poznal před více jak 20 lety. Vzpomínám si to přesně, jako by to bylo včera. V tehdejším \textit{Rudém právu} probíhala diskuse o životě Romů a velice mě tehdy zaujal příspěvek M.H. Požádal jsem redakci o adresu a obratem jsem Mileně v dopise poděkoval za krásná a upřímná slova, která vycházela ze srdce.

Pak přišla sametová revoluce, vznik \textit{ROI}, \textit{Romana kurka} a romských tiskovin a naše cesta se spojila. Byla mou velikou učitelkou a rádkyní, moje první verše přeložila do češtiny a také spolupracovala na mé knize romských básní. Několikát jsem naši sestru navštívil u ní doma a cítil jsem se tam jako doma. Ona rovněž  byla u nás  vždycky vítána, jezdili jsme spolu k Romům na Slovensko a všude ji měli lidé rádi.
Velice nám bude chybět.

Miro jilo – moje srdce bolí, dukhal pal tute Milen, stále na Tebe budeme vzpomínat jako na člověka, který našemu národu věnovala celý život, bojovala za Romy více jak samotní Romové. Posílám Mileně – amare pheňake, amara učitělkake  jekh lavoro jilestar – ŚOHA PRE TUTE NABISTERAHA, TIRI BUTI DUREDER BARARAHA U PRE AMARI ROMANI ČHIB THE PRE TUTE GONDOLINAHA. ÁČH DEVLEHA, LOKI PHUV TUKE, AMARI KEDVEŠNI MILENA.

\noindent
TIRO

\podpis{Jankus Horváth Bilovcistar}  