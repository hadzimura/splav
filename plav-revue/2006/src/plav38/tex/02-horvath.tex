\bigskip

\section{Nabisteraha}

\noindent
\textit{Leperav pre Milena Hübchmannova \\ Věnováno památce Mileně Hübschmannové}

\begin{verse}
O manuš, kana uľol, rovel, e daj the o dad lošanen. \\
Bararen les šukar, pativales, avka sar kampel. \\
Tu, amari phenije Milen, tu tiž amenge sikhadžal, \\
save dromeha te džas. \\
Savi giľi te giľavas, amari romaňi čhib te guľaras. \\
Tiri buti , tiro dživipen, tiri cali zor diňal amare \\
niposke, le čore Romenge. \\
Phirdžal  them themestar,  \\
romano dživipen sikhľiľal . \\
Sar te dživel, sar jekh avres  rado te dikhel. \\
Amaro šukar romipen  te ľikerel. \\
Tu Milen, salas amari čačikaňi romaňi dajori. \\
Tu salas amaro vudud, pre kada šilalo, šititno svetos. \\
Tu amenge sikhadžal o  drom, u amen kale  \\
 \hspace{\fill}dromeha džas, \\
the dživas. \\
,,Vakeren  romanes, na gadžikanes, ma ladžan pes \\
 \hspace{\fill}vaše duma romaňi\ldots`` \\
Ola tire lava, sar brišind andre bari truš, \\
 \hspace{\fill}sar balvaj andro baro keradžipen. \\
 Adadžives sal  pašo amaro somnakuno Devloro, \\
 \hspace{\fill}u amen dživas u tuke  soľacharas: \\
,,Amara guľa čhibaha vakeraha, romanes duma daha.`` \\
Amaro romipen bararaha, šukar romane giľa  giľavaha. \\
U pre Tute, amari bari romaňi pheňije the dajori šoha \\ 
N a b i s t e r a h a  !
\end{verse}

\podpis{Jan Horváth} 

\bigskip

\section{Amari luma / Náš svět}

\begin{verse}
Amari luma hiňi igen čori \\
Ňiko na džanel, katar o Roma avle, \\
Sar te dživel, sar peskeri bacht te rodel. \\
O čhavo barol the phučel la lumatar: \\
,,So te kerav? Kaj te džav, te rodel miri bacht? \\
The me kamav te asal,te khelel\ldots \\
The kola luluďora korkore baron, sar kamen, \\
O čirikle hurňisaľon, sar pes lenge pačinel. \\
Amari luma hini bari nasvaľi, rat, bokh, čoripen, \\
kaj dikhas, odoj o manuša cerpinen. Le čhavoren \\ 
O benga livinen, kaj  oda dogeľam?`` \\
Devla, šigitin amenge, savore manušenge, kaj te roden \\
Sako peskeri loš the bacht pre adi luma.  \\
Te dživel šukares, manuš manušeha – amara Lumaha, \\
Pre luma, la lumake.
\end{verse}

\medskip

\begin{center}
* * *
\end{center}

\medskip

\begin{verse}
Je velmi chudý, sám, nemocen. \\
A Romové v něm? Hledají své místo na slunci. \\
Co chtějí tito lidé? Čí jsou a odkud jsou? \\
Dětské oči se ptají světa: Co mám dělat, kam  jít \\ 
hledat štěstí a klid? Nechci více, jen se smát, \\
 \hspace{\fill}zpívat, radovat. \\
Pláče již bylo dost! \\
Svět náš je nemocen, krev a slzy, \\
 \hspace{\fill}hlad a bída je naší noční můrou. \\
Lidé z rodu Romů prosí Boha: \\
 \hspace{\fill}Dej nám sílu a světlo k uzdravení. \\
Místa v tomto světě tu pro nás není. \\
Chci žít, zpívat, tancovat -- s celým světem, \\
Pro svět. \\
A s ním.
\end{verse}

\podpis{Jan Horváth}

\section{Užarav / Čekám}

\begin{verse}
Užarav, pro khamoro, sar uštěla. \\
Peskere vastenca man taťarela. Mire čhavenge loš, \\
\hspace{\fill}kamiben anela. \\
Užarav, sar o čhonoro peskere jakha  \\
\hspace{\fill}phundravela u paš e jagori \\
raťi amenge svicinela. O čhaja le čhavenca khelena,  \\
\hspace{\fill}the šukar romane giľora giľavena. \\
Užarav paš e ľen, paňori žuži andro khoro mange lava, \\
\hspace{\fill}trušalo som the, \\
bokhalo, kaj džava, ko man maroro dela? \\
Paťiv, lačho lav, lačhes pre mande dikhela? \\
Užarav, paše jagori pre oda džives, so avela,  \\
\hspace{\fill}so man dureder užarela? \\
Užarav the duminav – Kamav feder dživipen  \\
\hspace{\fill}mre čhavorenge. \\
andre lengre kale jakha loš te avel, lengere kale bala e  \\
\hspace{\fill}balvan šukare te \\
chanel. \\
So kerav? Užarav. Na! Naužarav, imar džav –  \\
\hspace{\fill}užarel man baro drom, \\
Naužarav – džav, sar manuš! \\ 
Sar Rom! Džav.
\end{verse}

\begin{center}
* * *
\end{center}


\begin{verse}
Čekám, až sluníčko zlaté vstane,  \\
\hspace{\fill}svými paprsky mě ohřeje, mým dětem \\
černým radost a štěstí přinese. \\
Čekám, až měsíček bílý své oči otevře,  \\
\hspace{\fill}u ohníčka nám bude svítit. \\
Dívkám a chlapcům při tanci zanotuje. \\
Čekám u studánky, žízním, však i hlad mám, \\
Čekám, hladov, do ohně přikládám,  \\
\hspace{\fill}jsem zmaten a sám sebe se ptám: \\
Co proto udělám – aby mé černé děti jen štěstím žily,  \\
\hspace{\fill}budoucnost dobrou měly? \\
Dost bylo již slz, smutku a trápení –  \\
\hspace{\fill}i Rom je člověk a člověk je i Rom. \\
Co dělat mám? Bože, dál již nečekám! Jdu, \\
\hspace{\fill}vstanu a jdu, přede mnou cesta daleká, \\
\hspace{\fill}dlouhá, nelehká – na konci však naděje. \\
Pro mne, pro tebe, pro člověka.
\end{verse}


\podpis{Jan Horváth}

\section{Kaľi luluďori}

\begin{verse}
E balvaj le khameha la luluďora sadzinďa, \\
Korkori barol e kaľi luluďori, \\
Le apsenca barol andro bare bara. \\
Kala jakha, kale jakha, \\
Soske kejci roven? \\
Ňiko pre kaľi luluďori, paňori na čhivel, \\
Lačho lav na del, ajsi zoraľi, dikhen – mek dživel! \\
Kale jakha, kale jakha, soske kejci roven? \\
Šil, bokh,meriben,dukh,bari, oda savoro \\
E kaľi luluďori chaľas, ajsi zoraľi, -- mek dživel! \\
Kale jakha, kale jakha, soske jon asan? \\
O khamoro la taťarel, brišindoro šuťarel lakero gadoro. \\
So pes ačhiľas? \\
Paš late barol parňi pheňori, \\
Save šukar jekh paš aver. \\
O jile, o vasta peske den, savoreha pen ulaven. \\
Jekh avres bares bečeľinen. \\
O kham le čhoneha  čardašis khelel, \\
O čirikle pre lavuta šukar giľi bašaven. \\
Jekhetane imar dživena, sar phrala the pheňa. \\
Kaľi the parňi luluďori\ldots
\end{verse}

\podpis{Jan Horváth}

% \section{Meriben / Lety}
% 
% \begin{verse}
% Andro baro veš, dur romane gavestar adaj , adaj soven .
% O čhavore romane, peskere dajenca, sar o cikne čiriklore.
% Baro šilalo beng len chaľa dživipnastar.
% Sikhľonas te hurnisaľol, kamenas peskere dromeha te džal.
% Čit, kerestor, bari čar the luluďa pre lende barol, e phari kaľi phuv lengeri daj the dad.
% O apsa  mange čuľan andale mire kale jakha, bari choľi andro miro kalo jilo :
% So kerde kala cikne čhavore ? Murdarde len, čhide andre bari chev u daj soven.
% Soven, soven u te chal imar na mangen .
% Ko došalo ? Kaj o čačipen the pativ ?
% Sakoneske peskero !  The o Roma manuša, sar  aver, thovas lenge ajso baro bar, so cali luma dikhela  the o manuša adaj phirna. 
% Te šunel, sar oda kalo čirikloro giľavel žalostno giľi romaňi.
% Vazde  opre tiro kalo šero, ó Roma, Romale !
% Tiri lavuta mi bašavel , me vičinel adaj savoren nipen .
% Tiri giľi romani te šunďol pro agor lumake . Sako manuš me dikhel,
% hoj the o rom hin manuš,
% u  manuš hin the o Rom !  
% \end{verse}
% 
% Dlouhá cesta
% 
% \begin{verse}
% Jdou děti větru, jdou, cestou, necestou .
% Hledat své místo na slunci, dětem radost a štěstí .
% Vichr, déšť, zlobu a zášť – to věrná družka Romů .
% Hledám, hledám, úctu .Nemám nic, majetek,nejsem bohat .
% Jen otevřené srdce, lásku,romství,čest – Jak žít dál ?
% Můj vůz mě žene za sluncem .
% Déšť,bláto,chlad a nenávist lidí mne žene kupředu .
% Od vesnice k vesnici, můj chléb jsou mé písně, o žalu,
% Bolu tohoto světa .Bože, kam mám jít ?
% Ukaž mi cestu,chci žít jako člověk, s úctou, s láskou.
% Jdu, stále bloudím nocí, hvězdy a měsíc – mé věrné družky svítí 
% Mi do kroku . Síly ubývají,nemohu již více,
% Kde je to místo bez nenávisti,zloby a přetvářky ?
% Chci žít, tancovat,zpívat, pracovat.
% Smát se i plakat .
% To vše nalézám na mé cestě …
% \end{verse}
% 
% \podpis{Jan Horváth}

\section{Ježišis / Ježíši}

\begin{verse}
Uľiľal, kaj e luma te ratines, amare bini. \\
Uľiľal, sar o baro Dad phendžas, \\
So Tut pre luma bičhadžas. \\
Pre adi luma, pre vudud, \\
 tačipen, gulo kamiben \\
manušenge. \\
Sako nipos Tu kames, \\
Parnes, kales, the šarges. \\
Tiro dživipen našaďal prekal amende. \\
Tiro rat the tire masa amen chas sako džives. \\
O maro des amen adadžives,  \\
\hspace{\fill}zor amen de ó Devla Ježišis, \\
Pro kerestos. \\
De amen zor, mangas Tut bares, \\
Loš the asaben te avel andre amare jile.  \\
\hspace{\fill}Sikhav amenge o drom, sar andale bida avri.  \\
\hspace{\fill}Kijo feder dživipen amare čhavorenge. \\
Kamas Tut bares, paťas Tuke. \\
Mangav Tut – džaha Tuha, Tire dromeha,  \\
\hspace{\fill}miro calo dživipen Tuke dava, \\
Tiro nav, e suntno voďi andre amare jile  \\
\hspace{\fill}dži ko meriben ačhola. \\
Tu sal amaro Raj. \\
Kamas Tut the kamaha, aver drom nane. \\
Avka ela amaro dživipen aver, feder, lošaneder, \\
\hspace{\fill}šukareder -- Tu baro \\ 
Amaro Ježišis. \\
Amen.
\end{verse}

\begin{center}
* * *
\end{center}

\begin{verse}
Narozen, hříchy tohoto světa vykoupit a nás spasit. \\ 
Narozen, dle přání velikého Otce,jež Tě na svět poslal. \\ 
Pro tento svět, pro světlo, dobro a lásku  k lidem. \\ 
Všechen lid Ty miluješ, \\ 
bílého, černého i žlutého. \\ 
Svůj život však dal jsi nám v plen, \\ 
krev Tvá i tělo my denně přijímáme jen. \\ 
Ten Chléb , vzácný to dar! \\ 
Též sílu nám dej, prosím, \\ 
smích a radost, ať v našich srdcích vyžene bolest a žal. \\ 
Ukaž nám cestu, jak z bludu ven, tmy a nenávisti,  \\
\hspace{\fill}toho již bylo dosti! \\ 
Milujeme Tě, ó Pane Ježíši, věříme Ti, prosíme Tě! \\ 
Půjdeme s Tebou po celý svůj život, \\ 
Tvé jméno i duch svatý, nechť v našich srdcích nás  \\
\hspace{\fill}na věky hřeje, \\ 
Ty jsi náš Pán. \\ 
Milujeme Tě, jiná cesta není, náš život s Tebou  \\
\hspace{\fill}bude jiný, \\ 
Lepší, radostnější, krásnější. \\ 
Ó ty náš veliký, Ježíši. \\ 
Ámen.
\end{verse}

\podpis{Jan Horváth}

\section{Sikhľuvas / Učím se}

\begin{verse}
Amare phure Roma na džanenas te ginel,  \\
Te irinel, andre škola na phirenas,  \\
\hspace{\fill}ča phari buti kerenas. \\
Soske le Romeske e škola?  \\
\hspace{\fill}Mi kerel pre amende, na kamas sikhade \\
Romes, duminenas o raja. \\
Amare phure Roma andre škola na phirenas,  \\
\hspace{\fill}te ginel na ginenas, nikhaj našti phirenas. \\
Goďi, odi len bari has. Džanenas mišto savoro,  \\
\hspace{\fill}so kampel, sar te dživel. \\
Pašo maľi te kerel, andro veša, palo paňa,  \\
\hspace{\fill}o draba te kidel, \\ 
Palo vasta te ginel. \\
Ačhava korkoro, sar čhindo kaštoro, kim džanava  \\
\hspace{\fill}savoro. Andre škola phirava, \\
Odoj savoro man sikavna. \\
Andre škola džava, sa odoj sikhľuvava, sar o gadže,  \\
\hspace{\fill}sar o nipi dživava. \\
Čhavo romano avava, bachtalo, goďaver,  \\
\hspace{\fill}sar o manuša aver. \\
E luma ela the miri, sar tiri, leskeri, amari . \\
Sikhľuvav.
\end{verse}

\begin{center}
* * *
\end{center}

\begin{verse}
Naší předkové, staří Romové, číst, psát neuměli, \\
Do školy nikdy nechodili. \\
Práce, místo učení, to bylo jejich mučení. \\
Čemu ta škola Cikánovi bude? \\
Na poli ho je potřeba, na jaře, v létě, i v zimě,  \\
\hspace{\fill}práce pro něj \\ 
Je dosti. \\
Naši předkové, staří Romové, knihu neznali, nač také? \\
Práce a život  je naučil, rozumět lesu, slunci i měsíci. \\
Příroda byla jejich matkou, věrnou to kamarádkou. \\
Nyní je Rom sám, svět je daleko vpředu, věru Romové, \\ 
Musíme se hodně učit, žít, pracovat. \\
Hlupák je k ničemu, jen k smíchu, k legraci. \\
Tu budeme dělat po práci. \\
Škola, škola nás volá, učme se žít, společně. \\
Pro sebe, pro tebe, pro nás, pro vás, pro tento \\
Svět, kde i my jsme doma. Učím se.
\end{verse}

\podpis{Jan Horváth}

\section{Sikhľuvav}

\begin{verse}
Sikhľuvav te dživel, \\
Sikhľuvav buti te rodel,\\
Te khelel.\\
Rodav dživipen. \\
Sikhľuvav gadžikani čhib, kaj lenge te phenav: \\
Hoj som manuš, sar the jon. \\
Hoj kamav te dživel, na te merel. \\
Man romani cipa, romanes vakerav? \\
Oda bari bibacht? \\
Tu sal gadžo, baro manuš, me som tiž. \\
Jekh kham, the čhon pre sakoneste zabol, svicinel. \\
Barvale nasam, nane amen phuv či bare love, \\
Hin amen ča baro jilo. Sakones das pativ, manušes. \\
Ko hin lačho te šunel amari dukh the roviben. \\
 Našti avka dživas, sar ruva. \\
Sam manuša, kamas te sikhľol, te khelel, te asal. \\
Tuha, leha, amenca, jekhetane \\
Sikhľuvas.
\end{verse}

\podpis{Jan Horváth}

\section{Čit raťi / Tichá noc}

\noindent
\textit{Hraje romská hudební slupina City Boys \\ (Trnava, Slovenská republika).}

\begin{verse}
Imar avel, \\ 
oda džives, o Roma, užaren, \\
andre khangeri tumen džan, \\
akana mangen Devles,  \\
aven khere, chan, pijen, \\
pre odi karačoňa.

\medskip

Imar avel, \\
karačoňa, \\
kaštoro uraven, \\
o giľora peske giľaven, \\
kala šukar suntne giľora, \\
pal o skamind jon bešen \\
šukar pes ľikeren, pre kodi \\
karačoňa, pre kodi karačoňa.
\end{verse}

\podpis{Jan Horváth}

\bigskip

\section{Amaro drom}

\begin{verse}
Amaro drom phundrado či phandlo? \\
E luma bari, igen bari,  \\
Hiňi the amari? \\
Sakones hin than, peskeri phuv, ča o Rom dživel sar  \\
\hspace{\fill}andro veš  ruv. \\
Sar ruv rovel the rodel  bacht, kamiben, manušengeri  \\
\hspace{\fill}pativ, loš, tačipen. \\
Kaj amaro drom ehin, Devla? \\
Kaj amaro vudud, so amenge labola? \\
Pro than, kaj amen khelaha, paš amare jaga, lole sar  \\
\hspace{\fill}bengeskere drakha \\ 
Amare giľa, čorikane the barvale, Romale \\
Džava, džava lungone dromeha, o čhonoro  \\
\hspace{\fill}le khameha mange pro drom khelela. \\
E balvan la paňaha man korkoro na mukela. \\
Amaro drom, baro amen užarel, \\
Ta uščas upre, u džas, \\
La lumake anglal, o dukha, čoripena mukas dur, palal. \\
Angle amende baro drom, phundrado, džas, sidžaras. \\
O dživesa prastav, denašen, \\
Aven čhavale Romale, drom amaro \\ 
 phundrado
\end{verse}
\podpis{Jan Horváth}