\bigskip 

\noindent
\textit{Následuje dvojí překlad jedné básně, pocházející z pozůstalosti Emanuela Frynty.}


\section{Boris Pasternak (bez názvu)}

\begin{verse}


Neblahá věc -- slout proslulostí.

S~tou lásky sotva došel bys.

K~čemu nám archiv devotnosti,

nač uctívati rukopis?

\medskip

Cíl tvůrcův jeden -- odevzdat se,

nikoliv úspěch reklamy.

Je hanba: nebýt, jenom zdát se

a vlichotit se cetkami.

\medskip

Bez samozvanstva žít, jak prostí!

Tak žíti, abys přivábil

sem k~sobě lásku budoucnosti,

pro niž jsi prostor vydobyl.

\medskip

Ne papíry, však osud holý

zmapovat třeba, strast i strach.

Celého žití kapitoly

zaškrtat vroucně po stranách.

\medskip

A~ponořit se do neznáma

a stop svých skrýt v~něm živý sled,

jako je skryta krajina má,

když v~mlze na krok nevidět.

\medskip

Další si budou cestu klestit

houštím tvých zisků, snů i ztrát,

ty ale prohru od vítězství

pro sebe nesmíš rozeznat.

\medskip

Nezpronevěřit se ni vráskou

tváři své a své myšlence!

Jen zůstat živ, jen živ být s~láskou,

jen žít, jen žít až do konce!

\end{verse}

\podpis{přeložil Emanuel Frynta}