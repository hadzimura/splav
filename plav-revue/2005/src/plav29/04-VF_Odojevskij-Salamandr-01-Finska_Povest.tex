\section{Vladimir Fjodorovič Odojevskij: Salamandr}

\noindent
\textit{Úryvek pochází z~první části jménem ,,Fin Petra Velikého``.}

\medskip
\noindent
,,Na celém světě není země krásnější než naše Suomi; u~nás máme jak široké moře, tak hluboká jezera, tak borovice, jež se zelenají po celý rok; i v jiných zemích mají slunce, jenomže tam slunce vzejde, nakrátko zasvítí a zas se schová za obzor jako u~nás v zimě. Kdežto naše slunce půl roku odpočívá, zato půl roku pak svítí téměř bez ustání a naše nivy ještě ani neoschnou od večerní rosy a už je svlaží ranní rosa. Ale za dávných dob bylo u~nás ještě líp, neboť jsme měli divotvorný mlýnek Sampo, skvostné, vzácnými kameny nejrozličnějších barev vykládané paladium s tak mistrně zhotoveným víčkem, jaké by dnes ani ti nejdovednější kovotepci nevykovali. Tehdy se v Suomi žilo jako v ráji; lidé nemuseli nic dělat, všechno za ně dělal Sampo: i dříví nosil, i domy stavěl, i kůru na chleba mlel, i mléko dojil, i struny na kantele napínal, i písně zpíval, zatímco lidé jen leželi u~ohně a převalovali se z boku na bok; všeho bylo dostatek. Ale když se Väinömöinen (finská obdoba Apollóna, pozn. autora) na nás rozhněval, Sampo se skryl pod zem, nahrnul na sebe kamení a nám na zemi zůstala jen kantele. Tehdy lidé nebyli jako teď, ale mnohem urostlejší, statní. Chtěli se tím kamením k mlýnku prokopat, vynaložili na to nemálo času a sil, avšak nepořídili a jenom celé Suomi zavalili kamennou sutí. Od těch dob bylo také jiným národům známo, jaký poklad se skrývá v naší zemi; nejdříve Švédům, ale potom i Rusům; právě od těch dob vedou mezi sebou spor, komu má Sampo připadnout. Švédům panuje král, Rusům car. Oba jsou znamenití čarodějové. Znají způsob, jak dostat paladium ze země, ale ani jeden se ho nechce vzdát ve prospěch toho druhého. Zmocnit se mlýnku měli oba už dávno v úmyslu. Je vůbec něco, co by nevěděl náš věhlasný čaroděj Kukari? Jemu je jasné, co z čeho vzešlo, odkud se vzalo železo, odkud větrná smršť, odkud pocházejí tajemné síly země, ale před carem a králem bledne i jeho jasnozřivý rozum. Car je zřejmě mocnější než král, neboť car ví, jak se král narodil. Sotvaže král vyšel z matčina lůna, už dupnul na zem a pravil: ‚Co mi Jumala, vládce hromu, dal, to mi Perkele, vládce pekel, nemůže vzít.‘

A~vydal se do světa s železným mečem. Kamkoli přijde, mávne mečem a všichni kolem padají mrtví; a takovou moudrostí je obdařen, že ho ještě nikdo nespatřil, jak by jedl či pil, a když spí, má zavřeno jen jedno oko, kdežto druhým ustavičně pozoruje nebe i zemi a jasně vidí, co a jak z čeho pochází. Jen jedno mu zůstalo skryto: odkud se vzal car Rusů. Ten prý vystoupil přímo z mořských hlubin. Zuřila strašná bouře, vlny omývaly zemi, lodě se potápěly, skály padaly z břehů do moře. Král seděl na břehu, mával železným mečem, přikazoval skalám, aby vyvstaly z moře, ale skály ho neposlouchaly. Král se rozhněval, moře se vzdulo ještě divočeji. Vtom se však rozestoupilo a z vody vyšel car Rusů; jednou rukou vrátil skály zpět na jejich místo, druhou opsal kolem sebe kruh a pravil: ‚Vše, co obhlédnu, je moje.‘ Král se rozhněval ještě víc a střelil po carovi železnou koulí. Car mu odpověděl stejným způsobem. Nato král začal na cara střílet sírou a sanytrem. Ale car neměl ani síru ani sanytr. Boj se stal nerovný. Car shromáždil své Rusy a vydal se s nimi do širého světa na zkušenou; zchodil devatero moří, došel až tam, kde se nebe snoubí se zemí. Nakonec se zastaví na jednom místě, udeří žezlem do země a poručí: ‚Kopejte,‘ a ze země vyvstane železná ruda. Udeří na jiném místě -- ze země vyvstane síra a sanytr; udeří na třetím -- i tam ho čekají samé vzácné věci. Ale k divotvornému mlýnku Sampo se přesto neprokopal, protože Sampo máme jenom u~nás v našem Suomi. Všecko, všecičko, co car v širém světě nashromáždil, přinesl s sebou do své země. Ale protože byl tak dlouho pryč, všichni lidé v jeho zemi zatím zestárli, všem jim narostly dlouhé brady. Car se rozhněval. ‚Přikazuji,‘ řekl svým Rusům, ,abyste všichni omládli, poněvadž mně je zapotřebí zdravých a silných mužů.‘ A~jeho prozíravost a moc byly takové, že jeho jediné slovo stačilo, aby všichni Rusové omládli: v mžiku byli zdraví, silní a s bradami vyholenými. Tehdy car Rusům nařídil, že mají kovat zbraně proti jeho nepříteli, švédskému králi. Tři dny pracují robotězi usilovně pro cara, na zádech prachu na sáh, mouru na hlavě na aršín, po celém těle hustou vrstvu sazí. Jenže se o~tom doslechla carova sestra. Přijde, podívá se a řekne: ‚Pěkný houf kovářů sis, bratře, přivedl s sebou z dalekého světa; poruč jim, ať mi ukovají carskou čelenku, aby se mi všichni klaněli jako carevně; a ještě jim poruč, ať vykovají ze stříbra měsíc a ze zlata slunce, jež by kroužily kolem mne a svítily mi dnem i nocí. Neučiníš-li mi, bratře, po vůli, stihne tě má kletba.‘ Rozhněval se car, když uslyšel ta slova. ‚Není cara kromě mne, prohlásil; ‚ano, mám carskou čelenku, ne však pro tebe; ano, měsíc a slunce krouží po nebi, ne však kvůli tobě.‘ Carova sestra se zachmuřila a vzteky bez sebe si začla hřebenem rozčesávat svoje černé vlasy; vlasy padaly na zem, a všude, kde třeba jediný vlas dopadl, vyrazilo ze země jedovaté býlí. Pak nehodná sestra rozlámala hřeben na kousky, a z každého zubu vystoupil obr s lukem a šípy. Doslechl se to švédský král i Turčín, odvěký nepřítel všeho křesťanstva. Spolčili se a vytáhli do boje proti moudrému kováři. Jakmile kovář uviděl, co se děje, udeřil kladivem na kovadlinu, a po prvním úderu se rozpadli všichni obři v prach; udeřil podruhé -- od kovadliny odlétly úlomky železa a pohřbily pod sebou Turčína; kovář udeřil do třetice -- sprška jisker zapálila síru a sanytr a ožehla švédského krále. Král se vrhl do moře, aby uhasil oheň; car se žene za ním -- přiběhne k moři, ale král je už za mořem; car ho chce pronásledovat -- rozhlíží se kolem, ale nikde žádná loď, kam oko dohlédne, všude jen mořský písek, holé kameny, slatina, močál. Car shromáždil své Rusy a praví k nim: ‚Postavte mi zde město, kde bych mohl žít, já zatím postavím koráb.‘ Začali tedy stavět město, ale každý kámen, který položili, močál hned pohltil; nakladli už spoustu kamenů, kvádr na kvádr, trám na trám, ale bažina všechno spolykala a na povrchu se dál prostírala jen samá mokřaď. Car zatím postavil koráb, ohlédl se a vidí: město pořád nikde. ‚Vy jste mi pracanti!‘ řekl svým lidem, a řka to, začal vlastníma rukama zvedat vzhůru kvádr za kvádrem a tam nahoře je opracovávat. Postavil tak celé veliké město, a když je postavil, spustil je pěkně pomaličku na zem. Švédský král zatím chodí po protějším břehu -- a láme si hlavu, co má car asi za lubem. Potká ho měsíc. Král se mu hluboce uklání: ‚Ach, měsíci, měsíčku, neviděl jsi, co dělá car Rusů?‘ Ale měsíc mu neodpovídá. Krále potká slunce; král se mu klaní: ‚Ach, slunce, sluníčko, nevidělo jsi, co dělá car Rusů?‘ Ale ani slunce mu neodpovídá. Král potká moře, klaní se moři: ‚Ach, moře, moříčko, nevidělo jsi, co dělá car Rusů?‘ Až moře mu odpovědělo: ‚Vím, co dělá, zemi vysouší, zatlačuje mne zpátky, začíná mi být těsno v mých březích, tak jako tobě, králi, v tvém království.‘ -- ‚Pojď, napadneme ho,‘ řekl král, ‚uvidíš, že se nám hned bude volněji dýchat.‘ Domluvili se a vytáhli spolu na cara. Král schystal síru a rozžhavené uhlí, moře se vzdulo, vylilo se z břehů a vystoupilo až po střechy nového města. Car zrovna odpočíval po dni plném namáhavé práce, probudil se a vidí: moře ho chce spláchnout! Ze vší síly udeřil žezlem do moře, a moře se vylekalo, stáhlo se zpátky do svých břehů a jen ustrašeně omývalo carovy nohy. ‚Nauč se nosit moje koráby!‘ vzkřikl car strašným hlasem a moře nastavilo carským lodicím svá mokrá záda. ‚Ztuhni!‘ řekl car, a moře se potáhlo stříbřitým ledem. ‚Duj, bouře, do mých plachet!‘ přikázal car a koráby se s větrem o~závod rozletěly po hladkém ledu. Král zatím vidí, že moře zamrzlo, rozhlíží se kolem a raduje se. Moře zvítězilo, říká si, a pohřbilo Rusy pod svým ledovým příkrovem. Dál napíná oči, vidí -- něco se černá na bílém sněhu, blíž\ldots stále blíž\ldots běda!\ldots ruské koráby se řítí rovnou na něj; marně je král zaklíná, marně je zasypává ohnivými střelami, od korábů vane silný vítr, hořící síru sráží v hrozivé mračno a nemilosrdně spaluje švédského krále i celou švédskou zemi. Král se vyděsil a utekl se k Turčínovi prosit o~pomoc. Ale jako pravý čaroděj dokáže být zároveň jak u~Turčína za mořem, tak tady na našem břehu. Ach, jak jen skončí tahle krvavá vojna? Kdo se zmocní naší země? Kdo se zmocní našeho divotvorného mlýnku Sampo?``

Stařec se odmlčel, stařena už dávno podřimovala, Elza jen zřídkakdy vyhlédla ze svého úkrytu za chrastím a zas se schovala. Pouze Jaakko upíral na starce planoucí zraky a zřejmě si netroufal ani špitnout.

Stařec si svých posluchačů nevšímal; zcela se soustředil na to, co říká; slova mu plynula z úst bez zábran jedno po druhém; svému vlastnímu vyprávění naslouchal s opravdovým zájmem a zřejmě se mu nechtělo přestat.

,,Kukari mi říkal,`` pokračoval po nedlouhém odmlčení, ,,že se mu od jisté doby zdávají podivné sny: vídá prý, jak se z našich suomských břehů vynořují ohromné žulové kvádry, přeplouvají moře a skládají se pod nohy carovi Rusů, jak ten po nich stoupá stále výš a výš, jak na takovou vysokou hromadu žulových kvádrů vybíhají naši krajané, lidé ze Suomi, a ruský car je všechny přikrývá svou ohromnou dlaní. Jindy se mu zas zdá, že na mořském břehu pukají s rachotem skály a z nich vystupuje obrovské výstavné město; do města se sjíždějí černokněžníci ze všech světových stran a na hlučných shromážděních rokují se všemi lidmi ze Suomi o~mnoha tajemných věcech mezi nebem a zemí. A~vysoko nad městem se opět vznáší car Rusů se zlatou korunou na hlavě; pluje na oblacích, z jeho koruny padají na naše Suomi zlatisté jiskry a září jako tisícero sluncí. Divné! Neobyčejně podivné!``

\podpis{přeložil Jiří Honzík}