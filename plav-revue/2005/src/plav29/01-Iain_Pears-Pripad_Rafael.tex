\medskip 
\section{Iain Pears: Případ Raffael}

\noindent
\textbf{Kapitola 1}

\medskip

\noindent
\looseness+1 Generál Taddeo Bottando vystupoval po schodišti obloženém odcizenými uměleckými díly jako každý den krátce předtím, než zvon na Svatém Ignáci ohlásil sedmou hodinu ranní. Objevil se dole na náměstí už hezkou chvíli předtím, ale zapadl podle svého zvyku na deset minut do baru naproti své kanceláři, vypil tam dvě espressa a snědl panino s čerstvou šunkou. Místní štamgasti ho pozdravili, jak se slušelo a patřilo při setkání s jiným pravidelným hostem, přátelským ,,buon giorno``, pokynuli mu, ale žádné řeči s ním nezaváděli. Vstávání je v Římě stejně jako kdekoli jinde na světě osobní záležitost a nejlépe se odbývá v soukromí a nerušeně.

\looseness+1 Když absolvoval tenhle příjemný ranní rituál, přešel po kočičích hlavách náměstí a jal se stoupat do schodů, přičemž lapal po dechu a ztěžka oddychoval, ještě než zdolal první patro. Není to tím, že bych tloustl, ujišťoval se často. Uniformu jsem si nedal povolit už kdovíkolik let. Statný jsem, to snad ano. Působím důstojně, lépe řečeno. Měl bych přestat kouřit a omezit kávu a jídlo a začít sportovat. Jenomže život by mě pak zdaleka tolik netěšil. Krom toho mi táhne na šedesát a v tom věku je už pozdě ladit formu. Kdoví, jestli by mě takové úsilí nezabilo.

\looseness+1 Zůstal stát, jednak aby si prohlédl nový obrázek na stěně, ale spíš proto, aby nenápadně chytil dech. Vypadá to na kresbu od Gentileschiho. Kouzelná věcička. Škoda že se musí vrátit právoplatným majitelům, až se kolem toho vyřídí všechno papírování, viník bude obžalován a materiály předány kanceláři státního žalobce. Navzdory všemu mělo postavení šéfa italského celostátního odboru pro vyšetřování krádeží uměleckých památek své světlé stránky. Postrádaná díla se sice podařilo najít málokdy, ale pak obyčejně stála za to.

\looseness+1 ,,Pěkné, viďte?`` ozvalo se za ním, jak tam stál zahleděn do obrazu. Potlačil poslední projevy dýchavičnosti a otočil se. Flavia di Stefano patřila k oněm podivuhodným ženským bytostem, jaké se podle Bottandova názoru rodí jen v Itálii. Stanou se z nich pak buď manželky a matky, anebo se začnou věnovat práci. Pokud se rozhodnou pro práci, vrhají se do ní s nevídaným elánem, aby potlačily výčitky svědomí, že nezůstaly doma, a bývají dvakrát lepší než kdo jiný. Proto měl Bottando ve své desetičlenné pátrací skupině osm žen. Tím si jeho oddělení, jak dobře věděl, vysloužilo u~ostatních policejních jednotek posměšnou přezdívku. Ovšem Bottandův harém, jak kolegové očividně ze závisti nazývali jeho tým, vykazoval slušné výsledky -- na rozdíl od jiných, které by mohl jmenovat.

Pozdravil Flavii, mladou dívku, anebo spíše už ženu, dobrotivým ranním úsměvem; uvědomoval si, že sám dospívá do věku, kdy se mu každá žena mladší třiceti let jeví jako dívka. A~tuhle si zvlášť oblíbil, přestože zřejmě zcela postrádala schopnost projevovat mu úctu, na jakou měl vzhledem ke své hodnosti, věku a moudrosti nárok. Zatímco přátelé se občas taktně zmínili o~jeho rozložitosti, Flavia mu s láskyplnou upřímností a bez váhání říkala, že je kulaťoučký jak bochánek. Až na tohle však představovala téměř ideální subalterní kolegyni. 

Flavia, která zásadně chodila ve svetrech a džínách, aby bylo jasné, že nepatří do kategorie uniformovaných policistek ani ambiciózních manažerek, jeho pozdrav opětovala také úsměvem. Šel jí od srdce. Za posledních pár let ji generál hodně naučil, především tím, že ji nechal chybovat a pak to za ni dával do pořádku. Nepatřil k šéfům, kteří hledí na podřízené jako na příhodné obětní beránky, kdykoli dojde k maléru. Zakládal si spíše na tom, že své svěřence učí pracovat svědomitě, a dopřával jim značnou, ač vždy neoficiální nezávislost. Flavia na tuto metodu zareagovala s větším elánem než ostatní a ovládala už plnohodnotně detektivní řemeslo, i když jí příslušná úřední hodnost zatím chyběla.

,,Volali carabinieri z Campo dei Fiori, že nám někoho předají,`` oznámila mu. ,,Sebrali ho včera večer, když se vloupal do kostela v jejich rajónu, a má k tomu prý nějaké podivné vysvětlení. Usoudili, že se to týká spíš nás.``

Mluvila s drsným nosovým přízvukem severozápadní Itálie. Bottando si ji vybral přímo na univerzitě v Turínu a ona kvůli místu v Římě přerušila postgraduální studia. Opakovaně prohlašovala, že se časem na univerzitu vrátí, a uváděla to jako hlavní důvod, proč u~policie nenastoupí natrvalo. Odváděla však ve svém oddělení tak poctivou práci, že se to nejevilo jako příliš pravděpodobné. Jako mnozí Italové ze severu měla plavé vlasy a světlou pleť. I~kdyby nebyla pouze pěkná, ale přímo vyslovená krasavice, už pro své vlasy by poutala v Římě pozornost.

,,Říkali, oč jde?``

,,Ne. Týká se to snad nějakého obrazu. Možná je ten chlap cvok.``

,,Jak se s ním domlouvali?``

,,Anglicky a trochu italsky. Nakolik nevím.``

,,V tom případě s ním promluvíš ty. Víš, jak jsem na tom s angličtinou. Dej mi vědět, jestli vypoví něco zajímavého.``

Flavia posměšně zasalutovala tím, že krátce přitiskla dva prsty levé ruky k pečlivě načepýřené ofině, spadající do půli čela. Poté se oba rozešli do svých kanceláří, Flavia do přecpané místnůstky, kterou sdílela se třemi kolegy, a Bottando do luxusnějších prostor ve třetím patře, vyzdobených téměř výhradně dalšími kradenými uměleckými díly.

Zasedl za psací stůl a věnoval se obvyklému rannímu rituálu s došlou poštou, kterou mu sekretářka zanechala pečlivě urovnanou u~ruky. Běžná makulatura. Potřásl smutně hlavou, povzdychl si a celou hromádku hodil do koše.

Po dvou dnech se mu na psacím stole objevil objemný dokument. Byl to plod Flaviina rozhovoru se zadrženým, jehož jí přivedli carabinieri, a svědčil výmluvně o~její důkladnosti. Nahoře ležel lístek: ,,Myslím, že si počtete -- F.`` Správně měl výslech provádět pověřený policista, Flavia ale přešla rovnou do angličtiny a převzala iniciativu. Jak Bottando listoval stránkami, pochopil, že tenhle cizinec mluví italsky docela dobře. Službu konající policista však nebyl zvlášť bystrý a všechno zajímavé by nejspíš pominul.

Dokument obsahoval zhuštěný přepis výslechu: v takovém zpracování se podobné materiály předávají kanceláři státního zástupce, pokud policie usoudí, že by případ měl přijít k soudu. Bottando si došel pro espresso z automatu na chodbě -- vypěstoval si závislost na kávě už před mnoha lety a teď by ani večer neusnul, kdyby si na poslední chvíli nedopřál dávku oblíbené drogy --, pohodlně se uvelebil a začal číst.

\noindent
\looseness+1 Na prvních pár stránkách ho nic zvlášť nezaujalo. Zadržený byl Angličan, osmadvacetiletý, postgraduální student. Přijel do Říma na prázdniny a zadrželi ho pro potulku, když se zřejmě pokoušel přespat v kostele svaté Barbary poblíž Campo dei Fiori. Nic nebylo odcizeno, místní farář nenahlásil žádnou škodu.

\looseness+1 Tyto informace byly rozloženy do pěti stránek a Bottando se podivil, proč se carabinieri obracejí na jeho oddělení a proč toho Angličana vůbec zadrželi. Přespávat kde se dá, není vážný přestupek. V letních měsících se vyspávající cizinci najdou pomalu na každé lavičce a ve všech neohrazených prostorách po městě. Někdy nemají peníze, někdy opilí nebo zdrogovaní zabloudí a netrefí zpátky do svého penzionu, nebo nemají hotely široko daleko volný pokoj a nic jiného jim ani nezbývá.

\looseness+1 Nad další stránkou už Bottando zpozorněl. Zadržený, jistý Jonathan Argyll, tvrdil vyšetřovatelům, že se nešel do kostela utábořit, ale chtěl prozkoumat Raffaela nad oltářem. Navíc trval na podrobné výpovědi, protože došlo k neslýchanému podvodu.

\looseness+1 Bottando se zarazil. Jaký Raffael? Ten chlap se zbláznil. Nemohl se sice na tenhle kostel hned upamatovat, byl ale přesvědčen, že má spolehlivý přehled o~všech Raffaelových dílech v Itálii. Kdyby takový obraz visel v zastrčeném kostelíku, jako byla Svatá Barbara, věděl by o~něm.

\looseness+1 Přešel k počítači a zapnul jej. Otevřel databázi udávající pravděpodobné cíle zlodějů. Vyťukal ,,Roma``, a když byl vyzván k upřesnění, dodal ,,chiesi``. Poté zadal jméno kostela. Přístroj ho okamžitě informoval, že Svatá Barbara vlastní pouze šest předmětů, jimž může hrozit odcizení. Tři z toho byly ze stříbra, pak bible Vulgata ze sedmnáctého století v reliéfní kožené vazbě a dva obrazy. Raffael mezi nimi nebyl, ani takový, který by se za něj dal mylně považovat. V obou případech šlo o~díla vysloveně druhořadá, s kterými by žádný zloděj dbalý své pověsti neztrácel čas. Trh nevykazuje markantní zájem o~ukřižování devětkrát šest stop od anonymních římských malířů. Usoudil rovněž, že by se na mezinárodním ilegálním obchodu s uměleckými předměty sotva uplatnil oltářní obraz ze Svaté Barbary -- krajina s odpočívajícími na útěku z Egypta od proslule průměrného malíře z osmnáctého století Carla Mantiniho.

\looseness+1 Bottando se vrátil za psací stůl a přelétl očima dalších pár řádek v domnění, že tím ,,počtením`` Flavia měla na mysli charakter výpovědi jako dalšího dokladu lidské pošetilosti. Měla na tuto stránku lidské povahy názor velmi vyhraněný, zvlášť co se sběratelů umění týkalo. Už kolikrát zastavilo oddělení pátrání po díle zanedbatelné hodnoty, když zjistili, že je koupil -- jako Michelangela, Tiziana, Caravaggia nebo podobně -- zámožný cizí sběratel, vybavený spíše penězi než rozumem. Vyřídili si to s ním písemnou zprávou, že se stal obětí podvodu, a informovali místní policii. Soudili, že taková trapná lekce bude dotyčnému dostatečným trestem, a dílo samo nebylo natolik významné, aby stálo za nesnáze a výdaje spojené s mezinárodním zatykačem a příkazem k deportaci.

\looseness+1 Že by tenhle padesátistránkový dokument nebyl nic než záznam fantazírování vyšinutého prosťáčka, který uvěřil, že rychle zbohatne? Pár dalších odstavců však Bottanda stačilo přesvědčit, že vysvětlení nebude tak prosté. Od zápisu otázek a odpovědí přecházel text do souvislého prohlášení. Bottando četl a nepřestával se divit.

\podpis{přeložila Eva Kondrysová}