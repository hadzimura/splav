\pagebreak

\section{Tankred Dorst: Pěkné místo}

\noindent
\textit{Úryvek z~delší prózy.}

\medskip

,,Hallo Lízo!`` V~tlačenici, jak do něj někdo šťouchne, se tác s~naplněnými sklenkami rozkymácí.

,,Pozor! Ach to jsi ty, Albrechte! To jsem si mohla myslet, že mě vyvedeš z~rovnováhy.``

Rozzářeně se na něho usměje. Ale mladý muž s~nahou hrudí pod plátěným sakem nedá v~tváři nic najevo. No, tak mladý vlastně není.

,,Dej mi loka toho mizernýho vína.``

,,Poslyš, ty!``

Hned nato do sebe hodí sklenku a rychle ji zase vrátí.

,,No, pořád ještě lepší než ty krámy na stěnách.``

Teď se Líza rozhorlí.

,,Ty měly právě fantastický úspěch v~New Yorku!``

,,Nepapouškuj hovadinky, které tu slyšíš!`` Přiblíží se těsně k~jejímu obličeji. Opatrně zatáhne zuby za jeden pramen vlasů, který se vlní po její vysoké postavě. Jen protože chceš být všude u~toho a smíš roznášet sklenice, řekne zle.

,,Ty jsi tu taky!``

,,Bohužel! Však je to chyba!``

Ale samozřejmě nemá úmysl odejít, i on chce být u~toho! Obrazy na stěnách si už tak jako tak nikdo neprohlíží. Na začátku se ozývalo hraně překvapené, příležitostně i zamyšlené ,,Oh!`` a rychlý znalecký pohled klouzal z~obrazu na obraz. Pak se přešlo k~obvyklému klábosení. Objímání, vzájemné pozdravy, pronikavé smíchy.

,,Jsi šťastná, Lízo?``

Pitomé Albertovo dotazování! Jak může někdo položit takovou otázku, a k~tomu se tak tvářit! Jako by musel odpověď zaprotokolovat a jak by od ní něco záviselo! Žádný div, že ho má každý za únavného. Přitom je jen nešikovný a chce se svým dotazováním učinit zajímavým. Znám to přece! Musím mu už jednou vysvětlit, že jsou lidi jeho vyslýcháním vynervovaní, a proto mu jdou jak jen možno z~cesty.

,,Jsi tady šťastná?``

Samozřejmě je jí dobře, když tu může být a naslouchat všem těm hlasům, spleti hlasů, samé hlasy, které zná, těší ji být mezi lidmi, kteří se hladce dovedou pohybovat ve světě:

Rozparáděná dáma přikormidluje k~Mallmöhovi: ,,Tytyty, už jsem tě objevila, ty zlý, ty!`` A~líbá ho na obě uši, on se pokouší zachránit svou sklenku.

Že si ten doktor Mallmöh, prominentní státní zástupce, dá takové útoky líbit! Jestlipak by si to dal líbit i ode mne? Ach, jsou to všichni originální lidé. Žádní měšťáci! Jak komicky balancuje se skleničkou v~natažené ruce vysoko nad hlavami, nejraději bych mu ji vzala z~ruky, aby se mu víno nevylilo na sako.

,,Hallo Lízo!`` V~jeho kanceláři to určitě nechodí tak uvolněně, tam ho hlídá sekretářka. Tady je prostě taková zvláštní atmosféra. Musí se natáhnout, podaří se jí na špičkách chňapnout po sklence, ráda by přitom řekla něco vtipného, ale nic ji v~ten okamžik nenapadá, aspoň tedy na něho prostě nadšeně zasvítí pootevřenými, na tmavočerveno nalíčenými rty a vycení zuby, na které je tak pyšná od té doby, co jí Fred jednou řekl, že má zuby jako dravec\ldots nádherný dravčí chrup! Jen ho bohužel neskousneš. Jak to myslel? Já nejsem zkrátka agresivní typ, je mi líto.

,,Lízo! Lízo!``

Pospíší si, musí ještě roznést jednohubky.


,,Jak se to vyslovuje: Aleijadinho?``

,,Tak, jak se to píše.``

,,Ale-jadin-ho``, zkouší Anna opatrně.

,,Aleijadinho``, opraví ji Bonsack s~brazilskou melodičností, s~pečlivým přízvukem.

,,Znamená to Mrzáček``, dodává Anna spěšně. Adam napsal něco úchvatného o~tomto umělci. O~jeho tragickém životě.

,,Ne o~jeho životě, pokusil jsem se napsat něco o~jeho umění.``

Albrecht se vmísí do rozhovoru.

,,Dá se to oddělit?``

,,Dočista vzrušující text``, řekne Anna tiše se sklopenou hlavou, a v~jejím jemném tónu zaznívá zřetelná výtka vůči dotěravci.

Ale ten se nedá umlčet.

,,Zda je text vzrušující, to je třeba nejdřív přezkoumat.``

Bonsack se hrdě usmívá, brání se hranou skromností.

,,Je to jen malý essey, nic pro širší obecenstvo. Anna zase jednou přehání.`` A~protože Anna oddaně mlčí, dodá Bonsack: ,,Z lásky.``

Zní to posměšně.

\medskip

\noindent
ALEIJADINHO. Když mu bylo třicet, právé když začal dělat na svém velkém díle, zjistil, že onemocněl leprou. Uviděl bílé skvrny na kůži. Nedávaje se tím mást, pracoval dál na kamenných sochách, až jeho prsty ztratily cit. Když mu obě ruce uhnily, dal si nástroje připoutat na pahýly paží, dláto nalevo, kladivo napravo. Tak mohl pracovat dál. Když už ho nohy neunesly a posléze uhnily nohy i s~chodidly, zaplatil si černého otroka, posadil se mu na ramena a pracoval dál. V~té dobé vznikla kalvárie se spoustou soch. Byl teď nejslavnějším brazilským sochařem, přicházelo čím dál víc lidí, kteří obdivovali jeho umění a chtěli vidět, jak vytesává z~kamenných bloků své nádherné postavy: svaté i zločince, trpitele, vojáky, anděly i zvířata. Pro jeho zmrzačení ho teď jmenovali Aleijadinho: Mrzáček. Když ale lepra zpustošila i jeho obličej, uhnil mu nos, uši, divákům to bylo odporné, protože jim to připomínalo jejich vlastní pomíjivost. Jeho objednavatel chtěl, aby práci zastavil. Aleijadinho si však dal ušít kožený vak, v~něm dva otvory pro oči, ten si přetáhl přes obličej. Konečně byla práce hotova, kalvárie dokončena. Její tvůrce ale byl už jen hromádkou bláta a smradu. Zbožní lidé a jeho obdivovatelé odsunuli ten balík, který vzbuzoval hnus, na okraj schodiště. Jeho hlava však hlásala boží chválu, neboť Bůh mu dopřál, aby dokončil své dílo.

\podpis{přeložil Vladímír Tomeš}