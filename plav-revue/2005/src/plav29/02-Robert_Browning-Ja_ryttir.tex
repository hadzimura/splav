\section{Robert Browning: ,,Já, rytíř \\ Roland, přišel k~Temné věži!``}

\noindent
% \textit{Báseň vznikla téměř přesně před sto padesáti lety jako pozdně romantický nepřímý ohlas středověké rytířské epiky (Píseň o~Rolandovi -- ta ovšem žádnou Temnou věž nezmiňuje, či legend o~hledání svatého Grálu), kterým ovšem prolínají pocity hlubokého zmaru, podnícené zjitřenou atmosférou poloviny devatenáctého století a počátků zatuchlé doby viktoriánské. Báseň je literárními historiky většinou charakterizována jako narativní monolog, avšak jde spíše o~jakousi samomluvu člověka, který marně hledá smysl svého života a za každou cenu touží naplnit svůj osud. Sám básník ji vytvořil v~průběhu jediného dne a charakterizoval ji jako záznam jakési nutkavé snové vize (,,Když jsem tu báseň psal, Bůh a Browning věděli, co znamená. Teď už to ví jenom Bůh.``), vyjádřený dlouhou řadou barvitých obrazů, popisů fantaskní přírody a nejasných symbolů, které někdy vyznívají až apokalypticky. Záhadnost geneze a povahy této skladby podtrhuje i to, že její titul je přímou citací ze Shakespearova Krále Leara (věty z~Edgarova předstíraně pomateného blábolu ze závěru 4. scény III. jednání -- angl. ,,Child Rowland to the dark tower came`` -- Sládek: ,,Rek Rowland vešel v~tmavou věž``).
% V~poslední třetině básně Roland, hledající sám sebe a naplnění svého poslání, peripetiemi skličujících zážitků cesty nakonec dosáhne cíle -- dojde k~Temné věži, symbolu svého osudu, a postaví se jí. Čtenáři není jasné, jestli toto ukončení hrdinova putování znamená jeho smrt (možná i sebevraždu, v~souvislosti s~líčením jeho pocitů v~předcházejících pasážích) nebo vstup do nového života, osvobozeného od všech zničujících pochyb -- Roland jako by konečně znovu našel své rytířské druhy, které v~životě z~různých důvodů ztratil. Apel (,,poslání", ,,poselství") této básně asi spočívá -- primitivně řečeno -- v~podpoře teze, že ,,(vytrvalá, nezlomná, byť třeba sebezničující) trpělivost vskutku (jinak nedosažitelné) růže přináší`` a rozhodující měrou řeší (tak či onak, pozitivně, či negativně) existenciální problémy životem souženého jedince a osvobozuje ho od jeho trápení.}

%\medskip

\noindent
\textit{(pozn. red.: přetiskujeme prvních 10 strof)}

\medskip


%(,,Childe Roland to the DarkTower Came``)

%[Ze sbírky Men and Women -Muži a ženy (1855)]

\begin{verse}

I

Já věděl hned, že každým slovem klame,

ten mrzák chraplavý, v~očích zlý jas,

a lačně prahne poplést mě a zmást.

Rty zaťaté v~té zášti dobře známé,

jež bez milosti všechno živé láme --

že novou oběť zkosit chce jak klas.


\medskip

II

Co jiného by rád, ten s~křivou holí,

než zhoubnou Iží svou past mi nastražit,

jíž poutníkům všem život chtěl by vzít,

kdož se s~ním střetnou cestou mezi poli?

Já tušil, epitaf že v~duchu volí,

jímž stvrdit chce můj mrtvý klid.

\medskip

III

Pokud se pustím směrem, kterým kyne,

do rokle vkročím, o~níž pověst praví,

že nikdo nevyjde z~ní zpátky zdravý --

ukrývá Temnou věž, jež hrůzou slyne.

Však přesto vstoupím -- neznám cesty jiné:

nechť moje trápení tam osud zdáví.

\medskip

IV

Po všech svých toulkách celým světem,

po dlouhých rocích, kdy jsem hledal štěstí,

naděje zhasly -- dál už nechci vésti

prohraný boj a unaveným letem

dál mávat křídly za úspěchu květem.

Srdce mé nemá sil obstát v~něm se ctí.

\medskip

V

Už těžce nemocný se mrtvým může zdát

začasté dřív, než dostihne ho konec.

Žal přátel jako umíráčku zvonec

v~uších mu zní, byť do sebe by rád

vzduch čerstvý vsál. (Však ví: ,,Ten krutý honec

mi chystá to, co nelze odestát.``)

\medskip

VI

Když dobří lidé poblíž cizích rovů

nahlas si přejí, aby jejich drahý

ve vhodný den se v~míru vzdal vší snahy,

s~poctami nesen byl pak ke hřbitovu,

on sám si často přeje zas a znovu

nezklamat je a skončit s~žitím záhy.

\medskip

VII

I~já už takto strádám dobu drahnou

zklamaný, že můj život míjí cíl.

Tak k~rytířům mne osud přiřadil,

co po své spáse dlouho marně prahnou,

než paží svou si na Temnou věž sáhnou.

I~mne to čeká. Snad mám dosti sil.

\medskip

VIII

S~klidem, jenž spíše zove zoufání,

odbočím stezkou, jak ten netvor hadí

na sklonku pošmourného dne mi radí.

V~přízračně světlešerém smrákání

rovina širá strach mi nahání --

jen holá pláň můj pohled marně svádí.

\medskip

IX

A~hle! Když kroků pár jsem urazil

tou novou cestou, otočím se zpět,

naposled touže zhlédnout známý svět --

však nezřím nic. Jen prázdný obzor zbyl

a zrak můj nyní víc už nespatřil

než pustou planinu. Víc nevidět.

\medskip

X

A~tak jdu dál. Snad nikdy v~žití

v~krajinu bědnou tak mě nezaved můj krok --

vždyť nikde ani květ,

jen pýr a pryšec vítězně se cítí

v~té pustině, co všude vůkol svítí,

a bodláky by nikdo nespočet.

\end{verse}

\podpis{přeložil Miroslav Jindra}