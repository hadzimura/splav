\section{Su Tung-pchoe: \\ Fu o~Červeném útesu}

\noindent
Sedmého měsíce na podzim roku Žen-su, podle našeho kalendáře roku 1082, v čase, kdy začala luna právě ubývat, jsem se já, pan Su, vydal s přáteli na projížďku loďkou k Červenému útesu. Vanul svěží větřík, hladina Dlouhé řeky však byla klidná. Pozvedl jsem číši k přípitku, zarecitoval báseň o~jasné luně a zanotoval píseň o~půvabné krásce. Po chvíli se nad vrcholky kopců vynořil měsíc a vydal se na svou pouť mezi hvězdami. Nad řekou se vznášel bílý opar a ozářená hladina splývala s oblohou. Loďka klouzala po proudu v nekonečném prostoru. Cítili jsme se tak volní, jako by nás vítr unášel vzhůru do neznáma a my se stali, zanechavše svět za sebou, okřídlenými nesmrtelnými. Rozjařeni vínem dali jsme se do zpěvu a tloukli jsme při tom do rytmu o~okraje lodi. Zpívali jsme tuto píseň:

\smallskip

\begin{verse}

Naše kormidlo a vesla 

vonící po kasijích a orchidejích

se noří do hloubky pod vodou

a prodírají se paprsky měsíční záře.

Mé srdce však teskní po krásce,

dlící pod jinou oblohou.

\end{verse}

\smallskip

Jeden z našich druhů se nás jal doprovázet na flétnu. Zvuky, které jeho nástroj vyluzoval, se podobaly vzlykům plným touhy a smutku, jako by žalovaly a naříkaly, linouce se prostorem jako nekonečná nit. Byly tak jímavé, že by byly probudily i draka dřímajícího v temné jeskyni a vyloudily slzy v očích ovdovělé ženy, plavící se na osamělé bárce.

Smutek mě přiměl k tomu, abych si upravil šat, zpříma se posadil a zeptal se flétnisty, proč hraje právě tuto melodii.

A~on odpověděl: ,,Příteli, zarecituji ti báseň, kterou znáš``:

\smallskip

\begin{verse}

Měsíc jasně svítí,

na obloze hvězd pár.

Havrani a straky

odlétají na jih v dál.

\end{verse}

\smallskip

Napsal ji Cchao Cchao, zvaný též Meng-te, slavný vojevůdce dávných dob, když se tudy plavil poté, co dobyl mnohá vítězství s loďstvem tak početným, že zabíralo prostor jednoho tisíce li a jeho korouhve zastiňovaly oblohu. A~přece ho tu Čou Lang porazil\ldots Ach ano, Cchao Cchao býval kdysi hrdinou. Kde však je mu dnes konec? A~což teprve my! Jsme jen rybáři a drvoštěpové, společníci ryb a raků, přátelé jelenů a laní. Plavíme se po křehké, listu podobné bárce, připíjíme si tykvicemi naplněnými vínem. Je mi teskno z toho, že život trvá jen zlomek vteřiny. Závidím této řece, jejíž tok je bez konce. Jak rád bych se připojil k nesmrtelným a toulal se s nimi široko daleko, jak rád bych chtěl věčně objímat měsíc! Vím však, že to není možné a tak mi nezbývá, než svěřit svou melodii tesknému větru.``

,,Milý příteli,`` odvětil jsem, ,,pohleď zde na tu vodu a měsíc. Vidíš přece, že řeka neustále  plyne a přesto zůstává. Měsíc přibývá a ubývá a přesto se nemění. Díváme-li se na to z hlediska pomíjivosti, pak vesmír netrvá déle než mžiknutí oka. Chápeme-li však veškeré dění z hlediska stálosti, jsme my dva a vše, co nás obklopuje, věční. Nač tedy ten smutek.  Rozhlédneme-li se kolem, vidíme, že všechno má svého majitele. Nemůžeme si proto z toho, co nám nepatří, přivlastnit zhola nic. Nikdo nám však nemůže vzít svěží vánek nad řekou, stříbřitý měsíc nad horami, ani nic z toho, co naše ucho vnímá jako hudbu a zrak jako barvu. Všechna tato krása nám patří a můžeme se z ní těšit nekonečně dlouho. Je dílem tvůrčí síly, jež stála u~zrodu věci, a je proto věčná a neopakovatelná. Radujeme se z ní tedy společně.``

Přítel se zasmál, vypláchl číšku a naplnil ji vínem. Když bylo víno vypito, jídlo snědeno a kolem zbyly jen prázdné číše a misky, položili jsme se do loďky, opírajíce si hlavy vzájemně o~ramena. A~ani jsme nezpozorovali, že na východě začalo svítat.``

\podpis{přeložila Věna Hrdličková}

\section{Su Tung-pchoe: \\ Pozdější fu o~Červeném útesu}

\noindent
\looseness+1 ,,Téhož roku patnáctého dne desátého měsíce jsem se vracel ze Sněžné síně na Východním svahu domů do Lin – kao. Doprovázeli mě dva přátelé. Šli jsme cestou kolem  Kopce žluté hlíny, který byl pokryt mrazivým jíním. Listí stromů bylo opadané. Na zemi se odrážely naše stíny a na nebi svítil jasný měsíc. Z potěšení nad tím, co jsme kolem sebe viděli, jsme si v chůzi jeden druhému prozpěvovali.
Po chvíli jsem se však neubránil povzdechu: ,,Mám přátele, ale nemám víno.
A~i kdybych mělo víno, nemám k němu nic k jídlu. Měsíc jasně svítí a vane svěží větřík. Promarníme snad takový večer?``

\looseness+1 Jeden z mých přátel řekl : ,,Dnes v podvečer při západu slunce jsem rozhodil sítě a chytil rybu s velkou tlamou a drobnými šupinami. Podobá se proslulému okounu z řeky Sung. Kde však k němu seženeme víno?``

\looseness+1 Rozhodli jsme se zajít všichni ke mně  domů, kde jsem se svěřil  své ženě.

\looseness+1 ,,Mám žejdlík vína,`` řekla. ,,Schovala jsem ho před časem s tím, že by se mohl někdy hodit, jako třeba právě dnes.``

\looseness+1 Vzali jsme víno a rybu a šli jsme  k Červenému útesu. Proud řeky hřměl, strmé skály se tyčily  z břehů tisíc stop. Útes byl vysoký, měsíc nepatrný.

\looseness+1 Hladina řeky opadla a vyčnívaly z ní kameny. Ačkoliv od poslední mé návštěvy uplynul jen krátký čas, krajina se změnila k nepoznání!

\looseness+1 Z náhlého impulsu jsem nadzvedl suknici a začal jsem šplhat po skaliscích. Prodíral jsem se podrostem, lezl po kamenech připomínajících tygry a leopardy, klopýtal  přes kořeny stromů, jejichž kůra se podobala šupinatým drakům, až jsem se dostal tam, kde měl své  hrůzostrašné hnízd jestřáb a shlížel jsem hluboko dolů na  chrám boha řek. Mí dva přátelé mě v mém výstupu nenásledovali. Náhle jsem po vzoru taoistů vypustil z hrdla dlouhý hvizd. Traviny a stromy se zachvěly, hory se rozezvučely, údolí odpovědělo ozvěnou. Zvedl se vítr a vody se rozbouřily. Zachvátil mě smutek a úzkost. Byl jsem ztuhlý chladem a věděl jsem, že tady nemohu dál zůstat.

\looseness+1 Vrátil jsem se do lodě. Odveslovali jsme s ní doprostřed řeky, kde jsme ji nechali plout volně po proudu  a spočinuli tam, kde spočinula ona. Přiblížila se půlnoc. Obklopovalo nás hluboké ticho, když tu se náhle  objevil osamělý jestřáb, který letěl přes řeku směrem na východ. Křídla měl obrovitá jako kolo u~vozu, oděn byl v~černou péřovou suknici a bílý kabátec. Vydal dlouhý pronikavý hvizd, krátce se snesl  nízko nad naši loď a zamířil k západu.

\looseness+1 Po chvíli se mí hosté rozloučili a já jsem šel spát. Zdálo se mi o~taoistovi v péřovém šatu, jak jde kolem mého domu v Lin-kao. Uklonil se mi a pravil:

\looseness+1 ,,Líbila  se vám vyjížďka k Červenému útesu?`` Zeptal jsem se ho na jméno a příjmení, ale on jen sklonil hlavu a neodpověděl.

\looseness+1 ,,Ach ano,`` zvolal jsem. ,,Už  vím! To jste byl vy, kdo přeletěl s hvízdotem nad mou lodí.
Taoista se mlčky usmál. V tom okamžiku jsem se probudil. Otevřel jsem dveře, ale nikoho jsem neviděl.``

\podpis{přeložila Věna Hrdličková}