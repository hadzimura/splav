\section{Vladimir Fjodorovič Odojevskij: Salamandr}

\noindent
\textit{Úryvek pochází z~druhé části jménem ,,Elza``.}

\medskip

\noindent
Bylo už jedenáct hodin večer. V domě všichni ulehli ke spánku, ruch na ulicích také ustal; jen z požárních věží se občas ozvalo táhlé volání hlídek a doznívalo v dálce. Svíčky pomalu dohořívaly a chvějivé stíny se kladly na karnýzy, zdobené knížecími erby; všude zavládlo ticho.

Noviny mne okamžitě zaujaly; nad jejich stránkami jsem zapomněl na všechno kolem. Plně jsem se soustředil na onen veskrze pozitivní evropský svět s jeho činorodostí, průmyslem, náruživostmi a parními stroji, který z nich promlouval. Obzvlášť pozoruhodný se mi zdál článek o~železných drahách a má duše se bezděky zachvěla hrdostí, jakmile jsem pomyslil na gigantické průmyslové podniky naší doby. Zkrátka jsem se všecek ponořil do četby, když tu\ldots že by?\ldots ne, to není klam\ldots když tu se právě v strašidelném sále ozvalo, a to velice zřetelně, zasténání. Nikdy nezapomenu na tu chvíli; dodnes mi ty zvuky znějí v uších. Ten nářek se nepodobal ani lidskému hlasu ani zvířecímu skřeku, ale tajilo se v něm cosi nevýslovně smutného, pronikal až do nejskrytějšího koutu duše, nedalo se mu naslouchat bez zvláštního vzrušení. Měl jsem pocit, že ten zvuk rezonuje v samé hloubi mého srdce\ldots V tu chvíli odbilo dvanáct. Údery hodin mě probudily z ochromení a já se vrhl ke dveřím sálu – v sále bylo ticho. Svíčky, jež jsem tam rozestavěl po stolech, klidně hořely; všechny dveře byly zavřeny a v sále nebyl nikdo. Znovu jsem obešel celý sál, nahlédl i do sousedních místností – všude ticho a klid. Chtě nechtě znepokojen jsem se vrátil do kamrlíku ke strýci; seděl klidně na židli, pozorně se probíral svou knihou a dělal si do ní nějaké poznámky.

,,Slyšel jsi to?`` zeptal se strýc.

,,Slyšel.``

,,Nevíš, co to bylo?``

,,Ani v nejmenším.``

,,Možná, že to jen někde vrzly dveře,`` pokračoval strýc svým ironickým tónem.

Mlčel jsem. Strýc se mne otázal. ,,Chceš tu ještě zůstat?``

,,Třebas do rána. Ale neměli bychom raději čekat přímo v sále?``

,,Nejsem si jist, jestli by to nebylo na závadu našemu pokusu. Víš co? zkusíme to ještě jednou. A~uděláme to takhle: já půjdu do místnosti, odkud se vchází do sálu z protější strany; ty zůstaneš zde; oba budeme stát hned za dveřmi a jakmile se ozve zaúpění, okamžitě vstoupíme do sálu.``

Souhlasil jsem, i když se mne, přiznávám, zmocňoval dětinský strach, a pomyšlení, že zůstanu v kamrlíku sám, mi nahánělo hrůzu. Srdce mi silně bušilo a úpěnlivé stony mi nepřestávaly znít v uších.

Ve snaze uklidnit se zvažoval jsem v duchu všechny akustické jevy, jež by mohly být příčinou oněch zvuků. Přitom jsem se jednou rukou chopil svíčky, druhou položil na kliku dveří, abych v případě potřeby byl schopen okamžitě jednat. Nevím,jak dlouho jsem setrval v této pozici. Všude kolem bylo ticho. Zdálo se mi, že slyším svůj vlastní tep. Vtom, právě ve chvíli, kdy už jsem chtěl ode dveří odejít, se přímo u~nich, rovnou pod mým uchem znovu ozvalo zakvílení. Ale toto zakvílení mělo odlišný charakter. Rovněž se nepodobalo žádnému ze zvuků, které jsem znal, a jako by vyjadřovalo spíše hněv než smutek.

Krev mi ztuhla v žilách. Přesto jsem rychle otevřel dveře a div jsem necouvl zpátky, když jsem na opačném konci sálu uviděl lidskou postavu\ldots Teprve po chvíli jsem si uvědomil, že je to strýc, který podle naší domluvy otevřel dveře v téže chvíli, kdy já.

,,Slyšel jsi to?``opakoval svým obvyklým tónem.

,,Podivné, nanejvýš podivné!`` odpověděl jsem. ,,Teď ale, strýčku, musíme vyzkoušet poslední možnost: zůstaneme v tomhle sále a přesvědčíme se, zda k tomu strašidelnému úkazu dochází opravdu zde.``

,,Dobrá,`` souhlasil strýc, ,,i když, upřímně řečeno, bych zde z určitých důvodů raději nezůstával. A~za úspěch pak už vůbec neručím. Ostatně,`` dodal po krátkém zamyšlení, ,,zkusme to.``

Znovu jsem důkladně prohledal všechny okolní místnosti, zkontroloval všechny dveře, uvedl do pořádku svíčky – a abych přišel na jiné myšlenky, znovu jsem se pustil do svých novin; zaujali jsme místo u~kulatého stolku uprostřed sálu; strýc si na něm se soustředěným zaujetím čmáral jakási čísla a nesrozumitelné klikyháky.

,,Co to je?`` zeptal jsem se.

,,Nic zvláštního,`` odpověděl tónem vážnějším, než jaký byl u~něho běžný. ,,To se týká jenom mne. Ty se nacházíš mimo tuto sféru.``

,,Strýčku,`` zvolal jsem, ,,zanechte, proboha, těch tajuplností! Rád bych si teď uchoval jasnou mysl a pevného ducha.``

Zmlkli jsme. Déle než půl hodiny panovalo naprosté ticho, když vtom\ldots jakými slovy vypovědět svůj údiv?!\ldots se z hloubi sálu ozvalo sténání, zprvu tiché, pak hlasitější\ldots až nakonec naplno zaznělo přímo nad mým uchem. Tentokrát jsem jasně rozpoznával dva zvuky, z nichž promlouvalo jakési bezútěšné zoufalství, hněv, hoře, prostě všechno smutné, co se vůbec může v lidské duši zrodit. Vyskočil jsem ze židle a pohlédl na strýce – i on byl zjevně vyveden z míry a ztěžka opřen oběma rukama o~stolek znepokojeně sledoval pohyb zvuku\ldots Ale jakými slovy vypovědět svou hrůzu, když jsem pohlédl na protější stěnu a mezi stíny, které jsme vrhali já a můj společník, jsem spatřil ještě třetí stín, jenž byl naprosto zřetelný, jehož kontury se však nedaly postihnout, poněvadž se ustavičně měnily. Bylo to něco nepopsatelného, podobného lidské postavě, co se všelijak zmítalo, vlnilo i trhalo a bez ustání obměňovalo svůj tvar; vynořovala, se tam silueta hlavy, rukou, jež se hned protahovala do délky, hned smršťovala asi tak jako tomu. bývá se siluetami loutek v takzvaném stínovém divadle. Celá ta podívaná trvala sotva minutu\ldots Ohlédl jsem se: v sále nebyl nikdo kromě nás dvou; podíval jsem se opět na stěnu – záhadný stín bledl a spolu s ním zanikal na druhém konci sálu i nářek. Jako by prolétl kolem nás.

,,Chvála bohu, zmizel,`` řekl strýc a sundal ruce se stolu. ,,Chudáci nebozí!`` dodal s povzdechem. ,,Kdy konečně už splatíte poslední denár?``

Po chvíli se však uklidnil, jeho tvář zas nabyla svůj obvyklý ironický výraz a zeptal se mne:

,,Tak co? Slyšel jsi to?``

,,Slyšel,`` odpověděl jsem.

,,Viděl jsi to?``

,,Viděl,`` zněla má odpověď.

,,Byl ten pokus čistý, pane badateli?``

Mlčel jsem.

,,Teď se můžeme klidně odebrat domů,`` pokračoval strýc. ,,Nic už se dít nebude.``

,,Jak to, že jste si tím tak jist?``

,,Tvé jest údobí života – tvé jest i zakvílení.``

,,Proboha, zanechte už toho tajuplného tónu. Pokusme se raději společnýtni silami vyložit ten podivný úkaz\ldots``

,,Mně je velice jasný``

,,Tak mi to povězte.``

,,Jaký by to mělo smysl. Ty přece tak jako tak nic nepochopíš a dál budeš tvrdit, že se ti posmívám, že se to nedá dokázat a podobné věci, jak je tvým zvykem odpovídat na má poctivá objasnění – slyšíš: poctivá,`` opakoval strýc s ironickým výrazem ve tváři.

,,Ne, povězte mi to, strýčku, povězte, co víte a jak co chápete. U~takového podivného úkazu lze přece připustit všechno.``

,,Všechno?`` zeptal se strýc a zkoumavě si mě přeměřil.

,,Tedy\ldots myslel jsem, že při objasňování nutno využít všeho\ldots``

Strýc se usmál. Já jsem se odmlčel.

V strašidelném sále jsme zůstali až do rána a, jak strýc předpovídal, skutečně jsme už nic neslyšeli.

Hle, čím strýc objasňoval onen podivný úkaz. Pokusím se, nakolik paměť dovolí, reprodukovat zde jeho vyprávění v plném rozsahu.

,,Mám-li ti ten úkaz náležitě objasnit,`` pravil strýc, ,,musím začít hodně zdaleka. Časově celá záležitost spadá do třetího desetiletí osmnáctého století. Příběh, který uslyšíš, nenajdeš v žádném historickém spise, poněvadž vaše historie zachycuje pouze vnější děje, jen klamavé obrazy pravých vnitřních dějů. Kromě vašich filologů, archeologů, starožitníků a podobné čeládky existují na tomto světě také jiní historici. A~ti zaznamenávají právě ony jevy, které běžní historiografové ponechávají bez povšimnutí nebo vykládají zkresleně. Život mi dopřál dostat se do styku s těmito neznámými dějezpytci, a to, co ti budu vyprávět, jsem vyčetl z jejich tajuplných análů. Věř tomu nebo nevěř – to je tvá věc. Jestli se ti mé vyprávění bude zdát málo jasné, pokus se objasnit si je sám. Pokud jde o~mne, já žádná další objasnění nepotřebuji.``

\podpis{přeložil Jiří Honzík}