\section{Miha Mazzini: Maso}

\noindent
,,JSI TO NEJŠŤASTNĚJŠÍ DÍTĚ NA SVĚTĚ,`` zdůraznila máma velmi nahlas a potom tiše zasyčela:

,,Ale tobě to nestačí!``

Její hlas nesliboval nic dobrého, ačkoliv v~jedné větě měnila hlasitost běžně. To, co můžou slyšet sousedi, vždycky zdůraznila, a co bylo určené jen pro mě, říkala šeptem. Když se hodně rozzlobila, proměnil se šepot v prskání vzteklé kočky. 

Stáli jsme v kuchyni, bábí četla Život svatých a nás si nevšímala, duše umývaly nádobí a rachotily s talíři. 

,,Chceš ze mě udělat Madam X, ty zrádče!``

Přistoupila ke mně a její dech mi zasyčel do tváře. 

,,Co to s tebou je? Jak to, že jsi neustále nespokojený a nešťastný a pořád ti něco chybí?``

Velmi nahlas:

,,VŠECHNO JSEM TI OBĚTOVALA!``

Potichu:

,,A ty mě zradíš cizím lidem!``

Zadívala se do koupelny a položila si hřbet ruky na čelo.

,,Už v~první třídě jsi před doktorem předstíral, že jsi podvyživený a chudokrevný, jen proto, abych se cítila trapně!``

Začala se šklebit a napodobovat mužský hlas:

,,Dětem se musí dávat jíst, soudružko, vy to nevíte?``

Potom znovu zasyčela:

,,Nevaří ti snad bábí den co den oběd? Co? Tak řekni!``

Odpověděl jsem hodně nahlas, jak se slušelo:

,,ANO, MAMI!``

,,A co mě ten neschopný doktor pomluvil, máš každý den i večeři, ne? ZAJISTILA JSEM TI PLNOHODNOTNOU STRAVU!``

Obrátila se a ze skříňky nad sporákem vytáhla žlutou krabičku, jakých tam bylo nepočítaně.

,,Přečti, co je tady napsáno.``

,,INSTANTNÍ ŠPAGETY -- PLNOHODNOTNÁ STRAVA``

,,Ale tobě to nestačí, co? Jak máme být šťastní, když nejsi spokojený ani s tímhle?``

Otevřela krabičku a vyklopila obsah na linku.

,,Vidíš? Nic tomu nechybí! Makaróny, omáčka v prášku, sýr, dokonce nastrouhaný, takže se nemusíš namáhat, všechno ve zvláštním pytlíku, všechno je HYGIENICKÉ -- vysvětli mi, proč musíš jíst ty jejich materiály?``

Uhodila mě prázdnou krabičkou. Nebolelo to, ale trhl jsem sebou, protože jsem to nečekal. 

,,Dívej se mi do očí, když s tebou mluvím!``

Poslechl jsem.

,,Už když jsem tě nosila pod srdcem, věděla jsem, že v sobě chovám hada. A~tak jsem trpěla! Schválně jsi mě kopal, že mě od toho ještě dneska bolí ledviny!``

Podepřela si pravou tvář dlaní a něžně zašeptala:

,,Já vím, že nikdy nebudeš jako Heintje, ale řekni, copak jsem tě dala do domova?``

,,Ne, mami.``

Bylo vidět, že to v ní vře.

,,Když jsi trochu povyrostl, napadlo mě, že tě asi v porodnici vyměnili. Ale protože ušlechtilost je mi vlastní, rozhodla jsem se, že i když nejsi můj, pokusím se tě vychovat. Prostě jsem si řekla, že se už nedá nic dělat. A~zapřela jsem tě snad někdy? Jsi ošklivý, neschopný, hloupý a nešikovný, na nic se nehodíš, jsi prostě budižkničemu, ale já se za tebe nestydím. Vážíš si toho vůbec?``

,,ANO, MAMI!``

,,Ne, nevážíš.``

Útrpně přikývla.

,,TAK TY BYS RÁD MASO, SYNKU?``

Překvapeně jsem se na ni podíval.

Znovu mě uhodila krabičkou.

,,Řekni ano, mami.``

,,ANO, MAMI!``

,,DOSTANEŠ HO, SYNÁČKU. MAMINKA UDĚLÁ, CO TI NA OČÍCH VIDÍ.``

Otočila se na podpatku a ve dveřích mi ještě přikázala:

,,Počkej tady!`` Potom odešla.

Nevěděl jsem, jestli si smím sednout, ale raději jsem to neriskoval a zůstal jsem stát. Bábí přestala číst, pokynula mi a pustila se do modlení.

Máminy kroky jsem poznal hned, ale připadaly mi nějak zvláštní. Znělo to, jako by se potácela a něco za sebou táhla. První patro, chodba, druhé patro, schody, naše dveře.

Potom vešla. Vlekla nacpanou plátěnou tašku, ze které se sypala suchá hlína. Brambory? Vždyť máme ještě několik pytlů ve sklepě. Rozevřela ruku a taška spadla na podlahu. Podle zvuku v ní bylo něco kovového.

,,Sedni si,`` nařídila mi. Kapesníčkem si utřela z čela pot a počkala, až bude moct popadnout dech. 

,,Co všechno já pro tebe neudělám! Egone, synáčku můj, přišla jsem na to, že jsem ti křivdila. Když máš hlad, musíš jíst! Protože jsi ve vývinu!``

Seděl jsem za stolem a čekal.

Vytáhla z kabelky otvírák na konzervy a hodila ho přede mě na stůl.

Sklouzl po igelitovém ubruse, odrazil se od stěny, v rohu narazil do rádia a zastavil se. 

,,Na!``

Vzal jsem ho mezi palec a ukazováček.

Sklonila se, sáhla do tašky a vytáhla ven konzervu šunky ve vlastní šťávě. Zvysoka ji postavila na stůl. 

,,OTEVŘI TO, SYNÁČKU!``

Rychle jsem se do toho dal, ale nedařilo se mi to zrovna nejlíp. Po očku jsem mámu neustále pozoroval a viděl, jak ji zlobí, že jsem takové nemehlo. Znervóznilo mě to a šlo mi to ještě hůř. Příležitost otevírat konzervy jsem měl málokdy, a navíc jsem na otvírák moc tlačil. Byl tak malý, že se mi pod prsty skoro ztrácel. Ještě než jsem otevřel konzervu do půlky, bodal mě kov do naběhlých puchýřů. Na konci to už bylo lepší, protože jsem zjistil, jak to mám dělat. Puchýře mi ale stejně popraskaly. 

Pak jsem pod vrstvou rosolu uviděl růžovou šunku. Vypadala lákavě a omamně voněla. Každičké vlákénko masa čekalo, až se do něj zakousnu. Polkl jsem slinu.

Máma mlčela. Pozorně jsem si prohlédl obsah otevřené konzervy a cítil, jak mě stržené puchýře začínají pálit.

,,Dal by sis?``

,,ANO, MAMI!``

,,TAK DO TOHO, SYNKU!``

Vrhl jsem se na konzervu, zabořil prsty do měkkého masa a vyndal ho. Celé jsem ho naráz zhltl a dokonce jsem slízal ze stolu kousíčky rosolu.

Buch! Na stůl dopadla další konzerva.

Zopakovali jsme si to ještě jednou, a když jsem šunku snědl, už na mě čekala nová. 

Tlačil mě žaludek. Věděl jsem, že už do sebe nesmím dostat ani sousto, nebo mi bude špatně.

Ale smála se na mě třetí otevřená konzerva.

,,MÁŠ DOST, SYNKU?``

,,ANO, MAMI!``

,,UŽ SI NEDÁŠ? JESTLI NECHCEŠ, TAK TO ŘEKNI, NIKDO TĚ NENUTÍ!``

Žaludek říkal ne, ale oči… Lákavé maso, chuť v puse, možnost všechno zopakovat…

,,VEZMEŠ SI JEŠTĚ, NE?``

Přikývl jsem. 

,,Máma tě zná. Řekni ano, mami!``

,,ANO, MAMI!``

Nová konzerva. Otevřená, umazaná od krve. Otvírák mi pořád klouzal ze zkrvavených  rukou sedřených do živého masa. V prstech mi škubalo. 

,,KOLIK ŠUNEK JSI SNĚDL?``

,,ČTYŘI, MAMI!``

,,CHCEŠ JEŠTĚ?``

Bylo mi strašně špatně. Ale… taková šance se mi už nenaskytne. Chuť masa… Kousání, žvýkání…

,,ANO, MAMI!``

Po bradě mi stékal rosol a na stůl se s krví nalepila růžová vlákénka.

,,TAK KOLIK, SYNÁČKU?``

,,PĚT, MAMI!``

,,VIDÍŠ! JÁ SI UTRHUJU OD ÚST, ABYS BYL TO NEJŠŤASTNĚJŠÍ DÍTĚ NA SVĚTĚ! Kolik myslíš, že jako uklízečka v Železárně vydělávám, co? NECHCI TI SNAD VŠECHNO DOPŘÁT? Řekni ano, mami!`` 

,,ANO, MAMI!``

,,TAK SNĚZ JEŠTĚ JEDNU! MÁMA CHCE, ABYS UŽ NIKDY NEMĚL HLAD!``

,,ANO, MAMI!``

\podpis{přeložila Kristina Pellarová}