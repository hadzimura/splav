\section{Ve znamení přebytků}

\noindent
Číslo, které právě čtete, je číslem poněkud zvláštním. ,,Bubeníčci``, s nimiž se Plav revue pojí takřka láskyplným mutualismem květu a opylujícího hmyzu, jsou totiž tentokrát věnováni půlročnímu výročí měsíčníku pro světovou literaturu Plav. Nabízíme vám proto texty, které v tomto časopisu nebylo možno z kapacitních důvodů uveřejnit. Vzhledem k tomu, že ,,přebytky`` prvních čísel svou aktuálnost povětšinou ztratily -- leckdy totiž již vyšly v knižní podobě --, poskytujeme vám to, co se nevešlo do čísla současného, věnovaného slovinské literatuře. 

Jedná se tak zejména o básně – poprvé se můžete setkat s tvorbou Lucije Stupicy, která ač vcelku lyricky subtilní, svůj prostor ve ,,velkém`` Plavu bohužel nenalezla. Kromě toho máte možnost doplnit si několika básněmi znalost díla Uroše Zupana, Jany Putrle Srdić a Josipa Ostiho, s jejichž tvorbou jste se již seznámit mohli. Jediným prozaickým textem, který z přebytků slovinského čísla zařazujeme, je úryvek z románu Mihy Mazziniho \textit{Já, Tito a gramofon} nazvaný \textit{Maso}. 

Kromě těchto přebytků z čísla slovinského vám přinášíme ještě jedno překvapení. Jedná se o překlad ironické eseje Viktora Pelevina \textit{Mardongy}. Pozorného čtenáře to jistě zarazí, poněvadž jsme tento text již jednou publikovali – konkrétně v pátém čísle Plavu. Zde však došlo k poměrně závažnému pochybení, protože byla nedopatřením otištěna starší verze překladu. Překladatelce Ivě Dvořákové se proto hluboce omlouváme a správné, vybroušenější verzi jejího překladu poskytujeme kromě prostoru na našich internetových stránkách (\textit{www.svetovka.cz}), kde text naleznete i ve srovnání s originálem, možnost vyjít i v podobě tištěné.

Nezbývá tedy než vám popřát příjemné počtení. 

\podpis{Jan Chromý}