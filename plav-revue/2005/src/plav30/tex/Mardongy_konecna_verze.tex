\section{Viktor Pelevin: Mardongy}

\noindent
\textit{A~zvěst o~mně se rozlétne jak mrtvolný puch. \\ (N. Antonov)}

\medskip
\noindent
\textit{Mardong} je tibetské slovo, jež svými významy postihuje široký okruh představ. Původně toto slovo označovalo kultovní objekt, který vznikal následujícím způsobem. Nehledě na to, zda člověk ve svém životě vynikal svatostí, čistotou, nebo zda byl naopak, obrazně řečeno, takovým pověstným ,,květem zla`` (sepětí Baudelaira s~Tibetem se teprve začíná zkoumat), nebylo jeho tělo po smrti (kterou Tibeťané zároveň považovali za jedno ze stádií rozvoje osobnosti) pohřbeno, ale ,,opékalo se`` v~oleji (severně od Lhasy se pro tento účel zpravidla užíval jačí tuk). Potom bylo tělo slavnostně oblečeno do pláště a posazeno na zem -- obvykle někde u~cesty. Nakonec byl zemřelý kolem dokola obestavěn zdmi (z~kamenů a cementu) a vzniklo tak jakési staveníčko. Jeho tvar ještě připomínal obrysy člověka sedícího v~tureckém sedu. Tento výsledný objekt se pak obaloval nejprve hlínou (v~severních oblastech hnojem se slámou, to pak bylo nutné ještě jedno opálení), potom omítkou a nakonec byl mardong ,,pomalován``. Konečná malba měla být portrétem zazděného~-- zpravidla však nebylo možné osobu rozpoznat. Pokud zemřelý patřil k~sektě Dug-pa nebo Bon, byla mu přimalována ještě černá čapka. Mardong byl hotov a mohl se tak stát objektem nekonečného uctívání nebo naopak hanobení. To už záviselo na náboženské příslušnosti účastníků rituálu. Tolik tedy ke zrodu slova mardong.

\looseness+1 Druhý význam slova mardong je podstatně známější. Říkají si tak stoupenci Nikolaje Antonova a označoval se tak rovněž sám Antonov. Tato nevelká studie nám však nedává možnost prozkoumat celou historii sekty, a tak nás bude nejvíce zajímat průřez její ideologií a myšlení samotného Antonova. Zároveň je ale nutné konstatovat, že nesouhlasíme s~hypotézou, která se nedávno objevila a která tvrdí, že Antonov je osobnost vymyšlená a jeho práce jsou pouhé kompilace. Nesouhlasíme i přesto, že argumenty zastánců této hypotézy jsou často velmi důmyslné. Nesmíme totiž zapomenout, že ,,neexistence Antonova``, k~níž se sektáři často vyjadřovali, je jedním z~mystických dogmat, ale pro možné budoucí stoupence naprosto není určující. S~hypotézou ,,neexistence`` nelze pak souhlasit také proto, že všechna díla, která jsou známá jako ,,antonovská``, v~sobě nesou jasný otisk jedné jediné osobnosti. ,,Stačí pět šest stran,`` píše Gil de Chardin, ,,a člověku se začíná zdát, že mu noha uvízla mezi čelistmi nějaké příšery, které se svírají víc a víc, a všechno kolem tmavne\ldots`` Přepjatá emocionální hodnocení však přenechme tomuto dojatému Francouzovi. Pro nás je podstatné to, že Antonovovy práce jsou prostoupeny zcela nezaměnitelným nábojem, a jsou tak stylisticky výrazně odděleny od všeho, co bylo v~té době napsáno. I~kdybychom tedy přistoupili na tvrzení o~kompilaci, museli bychom autorovi přiznat, že byl opravdu jediný, a pak by se tato osoba pro nás stala Nikolajem Antonovem.

Začátek hnutí je spjat s~rokem 1993, kdy se objevuje Antonovova kniha \textit{Dialogy s~vnitřním umrlcem}. Její první část se nazývá \textit{Smrt není}. Myšlenka sama samozřejmě není nová, nezvyklá je však autorova argumentace. Ukazuje se, že smrt neexistuje proto, že už dávno minula, neboť v~každém člověku je přítomen vnitřní umrlec, který postupně dostává stále větší část naší osobnosti pod svou nadvládu. Život není dle Antonova nic jiného než proces vystupování mrtvolné podstaty na povrch, naše mrtvolné já se rozvíjí uvnitř člověka -- jako plod v~matce. Fyzická smrt je pak jen konečnou fází aktualizace mrtvolného já, a je tak vlastně porodem, zrozením. Živý člověk je budoucím zárodkem já mrtvolného, je to existence nízká a neplnohodnotná. Mrtvolné já se tedy nakonec jeví jako nejvyšší možná forma existence, neboť právě ona je věčná (samozřejmě ne fyzicky, ale kategoriálně).

Chyba běžného člověka spočívá v~tom, že v~sobě stále umlčuje vnitřní ,,mrtvolný`` hlas, bojí se chápat jeho existenci. Podle Antonova je VU (v~pracích současných antonovců je vnitřní umrlec obvykle označován touto zkratkou) nejcennější součástí lidské osobnosti a veškerý duchovní život musí být orientován právě na něj. K~této myšlence, jež byla dále rozvinuta v~následujících Antonovových pracích, se ještě vrátíme. Zatím však přejděme k~části druhé, tedy k~\textit{Dialogům}.

Tato část se nazývá \textit{Duchovní mardong Alexandra Puškina}. \linebreak Kromě seznámení se s~termínem samotným jsou tu popsány základní praktické metody (pořízené za života) o~probouzení naší mrtvolné podstaty. Antonov píše o~duchovních mardonzích, utvářejících se po smrti lidí, kteří zanechali zjevnou stopu ve společenském vědomí. V~takovém případě nastupují místo ,,opékání`` v~oleji především okolnosti smrti a to, jak je chápe společnost (Antonov připodobňuje Natálii Gončarovovou pánvičce a Dantese kuchaři). Roli cihel a cementu hraje v~tomto případě jednoznačnost interpretace myšlenek a motivů. Podle Antonova byl duchovní mardong Puškina definitivně dokončen na konci 19. století. Roli závěrečné kresby tu sehrály Čajkovského opery.  

Kultura je podle Antonova jistým společným hrobem, v~němž poklidně spočinuly duchovní mardongy ideologií, děl i velkých osobností. Přítomnost živého je v~této oblasti urážlivá a zcela nepřípustná -- stejně nepřípustná, jako by byla pro některá náboženství třeba přítomnost menstruující ženy v~chrámu Páně. Společný hrob je samozřejmě pojem abstraktní (po vydání knihy dostalo nakladatelství mnoho dopisů s~prosbou, aby prozradilo, kde se onen společný hrob nalézá).

Existence duchovních ,,mrtvol`` v~noosféře, jak dále praví Antonov, napomáhá utváření správného duchovně-emocionálního procesu, v~němž každý krok vede ke ,,zmrtvolnění`` (což je jeden ze základních termínů této práce). Všechny praktické rady, které byly uvedeny v~\textit{Dialozích}, byly později ještě dále rozvinuty, a proto považujeme za správné věnovat se i druhé Antonovově knize.

Kniha \textit{Noc. Ulice. Lampa. Lékárna} (1995) vypadá na první pohled jako nesouvislý popis aforismů a meditačních metod. Antonovovi stoupenci však tvrdí, že v~těchto výpovědích, stejně tak jako v~principech jejich vzájemného uspořádání, jsou skryty nejniternější zákony vesmíru. Vzhledem k~nedostatku prostoru se však této části knihy nemůžeme podrobně věnovat. Musíme se ovšem alespoň zmínit o~faktu, že poslední počítačové výzkumy prokázaly nepochybnou souvislost mezi častým opakováním slova harmonie v~Antonovově knize a rituálem přípravy vařené feny, národního jídla Indiánů kmene Navacho. (Legenda o~tom, že Antonov z mystických důvodů zkonzumoval svou vlastní fenku namaskovanou za Puškina, není nijak doložena a pravděpodobně jde jen o~jeden z~mýtů, které vznikly kolem jeho osoby. Nakolik je známo, Antonov nikdy žádného psa neměl.)

Praktické techniky vedoucí ke zmrtvolnění jsou velmi různorodé. Již v~první knize se objevuje \textit{Rozmluva o~Puškinovi}. (O~Antonovovi se tvrdí, že během posledních let svého života už otevíral ústa jen proto, aby mohl aktualizovat své výzkumy o~Puškinově velikosti; dle komentářů antonovců pracoval jejich mistr současně na dvou mardonzích -- upevňoval puškinovský a dokončoval svůj). Tato praxe je díky antonovcům dosti formalizovaná. \textit{Rozmluva o~Puškinovi} začíná úvodním ujištěním, že básník nevnímal období své nezralosti, a končí zpěvem mantry ,,Puškin je po puškinsku král\ldots``, v~níž jsou všechna slova i intonce voleny dle pevného vnitřího řádu. A~je dokonce možné připustit, že v~životě Antonova (antonovec by nás trochu poopravil -- v~době Antonovovy prvosmrti) existovaly nejrůznější odchylky, do současné podoby dospěl postupně.

Za druhou techniku, která vede též ke zmrtvolnění, považujeme studium staroruské kultury -- tedy vlastně nejen jí, ale také studium jejího mardongu. V~tomto ohledu vyslovil Antonov jednu velmi zajímavou myšlenku, díky níž se ke zmíněnému hnutí přidala i řada slavjanofilů. Prohlásil, že hliněný hrnec z~8. století, který byl nalezen v~Kyjevě, je prvním historickým mardongem. Byla v~něm také nalezena bércová kost dívenky Goruchšky -- toto slovo bylo zaznamenáno přímo na hliněném hrnci. Po takovémto vlasteneckém výroku získali Antonovi státní dotaci a pozice jejich hnutí tak zjevně posílila.

Kromě uvedených metod doporučuje Antonov rovněž studium nějakého mrtvého jazyka (například sánsktru) nebo také ležení v~rakvi.

Po vzniku sekty se způsob života jejích členů výrazně ritualizoval.  Vypráví se například, že Antonov nikdy svým stoupencům nedovoloval, aby za jeho přítomnosti vytahovali okurky z~láhve přímo rukou. Dle jeho mínění se totiž ,,mrtvost`` zeleniny mohla dotykem živého pošpinit. Bohužel se však nedochovala Antonovova práce \textit{Každodenní zázrak}, která právě o~těchto otázkách pojednávala. Kniha \textit{Majdan} (1998) je zdrženlivější a zadumanější -- i přes svou blízkost a stylistickou jednotu s~ostatními díly připomíná spíše medinské súry. Neobsahuje sice tolik nových myšlenek, ale ideje již dříve vyslovené jsou značně prohloubeny a rozpracovány. Například je tu probírána myšlenka o~množství vnitřních mrtvolných já, která jsou zasazena do sebe a  de facto tak fungují na principu matrjošek (ty jsou dle Antonova staroruským symbolem mardongu) -- každé z~následujících já pozoruje předchozí mrtvolné já s~nostalgií. Primární já stále touží po tom konečném, tedy po dokončeném aktualizovaném mrtvolném já; a kruh se uzavírá.

V~této knize se Antonov rovněž zříká těch ze svých stoupenců, kteří se dopouštějí sebevraždy a opovržlivě je nazývá nedonošenci. (V~antonovském systému se i císařský řez chápe jako vražda, zatímco sebevražda je vnímána jako předčasný porod. Smrt mladého člověka je pro něj rovna potratu.)

V~roce 1999 Antonov konečně dochází uznání. Vzniká i jeho mardong. Hned v~příkopě na devětatřicátém kilometru Možajské silnice.

Mardong stojí na tomto místě dodnes. A~kdykoli chcete, můžete u~něj vidět řadu antonovců -- ponurých mladých lidí v~tmavých pláštích, s~odbarvenými vlasy a s~gumičkami přetaženými přes ruce, tak, aby ruce vyhlížely co nejmrtvolněji. Čtou se tu verše -- často Sologub či Blok, z~nichž jsou však dle antonovského principu vybrány básně s~velkým množstvím šeplavých souhlásek. Někdy se dokonce čtou verše samotného Antonova:

\begin{verse}
	A~až jim svíce dosyčí

	a ve tmě zazní hodiny

	mrtví už hrůzou nekřičí

	a na zemi spí bez viny
\end{verse}

\noindent
Z~cesty se tak naskýtá neuvěřitelná podívaná.

\podpis{přeložila Iva Dvořáková}