\ifpdf
 \pdfinfo{
 /Title (PLAV 19.9.2005, c.5)
 /Creator (TeX)
 /Author (SPLAV! Team)
 /Subject (Prekladatele, Literati, Autori, Vykladaci)
 /Keywords (svetova literatura, preklady)
 }
\fi

\hlavicka{2005}{5}{POD DVOJÍM NEBEM}
\titulek{VIKTOR PELEVIN}
\setcounter{page}{20}

\nadpis{Viktor Pelevin: Mardongy}

\noindent
\textit{Následující text je mezi dalšími soutěžními překlady opět trochu netradiční. Jedná se o~přetlumočení ironické eseje Viktora Pelevina Mardongy, tedy textu spíše ,,exaktního`` než přímo poetického charakteru. Iva Dvořáková za tento překlad získala v~letošním roce čestné uznání v~kategorii prózy.}

\mymarginpar{\vspace*{3.58cm} Viktor Pelevin se narodil roku 1962. Vystudoval Moskevský energetický institut (specializace elektromechanik) a Literární institut. Má pověst uzavřeného člověka, nerad dává interview a se svými fanoušky raději komunikuje přes internet.

Do literatury vstoupil roku 1989, o~dvě léta později vydal svou první knihu, sbírku povídek Sinij fonar (Modrá lampa). Následují romány Omon Ra, (1992, česky Omon Ra, 2002), Čapajev i pustota (1996, česky Čapajev a prázdnota, 2001), Žizn' nasekomych (1996, Život hmyzu), Generation ,,P`` (1999, česky Generation P, 2002) a druhá sbírka povídek Žoltaja strela (1998, Žlutá střela). Jsou přitom dost těžko zařaditelné: kritikové je označují za alegorické, magicko-realistické či dokonce zen-buddhistické. Zpravidla si pohrávají s~přechody mezi různými světy a realitami a především s~literární kamufláží. V každém případě je ovšem Pelevin autorem, který je značně čten, přestože (nebo protože) podle některých kritik je to autor-ideolog.}

\bigskip
\bigskip

\begin{pairs}
\begin{Leftside}
\beginnumbering
{\rusky
\pstart
\nadpis{Мардонги}
\autor{Виктор Пелевин}
\noindent
Слух обо мне пройдет, как вонь от трупа.\\
Н. Антонов
\pend

\pstart
\noindent
Слово «мардонг» тибетское и обозначает целый комплекс понятий. Первоначально так назывался культовый объект, который получался вот каким образом: если какой-нибудь человек при жизни отличался святостью, чистотой или, наоборот, представлял собой, образно выражаясь «цветок зла» (связи Бодлера с Тибетом только сейчас начинают прослеживаться), то после смер\-ти, которую, кстати, тибетцы всегда считали одной из стадий развития личности, тело такого человека не зарывалось в зем\-лю, а обжаривалось в растительном масле (к северу от Лхасы обычно использовался жир яков), затем обряжалось в халат и усаживалось на землю, обычно возле дороги. После этого вокруг трупа и впритык к нему возводилась стена из сцементированных камней, так что в результате получалось каменное образование, в котором можно было уловить сходство с контуром сидящей по-турецки фигуры. Затем объект обмазывался глиной (в северных районах -- навозом пополам с соломой, после чего был необходим еще один обжиг), затем штукатуркой и разрисовывался -- роспись была портретом замурованного, но, как \linebreak правило, изображенные лица неотличимы. Если умерший принадлежал к секте Дуг-па или Бон, ему пририсовывалась черная камилавка. После этого мардонг был готов и становился объектом либо исступленного поклонения, либо настолько же исступленного осквернения -- в зависимости от религиозной принадлежности участников ритуала. Такова предыстория.
\pend}
\endnumbering
\end{Leftside}

\begin{Rightside}
\beginnumbering
\selectlanguage{czech}
\pstart
~\nadpis{\hspace{-0.52cm}Mardongy}
\autor{přeložila Iva Dvořáková}
\noindent
\textit{A~zvěst o~mně se rozlétne jak mrtvolný puch.\\
N. Antonov}
\pend

\pstart
\noindent
\textit{Mardong} je tibetské slovo, jež svými významy postihuje široký okruh představ. Původně toto slovo označovalo kultovní objekt, který vznikal následujícím způsobem. Nehledě na to, zda člověk ve svém životě vynikal svatostí, čistotou, nebo zda byl naopak, obrazně řečeno, takovým pověstným ,,květem zla`` (sepětí Baudelaira s~Tibetem se teprve začíná zkoumat), nebylo jeho tělo po smrti (kterou Tibeťané zároveň považovali za jedno ze stádií rozvoje osobnosti) pohřbeno, ale ,,opékalo se`` v~oleji (severně od Lhasy se pro tento účel zpravidla užíval jačí tuk). Potom bylo tělo slavnostně oblečeno do pláště a posazeno na zem -- obvykle někde u~cesty. Nakonec byl zemřelý kolem dokola obestavěn zdmi (z~kamenů a cementu) a vzniklo tak jakési staveníčko. Jeho tvar ještě připomínal obrysy člověka sedícího v~tureckém sedu. Tento výsledný objekt se pak obaloval nejprve hlínou (v~severních oblastech hnojem se slámou, to pak bylo nutné ještě jedno opálení), potom omítkou a nakonec byl mardong ,,pomalován``. Konečná malba měla být portrétem zazděného -- zpravidla však nebylo možné osobu rozpoznat. Pokud zemřelý patřil k~sektě Dug-pa nebo Bon, byla mu přimalována ještě černá čapka. Mardong byl hotov a mohl se tak stát objektem nekonečného uctívání nebo naopak hanobení. To už záviselo na náboženské příslušnosti účastníků rituálu. Tolik tedy ke zrodu slova mardong.
\pend
\endnumbering
\end{Rightside}
\Columns
\end{pairs}

\mymarginpar{\vspace*{-1.50cm}Iva Dvořáková se narodila roku 1980. Vystudovala obor český jazyk a literatura na FF UK a v~loňském roce na něj navázala doktorským studiem slovanské filologie. Tamtéž studuje také rusistiku. Pracuje v~lexikografickém oddělení Ústavu pro jazyk český AV ČR a v~překladatelské agentuře.}


\begin{pairs}
\begin{Leftside}
\beginnumbering
{\rusky
\pstart
Второе значение слова «мардонг» широко известно. Так называют себя последователи Николая Антонова, так называл себя сам Антонов. Наш небольшой очерк не ставит себе целью проследить историю секты -- нас больше интересует ранний \linebreak срез ее идеологии и мысли самого Антонова, кстати, мы не согласны с появившейся не\-давно гипотезой, что Антонов -- вымышленное лицо, а его труды -- компиляция, хотя аргументы сторонников этой точки зрения часто остроумны. Надо всегда помнить, что «несуществование Антонова», о котором многократно заявляли сектанты, есть одна из их мистических догм, а вовсе не намек неким будущим исследователям. Согласиться с этой гипотезой нельзя еще и потому, что все сочинения, известные как антоновские, несут на себе ясный отпечаток личности одного человека. «Пять или шесть страниц, -- пишет Жиль де Шарден, -- и начинает казаться, что ваша нога попала в медленные челюсти некоего гада, и все сильнее нажим, и все темнее вокруг\ldots» Оставим излишнюю эмоциональность оценки на совести впечатлительного француза, важно то, что работы Антонова дейст\-вительно пронизаны одним настроением и стилистически обособлены от всего написанного в те годы -- если уж предполагать компиляцию, то автор у подделок тоже должен быть один, и в таком случае под именем Николая Антонова нами понимается этот человек.
\pend

\pstart
Начало движения относится к 1993 году и связано с появлением книги Антонова «Диалоги с внутренним мертвецом».

«Смерти нет» -- так называется ее первая часть. Идея, конечно, не нова, но аргументация автора необычна. Оказывается, смерти нет потому, что она уже произошла, и в каждом человеке присутствует так называемый внутренний мертвец, постепенно захватывающий под свою власть все большую часть личности. Жизнь, по Антонову, -- не более чем процесс вынашивания трупа, развивающегося внутри, как плод в матке. Физическая же смерть является конечной актуализацией внутреннего мертвеца и представляет собой, таким образом, роды. Живой человек, будучи зародышем трупа, есть существо низкое и неполноценное. Труп же мыслится как высшая возможная форма существования, ибо он вечен (не физически, конечно, а категориально).
\pend

\pstart
Ошибка обычного человека заключается в том, что он постоянно заглушает в себе голос внутреннего мертвеца и боится отдать себе отчет в его существовании. По Антонову, ВМ (так обычно обозначается внутренний мертвец в изданиях нынешних антоновцев) -- самая ценная часть личности, и вся духовная жизнь должна быть ориентирована на него. Мы еще вернемся к этой мысли, получившей развитие в последующих работах Антонова, а пока перейдем ко второй части «Диалогов».
\pend

\pstart
Она называется «Духовный мардонг Але\-ксандра Пушкина». Уже здесь, помимо введения термина, обозначены основные \linebreak практические методы прижизненного пробуждения внутреннего мертвеца. Антонов пишет о духовных мардонгах, образующи\-хся после смерти людей, оставивших заметный след в групповом сознании. В этом случае роль обжарки в масле выполняют обстоятельства смерти человека и их общественное осознание (Антонов уподобляет Наталью Гончарову сковороде, а Дантеса -- повару), роль кирпичей и цемента -- утверждающаяся однозначность трактов\-ки мыслей и мотивов скончавшегося. По Антонову, духовный мардонг Пушкина \linebreak был готов к концу XIX века, причем роль окончательной раскраски сыграли оперы Чайковского.
\pend

\pstart
Культурное пространство, по Антонову, является Братской Могилой, где покоятся духовные мардонги идеологий, произведений и великих людей, присутствие живого в этой области оскорбительно и недопустимо, как недопустимо в некоторых \linebreak религиях присутствие менструирующей \linebreak женщины в храме. Братская Могила, разумеется, понятие идеальное (после выхода \linebreak книги в издательство пришло много писем с просьбой указать ее местонахождение).
\pend

\pstart
Существование духовных трупов в ноосфере, говорит дальше Антонов, способствует выработке правильного духовно-\linebreak-эмоционального процесса, где каждый \linebreak шаг ведет к «утрупнению» (один из ключевых терминов работы). Практические рекомендации, приведенные в «Диалогах», впоследствии получили развитие, поэтому будет правильно рассмотреть их по второй книге Антонова.
\pend

\pstart
Книга «Ночь. Улица. Фонарь. Аптека» \linebreak (1995) представляет собой на первый взг\-ляд бессвязный набор афоризмов и медитационных методик -- однако адепты ут\-верждают, что в этих высказываниях, а \linebreak также в принципах их взаимного расположения зашифрованы глубочайшие законы Вселенной. За недостатком места мы не сможем рассмотреть эту сторону книги -- отметим только, что последние исследования на ЭВМ ЕС-5540 установили несомненную структурную связь между повторяемостью в книге слова «гармония» и ритуалом приготовления вареной суки -- национального блюда индейцев Навахо. (Легендарный факт съедения Антоновым \linebreak в мистических целях своей собаки, предварительно якобы загримированной под Пушкина, никак не документирован и, по-видимому, является одним из многочисленных мифов вокруг этого человека, насколько известно, никакой собаки у Антонова не было.)
\pend

\pstart
Практические техники, ведущие к «ут\-рупнению», разнообразны. Еще в первой книге предложен «Разговор о Пушкине». (Утверждают, что в последние годы жизни Антонов открывал рот только для того, чтобы сделать очередное заявление о величии Пушкина, антоновцы комментиру\-ют это в том смысле, что мастер работал одновременно над двумя мардонгами -- укреплял пушкинский и достраивал \linebreak свой.) Эта практика среди антоновцев сейчас строго формализована: «Разговор \linebreak о Пушкине» начинается с вводного утверждения о том, что поэт не знал периода ученичества, и кончается распеванием \linebreak мантры «Пушкин пушкински велик» -- регламентированы не только все произносимые слова, но и интонации. Можно допустить, что при жизни Антонова (антоновец бы поправил, сказав -- при первосмертии) существовали отклонения от канона и в современной форме он сложился позднее.
\pend}
\endnumbering
\end{Leftside}

\begin{Rightside}
\beginnumbering
\selectlanguage{czech}
\pstart
Druhý význam slova mardong je podstatně známější. Říkají si tak stoupenci Nikolaje Antonova a označoval se tak rovněž sám Antonov. Tato nevelká studie nám však nedává možnost prozkoumat celou historii sekty, a tak nás bude nejvíce zajímat průřez její ideologií a myšlení samotného Antonova. Zároveň je ale nutné konstatovat, že nesouhlasíme s~hypotézou, která se nedávno objevila a která tvrdí, že Antonov je osobnost vymyšlená a jeho práce jsou pouhé kompilace. Nesouhlasíme i přesto, že argumenty zastánců této hypotézy jsou často velmi důmyslné. Nesmíme totiž zapomenout, že ,,neexistence Antonova``, k~níž se sektáři často vyjadřovali, je jedním z~mystických dogmat, ale pro možné budoucí stoupence naprosto není určující. S~hypotézou ,,neexistence`` nelze pak souhlasit také proto, že všechna díla, která jsou známá jako ,,antonovská``, v~sobě nesou jasný otisk jedné jediné osobnosti. ,,Stačí pět šest stran,`` píše Gil de Chardin, ,,a člověku se začíná zdát, že mu noha uvízla mezi čelistmi nějaké příšery, které se svírají víc a víc, a všechno kolem tmavne\dots`` Přepjatá emocionální hodnocení však přenechme tomuto dojatému Francouzovi. Pro nás je podstatné to, že Antonovovy práce jsou prostoupeny zcela nezaměnitelným nábojem, a jsou tak stylisticky výrazně odděleny od všeho, co bylo v~té době napsáno. I~kdybychom tedy přistoupili na tvrzení o~kompilaci, museli bychom autorovi přiznat, že byl opravdu jediný, a pak by se tato osoba pro nás stala Nikolajem Antonovem.
\pend

\pstart
Začátek hnutí je spjat s~rokem 1993, kdy se objevuje Antonovova kniha \textit{Dialogy s~vnitřním umrlcem}. Její první část se nazývá \textit{Smrt není}. Myšlenka sama samozřejmě není nová, nezvyklá je však autorova argumentace. Ukazuje se, že smrt neexistuje proto, že už dávno minula, neboť v~každém člověku je přítomen vnitřní umrlec, který postupně dostává stále větší část naší osobnosti pod svou nadvládu. Život není dle Antonova nic jiného než proces vystupování mrtvolné podstaty na povrch, naše mrtvolné já se rozvíjí uvnitř člověka -- jako plod v~matce. Fyzická smrt je pak jen konečnou fází aktualizace mrtvolného já, a je tak vlastně porodem, zrozením. Živý člověk je budoucím zárodkem já mrtvolného, je to existence nízká a neplnohodnotná. Mrtvolné já se tedy nakonec jeví jako nejvyšší možná forma existence, neboť právě ona je věčná (samozřejmě ne fyzicky, ale kategoriálně).
\pend

\pstart
Chyba běžného člověka spočívá v~tom, že v~sobě stále umlčuje vnitřní ,,mrtvolný`` hlas, bojí se chápat jeho existenci. Podle Antonova je VU (v~pracích současných antonovců je vnitřní umrlec obvykle označován touto zkratkou) nejcennější součástí lidské osobnosti a veškerý duchovní život musí být orientován právě na něj. K~této myšlence, jež byla dále rozvinuta v~následujících Antonovových pracích, se ještě vrátíme. Zatím však přejděme k~části druhé, tedy k~\textit{Dialogům}.
\pend

\pstart
Tato část se nazývá \textit{Duchovní mardong Alexandra Puškina}. Kromě seznámení se s~termínem samotným jsou tu popsány základní praktické metody (pořízené za života) o~probouzení naší mrtvolné podstaty. Antonov píše o~duchovních mardonzích, utvářejících se po smrti lidí, kteří zanechali zjevnou stopu ve společenském vědomí. V~takovém případě nastupují místo ,,opékání`` v~oleji především okolnosti smrti a to, jak je chápe společnost (Antonov připodobňuje Natálii Gončarovovou pánvičce a Dantese kuchaři). Roli cihel a cementu hraje v~tomto případě jednoznačnost interpretace myšlenek a motivů. Podle Antonova byl duchovní mardong Puškina definitivně dokončen na konci 19. století. Roli závěrečné kresby tu sehrály Čajkovského opery. 
\pend

\pstart
Kultura je podle Antonova jistým společným hrobem, v~němž poklidně spočinuly duchovní mardongy ideologií, děl i velkých osobností. Přítomnost živého je v~této oblasti urážlivá a zcela nepřípustná -- stejně nepřípustná, jako by byla pro některá náboženství třeba přítomnost menstruující ženy v~chrámu Páně. Společný hrob je samozřejmě pojem abstraktní (po vydání knihy dostalo nakladatelství mnoho dopisů s~prosbou, aby prozradilo, kde se onen společný hrob nalézá).
\pend

\pstart
Existence duchovních ,,mrtvol`` v~noosféře, jak dále praví Antonov, napomáhá utváření správného duchovně-emocionálního procesu, v~němž každý krok vede ke ,,zmrtvolnění`` (což je jeden ze základních termínů této práce). Všechny praktické rady, které byly uvedeny v~\textit{Dialozích}, byly později ještě dále rozvinuty, a proto považujeme za správné věnovat se i druhé Antonovově knize.
\pend

\pstart
Kniha \textit{Noc. Ulice. Lampa. Lékárna} (1995) vypadá na první pohled jako nesouvislý popis aforismů a meditačních metod. Antonovovi stoupenci však tvrdí, že v~těchto výpovědích, stejně tak jako v~principech jejich vzájemného uspořádání, jsou skryty nejniternější zákony vesmíru. Vzhledem k~nedostatku prostoru se však této části knihy nemůžeme podrobně věnovat. Musíme se ovšem alespoň zmínit o~faktu, že poslední počítačové výzkumy prokázaly nepochybnou souvislost mezi častým opakováním slova harmonie v~Antonovově knize a rituálem přípravy vařené feny, národního jídla Indiánů kmene Navacho. (Legenda o~tom, že Antonov z~mystických důvodů zkonzumoval svou vlastní fenku namaskovanou za Puškina, není nijak doložena a pravděpodobně jde jen o~jeden z~mýtů, které vznikly kolem jeho osoby. Nakolik je známo, Antonov nikdy žádného psa neměl.)
\pend

\pstart
Praktické techniky vedoucí ke zmrtvolnění jsou velmi různorodé. Již v~první knize se objevuje \textit{Rozmluva o~Puškinovi}. (O~Antonovovi se tvrdí, že během posledních let svého života už otevíral ústa jen proto, aby mohl aktualizovat své výzkumy o~Puškinově velikosti; dle komentářů antonovců pracoval jejich mistr současně na dvou mardonzích -- upevňoval puškinovský a dokončoval svůj). Tato praxe je díky antonovcům dosti formalizovaná. \textit{Rozmluva o~Puškinovi} začíná úvodním ujištěním, že básník nevnímal období své nezralosti, a končí zpěvem mantry ,,Puškin je po puškinsku král\dots``, v~níž jsou všechna slova i intonce voleny dle pevného vnitřího řádu. A~je dokonce možné připustit, že v~životě Antonova (antonovec by nás trochu poopravil -- v~době Antonovovy prvosmrti) existovaly nejrůznější odchylky, do současné podoby dospěl postupně.
\pend
\endnumbering
\end{Rightside}
\Columns
\end{pairs}

\begin{pairs}
\begin{Leftside}
\beginnumbering
{\rusky
\pstart
Другой техникой утрупнения является изучение древнерусской культуры -- разумеется, не ее самой, а ее мардонга. На этом пути Антонов высказал интересную мысль, благодаря которой к движению примкнула масса славянофилов. Антонов заявил, что найденный археологами под Киевом горшок VIII века является первым в истории мардонгом, а находящаяся в нем малая берцовая кость принадлежала девочке по имени Горухша -- это слово написано на горшке. После такого патриотического высказывания антоновцы получили государственную дотацию, и их движение заметно укрепилось.
\pend

\pstart
Кроме этих методик, Антонов рекомендует изучение какого-нибудь мертвого \linebreak языка, например, санскрита, а также лежание в гробу.
\pend

\pstart
С момента возникновения секты быт ее членов был подвергнут тщательной ритуализации. Рассказывают, например, что Антонов не терпел, когда при нем огурцы вынимали из банки пальцами -- по его мнению, мертвость овощей осквернялась живым прикосновением. Работа «Ежедневное чудо», где, может быть, рассмотрены эти вопросы, не сохранилась.

Книга «Майдан» (1998) при стилистическом единстве с остальными сочинениями сдержанней и задумчивей и чем-то напоминает суры мединского периода. В ней нет новых идей, но углублены и развиты ранее высказанные -- например, появляется мысль о множественности внутренних трупов, которые как бы вложены один в другой, наподобие матрешек (по Антонову -- древнерусский символ мардонга), причем каждый последующий труп созерцает предыдущий и испытывает по нему ностальгию, первичный внутренний труп тоскует по окончательному, то есть по актуализированному мертвецу -- круг замыкается.
\pend

\pstart
В этой книге Антонов отрекается от тех своих последователей, которые идут на самоубийство, -- он презрительно называет их «недоносками». (В системе Антонова убийство рассматривается как кесарево сечение, а самоубийство -- как преждевременные роды. Смерть в юности уподобляется аборту.)
\pend

\pstart
В 1999 году Антонов достигает актуализации. Его мардонг устанавливают на тридцать девятом километре Можайского шоссе, прямо у дороги.

Он и сейчас на этом месте, и в любое время там можно встретить антоновцев -- это хмурые молодые люди в темных плащах, крашенные под блондинов, с перетянутыми резинками -- чтобы трупно синели кисти -- запястьями. У мардонга читают стихи -- обычно Сологуба или Блока, отобранные по антоновскому принципу «максимума шипящих». Иногда читают стихи самого Антонова:
\pend

\pstart
\ldots а когда догорит свеча,

и во тьме отзвучат часы,

мертвецы ощутят печаль,

на полу уснут мертвецы\ldots
\pend

\pstart
С шоссе открывается удивительный вид.
\pend}
\endnumbering
\end{Leftside}

\begin{Rightside}
\beginnumbering
\selectlanguage{czech}
\pstart
Za druhou techniku, která vede též ke zmrtvolnění, považujeme studium staroruské kultury -- tedy vlastně nejen jí, ale také studium jejího mardongu. V~tomto ohledu vyslovil Antonov jednu velmi zajímavou myšlenku, díky níž se ke zmíněnému hnutí přidala i řada slavjanofilů. Prohlásil, že hliněný hrnec z~8. století, který byl nalezen v~Kyjevě, je prvním historickým mardongem. Byla v~něm také nalezena bércová kost dívenky Goruchšky -- toto slovo bylo zaznamenáno přímo na hliněném hrnci. Po takovémto vlasteneckém výroku získali Antonovci státní dotaci a pozice jejich hnutí tak zjevně posílila.
\pend

\pstart
Kromě uvedených metod doporučuje Antonov rovněž studium nějakého mrtvého jazyka (například sánsktru) nebo také ležení v~rakvi.
\pend

\pstart
Po vzniku sekty se způsob života jejích členů výrazně ritualizoval.  Vypráví se například, že Antonov nikdy svým stoupencům nedovoloval, aby za jeho přítomnosti vytahovali okurky z~láhve přímo rukou. Dle jeho mínění se totiž ,,mrtvost`` zeleniny mohla dotykem živého pošpinit. Bohužel se však nedochovala Antonovova práce \textit{Každodenní zázrak}, která právě o~těchto otázkách pojednávala. Kniha \textit{Majdan} (1998) je zdrženlivější a zadumanější -- i přes svou blízkost a stylistickou jednotu s~ostatními díly připomíná spíše medinské súry. Neobsahuje sice tolik nových myšlenek, ale ideje již dříve vyslovené jsou značně prohloubeny a rozpracovány. Například je tu probírána myšlenka o~množství vnitřních mrtvolných já, která jsou zasazena do sebe a  de facto tak fungují na principu matrjošek (ty jsou dle Antonova staroruským symbolem mardongu) -- každé z~následujících já pozoruje předchozí mrtvolné já s~nostalgií. Primární já stále touží po tom konečném, tedy po dokončeném aktualizovaném mrtvolném já; a kruh se uzavírá.
\pend

\pstart
V~této knize se Antonov rovněž zříká těch ze svých stoupenců, kteří se dopouštějí sebevraždy a opovržlivě je nazývá nedonošenci. (V~antonovském systému se i císařský řez chápe jako vražda, zatímco sebevražda je vnímána jako předčasný porod. Smrt mladého člověka je pro něj rovna potratu.)
\pend

\pstart
V~roce 1999 Antonov konečně dochází uznání. Vzniká i jeho mardong. Hned v~příkopě na devětatřicátém kilometru Možajské silnice.

Mardong stojí na tomto místě dodnes. A~kdykoli chcete, můžete u~něj vidět řadu antonovců -- ponurých mladých lidí v~tmavých pláštích, s~odbarvenými vlasy a s~gumičkami přetaženými přes ruce, tak, aby ruce vyhlížely co nejmrtvolněji. Čtou se tu verše -- často Sologub či Blok, z~nichž jsou však dle antonovského principu vybrány básně s~velkým množstvím šeplavých souhlásek. Někdy se dokonce čtou verše samotného Antonova:
\pend

\pstart
\textit{A~až jim svíce dosyčí}

\textit{a ve tmě zazní hodiny}

\textit{mrtví už hrůzou nekřičí}

\textit{a na zemi spí bez viny}
\pend

\pstart
Z~cesty se tak naskýtá neuvěřitelná podívaná.
\pend
\endnumbering
\end{Rightside}

\Columns
\end{pairs}
