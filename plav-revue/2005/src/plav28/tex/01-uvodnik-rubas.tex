\section{Laureáti soutěže aneb \\ ,,Z těch určitě něco bude!``}

Když si ode mě paní na poště přebírala dopisy pro překladatele, kterým patří dnešní večer, zvolala: ,,Jé, Kostelec nad Černými lesy! Vy to posíláte mojí sestřenici! Co to, prosím vás, je?`` Vida, že příbuzenské vazby boří všechny bariéry včetně listovního tajemství, vysvětlil jsem, že ,,sestřenice`` vyhrála v překladatelské soutěži a že jí posílám pozvánku na večer laureátů. ,,Vážně?`` pravila paní za přepážkou a nadšeně prohlásila: ,,No, vona je dobrá! Z tý určitě něco bude!``

Ta paní na poště v podstatě shrnula ideální výsledek několikaletého úsilí Obce překladatelů zaměřeného na vyhledávání překladatelských talentů, kterému se říká Soutěž Jiřího Levého. Letos proběhla už podvanácté a sešlo se v ní sedmačtyřicet překladů z celkem třinácti jazyků, od běloruštiny po čínštinu, a také dvě práce na téma kritika překladu... 

Nabízíme vám tedy výběr toho nejlepšího, co se v letošní soutěži sešlo -- jako sondu, co si překladatelé vybírají sami od sebe, z čirého zájmu o~věc i jako ukázku, jak na tom je generace těch mladých a nejmladších, kterým patří budoucnost českého překládání... 

Popřejme jim, aby se určitě vyplnilo to malé poštovní proroctví!

\podpis{Stanislav Rubáš}