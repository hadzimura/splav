\section{Jak se z Plav!u stala Plav Revue?!}
Když v září roku 2002 vyšlo první číslo tohoto časopisu, nikdo z těch, kdo se na jeho vzniku podíleli, jistě nemohl tušit, že se o dva a půl roku později objeví v celorepublikové distribuci časopis stejného jména, jen s podtitulem ,,měsíčník pro světovou literaturu``. A že se budou podílet na jeho výrobě. Po delší debatě jsme se rozhodli zachovat i tento menší časopis, který vycházel a vycházet bude jako doprovod \textit{bubeníčkovských překladatelských večerů}. Aby byla zachována názvová kontinuta a nedocházelo k lezení si do jmenného zelí, přejmenovali jsme časopis na \textit{Plav Revue}. Budete se v něm i nadále setkávat s překlady textů, které se na \textit{večerech} objevují, a budete si ho moci i nadále stáhnout v elektronické formě z webové stránky \textit{www.splav.cz}. A má-li tento časopis nenávratně zmizet v propadlišti dějin, stane se tak ve chvíli, kdy nebude žádný zjevný důvod k jeho vydá(vá)ní. 

Právě probíhající \textit{bubeníčci} jsou odrazem překladatelské soutěže Jiřího Levého. V říjnovém čísle časopisu \textit{Plav - měsíčník pro světovou literaturu} se setkáte s výběrem nejzajímavějších textů z Levého soutěže za posledních několik let, v časopise \textit{Plav Revue} nahlédněte do textů z ročníku právě ukončeného. A dobře se bavte.

\podpis{Radim Kučera}