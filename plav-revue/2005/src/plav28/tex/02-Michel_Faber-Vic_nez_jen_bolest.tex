\section{Michel Faber: Víc než bolest}


Morpheus, bubeník přední deathmetalové kapely ze skotského North Ayrshiru, se v den, kdy měli s Corpse Grinder vyjet na východoevropské turné, probudil se zvláštním pocitem v hlavě. Z přemíry zimního světla, které do bytečku dopadalo nezataženým arkýřovým oknem, mrkal a mhouřil oči. Na autech zaparkovaných před domem ležela bílá vrstva nočního sněhového přídělu. Oslnivě, agresivně bílá. Zasněžená auta dřív Morphea nikdy netrápila, ale teď to bylo jinak.

,,Je mi nějak divně,`` řekl maďarsky své přítelkyni Ildiko. Na tom nic zvláštního nebylo. Postel, ve které spolu leželi, se nacházela v Budapešti.

,,No jo, vždyť jsi divnej kluk,`` odvětila Ildiko a střapatou hlavou se mu otírala o~rameno. Pohladil ji pod dekou a trochu ho zmátlo, že už není nahá, ale zabalená v teploučké bavlně. Jeho dlaně, samý mozol po letech bubnování rychlostí 240 úderů za minutu při písních jako ,,Inferno Express`` a ,,Sejdem se v Gomoře``, spočinuly na podivných strupovitých vzorech na jejím oděvu.

,,Co to máš na sobě?`` zeptal se a ona se posadila, aby mu ukázala jeho vlastní černé triko s logem Corpse Grinder, na zádech potištěné stříbrnými písmeny: Evropské turné 2002 – Budapešť, Bratislava, Praha, Wroclaw, Varšava\ldots následovala další města, jejichž jména se v pračce už napůl rozložila.

,,Prát ručně,`` stálo na výrobku, ale na rovinu, kdo byl pral trička v ruce?

,,Sluší ti to,`` pochválil Morpheus. Byla to úleva, koukat na něco tmavého.

,,Jestli chceš, mohla bych ti půjčit hezkou růžovou košilku.``

,,Hahaha,`` zabručel a v duchu si říkal, jestli se ,,hahaha`` maďarsky náhodou neřekne trochu jinak.

Ildiko byla nejvtipnější holka, s kterou kdy chodil. Nebyla zanícenou fanynkou, vlastně se jí Corpse Grinder ani moc nelíbili. Její parketou byl ambient. Ale jeho měla ráda.

,,Mám takovej zvláštní pocit v hlavě,`` řekl.

,,Je to nepříjemný?`` zeptala se a vstala z postele, lem obřího trika jí bohatě zakrýval zadek.

,,TŘESTE SE!`` hlásal slogan pod seznamem vystoupení.

,,Jo, nepříjemný,`` připustil zamračeně.

,,Takže\ldots tě bolí hlava?`` řekla a ze starého litinového topení sundala vyhřáté punčocháče.

,,Mě nikdy hlava nebolí,`` prohlásil Morpheus a ve tváři mu škublo, protože jak si stoupla před okno, oslepil ho prudký sluneční svit, který jako svatozář orámoval její postavu.

,,Tak teď tě asi bolí.``

,,Možná že mám\ldots ten\ldots`` Nevěděl, jak se maďarsky řekne mozkový nádor. ,,Možná že umřu.``

Hodila mu tričko zpátky, aby si mohla vzít podprsenku. ,,Pro začátek zkus aspirin,`` poradila mu.

\begin{center}
* * *
\end{center}

,,Víš, že na drogy nedám,`` pokáral ji. Masivníma, sluncem zalitýma rukama si clonil obličej.

To byl fakt, že Morphea nikdy hlava nebolela. I~když jako teenager žil v Maybole – to se ještě jmenoval normálně Andy Wilkie, nikdy v žádné části těla bolest nepocítil, tedy kromě puchýřů na rukou, když se poprvé přidal k The Unbelievably Uglies (později The U.U., potom Judas Kiss a nakonec Corpse Grinder). 

,,Bolest je iluze,`` říkával. ,,Síla ducha, kámo!``

Taky byl fakt, že Morpheus nedal na drogy. Málokterý z jejich fanoušků to věděl, ale Corpse Grinder byla skvadra neobvykle slušných týpků, kteří už dávno vyhodili a nahradili spoluhráče, jejichž zlozvyky jim znemožňovaly zkoušet všechny ty hudební libůstky a zvládat smrtící tempo jejich koncertů. Když ještě Corpse Grinder působili ve Skotsku, kytarista Neil (alias Cerberus) si čas od času přihnul a basák Charlie (Janus) si někdy na nočních akcích dal éčko, ale teď, když už vyrostli a usadili se v Budapešti, byli čistí jako padlý sníh.

,,Zvláštní, jak se to všechno vyvrbilo,`` říkal vždycky Cerb. ,,Z~Ayrshiru do Maďarska. Tam doma se k nám nikdo neznal, ještě dneska bysme hráli v místní putyce. Tady na nás chodí stadiony.``

Když Cerb začal na tuhle notu, Morpheus se vždycky zvedal. Ve dvaadvaceti byl ještě trochu mladý na nostalgické vzpomínání. Navíc nebyla tak úplně pravda, že na Corpse Grinder chodí stadiony – stadiony objížděli, jenom když se jim poštěstilo dělat předskokana větší kapele, třeba Panteře nebo Metallice. Stejně se to mělo i s touhle jejich východoevropskou šňůrou. Ačkoli na svých tričkách byli jako hvězda uvedeni výhradně Corpse Grinder, ve skutečnosti byli jenom jednou z předkapel Slayeru, těch zasloužilých heavymetalových harcovníků. Celé tisíce východoevropských adolescentů byly připraveny vylézt z úkrytů, aby spatřili Slayery, a při troše štěstí snad zajásají i při Corpse Grinder a koupí si jejich cédéčko nebo tričko (co se má ,,prát ručně``).

,,Možná sis, Morphe, přeležel krk,`` nadhodila Ildiko. ,,Třeba jsi spal ve špatný poloze.``

,,Jo, vedle tebe,`` zašklebil se a zkusil si třít spánky.

,,Přestaň skuhrat,`` řekla, teď už zcela oblečená a v plné práci. ,,Přinesla jsem ti kafe.``

,,Snad ne ten portugalskej šunt z modrožlutý krabice?``

,,Ne, tohle je holandský. Špičková značka. Inferno Espresso.`` Shlížela na něj s kamennou tváří, dokud mu neseplo, že si dělá legraci.

,,Hahaha,`` pronesl.

Po chvíli ho přesvědčila, aby se šli projít na čerstvý vzduch. Jeho ,,mrzuté hlavě``, jak ji diplomaticky nazvala, by kyslík a pohyb mohl prospět. A~tak si oba vzali bundy, rukavice a polské boty lemované kožešinou a vyrazili do ulic v okolí jejího bytu. Morpheus si nasadil tmavé sluneční brýle. Obvykle se takovému pozérství ve stylu populárních rockových hvězd vyhýbal, ale slunce na sněhu bylo pořád děsivě ostré.

,,Ildiko, to máme dneska ale krásně!`` zavolala květinářka Hajnalka.

,,Nádherně!`` odpověděla jí halasně Ildiko.

,,O tom se ve Skotsku taky všichni baví,`` mumlal Morpheus. Očima sledoval chodník, kde podrážky chodců sníh rozhnětly v mnohem koukatelnější břečku. ,,O počasí.``

,,Tak to podle mě určitě bude lidskej povahovej rys,`` řekla, zatímco ho vedla pod plátěné přístřešky pouliční tržnice.

Prodejci tu dnes byli v hojném počtu. Obvyklé stánky nabízely mobily, nemoderní italské kožené bundy, padělané oblečení od Gapu a Adidasu, pirátské videokazety s hollywoodskými filmy, portugalskou kávu v modrožlutých krabicích, kalendáře s Britney Spears a zlevněné cukrovinky; byly tu však k mání i tradičnější tovary: domácí jahodová marmeláda, prasklé žárovky, které si lidi za pár forintů mohli koupit, aby je v práci ,,prohodili`` za funkční, stohy kancelářského papíru, obrovské plesnivé salámy.

,,Nechtěl bys Bounty?`` zeptala se Ildiko, když její pohled dopadl na vachrlatý stůl naložený americkými čokoládovými tyčinkami dováženými přes Spojené arabské emiráty.

,,Je mi\ldots Mám zvláštní pocit v žaludku,`` řekl Morpheus.

,,Jinými slovy – je ti špatně,`` upřesnila Ildiko a sama si koupila marsku.

,,Mně nikdy není špatně,`` trval na svém Morpheus. Zatímco se přehraboval pirátskými cédéčky, posunul si brýle až nahoru pod kapucu. Měli tu výběr největších hitů Slayeru, nazvaný  \textit{The Greatest Hits of Slayer}, a (to snad ne) taky poslední album od Cradle of Filth. Od Corpse Grinder samosebou nic.

,,A ty bys chtěl, aby tu něco bylo?`` podivila se Ildiko. ,,Za černý kopie byste nedostali žádný peníze.``

,,Za oficiální cédéčka jsme taky nikdy žádný peníze neviděli,`` zabručel. ,,Piráti si aspoň platí vlastní náklady.``

Ildiko si nenačatou tyčinku schovala do kapsy u~bundy. ,,Nejdřív si dám něco výživnýho,`` rozhodla.

,,Pojď, skočíme ke Kálvinovi a dáme si \textit{halászlé}.``

,,Nemám hlad!``

,,Nemůžeš být dneska večer na lačno.``

To bylo poprvé, co se zmínila, že má Morph dnes večer koncert, první z dvaadvaceti akcí, našlápnutý rozjezd nejvýpravnějšího turné v historii Corpse Grinder.

,,To má čas,`` pravil Morpheus a jeho zrak upoutal jakýsi leštěný magazín, který vypadal, že by mohl být o~thrash metalu. Ukázalo se, že je to porno pro úchyláky, kteří ulítávají na kůži.

\begin{center}
* * *
\end{center} 

,,Morphe, vem si ode mě trochu polívky,`` naléhala Ildiko, zatímco si do \textit{halászlé} míchala smetanu.

,,Na cokoli z ryb je ještě moc brzo,`` řekl. Světla v kavárně Kálvin byla příjemně tlumená, ale když viděl, jak se v tmavé polévce kolem kroužící lžíce točí světlá smetana, zamotala se mu trochu hlava.

,,Už je jedna,`` připomněla mu. Koncert na hradě měli v 19.30 odpálit tuzemští ctižádostivci Férfiak, Corpse Grinder budou následovat ve 20.15. Jak Morpheus stále bezmocně civěl na smetanu rotující v polévce, dostal náhle vizi, jaká světelná show by se nejlíp hodila k jejich kapele: blikající červené stroboskopy a smyčky omračující bílé záře prohánějící se po pódiu.

,Dneska roznesem všechny na kopytech,`` prohlásil, popadl lžičku a vidličku a na hraně stolu zabubnoval zběsilý tuš. I~takhle na ubruse byl jeho výkon a um nezaměnitelný.

,,Tak už máte vybráno nebo co?`` ozvala se servírka.

\begin{center}
* * *
\end{center}

Morph prošel dveřmi pod vývěsním štítem s nápisem GYOGYSZERTÁR. Klidně se tak mohla jmenovat nějaká východoevropská thrashmetalová nebo gotická skupina, ale znamenalo to ,,lékárna``.

,,Bolí mě hlava,`` oznámil staré paní v bílém plášti za pultem.

,,Mluvte nahlas,`` vyzvala ho, uzlovitou ruku si dala za ucho.

,,Hlava mě bolí,`` zopakoval zahanbeně.

,,Co jste už zkusil?``

,,Nic. Co máte?``

Ukázala za sebe, na stěnu plnou malých kartonových krabiček. Analgetika pro každého muže, ženu a dítě v Budapešti, jak se zdá. Vážně se v patřičném množství lebek skrývá dost bolesti, která by ospravedlnila existenci všech těch pilulí?

,,Aspirin je prej docela dobrej.`` prohodil Morpheus v naději, že se o~něj stará paní postará.

,,V tom případě nemusíte platit nekřesťanské peníze za nějakou vyhlášenou značku.`` Připadala mu laskavá, taková mateřská. ,,Vzadu máme haldy laciného aspirinu. Dostanete jich sto za stejnou cenu jako dvacet od Bayera.``

,,Potřebuju jich jen pár,`` požádal Morph. Říkal si, čím si to zasloužil, že se musí dohadovat s jediným žijícím prekapitalistou v celém Maďarsku.

,,Dám vám jich padesát.`` Usmála se a už si to šinula do skladu, jako by byl drzým chlapečkem v pekařství a ona mu chtěla tajně podstrčit sáček koblih od včerejška.

Za chvíli před ním stála s umělohmotnou lahvičkou a sklenicí vody.

,,Cože, to jako tady?`` zeptal se Morph vyplašeně.

,,Jistě. Co můžeš udělat hned\ldots``

Vyklepl z lahvičky dvě tablety, hodil si je do pusy a rychle je spláchl hltem vody.

,,Vy jste to ještě nikdy nedělal?`` podivila se, když se začal skoro dusit. Zašklebil se a ještě se napil.

,,Brrr,`` odvětil a zavrtěl hlavou.

,,Pracujete v Budapešti?`` Samozřejmě poznala, že je cizinec.

,,Jsem muzikant.``

,,Vážně? Jak se jmenujete?``

,,Ee\ldots Andy.``

,,Z Anglie?``

,,Ze Skotska.``

,,Krásná země. Co vás přivádí do tohohle zlodějského doupěte?``

Malá poptávka po Corpse Grinder ve Skotsku, pomyslel si. ,,Mám tady přítelkyni.``

,,To je hezké,`` řekla a vlídně svraštila koutky očí. ,,Tak to máme sto forintů.``

\begin{center}
* * *
\end{center} 

,,Co hlava?`` starala se Ildiko, když se vraceli do bytu. Drobný déšť se do sněhu zakusoval jako kapky kyseliny. Zaparkovaná auta začala zpod svých bílých přístřešků vykukovat jako velikánské plechové houby.

,,Čím dál horší,`` řekl Morph. ,,Neměl jsem si ty prášky brát. Sí1a ducha, ta je potřeba.``

Když byli doma, nechal Ildiko, aby mu masírovala krk a ramena, zatímco se díval na televizi. Dálkovým ovladačem snížil jas natolik, že všichni vypadali jako černoši.

,,Možná bys měl dát klukům vědět,`` navrhla Ildiko.

,,Dát vědět co?``

,,Že dneska možná nebudeš moct hrát.``

,,Jasně že budu moct hrát. Duch nad hmotou.``

Políbila ho na hlavu. Prsty měla unavené, jak mu hnětla napjaté svalstvo.

,,Hele, koukej!`` zavolal. ,,To je zpěvák z Férfiak!`` Na obrazovce silně potetovaný mladík nějakému novináři povídal, že jeho kapela to na turné hodlá Slayerům pěkně natřít. Pak se před kameru protlačil basák, prostředník provokativně zdvižený, a anglicky zařval: ,,Nakopem je do prdele!``

Morph a Ildiko se pobaveně rozřehtali.

\begin{center}
* * *
\end{center}

V šest už byl Morpheus na cestě na hrad, na dostaveníčko se svými druhy z Corpse Grinder. Autem to bylo dvacet minut. Ildiko seděla za volantem svého pirátského Volva, Morph vzadu, protože přední sedadlo obsadil pečlivě vyvážený a chvějící se průhledný plastový pytel naplněný vodou a tropickými rybami. Exotičtí živočichové pluli v obrovské igelitové placentě sem a tam, voda vibrovala do rytmu s vrčením motoru.

Morpheus teď do sedadla zezadu bubnoval opravdickými paličkami, aby se dal pořádně do kupy.

,,Sejdem sééé v Gomořééééé!`` zpíval monotónně a zároveň se snažil vymlátit z kožené opěrky duši.

,,To těm rybám asi nedělá zrovna dobře,`` zavolala Ildiko přes rameno. Její otec čekal celé měsíce, až tyhle malé krásky dorazí, a nebyl by tedy právě nadšený, kdyby jeho rozhodnutí, že mu je ve městě vyzvedne dcera, mělo za následek, že nakonec obdrží mrtvolky.

,,Prá-á-ávo silnějšííího-o-o!`` notoval si Morpheus titulní skladbu z jejich prvního alba. Ale každopádně už nebubnoval tak agresivně.

Upřímně řečeno, bylo mu pod psa, a i když bil paličkami jen do sedadla, z toho vypětí mu ve spáncích bušila krev. Svíral paličky pevně v rukou, zhluboka dýchal, klouby na rukou zavrtával do kolen. Kožené motorkářské kalhoty, ve kterých byl jindy jako ryba ve vodě, studily a lepily se na nohy. Na bledých holých pažích mu naskočila husí kůže, ale bundu měl na předním sedadle, kde s ní byla podestlaná ta vratká rybí bublina, a představa, že ji bude muset vytáhnout, mu přišla naprosto neúnosná. V minulosti jezdil na koncerty vždycky jenom ve svém speciálním tričku; adrenalin ho hřál, i když auto nemělo topení a venku bylo pod nulou. Dnes klepal kosu.

,,Vážně jsi v pohodě?`` starala se Ildiko, když zhluboka vzdychl, protože zrovna projeli ostrou zatáčkou.

,,Otrava aspirinem,`` zaskuhral. Žaludek a střeva se mu v břiše proměnily v tvrdou gumu, v pevnou hmotu zcela neprůchodné anatomické skulptury. Ohromná temná bolest mu tepala v oblasti za levým okem a obočím. Ukazováčky tlačil do kořene nosu, čím dál silněji, až se zdálo, že by mohly prorazit lebku (což by se tedy, jak musel uznat, na obalu nějaké heavymetalové desky skvěle vyjímalo).

,,Zastav.`` Nepoznával svůj hlas, přiškrcený nasální zvuk přehlušený vrčením motoru a hukotem v hlavě.

,,K našim už je to jenom pár minut.`` Ildiko pohlédla do zpětného zrcátka a zjistila, že těsně za nimi stále jede auto natřískané postadolescentními Maďary, Peugeot po střechu narvaný mladíky natěšenými, jak si zapaří.

,,Budu nejspíš --``

Morph sebou trhl, zběsile stáhl okýnko a vychrlil ven proud horkých zvratků. Vytryskly mu z pusy a z nosu jako pivo z protřepané plechovky, a jak je unášel vítr, udělaly na boku auta dlouhou žlutou šmouhu. Ildiko hladce přibrzdila a sjela ze silnice, aby uvolnila cestu burácejícím vozidlům.

Morpheus horko těžko otevřel dveře a svalil se na všechny čtyři do vlhké, zmrzlé trávy. Zvracel dál, křečovitě dávil, až se zdálo, že se mu samou bolestí rozskočí hlava. Když ho Ildiko vzala kolem zad, dala se do něj šílená zimnice.

,,Ja-jak jsme na tom s časem?`` zalapal po dechu.

\begin{center}
* * *
\end{center}

Morpheus se probudil v tmavé místnosti, obklopený keramickými děvečkami a vyřezávanými soškami sobů. Ležel v posteli -- ne u~Ildiko v jejím voňavém hnízdečku, ale na cizím, rozměrném rokokovém loži, několikapatrovém dortu složeném z přikrývek, vyšívaných přehozů, nažehleného bavlněného povlečení a podušek lemovaných kožešinou. Klidně mohl být dávnověkým válečníkem na pohřebním voru, jenž pluje po temném jezeře těsně předtím, než ho zapálí.
Škvíra ve dveřích vpouštěla do ložnice bledou, starodávnou zář z haly. Rodiče Ildiko byli vždycky bohatí, dokonce ještě než byla chatrná maďarská železná opona netrpělivě roztažena. Obývali úděsný přízemní dům ve vídeňském stylu, který měl tři ložnice a interiér lovecké chaty z devatenáctého století -- nazdobený šlehačkový dort chytrácky ukrytý v krabici od sucharů.
Všude bylo hrobové ticho -- obvykle, když přijel Morph na návštěvu, klokotal z předpotopní hifi-věže Pavarotti nebo Carreras. Posadil se v posteli, načež zjistil, že má na sobě jen tričko a trenky, točí se mu hlava a je slabý jako kotě.

,,Ildiko!`` zavolal potichu.

Téměř okamžitě se objevila ve dveřích, v ruce hrnek ve tvaru veverky.

,,Byl tu u~tebe doktor,`` vysvětlovala. ,,Dostals injekci proti bolesti a něco na zvracení. Že prý migréna.``

,,A koncert\ldots``

,,Už je dávno po. Zoltán z Férfiak za tebe zaskočil.``

,,Zoltán? Kristova noho, ten by měl radši sedět na poště a razítkovat dopisy\ldots``

,,Možná že jo. Ale zaskočil za tebe. Teď už jsou na cestě do Bratislavy.``

,,Do Bratislavy? Jakto?`` Vyhodil nohy z postele a zkusil se postavit, ale zdálo se, že mu na to dvě nohy nestačí.

,,Už je druhej den,`` řekla Ildiko a poodtáhla závěsy. Dovnitř dopadlo nefalšované denní světlo.

Předložky z barvené ovčí kůže na zemi se karamelově a zlatě rozzářily. Morpheus si všiml, že s tímhle se teď už dokáže vyrovnat, že slunce pro něj už je v mezích snesitelnosti. Měl žízeň a byl trochu vyhládlý; vnitřnosti měl prázdné jako kopák.

,,Musím do Bratislavy,`` prohlásil. Na stehně měl strup zaschlé krve, tam, kde kůží projela jehla.

Doktora si vůbec nepamatoval, i když si matně vybavoval, jak dorazili k rodičům Ildiko: extatický štěkot psa, jeho rozpaky, když ho s rukama zaháknutýma kolem jejich ramen museli malinkatí manželé středního věku skoro nést přes práh, fantaskní cesta podél knihovny nacpané Goethem a všelijakými cetkami, vycpané antilopí hlavy, háčkované selské výjevy v týkových rámech, dveře pokoje pro hosty a na nich rozesmátá fotka z promoce, rezervní kulaté plastové akvárko, které mu dali na zvracení, božská úleva, že je v klidu a v teple a potmě.

,,Nejdřív zkus, jestli zvládneš dojít na záchod,`` navrhla Ildiko. Jak tam stála a v rukou chovala porcelánovou veverku, vypadala šťastně a vyrovnaně -- vždycky se ráda podívala na svůj starý pokoj, aby si připomněla, proč teď vede život ve znamení umolousaného minimalismu.

,,Poblil jsem vám v hale koberec,`` vzpomněl si najednou Morpheus.

,,S tím se netrap, to se spraví,`` uklidňovala ho Ildiko. ,,Navíc jsou naši úplně u~vytržení z těch nových cichlid.``

Morph si představil manžele Flepsovy, jak sedí v předním pokoji, Pavarotti byl umlčen, a oni zbožně, téměř v transu zírají na své poslední exotické přírůstky. Zachechtal se a klidně se zase svalil na deku.

,,Vypadáš na tý velký posteli moc roztomile,`` řekla Ildiko, zatímco zpod něj tahala pokrývky a znovu ho jimi přikrývala.

,,Jsem z toho nějak\ldots na kaši,`` zabrumlal, ale takovýhle obrat v maďarštině určitě neměli, protože Ildiko opáčila: ,,Myslím, že tu naši žádnou nemají. Co takhle trochu kuřecí polívky?``

,,Ne, dík\ldots jenom kafe.``

,,A co třeba čaj ve veverčím hrnečku? Drahý anglický čaj, co můj statečný tatínek zakoupil v roce 77 na černém trhu. Od té doby zraje v plechovce a čeká jen na tebe, až se zastavíš.``
Znaveně zavřel oči. 

,,Moje kapela mě opustila,`` pravil hlasem až ze samých hlubin Hádu.

\begin{center}
* * *
\end{center}

Do večera byl Morpheus na nohou, i když trochu vratkých. Za přítomnosti milých a pozorných Flepsových, přehnaně přátelského vlčáka a zdárně začleněných tropických ryb mluvil po telefonu s Cerberem.

,,Slayeři sou v poslednim tažení,`` oznamoval nadšeně Cerb se svým silným ayrshirským přízvukem. ,,Sou to starý dědkové. Naděláme z nich fašírku.``

,,A co já?``

,,Tobě necháme poslední ránu. Doraž za náma, až budeš moct. Sejdem se v Gomoře!`` Cerb šílel, nakopnutý adrenalinem. Zrovna se chystal vylézt na bratislavské pódium a mluvil, jako by ho ohodila obrovská vlna fanouškovského obdívu nebo se nechal zlákat dealerem kokainu.

,,Jak si vedou?`` zeptala se Ildiko, když zavěsil.

,,Nevím,`` řekl. ,,Asi dobře.``

Morph dostal hlad, ale nechtělo se mu prodělat večeři o~třech chodech pod bedlivým dohledem jejích rodičů, psa, dvou El Grecových svatých a vycpané liščí hlavy.

,,Vezmu tě někam na jídlo,`` zašeptala Ildiko a rodiče ujistila, že ,,Andreas`` potřebuje na chvíli na vzduch.

Vyšli z domu a oslněni světly nad vchodem váhavě klopýtali, dokud nenašli pevnou půdu pod nohama na hlavní, měsícem osvětlené ulici. Morphovy nohy fungovaly, jako by si je byl právě pořídil a ještě se nerozchodily. Pohlédl na oblohu, která tu byla jasnější než ve městě. Hvězdy tu tvořily neznámé vzory, úplně jiné než nad domem jeho rodičů v Ayrshiru.
Na konci ulice byly potraviny, zavřené, a hospůdka s názvem Blaha. Vešli dovnitř a usadili se u~stolu, čímž zdvojnásobili počet lidí, kteří sem přišli povečeřet, i když tu byla ještě hrstka těch u~piva a vína. Trio místních muzikantů -- kytara, akordeon a bicí -- hrálo osekané verze popových standardů. Po příchodu Morpha a Ildiko usoudili, že se demografické složení v taverně Blaha změnilo natolik, aby mohli přejít od Abby k U2. Dívka s kočičíma očima, s níž Ildiko chodila do školy, se přiloudala pro objednávku.

,,\textit{Somlói galuska},`` řekla Ildiko, aniž se podívala na jídelní lístek.

,,\textit{Bácskai rostélyos},`` pravil Morph po chvíli přemítání.

,,To myslíš vážně?`` zašeptala mu Ildiko. ,,Tak brzy po tý šavli?``

,,Duch nad hmotou,`` ušklíbl se.

Než jim přinesli jídlo, hrála kapela U2, pak se sestava ztenčila na duo a spustili ,,Nothing Compares 2 U``. Morpheus pozoroval, jak si Ildiko do dokonalé pusinky nabírá vanilkový dort se šlehačkou; Ildiko pozorovala, jak Morpheus hltá roštěnou s rajskou omáčkou.

,,Vadí to vašim,`` pronesl mezi dvěma sousty, ,,že jsme se ještě nevzali?``

,,To víš, že jim to vadí, ty pitomče,`` odvětila a dlouhým růžovým jazykem si z prstů slízla moučkový cukr.

,,Tak se vezmeme,`` řekl.

Starší osazenstvo na baru radostně zajásalo. Sem tam někdo zatleskal. Morpheus usoudil, že tím dávají muzikantům najevo své uznání.

,,Ty blboune pitomá,`` povzdechla Ildiko a zavrtěla hlavou. ,,Hele, zatancujeme si.`` A~zvedla ho ze židle.

,,Mám z těch prášků pořád motolici,`` zasyčel ji do ucha, když si ho přitahovala k sobě. Akordeonista zahrál roztřesený tuš. 

,,Podržím tě, dokud nezabere ta roštěná,`` pošeptala mu.

Kapela, opět v plné sestavě, spustila waltzovou verzi skotské ,,Loch Lomond``.

,,Na takovouhle věc já neumím tancovat,`` zamumlal jí ustrašeně do vlasů.

,,Jenom se mě pevně drž,`` říkala mu přímo do toho ucha, které nebylo tak ohluchlé. 

,,Zavři oči a dělej, jako že si ke mně leháš.``


\podpis{z~angličtiny přeložila Lucie Sedláčková} 
\podpis{3. cena v~kategorii próza} 

