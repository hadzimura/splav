\section{Vasil Bykav: Zeď}

\noindent
\textit{Věnováno Ryhoru Baradulinovi}

\smallskip

Zeď byla z kamene, začouzená a velmi stará, jako celá tahle vězeňská pevnost, kterou zdejší diktátor zdědil po svém předchůdci -- a ten ji zdědil po ještě starších diktátorech. V této zemi diktátoři nikdy nevymřeli. Kdysi dávno se tu konaly hostiny, tryzny nebo se zde strojily úklady a různé jiné zločiny -- proti vládě, proti lidu nebo proti tomu, kdo měl v nikdy nekončícím boji o moc zrovna navrch. Nebo kdo zrovna prohrál. A teď tu seděl on a jemu podobní. Ostatní tu možná byli zavřeni již delší dobu, on se zde ocitl teprve nedávno. Poslední roky a měsíce svého života strávil na jiném místě, které se však velice podobalo tomu zdejšímu, takže ho změna žaláře nijak zvlášť neznepokojovala -- tady jako tam pro něj byla uchystána kamenná šachta, vlhká a temná, pět kroků na délku a tři kroky na šířku. Železné dveře na závoru, páchnoucí kýbl v rohu místnosti, stejné jako všude. Jediné, co ho zde zaujalo, byla zeď. Ta naproti dveřím.

Kdekoli a kdykoli byl zavřený, vždycky ho zajímaly zdi -- to, jak a z čeho jsou postavené. Především pak, co je na druhé straně, za zdí. Obvykle tam byla vysněná, ale nedosažitelná svoboda, po které nikdy nepřestával toužit. Zdi byly postaveny poctivě, tak, jak to kdysi uměli: kámen, vápno a cihly. A všechno dokonale tvrdé a pevné, ani kousíček se neodlomí. Ani klepání není slyšet. Hned první den zkoušel obvyklou vězeňskou morseovkou vyťukat na zeď zprávu, žádnou odpověď však nezaslechl. Možná, že za zdí nikdo nebyl. To by znamenalo, že tou zdi tahle kamenná věznice končí a začíná svoboda. Že se tam na slunci zelená nějaký trnový keř, tráva nebo že u zdi rostou věkovité stromy zámeckého parku\ldots

Klekl si na zem vedle svých roztrhaných hadrů a přiložil ucho ke stěně. Bylo naprosté ticho. Za dveřmi před chvílí přešel dozorce a jeho kroky již pomalu dozněly. Slabá žárovka vysoko pod tmavým stropem stěží rozptylovala šero u kamenné podlahy. Nastávala hluboká vězeňská noc. A on se pomalu sunul po kolenou podél zdi naproti dveřím a začal ji ohmatávat, ohledávat, zkoumat.

Masivní kamenné kvádry ve spodní řadě byly k sobě postaveny natěsno a tak pevně, že jakékoli zaklepání -- kloubem nebo pěstí -- se v nich ztratilo. Stejné to bylo o řadu výš. Na kamenech bylo sice plno rýh, vyrytých zřejmě člověkem, ty mu toho však moc neříkaly. Mohl to být klidně nějaký nápis, ale v pološeru by stejně nic nerozluštil. A ani se o to nesnažil. K čemu by mu byly záznamy cizích osudů, když měl dost toho svého -- nešťastného, plného utrpení a nadmíru nepovedeného. Těžko si už ve svém osudu dokázal představit něco jiného než tuhle kamennou noru. Nic jiného už nejspíš nikdy nebude a jeho život tu nakonec skončí. Bez sebemenší naděje.

Přitom tolik toužil po lepším, po nějaké změně. Jenže jaké? Bylo mu vcelku jasné, že nemá v co doufat, že se musí smířit s nejhorším. A s tímhle vědomím musí dožít až do konce Bohem a soudci vyměřený trest. Stejně však nedokázal vzdorovat naději, která na něj, jak se zdálo, vytrvale a neodbytně působila.

Zeď byla studená a tvrdá. Postavená ze samých drsných žulových kvádrů s nánosy vápna, časem dočista zkamenělého. Pravda, místy se daly nahmatat i měkčí kousky vápence. A jenom v koutě, ve vlhkém, ba mokrém koutě u země\ldots Kámen a tvrdost vápna tam podléhaly času a patrně i stálé vlhkosti a stěna se v těch místech zdála být trochu měkčí. Samozřejmě ne tak, aby se dalo něco vyrýpnout, nějaký kousek. Ale kdyby si dal tu práci\ldots Musí se o to pokusit.

Zkusil to. Nejprve nehty, ty si ale ihned všechny ulámal, potom bříšky prstů. Prsty kámen pochopitelně příliš neodrolil, ale přeci jen něco\ldots

Svitla mu naděje. Jeho život jako by dostal jakýsi zatím nejasný smysl. Nemyslel již tolik na minulost a dokonce již ani nemyslel jen na utrpení. Počkal vždy, dokud za dveřmi neutichly kroky dozorců, a hned se vrhl do kouta, sedl si na bobek a rýpal, dloubal a škrábal. Ani se nemusel dívat pod ruce, nepotřeboval světlo, jeho prsty dobře znaly všechny detaily kamene, který se ale pořád zdál tak tvrdý, že jej žádné jeho úsilí nemůže zdolat. V rohu, v koutě to šlo poněkud snadněji, kámen zde byl měkčí nebo co. Právě tady se nyní koncentrovala jeho vůle a všechny za ta léta nenaplněné naděje.

Každý den spal jen pár hodin před budíčkem. Schoulil se do hadrů u nejsušší části zdi a trošku si zdříml, dokud k němu zdáli nedolehl zvuk gongu -- znamení, že má vstávat. Počkal na snídani, misku čočkové kaše, kterou snědl během chvilky. Zbytek dne a noci trávil ve svém koutě a jako nějaký červotoč zpracovával kamennou zeď. Na reliéfu stěny se začínal rýsovat první cípek kamene. Možná jej již brzy bude moci zkusit vytáhnout a po něm snad zůstane malá škvíra. Kdyby tak byla stěna tenčí. Stačila by jen úzká skulina a protáhl by se. Jako se vždy protáhl plotem, když byl ještě chlapec, uličník, a lezl sousedovi do zahrady. Starší kluci ho posílali prvního, protože byl nejhubenější a nejmenší a prolezl všude, kde to jen šlo. A tam, kde to nešlo, prolezl také. Od těch dob si pamatoval, že když projde hlava, projde i celé tělo. Takový to byl čiperný uličník. Dokud nevyrostl a nezačal se zajímat o politiku. V politice to bylo jinak. Stávalo se, že neprolezl nikdo. Ani on ne. Proto jej chytili.

Nepočítal dny a neměl představu, kolik jich uběhlo od té chvíle, kdy se pustil do své nekonečné práce. Den se pro něj dělil na budíček, jídlo -- dvakrát denně -- a dozorcovy kroky za dveřmi -- tam a zpátky. Měl cíl a s cílem se zrodila netrpělivost, která ho záhy spalovala tak silně, až se jí sám lekal. Bylo mu jasné, že tak to nejde. Má-li se mu něco podařit, nepůjde to zdaleka tak rychle, výsledek se dostaví až po nelehkém zdolávání kamene a času. K tomu potřebuje dny, měsíce a možná i roky. Musí se obrnit nesmírnou trpělivostí a schopností přemáhat čas. A tak se obrnil. Naučil se pracovat pomalu a systematicky, nespěchat. Nebylo snadné si na to zvyknout, osvojit si onen klid, ďábelské sebeovládání, a učinit z něj svou druhou přirozenost. Ale snažil se, poněvadž jinak by neměl naději. Netrpělivost hubí naději, tím si byl jist. Naděje někdy skomírala, jindy ho zas rozpalovala k nevydržení. Neustále se zaobíral vyčnívajícím růžkem kamene, když najednou z druhé strany za zdí zaslechl jakési zvuky, známky života, jakoby pískot ptáčete. Vnitřním zrakem již viděl ptačí hnízdo, někde blízko, ve větvích trnkového keře, malinká ptačí mláďata se žlutými zobáčky, a považoval to za velmi dobré znamení. Zmocnila se ho zoufalá netrpělivost. Teprve logickou úvahou se mu podařilo netrpělivost trochu zchladit. Pískot přeci vůbec nemusel patřit ptáčeti, ale například myši. Mohla zde být obyčejná díra plná odporných vězeňských krys. Ale i myší díry musí vést někudy ven ze zdi. A možná i na svobodu.

Jednoho rána po dlouhé hluboké noci, kterou téměř celou probděl ve svém temném koutě, začal cítit, jako by se ten cípek, růžek kamene pod jeho prsty, pohyboval. Ne moc, avšak jeho citlivé prsty přeci jen zaznamenaly neznatelný pohyb -- dolů a nahoru. To mu dodalo ohromnou energii. Napnul síly, zabral, ale v ten okamžik se zarazil: dnes ještě ne. Dnes ať ještě zůstane vše při starém. Úlomek musí vytáhnout z kraje noci, ne v jejím závěru. Na počátku noci bude mít před sebou přeci jen mnohem více času a možností. Po ránu navíc obvykle chodí dozorci.

Před příchodem dozorců byl celý nervózní z toho, že by mohli v jeho tváři něco postřehnout, nějaké oživení nebo stopy naděje. V cele bylo sice temno, ale on si uvědomoval, že ve výrazu jeho zarostlé tváře, do které se dozorci pokaždé, ze zvědavosti nebo z povinnosti, pozorně dívali, se mohlo cosi odehrávat. Snažil se proto nasadit co možná nejklidnější výraz, soustředěný kamsi dovnitř, zastřený, uzavřený pohledům zvenčí. Sám nevěděl, jak se mu to podařilo, nicméně dozorci si zatím, zdá se, ničeho nevšimli. Už jen to znamenalo významný úspěch. Úspěchů však teď bude potřebovat mnoho, prahl po nich. V prvé řadě prahl po úspěchu ve vlhkém koutě.

Osud mu byl očividně nakloněn. Brzy se mu podařilo vytáhnout růžek kamene -- úzký, s ostrými hranami. Dále však pokračoval tvrdý, drsný a studený kámen. Ten růžek se dal přinejmenším použít jako nástroj, mohl jím rýpat do dalších kamenů. Ihned se do toho pustil a rýpal, odlamoval a drhl kámen o kámen. Tlouci nemohl, bylo nutné se pokud možno vyvarovat sebemenšího hluku, aby nikdo ani za dveřmi ani tam za zdí nic neslyšel. Pořád, stejně jako dřív, nevěděl, co ho tam čeká. Zda svoboda, nebo třeba další vězení. Naděje mu však vytrvale vykreslovala jasný obraz svobody: město, setkání s přáteli, s ní\ldots Stačí málo -- překonat tuhle kamennou zeď.

A tak dál rýpal, tiše, ostražitě, a přitom se neustále zaposlouchával do zvuků a šelestů za dveřmi. Co chvíli se napřímil a přesunul se mezi své hadry do druhého kouta. V noci již bylo klidněji, v noci jej dveřmi skoro nekontrolovali, mnohem více se dívali zvečera. Nad ránem byli dozorci zřejmě unavení a chodili méně a pomaleji. Jednou ho napadlo, že by si mohl posunout postel blíž k rohu, aby mohl pracovat vleže. Byl by tak méně nápadný. Jenže jak to udělat? Když posune postel o metr, okamžitě si toho všimnou. Co kdyby ji posouval po menších kouscích, jen pár palců denně? Posunul se. Nejprve o tři palce, pak o pět. A ještě kousek. Kdykoli přišli dozorci, sledoval nenápadně, zato pečlivě jejich pohledy -- zda si něčeho nevšimli. Ne, zatím, zdá se, ničeho. Nebo že by snad tak dovedně tajili svá podezření? Co když ho prohlédli a teď jen čekají, až ho přistihnou v úplném závěru, v posledním okamžiku? Aby ho chytili zezadu za nohy, až se bude prodírat na svobodu! Hrozná představa.

Ke konci léta už měl postel posunutou až těsně do rohu. To, že léto skončilo, poznal podle teplejšího oblečení dozorců. Když některý z dozorců přišel zvenku, byla z něj cítit svěžest podzimního chladu a na uniformě se mu leskly kapičky deště. V cele se přitom teplota neměnila, trochu chladno na léto a trochu teplo na zimu. Žádné topení tu nebylo.

Po nějakém čase se mu podařilo vydrolit mezi kameny úzkou skulinu na čtyři prsty. Byla tam skutečně chodbička nebo možná myší díra, úzká a patrně i hluboká. Nasypal do ní všechno smetí ze zdi, odrolené během noci, a smetí zmizelo a nezanechalo po sobě ani stopy. Ráno, před budíčkem, zakrýval pečlivě otvor kamenným odštěpkem. Dobře pasoval na své původní místo a dokonce ani zblízka nebylo patrné nic podezřelého.

Ještě pořád jako by se bál uvěřit ve svůj úspěch. Jedinkrát částečně povolil v ostražitosti a málem už mohla být celá věc prozrazena. Na vině byly jeho ruce. Dozorce, postarší muž s nezdravou pletí, když mu jednou přinesl jídlo -- misku čočkové kaše --, utkvěl zrakem na jeho rukách. Postřehl to, ale nestihl již ruce schovat, už v nich měl misku. „Zedník, co?“ zeptal se dozorce a pokývnutím ukázal na jeho ruce. „Zedník,“ odpověděl co možná nejklidněji, ačkoli uvnitř byl celý napjatý nečekanými obavami: snad se nedovtípil? Ale ne, nedovtípil se, neboť dozorce jen se zjevným pochopením odvětil: „Taky jsem dělal zedníka.“ Když pak zůstal o samotě, pozorně si prohlédl své zdrsnělé ruce plné zatvrdlých mozolů. Takhle nikdy dříve nevypadaly, na svobodě fyzicky skoro vůbec nepracoval. Zedník se z něj stal až tady, v cele. S tím rozdílem, že on nestavěl jako obyčejní zedníci, ale boural starobylou kamennou zeď. A možná to bylo těžší než stavět. K čemu všemu se člověk neodhodlá kvůli vidině svobody. To člověk pochopí jen tehdy, když ji, svobodu, navždy ztratí.

Snad by bylo lepší, kdyby raději zahynul, zemřel, kdyby ho pochovali spolubojovníci a ona by si poplakala a spolu s ním by pochovala i svou lásku. Bylo by to lepší, jenže se tak nestalo. Dopadlo to hůř a teď ho tížila nejistota, neukončenost, nedořečenost. Onoho jarního večera se loučili, jak se domnívali, jen na pár hodin. K půlnoci se slíbil vrátit do jejího podkrovního pokojíku na kraji městského parku. A ona -- to si pamatuje -- se ani nechystala k spánku. Řekla mu: „Budu čekat.“ Ani se s ním pořádně nerozloučila, jen se k němu lehce přitiskla, otevřela dveře -- a běž! A on šel. Spěchal, neboť měl jít až na druhý konec města, času bylo málo a on měl zpoždění. Kdyby tak tušil, kam tolik spěchá! Měl za to, že spěchá pro dobro jejich svaté věci, za přítelem. A zatím na něj čekala podlá zrada. Neštěstí jej stihlo nečekaně, ani nemohl zanechat své milované sebemenší vzkaz. Zmizel navždy. Kdekdo si mohl pomyslet, že prostě utekl -- před druhy, od ní, pryč z města. A ještě k tomu, že zradil. Je možné, že všechno nastražili jejich nepřátelé, tak, aby si jeho druhové a ona mysleli, že je prodal. A ona ho teď nenávidí. Copak lze milovat zrádce? On jí navíc dosud neřekl to hlavní -- pořád o tom přemýšlel, chystal se to vyslovit a znovu to odkládal. Buď nebyla vhodná chvíle nebo příhodný čas. Říkal si: potom. Ne. Musí se stůj co stůj dostat na svobodu.

Za celé ty roky, které strávil v samovazbách nebo ve společných celách, ani nezadoufal, že by se někdy mohl dočkat svobody. Ovšemže mohl očekávat amnestii nebo čekat, až zemře diktátor. Jen\-že v této zemi diktátoři neumírali často, ti měli železné zdraví, jako koneckonců všichni diktátoři. Dříve umře on, jeho zdraví nebylo nijak pevné. Venku jej podněcoval a držel na nohou boj za velkou myšlenku. Tady pak zeslábl, často míval kašel a pálilo ho na prsou. Jakmile mu však svitla ona nová naděje, cítil se hned lépe, jako by chytil druhý dech, a kašlal už jen, když ležel na posteli. Pokud se pachtil v koutě, nezakašlal ani jednou. Nemohl si to zkrátka dovolit.

Jedné hluboké noci, když za dveřmi prošel dozorce a jeho kroky utichly, nahmatal u díry něco nového. Novou hranku kamene, růžek, za který se dalo vzít. Zatím jen jednou rukou, ale když se bude ještě chvíli namáhat, možná to půjde i oběma. Tento objev jej nadmíru vzrušil. Byl to plod, nové možnosti, vzrůstající naděje, přiblížení vytouženého cíle.

Nesměl ovšem nic uspěchat. Vytáhnout ten kámen se mu hned tak nepodaří, na to měl prostě málo síly. Nejprve bylo potřeba vydrolit a rozšířit skulinu, případně vyrýpat vedle druhou. Ostatně zřejmě přecenil své možnosti, velmi se potil a padala na něj únava. Zvláště nad ránem. Bylo to pochopitelné, vždyť vlastně vůbec nespal. Nějaké dvě hodiny zvečera nebo před budíčkem nestačily na to, aby si odpočinul a znovu nabyl síly. Svou porci kaše snědl vždy celou, ale najednou mu to bylo málo. Když požádal o přídavek, jen se na něj udiveně podívali: co tak najednou? Tak raději zmlkl. Musel se spokojit s tím, co mu dali.

Od chvíle, kdy nahmatal hranku kamene, myslel stále častěji na ni. Když ve svém koutě rýpal, škrábal a drolil, vyvstávala mu před očima její půvabná tvář. Zjevovala se mu ve snu během krátkého spánku, a pokaždé to byl obraz smutný a hořký. Nikdy veselý, radostný, jak ji často vídal ve skutečnosti, zejména když byli sami v jejím podkroví. Venku za dveřmi šustil v korunách stromů vítr, do oken nahlížely sukovité větve javorů a jim bylo tak dobře jen ve dvou. I beze slov. Mlčky se věnovali jeden druhému a starali se o společnou svatou věc. Teď si jen přál jí říct, že ji miluje a že není žádný zrádce, ale oběť. Ať ví všechno po pravdě. Aby o něm nesmýšlela špatně, to si nezaslouží. Kdyby jen věděl, jak všechno nakonec dopadne, choval by se tehdy jinak. Byl by stokrát lepší, odvážnější, rozumnější. Každý den, kdy byli spolu, by jí říkal, že ji miluje. Jako nikdy nikoho nemiloval. Kdyby to jen věděl! Jenže to nevěděl a nic neřekl. A to nevyřčené slovo ho velmi trápilo. A teď má naději\ldots

Doposud šlo všechno poměrně hladce. Žalářníci si zatím ničeho nevšimli a nic nepoznali. Z vlhkého kouta vanul chlad, kameny byly ledové, až z nich mrzly ruce. Podle toho usoudil, že už mnoho nezbývá. Několik posledních kamenů -- a svoboda! Jednoho dne, zrovna když ležel na posteli, zaslechl nějaké hlasy. Odtamtud, zpoza zdi. Docela blízko jeho kouta. Slovům nebylo rozumět, ale byly to hlasy lidí na svobodě. Svoboda byla nesmírně blízko! Právě v té chvíli cítil jistý neklid, který se objevil již po ránu. Za dveřmi se rozléhal křik, zřejmě někoho bili. Posléze se křik ztlumil, asi jak křičícího zavřeli do cely nebo co. Seděl, napínal uši a vyčkával se zatajeným dechem. Měl dojem, že se to nějakým způsobem může týkat i jeho. A taky že ano. Najednou se dveře jeho cely doširoka otevřely, dovnitř se nahrnula celá tlupa žalářníků, s nadávkami jej vystrčili na chodbu a začali prohledávat celu. Prohledali jeho hadry, prohmatali zdi. Sžírala ho zlá předtucha, vypadalo to, že něco vědí. Dozajista zbledl ve tváři a bál se, že si toho všimnou. Naštěstí si zřejmě ničeho nevšimli. Postrčili ho na rozházené zmuchlané hadry a vypadli. Tu noc nepracoval, raději jen čekal a naslouchal. Noc byla neklidná, skoro až do rána za dveřmi chodili a pobíhali žalářníci, opět bylo slyšet křik, nadávky, výhrůžky. Teprve nad ránem se vše uklidnilo a utichlo. Před budíčkem si ještě trochu zdříml.

Příští den vypadal klidněji a on, ačkoli byl pořádně vyhladovělý, se zase pustil do díla. Snažil se tentokrát rýt do hloubky, na tloušťku stěny, a ne rozrývat škvíru do šířky. Jakmile bude moci sáhnout hlouběji, ještě navíc oběma rukama, může se mu podařit vytáhnout nebo vyviklat větší kámen a vznikne průlez. Třeba sebemenší otvor. A už by se protáhl. Hodně už toho vyryl, jenže nevěděl, kam s odpadem. V koutě ho nechat nemohl, to by okamžitě zpozorovali. Jednou dvakrát vysypal odpad do kýble, který vynášeli vězňové pod dohledem dozorců. I to bylo riskantní. Či ho snad měl sníst? Zatím sypal odpad do bot, co měl postaveny stranou vedle dveří. Byly sice přímo na očích, ale v šeru nebylo vidět, že v nich něco je. Sám chodil bos. Prý že mu je v botách horko.

Jednoho rána, ještě před snídaní, když pohlédl do svého kouta, málem ztratil vědomí -- v šeru matně probleskovala uzounká krátká skulina. Jako světlá čárka na temně černé zdi. V obavě, že by si jí mohli dozorci všimnout, hodil do kouta svou vězeňskou halenu a světlou čárku zakryl. Dozorci si sice ničeho nevšimli, každopádně mu ale bylo jasné, že už má nejvyšší čas. Musí co nejrychleji prásknout do bot. Jinak je vše ztraceno.

V noci dosáhl dalšího úspěchu. Podařilo se mu rozšířit skulinu tak, že jí mohl protáhnout obě ruce a uchopit kámen. Zbývalo jen vytáhnout jej k sobě. Podaří se mu ho vyrvat? Bude mít dost sil? Sil měl přeci jen málo. Začal tedy kamenem nejprve viklat, tahal a tlačil jej ze strany na stranu, odshora dolů. Zabralo mu to dost času, ale výsledek stál za to. Bylo zřejmé, že pomalu zdolává kamennou masu -- svými slabými lidskými silami a mohutnou touhou po svobodě.

Kámen nakonec vytáhl, i když se mu z toho vypjetí míhaly barevné kruhy před očima a sotva popadal dech. Ve zdi zůstal úzký a hluboký otvor, z něhož do cely zavanul chlad a temno bezhvězdné noci. Neměl času nazbyt. Do rána zbývaly jen minuty, tušil, že to nestihne. A v posledním okamžiku se může stát neštěstí.

Opatrně vrátil kámen na původní místo, dlouho jím otáčel, dokud nezapadl přesně tam, kde po léta ležel. Škvíry zakryl svými hadry. Ani zblízka by nemělo být nic vidět. Jen aby nedělali prohlídku! Jestli bude prohlídka, je se vším konec.

Ten den byl jako na trní. Chvilku poseděl, hned zas vyskakoval, přecházel sem a tam po té protivné cele. Dozorce, který mu nesl kaši, měl na sobě vlhký plášť a promočené boty. Znamenalo to, že tam, na svobodě, prší. Pravděpodobně začínalo jaro. A stále ho neodhalili. Už jen to byl úspěch, nevídaný, nebývalý. Hrůza pomyslet, kdyby se všechno pokazilo v samém závěru. V posledních okamžicích. Už aby se dočkal noci\ldots

Konečně nastala noc. Žalářníci se chovali jako vždy, přinesli večeři, vyměnili kýbl. V jejich chování ani v jejich pohledech nebylo patrné nic, co by jej mohlo znepokojovat. Zajisté by si něčeho všiml. Jeho oči a uši byly teď napjaté jako struny, jako laserový paprsek, nic jim neušlo. Mnohé teď navíc už nezáleželo na něm, ale na druhých. Tak dával pozor na ty druhé, na své věznitele. Vše, co záviselo na něm, již vykonal. Dokonce víc, než doufal.

Spěchal. Večer ani nemohl dojíst svou porci chleba (vázla mu v krku) a nevěděl, kam ji schovat. Nesnědený chleba mohl vyvolat podezření a prozradit ho. Strčil jej do bot, zpola již zaplněných odpadem ze stěny. Sám byl již zvyklý chodit bos. Jak se mu půjde bez bot tam, na svobodě, o tom ani nechtěl přemýšlet. Hlavně aby se dostal ven. Hned jak se setmělo a za dveřmi utichl veškerý ruch a hluk, poklekl na kolena a pokřižoval se. Nebyl sice nijak zvlášť nábožensky založený, ale přeci jen\ldots Pro tak obrovský cíl se hodí všechny prostředky, ďáblovy i boží. Třeba pomohou. Odvalil stranou kámen, shrbil se, předklonil se -- hlava se zatím vešla. Pořádně si přitom odřel ucho a sedřel si kůži na čele. Prolézt tělem však bylo náročnější. Hlavně nemohl protáhnout rameno a ruku, ať jí v tom kamenném tunelu kroutil všelijak. Chvíli se vrtěl a natáčel, až se nadobro zasekl a zhrozil se, že nemůže ven ani zpátky. Nakonec se mu to nějak podařilo, protáhl ruku i hlavu a přitom se pod košilí odřel do krve. Horší bylo, že ani potom to nešlo snadněji. Pomalu mu docházelo, že asi nemá dost sil na to, aby se protáhl na jednu nebo na druhou stranu. Chvilku ležel jen tak, zaklesnutý mezi kameny, a nehýbal se. Tehdy si zřetelně uvědomil, že je s ním konec, pokud nenajde sílu k poslednímu náporu. Zmocnil se ho strach. A strach jako by mu dodal síly, jako by ho přinutil se rvát. Uchopil rukou cosi vpředu a o kousek se posunul. Znovu se vzepjal a ještě o kousek se posunul. A ještě. Nakonec vytáhl z kamenného tunelu hrudník a dále už to bylo snadné.

Ocitl se venku, u zdi, na mokré studené dlažbě, kde mezi kameny prorůstala jemná tráva.
Venku byla tma, chladno a klid. Nějakou dobu ležel nehybně na dlažbě, oddychoval, neschopen se zvednout, aby se posadil. Po chvíli zvedl hlavu. Proti černému nebi se tyčila šedá kamenná stěna, kolem které se táhla další zeď -- zřejmě jedna z věží této staré budovy. Stěna ubíhala do strany, kolem jeho díry, která mu jako černá skvrna zela nízko u nohou. Kolem něj byla noc a svoboda -- konečně jí dosáhl. Proč ale necítil radost?

Něco mu neustále kazilo radost z nabyté svobody. Možná jakási temná nejistota z okolí? Kde to vlastně je? Ulice, dvorek, zanedbaná zahrada? Nikde neviděl žádné stromy. Vždyť přeci podél staré vězeňské zdi musely nějaké růst! Místo stromů se do výše strmě vypínala černá věž a za ní se ve tmě rýsovalo šedé pokračování stále stejné zdi. Se zlou předtuchou se postavil na nohy, vrávoravě popocházel podél stěny a ohmatával ji zakrvácenou rukou. Bosýma nohama našlapoval na kluzkou studenou dlažbu. Brzy došel až do rohu, kde vysoká, třípatrová zeď uhýbala. Zabočil podél ní a skoro se dal do běhu. Stále rukou ohmatával hrubou omítku stěny, čekal, že narazí na nějaká vrata, bránu nebo východ. Avšak marně. Chvíli běžel a skoro se uhodil o výstupek nerovného kamenného zdiva, za kterým se cosi černalo. Byl to široký a vysoký výklenek a v něm východ z této kamenné pasti. Ano, ve výklenku se černaly obrovské dveře s železným kováním, ale ani podél dveří ani pod nimi neprosvítala jediná skulinka. Nebyla na nich ani závora. Jemně na dveře zatlačil, opřel se do nich ramenem, ale ani se pod jeho slabou silou nepohnuly. Vrhl se na druhou stranu. Do něčeho narazil a upadl na dlažbu. Rozbil si koleno. Vstal a vzápětí strnul v němém úžasu. Nad ním se proti temně šedému nebi rýsovaly známé obrysy šibenice. Přímo nad jeho hlavou visela oprátka. Pod ní stály dřevěné schůdky, o které se prve ve tmě uhodil. Nebylo úniku.

Zdrcený a vysílený klesl tiše na zem. Dostal se do slepé uličky, spadla klec. Naděje, která jej takovou dobu vedla, mu před očima pohasínala, až náhle skonala. On přitom ještě žil, ale už jen částí své bytosti. Žít ve vězení nemělo smysl, zemřít nebylo o nic rozumnější. Lépe by bylo vůbec se nenarodit. Jenže to bohužel nezáleželo na něm\ldots

\podpis{z běloruštiny přeložil Adam Havlín}
\podpis{čestné uznání v kategorii próza}
