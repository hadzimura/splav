\section{Umberto Saba: básně}

\begin{verse}
\textbf{Mrtvý list}

\smallskip

Jak rudý mrtvý list, \\
jejž žene vítr,\\
vítr a metař –  \\
padá pod třpytným nebem, barví krví \\
tu cestu s ostatními –

\medskip

chtěl bych padnout. Zhnusen\\
z těch marných slov, \\
 z těch tváří bez světla.

\medskip

Ale tvůj hlas, ó laskavá, mi říká:\\
Ne, ještě nepadej!

\bigskip

\textbf{Strůmek}

\smallskip

Dnes je čas deště.\\
Den vypadá jak večer,\\
jaro vypadá\\
jak podzim a vichr pustoší\\
strůmek, jenž stojí pevně, ač se nezdá.\\
Působí mezi těmi stromy jako výrostek,\\
který se na svůj zelený věk příliš vytáhl.\\
Díváš se na něj. A je ti zřejmě líto\\
těch běloskvoucích květů,\\
které mu bere bóra; padají v nich\\
do trávy plody, sladká zavařenina\\
na zimu. A tvoje širé mateřství\\
jich lituje.

\bigskip

\textbf{Zima}

\smallskip

Je noc, ničivá zima. Rozhrneš\\
závěs a vyhlížíš. Tvé divoké\\
vlasy se chvějí, černá zornička\\
se rozšiřuje náhlou radostí:\\
neboť ten obraz – obraz konce světa –\\
který jsi zahlédla, tě blaží\\
a hřeje v samé hloubi srdce.

\medskip

Pod křivou lucernou se ubírá\\
jezerem ledu osamělý muž.

\bigskip

\textbf{Odysseus}

\smallskip

Za mlada jsem se plavil vodami\\
dalmatských břehů. Těsně nad hladinou\\
se nořily z vod drobné ostrůvky, \\
vždy plné chaluch, kluzké, zářící\\
v slunci jak smaragdy, kam se jen pták\\
snes občas za lovu. A když je příliv\\
a noc zas smazaly, loď s větrem v plachtách\\
hned prchala před jejich nástrahami\\
dál do moře. Ta země nikoho\\
je dnes mou říší. Přístav zažíhá\\
svá světla pro jiné; mne dosud žene\\
na širé moře nezkrocený duch\\
a bolestivá láska k životu.
\end{verse}

\podpis{Jan Vladislav}
