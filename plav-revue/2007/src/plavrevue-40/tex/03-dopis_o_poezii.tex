\section{Dopis o poezii}

\noindent
Díky za dopis, i když v něm kladeš takové otázky. Jistěže, když píšu, píšu pro někoho – každá báseň je zároveň cosi jako dopis. Jedna z mých sbírek má název \textit{Samomluvy} – a byly to skutečně samomluvy, jež jsem vedl v době, kdy jsem byl na tom tak, že jsem neměl ani ke komu mluvit. Přitom však byla i tenkrát většina těch textů adresována určitým konkrétním osobám, jen jsem jim je mohl dát až později. A několikrát se dokonce stalo, že jsem ty dopisy psal naslepo, do zásoby, a že se pravý adresát ukázal na obzoru nebo ve dveřích teprve dodatečně. Byl to však on, koho jsem měl na mysli, na koho jsem čekal.

Ale já vím, že míříš ještě jinam a že ti to připadá všechno příliš složité. Inu, je to pravda a není to pravda, rozdíl mezi prózou a poezií není tak velký, jak si myslíš. Když někdo píše dejme tomu o Africe, jako třeba Hemingway, nebo o Michelangelovi, jako třeba Karel Schulz, každý čtenář má dojem, že jejich próze dokonale rozumí. Ale je tomu skutečně tak?  Může někdo skutečně porozumět popisu Kilimandžara, když ho nikdy zblízka neviděl, může pochopit, co to je lov na lva, když to nezažil, chápe člověk líčení býčího zápasu, když už jen ta představa vraždění vydráždě­ného zvířete je mu odporná? A může někdo skutečně rozumět životu takového umělce, jako byl Michelangelo, když ve skutečnosti nikdy neviděl Florencii a Řím, nemluvě o Caprese, té krásné horské vesnici nad Arezzem, kde se Buonarroti narodil? A to jsem zatím ani nepřipomněl jeho malby a sochy. Ani odborníci se dodnes nedo­kázali shodnout na tom, co vlastně přesně znamená soubor jeho fresek na stropě Sixtovy kaple, nebo jaká byla skutečně vůdčí myšlenka jeho nikdy nedokončeného náhrobku Julia II.: opravdu, co představují vlastně ti jeho \textit{Zajatci} nebo \textit{Otroci}, roztroušení po muzeích Florencie a Paříže? Ale i když to čtenáři Michelangelova životopisu nevědí, i když se to lidé z žádné románové či jiné evokace nedozvědí, přece jen něco z toho příběhu velmi dobře chápou – totiž to, že to je příběh o jednom z nich. 

S hudbou je to patrně dost podobné, i když v rovině mnohem abstraktnější. Tak jako nachází čtenář uspokojení v střídání různých motivů v příběhu, který čte a ve kterém poznává příběh o člověku, podobně nachází uspokojení ve sledu hudebních motivů, jež mu zdálky evokují rovněž jakýsi „příběh“, slovy nepostižitelný, ale tím také tajemnější, přitažlivější a navíc dramatizovaný řadou splněných nebo zase překvapivě nesplněných opakování, přechodů, zvratů, jako když se dívá na linie a orchestrální doprovod jistých arabských kaligrafií.

Poezie je něco mezi tím, mezi příběhem prózy a příběhem hudby: je to sled nápovědí, nadhozených příběhů, naznačených myšle­nek, malých i velkých pravd, čekajících, aby si je čtenář objevil sám a po svém; jednu z pravd, naznačenou v Sofoklově \textit{Oidipovi}, si lidé uvědomili až po dvou tisíci letech.

S poezií a s uměním vůbec je ovšem jedna potíž: je třeba dát se jimi vést. Patrně asi znáš tu známou existenciální scénu z \textit{Vojny a míru}, kdy se jedna z postav románu začne v opeře divit, co to vlastně ti lidé tam na těch prknech dělají: místo aby mlu­vili, zničehonic pořád křičí a zpívají a jiní jim k tomu vržou hůlkami po ovčích střevech napnutých na kusech dřeva. Krátce, Tolstého hrdina si najednou uvědomí absurdnost počínání, kterému se v úhrnu říká operní představení.

Kdo nepřijme základní předpoklad, nebo spíš nevyslovenou dohodu, že se v opeře zpívá, že herci v divadle i po smrti ožívají, jakmile spadne opona, a že místo pískání a pokřikování lidé vyjadřují své city pomocí fléten, klarinetů, houslí, bubnů a tak dál, ten samozřejmě nemůže pochopit ani \textit{Traviatu}, ani \textit{Hamleta}, ani žádnou Beethovenovu symfonii. A kdo není ochoten přijmout při čtení takový zdánlivý nesmysl, totiž že v jistých situacích mohou mít řádky jen jistý předem stanovený počet slabik a navíc že tyto řádky končí v jistém pořádku stejně znějícími hláskami, kterým se říká rýmy; kdo není ochoten přijmout dohodu, že v tomto druhu slovesného projevu se užívá zcela osobité logiky – asi tak jako se užívá osobité logiky u šachů, ve fotbalu nebo v matematice; krátce kdo se touto bizarní, a přitom velice lidskou logikou nedá vést, ten prostě nemůže dobře číst poezii, natož jí „rozumět“.

V této souvislosti je jistě pozoruhodné, ba příznačné, co filozofů, zvlášť moderních, je přímo fascinováno poezií – Martin Heidegger Hölderlinem, Jan Patočka K. H. Máchou, ale také Thomasem Mannem a Jaroslavem Durychem – a to jsou jen dva z mnoha dalších. Tito filozofové totiž vědí, že filozofické myšlení je hluboce spjato s jazykem a že básníci mají jeden z klíčů k tomuto mocnému nástroji, který vlastní jen lidé. Ostatně to je rovněž důvod, proč představitelé totalitních režimů věnují takovou pozornost poezii a literatuře vůbec, proč o básníky a spisovatele tolik stojí, ale také se jich obávají, umlčují je, soudí je, posílají je do vězení, do koncentračních táborů a dokonce na smrt.

Člověk, který nikdy nebyl na koncertu vážné hudby a neslyšel symfonický orchestr, si sotva může zamilovat napoprvé Janáčka, natož některého z mladších a nejmladších skladatelů. Neocení asi ani komorní nebo symfonickou hudbu Smetanovu, i když třeba zná a má docela rád \textit{Prodanou nevěstu}; jenže u té se může dát vést příběhem, který dovede zaujmout i toho, kdo se jinak o tento druh hudby nezajímá. U Smetanových symfonických básní nebo kvartetu \textit{Z mého života} už ten příběh nenachází, už jej nestačí sledovat ani tam, kde vlastně je, například v programových skladbách \textit{Mé vlasti} a podobně. Jenomže tam už jde jen o narážky, nápovědi, vzdálená echa, vybízející k vlastním představám; vnucuje se mi přitom přirovnání k zvukovému filmu, při němž potlačili obraz a ponechali plátno prázdné k hře vlastní imaginace, podněcované proudem zvuků…

Je to ovšem přirovnání silně pokulhávající a nakonec zbytečné, zejména pro toho, kdo se už jednou opravdu ponořil do moře hudby. Je to osobitý \textit{živel}; a právě takovým osobitým, jedinečným živlem je i poezie. Její proud přináší a odnáší s sebou všechno možné, konkrétní situace ze života básníkova i těch kolem něho, skutečné i pomyslné krajiny, někdy hluboce prožité, jindy jen zahlédnuté z vlaku nebo auta, reminiscence z četby, z obrazů, soch, hudebních skladeb i popěvků, vzpomínky, citáty, úvahy, slovní hříčky a vůbec hry, protože poezie je vždy zároveň jakýsi pokus i jakási hra. Romantikové sice vyhlašovali, že nejkrásnější zpěvy jsou ty nezoufalejší, ale už Shakespeare před nimi ukázal, že ještě působivější je směs nebo střídání slz i smíchu, vážnosti i veselí, hoře i radosti, tak jak to chodí i v životě.

Čtenář básní nemusí a většinou ani nemůže znát všechny podrobnosti, z nichž roste poezie; nezná, nemůže je znát ostatně ani čtenář prózy. Neví, co vedlo Tolstého k psaní \textit{Vojny a míru}, nemůže konfrontovat jeho líčení s historickou skutečností, a ani po tom netouží. Dává se unášet proudem prózy, věří autorovi na slovo, a to i tam, kde si případně uvědomuje jeho licence. Hlavní je přece, že v tom příběhu poznává něco známého, něco blízkého, něco, co vypravuje o člověku, to znamená i o něm samém. A poezie to dělá rovněž, i když poněkud jinými prostředky. Ta se ještě víc než \pagebreak próza podobá těm velkým ulitám, které si lidé přikládají k uším, aby v nich slyšeli šum moře. Ve skutečnosti v nich slyší šum vlastní krve…

Poslouchal jsem ten šum velkých mořských ulit s napětím v dobách, kdy jsem si moře vůbec nedovedl představit. A zrovna tak napjatě jsem poslouchával bzučení, které oživovalo telegrafní sloupy, když jsem k nim jako chlapec přiložil ucho. Dlouho jsem docela vážně věřil, že budu-li poslouchat dobře, uslyším nakonec i ty skutečné zprávy, které probíhaly dráty nad mou hlavou. Dnes, po tolika letech, si občas říkám, zda věda není něco podobného, zda to není také jakési přikládání ucha k telegrafním a telefonním sloupům v naději, že konečně uslyšíme tu pravou zprávu. A přitom je to možná zase jen pouhý šum, způsobený něčím zcela vedlejším, nárazy větru do drátů, v nichž probíhají signály úplně jiného rázu. Na rozdíl od vědy poezie s touto možností neustále počítá; ví o sobě, že je fikcí, že z její ulity nezní skutečné moře, ale jen naše představa o něm – naše pulzující krev.

Nikdo mě nikdy nevedl k tomu, abych poslouchal ulity a telegrafní sloupy. Hudbu ulity mi ukázal poprvé tuším otec, když mi ji doma přiložil k uchu a já uslyšel to imaginární moře. Potom jsem je už chodil poslouchat sám, co chvíli jsem se vkradl do pokoje, kde ta čarovná ulita ležela na knihovně. A hudbu telegrafních sloupů mi ukázali zřejmě kluci, kterým je ukázali zase jiní kluci. S poezií to bylo dost podobné, tu mě taky nikdo neučil znát, musil jsem k ní přiložit ucho sám. Co mi z ní ukazovali ve škole, to mi dlouho připadalo spíš podivné, ne-li komické: proč se například u veršů nechává tolik volného místa, když by se dalo psát až do konce řádky? Nechápal jsem, že v básni mluví i ticho. Tenkrát, to mi bylo asi čtrnáct, patnáct, jsem se zabýval vášnivě matematikou a fyzikou, představoval jsem si, že budu astrofyzikem. Předtím jsem ovšem stejně vášnivě stavěl v duchu lodě, kajaky, kanoe, a taky letadla: přímo jsem se \textit{viděl}, jak se snáším ze stráně na svém létacím stroji, ty představy byly stejně živé jako líčení, které jsem později našel v zápiscích Leonarda da Vinci. Některé sny se zřejmě vracejí, a to i po staletích.

Jistě sis všiml, že rád pracuju rukama a že dovedu udělat leccos uspokojivého nejen pro sebe, ale i pro jiné. Ale ještě krásnější než takové práce je snění o tom, jak je budu dělat,a dokonce i snění o věcech, které vůbec dělat nemohu, protože na ně nestačím silami ani prostředky. A když se teď na to dívám pozpátku, vidím, že to snění nad kusem prkna, které jsme přiřízli zhruba do tvaru vrtule, že tohle snění o příštích letech nad naší vesnicí bylo prvním projevem poezie. Děti takhle dovedou oživit všechno, polínko se pouhým magickým slovem změní v princeznu nebo rytíře, případně letadlo; a poezie zrovna tak vytváří z hrstky slov leccos nevídaného a neslýchaného, co se podobá málem zázraku. Chce to jen trochu souhlasu ke hře a trochu fantazie.

Ptáš se, pro koho píšu: pro ty, co jsou ochotni přistoupit alespoň na chvilku na hru, kterou jim nabízím, a co z několika slov, z několika řádek, jež jim předkládám, dovedou sami sestavovat města, hrady, hory, lesy, Afriky, Ameriky, severní a jižní točny, a ovšem taky létací stroje nebo lodičky pro dva. Možná že se tomu říká poezie. V každém případě to jsou drobné návody k zasnění, i když ty sny nejsou vždy jen radostné a nevinné. Ale vždyť i naše noční sny, ať už úlevné, nebo tíživé, v nás přímo či nepřímo probouzejí podřimující svědomí.

\podpis{Jan Vladislav, březen 1983 – prosinec 1986}



