\section{Galgócz}

Tenkrát, když jsem se tam narodil, nebyla řeka ještě regulovaná. Obtékala z jihu město, pokud se tak dalo říkat těm několika křivolakým ulicím kolem čtvercovatého náměstí na návrší, i když se odedávna pyšnily městským, trhovým a hrdelním právem, obtékala městečko širokým obloukem a prosazovala vůči němu svá daleko starší a skutečnější práva, jak se jí zamanulo. Patřilo k nim především právo velké vody. Nejmíň dvakrát do roka se vylila z břehů a zalila nejen pastviny, ale často i Luka, chudší část města u samé řeky, a tak se stalo, že jsem se několikrát mohl projíždět v neckách rovnou na dvoře.

Jen s mosty toho řeka moc nesvedla. Nahoře, proti proudu, ji sevřel první, železniční, dunící párkrát za den pod vlaky, které většinou ani na místním nádraží nestavěly, dole, po proudu, se přes ni položil druhý, silniční, který duněl nanejvýš pod náporem jejích vlastních vod, vzedmutých zjara nebo na podzim. Auto po něm přejelo zřídka, a když se přece jen objevilo, sbíhaly se děti, provázely je společně se psy úprkem, dokud jim stačil dech, a pak se jim ještě v noci zdávalo, jak ten div pokroku poslušně zastavuje u jejich nohou a zve je k projížďce. Hrčelo-li něco po mostě k Berekseku, byla to nanejvýš nějaká bryčka nebo fůra s dřevem. A tak byly pravidelnými uživateli mostu jedině krávy, co se po něm večer co večer vracely z pastvin za řekou. Vířily oblaka prachu, bučely své nostalgické fanfáry a zanechávaly za sebou koláče kravinců, zatímco je pastevci marně, ale vytrvale sháněli do houfů. Rány jejich dlouhých kožených bičů s kratinkými násadami se rozléhaly jako výstřely o podzimním honu. Ale co kluci pasákům nejvíc záviděli, to byla výsada přecházet most ne po bílé, udusané vozovce jako kdokoli, ale po jeho vysokých obloucích. I já sledoval tu podívanou žádostivě večer co večer: po hřebenu tří velbloudích hrbů mostu běžela pěšinka, právě tak široká, aby se na ni vešly vedle sebe dvě bosé klukovské šlapky, a po té pěšince se třepetala nahoru a dolů, nahoru a dolů a ještě jednou drobná postavička s vlající halenou a dlouhým bičem. Stavitelé mostu zřejmě s tímto praktickým využitím oblouků počítali, protože je pro větší bezpečnost posázeli dvěma řadami kulatých hlav nýtů, spojujících jednotlivé díly konstrukce. 

Dětem z města bohužel rodiče tyhle nádherné procházky dvacet třicet metrů nad řekou zakazovali a pokusy překročit ten nesmyslný zákaz dokonce trestali, nejčastěji lískovým prutem uříznutým přímo na místě ve škarpě. A přitom byly jejich obavy zcela zbytečné, protože pokud jsem pamatoval, žádnému z těch akrobatů se nikdy nic nestalo. Stádo i s pasáky prošlo pokaždé bez nejmenší úhony soutěskou mostu, za mostem se znovu rozlilo přes celou ulici a teprv dál, cestou do návrší, se začalo postupně roztrácet. Když mě matka poslala, vždy pozdě, zavřít okenice, aby se nenaprášilo do pokojů, zastihl jsem za okny už jen kotouče teple páchnoucího prachu, jimiž plula poslední dobytčata. To bylo znamením, že dlouhý letní večer neodvratně končí. Na bledé obloze vysoko nad střechami houstl přízrak věže. Dva ciferníky ze zlatými římskými číslicemi, oddělené ostrou hranou zdi, zářily nehybnýma zvířecíma očima sovy, která mě fascinovala v Brehmově \textit{Ilustrovaném životě zvířat}. Chvíli jsem napjatě sledoval její pohled, ale než jej mohla stočit ke mně, přirazil jsem okenice jako knihu, která mě příliš vzrušovala. „Nezahlídla mě!“ Teď už jsem byl i já unaven tím nekonečným letním dnem. Ale ještě v posteli, než jsem usnul nadobro, jsem počítal daleká bučení a slábnoucí práskání biče, který zůstal navždy jedním z mých nesplněných přání.

\pagebreak

\noindent
Čas měl tenkrát příchuť věčnosti. Všechno v něm bylo na svém místě, podobně jako u řeky a v řece samé. Někde se hnala mělčinou tak prudce a přilnavě, že se krabatila nad každým kamenem, jinde skoro stála a točila se hlubinou dokola i s tím, co přinesla zdaleka, z hor kdesi na severu, s kusy dřeva, větvemi, slamníky, utopenci. Kousek pod městem shovívavě tlačila obrovská kola mlýnů na spodní vodu a zrcadlila s přeludnou přesností, ale hlavou dolů muže s pytli zrní přes rameno, kteří nad ní balancovali od rána do večera po dlouhé úzké lávce z dvou kmenů, spojující mlýny s břehem a roubenými sýpkami. Ale i tento ráj byl přístupný jen vyvoleným a na vetřelce měli mlynáři pořád po ruce důtky s uzlíky na dlouhé žíle. Když přilehla k horké mokré kůži, zůstávala po ní několik dnů tenká jasně červená stužka, která se nedala umýt ani odpárat. Mlynáři vysvětlovali svou nepřejícnost všelijakým pomyslným nebezpečím, ale byly to jen výmluvy. Většina kluků se vozila na dlouhých prknech mlýnských kol nahoru a dolů s naprostou jistotou. Jen úplný hlupák nebo nešika se na druhé straně, po proudu, nepustil včas, dřív, než ho kolo vtáhlo pod sebe, takže ho pak museli z té mělčiny s křikem tahat napůl zalknutého vodou a strachem. Já sám takovou pohanu nezažil nikdy, jenže právě tak byla pravda, že se mi při každé takové jízdě srdce div nerozskočilo vzrušením. Něco z toho jsem zakoušel později na ruském kole, ale to byla už jen slábnoucí ozvěna. V plné síle se to ozvalo znovu až po letech a v souvislostech naprosto jiných a nečekaných. O tom však až jindy. 

Ale k řece patřilo i velikánské smetiště na srázu za posledními domky předměstí, taková malá Tarpejská skála vysloužilých věcí a hlavně cíl dychtivých výprav za poklady, za něčím, co na člověka čeká, co stačí najít, vyhrábnout z popele a co pak jako Aladinova lampa jedním rázem všechno změní. Nejdřív jsem ty rozbité a odhozené krámy, plechovky, součástky neznámých strojů, lahvičky od neznámých nápojů lásky a jedů, knihy v neznámých řečech a časopisy s neznámými tvářemi nosil domů a strhl jsem do své sběratelské vášně dokonce i otce, ale pak nám to vetešnictví, jak se nemilosrdně vyjádřila, s konečnou platností zakázala matka. Nezbylo mi, než si najít tajnou skrýš ve sklepě sousedního domu, který chátral neobydlený, a pevně doufat, že se jednoho dne ocitnu i se svými poklady na pustém ostrově uprostřed jižních moří, a pak přijde Robinsonovi vhod každý hřebík, každý drátek, natož starý mlýnek na kávu, prasklá petrolejka nebo propálená alpaková konvice s drobnou rytinou jelení hlavy. Uzná to určitě i matka a bude trpce litovat, co všechno kdysi vyhodila. Ale už bude pozdě.

Zatím jsem byl docela spokojen s písčinami a štěrkovými ostrůvky v řece, nad nimiž se kdesi až u samého nebe třpytily mezi stromy bílé zdi a věžičky \textit{zámku}. Ten přitahoval především svou nedostupností, kterou zhmotňovala černá kovaná mříž zakončená špalírem kopí. Střežila park rozložitých dubů, palouků a pískových cestiček, na nichž se jednou dvakrát do roka ozýval dusot koní s drobnými cínovými vojáčky v sedle. Ti nejmenší z nich se smáli a volali na sebe dětskými hlasy, jenže to všechno bylo tak vzdálené, že kluci, přimáčknutí z druhé strany k chladivé mříži, se po chvíli bez lítosti znovu rozbíhali k řece. Daleko bližší, přítomnější, skutečnější byla jiná podívaná, kovárna na plácku naproti věčně zavřené kapli, hned u silnice, aby ji vozkové s prasklou obručí bryčky nebo herkou bez podkovy nemuseli dlouho hledat. Všechno to tam páchlo ohněm, železem a koží, i sám pán té černokněžnické kuchyně. Jeho světélkující oči a červená ústa plná velkých dravčích zubů zároveň lákala i děsila. Vydržel jsem se na něho dívat celé hodiny, patrně proto, že jsem už tenkrát nejasně narazil na to, co mě později znovu a znovu uvádělo v úžas: jak je člověk nezbadatelný. Co když to byl taky jeden z těch strašných \textit{černých kovářů}, co chodili o půlnoci k řece a tam se svlékali nejen z kožené zástěry, ale i z lidské kůže? Co když i on se proháněl celou noc řekou v podobě velké vydry a chytal ryby svými velkými dravčími zuby? A ráno při prvním kokrhání zas vylézal na břeh a znovu na sebe navlékal lidskou kůži i s těmi černými vlasy a dragounskými kníry? 

A kovárna! Kromě výhně, měchů, kovadlin, štoudví s vodu a celého arsenálu ještě tajemnějšího nářadí a nástrojů v ní bylo navíc cosi nepatřičného, a přece jaksi nezbytného. Na zdi naproti dveřím se černal pod sklem obraz. Zašlá rytina měla představovat město za tureckého obležení, ale Turky nebylo vidět, ani když jsem se postavil na špičky a povystrčil samým úsilím jazyk. A zrovna tak jsem nerozpoznal ani jediného z těch dlouhokrkých psů, zralých pro rožeň, kteří se odmítali vzdát a troufale hájili své domovy. Jediné, co se dalo rozluštit, byl nápis a letopočet. GALGÓCZ, slabikoval jsem, MDCLXXXIII. Nikdy jsem se však nezeptal, co to znamená. Teprve později, jak to bývá, když mě už dávno hnětly docela jiné otázky, jsem přišel na to, že to je jen jedna z obměn jména, které jsem vždy a všude vpisoval do rubriky \textit{místo narození}.

Smysl věcí se vyjevuje pomalu, i když byl do nich vepsán odedávna. Na sever, k půlnoci, se zvedaly soumračné končiny Starého lesa a za ním Javory a ještě dál kamenitý Ptáčník, poslední val proti útočící nížině. Na jih, k poledni, od prahu kovárny přes řeku až kamsi na obzor, se pod sluncem tetelila a modrala rovina. Její nekonečnou linii rušil uprostřed jen obrys obrovité pyramidy, kterou stavitelé bůhvíproč opustili, ještě než dostavěli první stupeň. Později jsem tu siluetu bezpečně poznal na fotografiích zigguratů, vykopaných z mezopotamského písku; tady to však byla jen jedna z těch cihlových pevností, které se stavěly proti Turkům, Švédům, Prusům, po celém císařství. Císařská byla i jejich jména, Leopoldov, Josefov, Terezín. 

Za touhle leopoldovskou končil svět a začínalo průzračné nic. Rozpouštěla se v něm oblaka, mizely v něm vlaky, ztrácely se v něm známé cesty i tajné stezky, po kterých prchali z pevnostních kobek lupiči a vrazi. A jednou za rok, prostřed léta, se na tom pomezí světa rozhořela večer hranice na paměť muže, kterého kdysi dávno kdesi daleko upálili. Praskot velikých smrkových soušek ze Starého lesa nebylo na tu dálku slyšet, ten slyšeli pouze rodiče, kteří se vlastenecky odebrali přes most přímo na místo. Děti zůstaly doma. Pozorovaly napjatě ohnivé arabesky uhlíků, unášených horkým vzduchem k černé obloze, dokud se kolem půlnoci dospělí, celí rozjaření, nevrátili a nezahnali je spat. Ale hranice žhnula na obzoru až do rána, jako by chtěla co nejvíc prodloužit okamžik, kdy se přesýpací hodiny roku obracejí. 

A pak to všechno skončilo. Naráz a nadobro, jak mi to několikrát názorně ukázala řeka. Znenadání se na ní zvedla vlna, přelila se přes plochý břeh a jediným šplouchnutím smazala hrady, mosty a tunely z písku, které jsem pracně stavěl celý den. Vzala s sebou všechno, dokonce i barokní františkánský klášter, jehož světlým ambitem jsem prchával před obrazy plnými sražené mučednické krve, i parní velkomlýn uprostřed města, jehož třídenní požár byl nepochybně největší podívanou, jakou jsem kdy zažil, i pusté dvory a opuštěné haly vagónky, odsouzené po krachu vzdálených burz k potupné smrti prachem a rzí, jež zvířili občas jen kluci, když tam s Tomem Sawyerem pátrali po zavilém Indiánovi. To všechno končilo. Otec postoupil a to znamenalo, že jsme se stěhovali pryč.

Tisíckrát horší než loučení samo bylo očekávání té nenapravitelné události. Ještě teď cítím, jak jsme se s kamarády rozechvěle tiskli k sobě v kůlně na dvoře nad mapou vytrženou ze školního atlasu. To byla jediná zbraň, ten kus papíru, proti doléhajícímu osudu. Prstem jsme si na ní ukazovali, kudy to bude nejkratší a nejjistější, až o nejbližších svátcích vyrazíme, abychom se setkali někde v půli cesty. „Tady“, upoutala naši pozornost kóta se značkou zříceniny. „A nejlíp podle řeky a pak proti proudu přítoků, tak se to vždycky dělá v pralese.“ 

Mapa byla příliš hrubá, aby se podle ní dalo chodit. Jako dětská fantazie; ani ta nebrala na vědomí podrobnosti, jimiž každodenní skutečnost prosazuje svou. Nikdy jsme se nesešli. Nikdy jsme si nenapsali. A nakonec jsem zapomněl i jejich jména. 

\podpis{Jan Vladislav}
