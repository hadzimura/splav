\section{Malá antologie japonských \\ pětiverší}

\begin{verse}
Co jiného je\\
tento svět nežli vlna, \\
která se bělá\\
za lodí, když se zrána\\
rozjede po hladině?

*

Když moje srdce\\
pomyslí, jak je krásný,\\
vzedme se celé\\
jak řeka, která boří\\
přehradu za přehradou.

*

Sotva se touha\\
celého roku ztiší\\
v jediné noci, \\
začnou hned ráno oba\\
po sobě toužit znova.

* 

Slyšel jsem cosi,\\
že po té cestě jednou \\
musíme každý, \\
nemyslel jsem však nikdy, \\
že to má být tak brzy.

*

Vydala jsem se \\
na cestu lodí vratší\\
než jeho srdce\\
a nebylo dne, aby\\
nás nezalily vlny.

*

Jako klas rýže,\\
prázdný klas, kterým zmítá \\
podzimní vítr,\\
schnu prázdná, bez naděje, \\
že bychom byli spolu. 

*

Ten odlesk luny,\\
jenž vězí v trošce vody\\
nabrané dlaní,\\
co to je, pravda, zdání?\\
A jakpak je to se mnou?  

*

Ten matný měsíc, \\
když vyjde v jarní noci\\
a blankyt nebe\\
i barvy květů mizí\\
v závoji lehké páry!

*

Vy si už dávno\\
myslíte bezpochyby,\\
že nejsem živa,\\
ale já pořád ještě\\
pláču a v pláči žiju.

*

I když mám srdce\\
už skoro slepé od slz, \\
jež stále kanou,\\
dnes v noci září měsíc\\
tak, že jej přece vidí.

*

Smáčena rosou\\
pelyňku, který bují,\\
stojím a čekám, \\
čekám a naříkám si, \\
že nikdy nikdo nejde.
 
*

U mne už třešně\\
shodily všechny květy, \\
a přitom u vás,\\
i když je konec jara,\\
jsou pořád v plném květu.

*

Už tolik nocí \\
mě bambusové listí \\
budí svým šumem\\
a ani sama nevím,\\
proč mě tak plní steskem.

*

Další rok končí,  \\
noc se už zvolna jasní,\\
rukáv se leskne \\
ve svitu ranní luny\\
a všechno je jen marnost.

*

Ach, podívej se,\\
zamženým nebem letí\\
divoké husy.\\
Jako když někdo zlehka\\
načrtne tuší dopis. 

*

Víš-li ty vůbec, \\
jak může noc být dlouhá,\\
když ležím sama\\
a bez přestání vzlykám\\
až do bílého rána?

*

Ne, lidské srdce \\
se nikdy nedá poznat, \\
ale ty květy \\
v rodné vsi voní stále, \\
jak vonívaly kdysi.
\end{verse}

\podpis{Jan Vladislav}