\section{Lev Nikolajevič Tolstoj: \\ Vojna a mír}

\noindent
\textbf{I.}

\noindent
„Eh bien, mon prince. Gênes et Lucques ne sont plus que des apanages, des panství, de la famille Buonaparte. Non, je vous préviens, que si vous ne me dites pas, que nous avons la guerre, si vous vous permettez encore de pallier toutes les infamies, toutes les atrocités de cet Antichrist (ma parole, j'y crois) – je ne vous connais plus, vous n'êtes plus mon ami, vous n'êtes plus, můj věrný otroku, comme vous dites. Tak buďte zdráv, buďte zdráv. Je vois que je vous fais peur, posaďte se a vyprávějte.“

Takto vítala v červenci 1805 proslulá dvorní dáma a vůbec jedna z nejbližších carevny Marie Fjodorovny Anna Pavlovna Schererová důležitého a úřadem obtěžkaného knížete Vasilije, který na její večer dorazil jako první. Anna Pavlovna už několik dní kašlala, měla chřipku, jak sama říkala (a chřipka bylo tehdy slovo zcela nové, slovo, jehož užívali opravdu nemnozí). V pozvánkách, které již ráno rozvezl posel, se pro všechny bez rozdílu psalo:

"Si vous n'avez rien de mieux à faire, M. le comte (nebo mon prince), et si la perspective de passer la soirée chez une pauvre malade ne vous effraye pas trop, je serai charmée de vous voir chez moi entre 7 et 10 heures. Annette Scherer".

„Dieu, quelle virulente sortie,“ odpověděl právě vcházející kníže, aniž by se takovýmto setkáním dal vyvést z míry: Na sobě měl dvorskou vyšívanou uniformu, na nohou punčochy, boty, na prsou hvězdy, v ploché tváři přívětivý výraz.

Mluvil vybranou francouzštinou, jíž nejen hovořili, ale v níž i přemýšleli naši dědové, a s oněmi tichými, protektorskými intonacemi, jež jsou vlastní významným lidem, kteří zestárli u dvora a ve velkém světě. Přistoupil k Anně Pavlovně, políbil jí ruku, pak jí nastavil svou navoněnou, zářící pleš a poklidně usedl na pohovku.

„ Avant tout dites moi, comment vous allez, chère amie? Uklidněte mě,“ řekl naprosto nezměněným hlasem a tónem, jímž za zdvořilostí a přátelskou účastí prosvítala lhostejnost a dokonce lehký výsměch.

„Jak může být člověk zdravý…, když prožívá takové mravní útrapy? Copak někdo, kdo v sobě má cit, může v našich časech zůstávat klidný?“ odvětila Anna Pavlovna. „Pevně doufám, že u mě hodláte strávit celý večer.“

„A co slavnost u anglického vyslance? Je středa. Musím se ukázat i tam,“ řekl kníže. „Zastaví se pro mě dcera a odveze mě.“

„Já myslela, že oslavy byly zrušeny. Je vous avoue que toutes ces fêtes et tous ces feux d'artifice commencent à devenir insipides.“

„Kdyby věděli, že je to vaše přání, určitě by je zrušili,“ řekl kníže jako právě natažené hodiny obvyklým tónem, jímž pronášel věci, v jejichž případě si nepřál, aby jim ostatní věřili.

„Ne me tourmentez pas. Eh bien, qu'a–t–on décidé par rapport à la dépêche de Novosilzoff? Vous savez tout.“

„Jak bych vám to řekl?“ pokračoval kníže chladným, znuděným tónem. „Qu'a–t–on décidé? On a décidé que Buonaparte a brûlé ses vaisseaux, et je crois que nous sommes en train de brûler les nôtres.“  

Kníže Vasilij hovořil vždy lenivě, jako když herec deklamuje text starého kusu. Anna Pavlovna Schererová byla naopak přes svých čtyřicet let plná života a elánu.

Být plná elánu se pro ni stalo společenskou nutností, a občas, dokonce když se jí do toho vlastně nechtělo, předváděla ten elán jen proto, aby nezklamala očekávání lidí, kteří ji dobře znali. Zdrženlivý úsměv, který neustále pohrával na její tváři, se sice k jejím povadlým rysům příliš nehodil a spíš jako u rozmazlených dětí vyjadřoval neutuchající povědomí vlastního roztomilého kazu, jehož se nechce, nemůže a ani nepovažuje za nutné se zbavit.

Uprostřed debaty o politických záležitostech se Anna Pavlovna rozhorlila:

„Proboha nemluvte mi tu o Rakousku! Já vůbec ničemu nerozumím, ale Rakousko nikdy nechtělo a nechce válku. Zrazuje nás. Rusko se musí stát spasitelem Evropy samo. Náš panovník zná své ušlechtilé poslání a dostojí mu. To je jediné, več věřím. Našeho dobrého a vzácného vladaře čeká ve světě úloha zcela mimořádná, a on je tak dobrotivý a skvělý, že ho bůh neopustí a on vyplní svůj úkol – zadusit hydru revoluce, která je dnes v osobě toho vraha a ničemy ještě příšernější. Jsme to jen a jen my, kdo dokáže vykoupit krev toho svatého muže. Protože v koho my ještě můžeme doufat, ptám se vás…? Anglie se svým obchodnickým duchem velikost duše cara Alexandra nepochopí, a ani pochopit nemůže. Vždyť Britové odmítli vyklidit Maltu. Chtějí v našich krocích vidět nějaké postranní úmysly a pátrají po nich. Protože co řekli Novosilcovovi? Nic. Nepochopili a ani nemohou pochopit obětavost našho panovníka, který pro sebe nechce nic a vše dělá jen pro blaho míru. A co slíbili? Taky nic. A nesplní ani to, co slíbili! Prusko už prohlásilo, že Bonaparte je neporazitelný a že proti němu nic nezmůže ani celá Evropa. A já ani Gardenbergovi, ani Haugwitzovi nevěřím jediné slovo. Cette fameuse neutralité prussienne, ce n'est qu'un piège. Já věřím v jediného boha a ve vznešený úděl našeho panovníka. On Evropu zachrání…!“ S posměšným úšklebkem nad vlastním horlením najednou ustala.

„Myslím,“ poznamenal s úsměvem kníže, „že kdyby místo našeho milého Vincengerodeho vyslali vás, tak byste souhlas pruského krále vybojovala ztečí. Jste tak výmluvná. Nedopřála byste mi čaj?“

„Zajisté. A propos,“ odpověděla už zase klidněji, „mám tady dva velice zajímavé muže, le vicomte de Mortemart, il est allié aux Montmorency par les Rohans, což je jedno z nejvznešenějších jmen ve Francii. Je to jeden z nejlepších emigrantů, z těch opravdových. A opak je tu l´abbé Morio; to je hluboký duch – znáte ho? Přijal ho i panovník. Tak znáte ho?“

„Ach ano! Bude mě velice těšit,“ odvětil kníže. „Poslyšte,“ prohodil vzápětí s hranou bohorovností, jako by si právě letmo vzpomněl na cosi nedůležitého, i když ve skutečnosti to byl hlavní cíl jeho návštěvy, „je pravda, že l´impératrice mère hodlá jmenovat barona Funka prvním tajemníkem do Vídně? C'est un pauvre sire, ce baron, à ce qu'il paraît.“ Kníže Vasilij hodlal místo, jež mělo připadnout prostřednictvím carevny Marie Fjodorovny baronovi, získat pro svého syna.

Anna Pavlovna na znamení toho, že ani ona, ani nikdo jiný nemůže posuzovat, co se zlíbí či zamlouvá panovnici, téměř zavřela oči.

„Monsieur le baron de Funke a été recommandé à l'impératrice–mère par sa soeur,“ řekla jen smutným, suchým tónem. Ve chvíli kdy Anna Pavlovna pojmenovala panovnici, se v její tváři náhle rozhostil výraz hluboké a upřímné oddanosti a úcty, spojené se smutkem, což se jí stávalo pokaždé, když se během rozmluvy zmínila o své vznešené příznivkyni. Dodala ještě že Její Veličenstvo ráčila baronovi Funkemu prokázat beaucoup d´estime, a její pohled opět zastřela zasmušilostí.

Kníže lhostejně zmlkl. Anna Pavlovna se vší dvorskou a k tomu i ženskou obratností a břitkým smyslem pro takt hodlala za to, jak opovážlivě se vyjádřil o osobě, jež byla carevně doporučena, knížete vyplísnit a zároveň utěšit.

„Mais à propos de votre famille,“ řekla, „jestlipak víte, že od té doby, co vaše dcera začala chodit do společnosti, fait les délices de tout le monde. On la trouve belle, comme le jour.“

Kníže se na znamení úcty a vděku lehce uklonil.

„Často přemýšlím o tom,“ pokračovala Anna Pavlovna po minutové odmlce, naklánějíc se s laskavým úsměvem ke knížeti, jako by tím chtěla naznačit, že řeči na politická a světská témata jsou u konce a teď že je oba čeká rozmluva opravdu upřímná, „jak nespravedlivě se někdy v životě rozděluje štěstí. Zač vás osud odměnil, že vám dal dvě tak báječné děti? (Vyjma mladšího syna Anatolije, toho tedy v lásce nemám,“ dodala tónem nepřipouštějícím námitky a povytáhla obočí.) „Věřte mi, že vy si jich vážíte ze všech nejméně, a proto jich nejste hoden.“

A obdařila ho dalším zářivým úsměvem.

„Que voulez–vous? Lafater aurait dit que je n´ai pas la bosse de la paternité,“ řekl kníže

„Těch špásů si laskavě nechte. Chtěla jsem si s vám promluvit vážně. Musím vám říct, že s vaším mladším synem jsem opravdu nespokojena. Mezi námi (její tvář se znovu zasmušila) – mluvilo se o něm i u Jejího Veličenstva a byl jste litován…“

Neřekl nic, ale ona si ho mlčky a významně měřila pohledem a čekala na odpověď. Kníže Vasilij se zaškaredil:

„A co mám dělat?“ řekl po chvíli. „Věřte mi, že jsem pro jejich výchovu učinil vše, co otec učinit může – z obou jsou nakonec les imbéciles. Ippolit je naštěstí ňouma docela klidný, zato Anatol je rošťák. To je ten rozdíl,“ dodal s úsměvem o něco nepřirozenějším a oduševnělejším než obvykle, ale v rýhách kolem úst přitom hned a obzvlášť ostře vyvstalo cosi nečekaně hrubého a nepříjemného.

„Proboha proč se lidem jako vy vůbec rodí děti? Kdybyste nebyl otec, nemohla bych vám vyčíst ani to nejmenší,“ odtušila Anna Pavlovna a zvedla k němu zamyšlený pohled.

„Je suis votre věrný otrok, et à vous seule je puis l´avouer. Mé děti – ce sont les entraves de mon existence. Jsou můj kříž. Alespoň tak to sám sobě vysvětluji. Que voules vous…?“ Odmlčel se a jen gestem naznačil svou pokoru před krutým osudem.

Anna Pavlovna se zamyslela.

„Ještě nikdy jste neuvažoval o tom, že byste toho vašeho ztraceného syna Anatolije měl oženit. Říká se,“ poznamenala lehce, „že staré panny ont la manie des mariages. Já v sobě tuhle slabost ještě necítím, ale vím o jedné petite personne, která je z vlastního otce zoufalá, une parente à nous, une princesse Bolkonská.“ Kníže Vasilij neřekl nic, i když s pohotovostí a schopností hodně si pamatovat, lidem velkého světa vlastní, lehkým pohybem hlavy naznačil, že tohle sdělení bere na vědomí.

„Asi mi to nebudete věřit, ale ten Anatol mě stojí čtyřicet tisíc ročně,“ řekl najednou, protože v sobě tok truchlivých úvah asi nedokázal udržet. Znovu se odmlčel.

„Co bude za pět let, když to takhle půjde dál? Voilá l´avantage d´être père. Je bohatá, ta vaše kněžna?“

„Její otec je velice bohatý a lakotný. Žije na venkově. Jde o známého knížete Bolkonského, který byl do výslužby poslán už za zesnulého cara a jemuž se dostalo přezdívky pruský král. Je to muž velice moudrý, ale zároveň podivín těžké krve. La pauvre petite est malhereuse, comme les pierres. Má bratra, který si nedávno vzal Lisu Meinenovou, je to Kutuzovův pobočník. Za chvíli tu bude.“

„Ecoutez, chère Annette,“ řekl kníže, náhle svou hostitelku uchopil za ruku a neznámo proč ji stlačil dolů. „Arrangez–moi cette affaire et je suis votre nejvěrnější otrok à tout jamais (pan – comme mon šafář píše ve svých hlášeních). Je z dobré rodiny a bohatá. Což je přesně to, co potřebuji.“

A oněmi volnými, graciézními gesty, jimiž ve společnosti proslul, uchopil dvorní dámu za ruku, políbil ji, poté tou rukou ještě potřásl, rozvalil se ve svém křesle a pohlédl kamsi stranou.

„Attendez,“ pochopila okamžitě Anna Pavlovna. „Hned promluvím s Lise (la femme du jeune Bolkonskij). Je možné, že se to zdaří. Ce sera dans votre famille, que je ferai mon apprentissage de vieille fille.“

\podpis{přeložil Libor Dvořák}