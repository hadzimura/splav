\section{Michel Houellebecq: Možnost ostrova}

\textit{„Co tedy dělá krysa při probuzení? Čmuchá.“}

\begin{flushright}
Jean-Didier, biolog
\end{flushright}

\bigskip

\noindent
Nakolik jen mi ty první okamžiky mé budoucí role tajtrlíka utkvěly v paměti! Bylo mi tehdy sedmnáct a dost pošmourný srpen jsem trávil v Turecku v klubu  \textit{all inclusive} – naposledy, co jsem musel odjet na prázdniny s rodiči. Ta píča moje sestra – bylo jí třináct – rozpalovala už pomalu každýho chlapa. Seběhlo se to u snídaně; jako každé ráno se utvořila fronta na míchaná vajíčka, pro letní hosty snad největší lahůdku. Stará Angličanka vedle mě (vyzáblá, zlá, ten typ, co ani nemrkne a picne lišku, jen aby si mohla vyzdobit obývák) už si vajec pořádně naložila a bez zaváhání shrábla i poslední tři párky, zastupující na stříbrném podnose oblohu. Přitom bylo za pět minut jedenáct, snídaně u konce, a tudíž nemyslitelné, že by číšník mohl ještě další párky donést. Němec stojící za ní ztuhl; prve už na jeden z párků mířil vidličkou, ale napřáhl se marně; ve tváři rozhořčením celý zbrunátněl. Byl obrovský, hotový německý obr vysoký přes dva metry, měl nejmíň sto padesát kilo. Chvilku jsem si myslel, že vidličku zabodne osmdesátnici do očí nebo ji chytne pod krkem a rozdrtí jí hlavu o servírovací pult na teplá jídla. A ona, jako by se nic nedělo, s onou senilní sobeckostí, jež je u stařečků už neúmyslná, cupitala ke svému stolu. Němec se držel, seč mohl, bylo vidět, že se musí přemáhat, aby nevybuchl, pak to ale vydýchal, párky vzdal a zamířil smutně ke svým nejbližším.

Po téhle příhodě jsem si vymyslel malý skeč o krvavé vzpouře v jistém prázdninovém klubu vyvolané nepatrnými prkotinami popírajícími model  \textit{all inclusive}: nouze o párky při snídani a potom poplatek za minigolf. A skeč jsem hned sehrál na společném večírku „Máte talent!“ (jednou týdně sestával večerní program z představení, která namísto profesionálních bavičů připravovali sami hosté); předváděl jsem všechny postavy najednou, a tak jsem vlastně vykročil na dráhu  \textit{one man show}, již jsem následně prakticky po celou svou kariéru už neměl opustit. Na večerní program chodili skoro všichni, až do začátku diskotéky v hotelu nebylo moc co dělat; publikum tedy čítalo dobrých osm set lidí. Mé vystoupení mělo pořádný úspěch, spoustě diváků smíchy až tekly slzy, aplaus byl docela výživný. Ještě ten večer mi pěkná brunetka jménem Sylvie na diskotéce řekla, že se díky mně pěkně nasmála a že má ráda kluky se smyslem pro humor. Zlatá Sylvie. Přišel jsem o panictví, o mé budoucí dráze bylo rozhodnuto.

Po maturitě jsem se zapsal do hereckých kurzů, následovaly nepříliš slavné roky, kdy jsem byl postupně pořád víc a víc zlý, a tím pádem i jízlivější; za těchto okolností nakonec úspěch musel přijít – a dokonce takový, až mě to překvapilo. Začínal jsem malými skeči o sloučených rodinkách, o novinářích z \textit{Le Mondu}, obecně o omezenosti střední třídy – zdatně jsem zpodoboval incestní choutky intelektuálů v kariérním prostředí a konfrontoval je s jejich vlastními či nevlastními dcerami, co ukazují pupek a z kalhot jim koukají tanga. Souhrnně vzato, byl jsem  \textit{uštěpačný pozorovatel současné reality}. (…) I když jsem se nadále věnoval  one man show, občas jsem přijal pozvání do televizních pořadů, které se vyznačovaly velkou sledovaností a všeobecnou podprůměrností. Podprůměrnost jsem nikdy neopomněl zdůraznit, ačkoli jsem to dělal jemně; nejlépe, když se moderátor cítil trochu ohrožen, ale ne zase moc. Byl jsem zkrátka  \textit{skutečný profesionál}; jen trochu přeceňovaný. Ale nebyl jsem sám.

Netvrdím, že by mé skeče postrádaly vtip, kdepak. Byl jsem přece uštěpačný pozorovatel současné reality; pouze jsem měl dojem, že jdu čistě po tom základním, že v současné realitě zbývá tak málo věcí k pozorování: natolik jsme zjednodušovali, natolik omezovali a bořili hranice, tabu, mylné naděje, falešné tužby; zbývalo skutečně jen velmi málo. Z pohledu sociálního byli bohatí a byli chudí, a pouze pár vratkých můstků –  \textit{společenský výtah} byl námětem, který se patřilo ironizovat; každé takové téma vás ale mohlo též spolehlivě odrovnat. Z pohledu erotického byli ti, kteří vyvolávali touhu, a ti, kteří nevyvolávali nic: skrovný mechanismus v některých případech složitější (homosexualita apod.), přesto však dovolující zjednodušení na ješitnost a narcistické soutěžení, dobře popsané už o tři století dříve francouzskými moralisty. Samozřejmě že byli i  \textit{správní lidé}, takoví, co pracují, zajišťují efektivní výrobu potravin, a také ti, co se – způsobem poněkud komickým či patetickým, chcete-li (já ale byl především komik) – obětují svým dětem; takoví, co nejsou ani zamlada krásní, ani později ctižádostiví a nikdy bohatí; již ovšem celým srdcem lnou – a dokonce stojí v popředí, myslí to upřímněji než kdokoli jiný – k hodnotám krásy, mládí, bohatství, ctižádosti a sexu; jsou zkrátka  \textit{solí země}. Je mi líto, že to tak musím říct, ale tihle mi nemohli posloužit jako  \textit{námět}. Pár podobných jsem do svých skečů zapracoval, čistě pro zpestření, k navození  \textit{reálného obrazu}; ale jinak mi lezli krkem. Nejhorší je, že mě měli za \textit{humanistu}; sice humanistu  \textit{jízlivého}, ale humanistu. Pro dokreslení uvádím jeden z žertů, kterými byla má představení vyšperkovaná:



 „Víš, jak se říká tomu tučnému kolem vagíny?“

 „Ne.“

 „Ženská.“



 Tenhle typ vtípků se mi kupodivu dařilo používat pořád dokola, a stejně jsem měl stále dobré kritiky v časopisech jako  Elle či Télérama; pravda je, že všechny machistické úlety lehce rozvrátil nástup arabských komiků, a to já přitom ujížděl spanile: pořádně zahranit, pak odlehčit a zejména neztratit balanc. Největší zisk z povolání humoristy, a vůbec humoristického postoje v životě nakonec je, že se člověk může ve vší beztrestnosti chovat jako dobytek a ještě dokáže svou podlost tučně zhodnotit, ať už ji přeleje do úspěchů v sexu, anebo promítne do výše nastřádané hotovosti, hlavně že se jeho jednání těší všeobecnému souhlasu.

 Můj takzvaný humanismus stál ve skutečnosti na hodně hubených základech: neurčitý výpad proti daňařům, připomínka mrtvol ilegálních černochů vyvržených na španělském pobřeží mi stačily ke získání pověsti  levicově smýšlejícího a  obránce lidských práv. Já a levicově smýšlející? Možná jsem příležitostně ve svých skečích použil i nějaké odpůrce globalizace, spíš mladé, a nedával jim zrovna nesympatické role; možná jsem příležitostně mohl sklouznout k jisté demagogii: byl jsem přece, znovu říkám, skutečný profesionál. Krom toho jsem měl ksicht Araba, což mi mnohé ulehčilo; antirasismus, či přesněji antibílý rasismus totiž představoval jediný námět, který v těch letech ještě levici zbyl. Vlastně ani netuším, kde jsem k té arabské vizáži přišel, a mimochodem, jak mi přibývala léta, projevovala se stále nápadněji. Matka měla původ španělský a otec, pokud vím, bretaňský. Moje sestra, ta malá coura, byla nepopiratelně středozemní typ, přitom ale zdaleka ne tak snědá jako já, a navíc měla rovné vlasy. Kdoví, jak na tom byla moje matka s věrností. Třeba mě zplodil nějaký Mustafa. Či jiná hypotéza – nemohl to být Žid?  Fuck with that: Arabové na má představení chodili v hojném počtu – a Židé ostatně taky, i když trochu míň; a platili všichni jak diví, cálovali plné vstupné. K okolnostem své smrti člověk přirozeně nebývá lhostejný; s okolnostmi zrození je to ovšem nejasnější.

 A  lidská práva? Nezájem; vždyť já se sotva dokázal postarat o práva svého penisu.  



 V tomto směru pokračování mé kariéry v jistém smyslu můj první úspěch na dovolené potvrdilo. Ženám se obvykle smyslu pro humor nedostává, a proto ho považují za součást mužských kvalit; příležitosti, jak umístit úd do náležitých otvorů, mi tedy během celé kariéry nechyběly. Po pravdě řečeno však na těch koitech nebylo nic skvostného: o komiky se většinou zajímají ženy starší, když kolem čtyřicítky začínají cítit, že všechno už jde z kopce. Některé měly velkou prdel, jiné prsa jak vyždímaná a někdy obojí dohromady. V podstatě na nich nebylo nic zrovna rajcovního; navíc když poklesne erekce, přece jenom se sníží i zájem. Ne že by byly moc staré, to zase ne; věděl jsem, že jak se jim blíží padesátka, začnou hledat něco falešného, uspokojivého a snadného – což ovšem nikdy nenajdou. A já jim jen mohl potvrdit – zcela nedobrovolně, věřte, nikdy to není příjemné –, že jejich erotická hodnota upadá; mohl jsem pouze kvitovat jejich první pochyby, proti své vůli jim poskytnout beznadějný pohled na svět: kdepak, žádná vyzrálost je už nečeká, jenom stáří; na konci cesty není nový rozpuk, nýbrž souhrn frustrací a soužení zprvu nepatrného, pak velmi brzy nesnesitelného; žádná legrace. Život začíná v padesáti, pravda; akorát že ve čtyřiceti končí.

\podpis{Jovanka Šotolová}