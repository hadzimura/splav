\section{Brána bez dveří (Wumenguan)}

\noindent
\textbf{K překladu}

\noindent
Wumenguan (jap. \textit{Mumonkan}) čili \textit{Brána bez dveří} je sbírka 48 zenových kóanů ze 12. století. Zen je sinojaponská verze čínského slova chan, kterým čínští buddhisté ve 4. století n. l. transkribovali sanskrtské \textit{dhjána} ve významu meditace, kontemplace. Kóan je sinojaponská verze čínského slova \textit{gong´an} (vyslov kung-an; ng- je nosovka). To slovo znamená případ, kauza.

Poprvé jsem se pokusil překladatelsky srovnat s \textit{Bránou bez dveří} počátkem devadesátých let. Zkoušel jsem to již dříve, ale teprve obnovená tradice odeonovské \textit{Světové literatury} a práce na zvláštním čísle věnovaném „zenu“ dala těm pokusům smysl. S odstupem jsem si uvědomil, že pro úplnější překlad musím zjednat pro sebe i pro čtenáře potřebný kontext, širší a hlubší, než může dát obvyklý úvod a komentář. To se stalo až \textit{Čínskou filosofií: Pohledem z dějin} (Maxima 2005) a \textit{Sebranými spisy Mistra Zhuanga} (Maxima 2006). Ta první představila horizont, na kterém dokážeme lépe číst odkazy Wumenových promluv, ta druhá ukázala k pramenům Wumenových promluv. 

Základní, v podstatě neřešitelný problém překládání těchto kóanů je v tom, že to, co se tu předvádí, vůbec není literatura, a textu je v tom, co by semenem dohodil. Čirá performance, literárnost zhola bezděčná, vlastně nežádoucí. Všechno za slovem, všechno za obrazem. 

\begin{verse}
Kóan je ,,případ, kauza, událost``.\\
Wumenovy kóany jsou návraty k holým větám \\\hspace*{\fill}bytí v tom.\\
Žádná česká věta není dost holá.\\
\end{verse}

\bigskip

\noindent
\textbf{Pátý případ}

\noindent
Řekl mnich Xiang´yan (vyslov měkce, „polsky“: Siang-jen): „Představ si, že lezeš na strom, zuby se držíš větve, ruce se nemají čeho chytit, nohy nemají, o co by se opřely. A najednou se tě někdo zdola pod stromem zeptá na smysl Bódhidharmova příchodu ze západu. Neodpovíš mu, zpronevěříš se otázce, odpovíš mu, zřítíš se a přijdeš o život. Co si jenom počneš, abys obstál?“ 

\medskip
\noindent
\textit{Wumenův komentář}

\noindent
Můžete být výmluvní jako řeka, stejně vám to nebude k ničemu. Můžete vyložit kompletní kánon našeho učení, stejně vám to nebude nic platné. Učinit všemu a všem zadost, to je jako dát život jednou mrtvým, dát smrt jednou živým. Jestli ani teď nedokážete správně odpovědět, počkejte si na Maitréju a zeptejte se jeho. 

\medskip
\noindent
\textit{Gáthá:}

\begin{verse}
Mnich Xiang´yan (vyslov: Siang-jen) blafoval,\\
však jedovatý byl až tak,\\
že mnichům zavřel pusu, \\
démonům vytřel zrak. 
\end{verse}

\medskip
\noindent
\textit{Z komentáře překladatele:}

\noindent
Ptát se po smyslu Bódhidharmova příchodu znamená navodit mezní situaci, kritický bod můstku. Krajnost radikální a mezní, nikoli však transcendentní. V chanu je konec konců všechno konkrétní! Vždycky je tu nějaké učení, vždycky je tu nějaká otázka. Ale vždycky jde o život! A dokud nejde o život, jde o hovno. Takže se nebudeme divit, uslyšíme-li odjinud, že ten milý Xiang´yan spálil všechny své knihy se slovy, prázdný žaludek nenasytíš malovánkami jídel. Podobně jako o něco starší Deshan (Tešan) z Wumenova případu 28.

Také je dobré vědět, že dále zmíněný Maitréja, budoucí buddha, na kterého si máte počkat, by se podle přesných propočtů měl objevit 5,670.000.000 roků po smrti historického Buddhy Šákjamuniho. Dovedete si představit, jak brzo se dočkají odpovědi ti, co dají na Wumena? Prameny ze Šrí Lanky nás navádějí, že Buddha Šákjamuni zemřel mezi léty 420 až 380 před n. l. Každému Číňanu to nutkavě připomene, že byl nejméně o dvě až tři generace mladší než Konfucius. Tak tedy, co se Maitréji týče, 5,670.000.000 roků bez 2.400 roků, zbývá 5.669.997600 let. 


\podpis{přeložil Oldřich Král, vydává Maxima 2007}