\section{Shel Silverstein: Poezie}

\begin{verse}
\textbf{Nedostupné místo}\\
\smallskip
Na zádech, hned u lopatky, \\
někde tam to nejspíš bude,\\
jako když se ze skořápky\\
něco pořád marně klube,  \\
ať jsi žena nebo muž,  \\
zavrtíš se, ohneš záda,\\  
v lokti lupne, brada padá,\\
natáhneš prst a už už  \\
tam budeš – ne, to je pech,  \\
ještě kousek, kristepane,  \\
nikdy se tam nedostaneš,\\  
sám spíš chytíš vlastní dech,\\
než bys dosáh na to místo,  \\
co tě svědí na zádech.  
\end{verse}
\podpis{přeložil Standa Rubáš}

\bigskip

\begin{verse}
\textbf{Klaun Konrád}\\
\smallskip
Byl jeden klaun jménem Konrád a ten \\
přijel k nám do města hrát s cirkusem.\\
Boty měl obří a buřinku malou –\\
jenže to nikoho nerozesmálo.\\
Legrační písničky loudil ven z trombónu,\\
zelenou dogu měl a tisíc balónů,\\
byl jak lunt, nos jak špunt, co vlas to drát –  \\
jen nesved nikoho rozesmát.

\medskip

Vždycky když metal kozelce,\\
lidem se udělalo zle.\\
Při pointách všech jeho vtípků\\
smrkali, ač neměli chřipku.\\
Pokaždé, když mu spadla bota,\\
křičeli: ,,Co se tady motáš!``\\
Ze stojek byli zoufalí,\\
namísto aby tleskali.\\
Když začal skákat po manéži,  \\
usínal i ten, kdo byl svěží.\\
A když si snědl motýlka,\\
sál jenom tiše zavzlykal.\\
Jak si měl Konrád vydělat?\\
Nikdo se zkrátka nechtěl smát.

\medskip

Jednou si řek, že všem vypoví,\\
jak zle je nudnýmu klaunovi:\\
proč se mu věčně třese brada,  \\
jak na něj pořád všechno padá,\\
o strachu mluvil, o bolesti,  \\
i jaký to je nemít štěstí…\\
A když jim všechno vypověděl,\\
plakali? Ba ne, co vás vede!\\
Popadali se za břicho –\\
chachachá, cheché, chochochó,\\
řehtali se a řvali, hohó,  \\
celý den… týden… až pak z toho\\
dostali záchvat koliky,\\
ulítaly jim knoflíky\\
a jejich smích se rozléhal\\
po horách, za moře a dál,  \\
do městeček i velkoměst,\\  
od Bjundee po Litevský Brest…\\
Celý svět brzy duněl smíchy,\\
jen Konrád v něm stál divně tichý.\\
Shrbený, se svěšenou hlavou\\
(zatímco manéž řvala ,,bravo!``)\\
zašeptal: ,,Nesmějte se mi.\\
To NEMĚLO být legrační.``\\
A když smích vylét` nad oblaka,\\
Konrád sed do písku a plakal.
\end{verse}
\podpis{ přeložila Zuzana Šťastná}


\bigskip

\begin{verse}
\textbf{Navlíkli velblouda do podprsenky}\\
\smallskip
Navlíkli velblouda do podprsenky,\\
ať nemá hrby nahatý.\\
Navlíkli velblouda do podprsenky,\\
prý jakýpak s tím ciráty.\\
Mají i další chvályhodné plány,\\
spoďáry pro psy a kočkodany,\\
kdoví co za lubem mají ti páni,\\
co navlíkli velblouda do podprsenky.

\medskip

Navlíkli velblouda do podprsenky,\\
nahotu je třeba zahalit.\\
Navlíkli velblouda do podprsenky\\
a na nic se ho neptali.\\
Cpali ho do ní zleva zprava,\\
jen ať si ji prý nesundává.\\
Kdoví jak dopadne chudák kráva,\\
když velblouda navlíkli do podprsenky.
\end{verse}
\podpis{přeložil Lukáš Novák}

\bigskip

\begin{verse}
\textbf{Želé}\\
\smallskip
Král Midas se měl vážně skvěle –\\
na co sáh, to změnil v zlato.  \\
Zato na co sáhnu já, to  \\
změním v malinový želé.\\
Dotknul jsem se zdi v obýváku (blemc),\\
pak jsem chtěl bráchovi dát páku (šlemc),\\
na kole chtěl jsem vyměnit duši (glemc),  \\
mámu jsem pohladil, jak jí to sluší (flemc),\\
chtěl jsem si zavázat tkaničky (plemc),\\
naskládat nádobí do myčky (hlemc),\\
prohrábnul jsem si rukou patku (chlemc),\\
chtěl jsem si dáchnout na lehátku (žlemc),\\
teď jsem si oplách ruce v moři (mlemc),\\
nechceš se seznámit? Jsem Bořík (žblemc).
\end{verse}
\podpis{přeložili S. Rubáš, L. Novák a Z. Šťastná}