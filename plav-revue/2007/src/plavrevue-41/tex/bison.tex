\section{Terry Bisson: \\ Medvědi objevují oheň}

\noindent
PÍCHLI JSME V NEDĚLI VEČER na dálnici I-65, severně od Bowling Green. Já, můj bratr kazatel a jeho syn jsme byli navštívit maminku v Domově a teď jsme se vraceli domů. Auto patřilo mně a prasklá pneumatika vyvolala pár významných povzdechů, protože jako nemožně staromódní člověk (aspoň to o mně oni tvrdí) si pneumatiky spravuju sám, i když mi můj bratr pořád radí, abych si pořídil radiálky a přestal skupovat ty staré.

Ale když víte, jak pneumatiky nasazovat a spravovat, dostanete je skoro zadarmo.

Protože šlo o levou zadní pneumatiku, zaparkoval jsem nalevo, na trávě zalesněného pruhu mezi dvěma dálnicemi. Pneumatika byla vyřízená, to jsem poznal už podle toho, jak trhaně můj cadillac zastavil. „Asi se ani nemusím ptát, jestli máš v kufru lepidlo na píchlé duše?“ položil Wallace řečnickou otázku.

„Tady mi, synku, podrž baterku,“ obrátil jsem se na Wallace mladšího. Je už dost starý, aby mi chtěl pomoct, a ještě ne dost starý, aby věděl všechno líp než já. Kdybych se býval oženil a měl děti, takové bych si určitě přál.

Staré cadillaky mají velké kufry, které se postupně plní nejrůznějším haraburdím až po okraj. Ten můj je ročník ’56. Wallace měl na sobě sváteční košili, takže se mi nehnal na pomoc, když jsem při hledání heveru začal hrnout na stranu časopisy, rybářské náčiní, dřevěnou krabicí na nářadí, nějaké staré šaty, vrátek v sisálovém pytli a postřik na mšice. Rezerva vypadala trochu splaskle.

Baterka zhasla. „Zatřepej s ní, synku,“ poradil jsem mu.

Baterka se zase rozsvítila. Pořádný hever byl už dávno pryč, já s sebou ale vozím ještě jeden – malý, čtvrttunový a hydraulický. Našel jsem ho pod matčinými časopisy  Jižanské bydlení, ročníky 1978–1986. Už dlouho jsem je chtěl vyhodit na skládku. Kdyby se mnou bratr nejel, nechal bych Wallace mladšího umístit hever pod nápravu, ale radši jsem si klekl a udělal to sám. Vůbec neškodí, když se kluk naučí měnit pneumatiky. I když je nebude spravovat a nasazovat, pár si jich v životě vymění. Baterka vypověděla službu, ještě než jsem stačil kolo zvednout. Překvapilo mě, jak rychle se setmělo. Konec října se blížil a už bylo chladno. „Zatřepej s ní znovu, Wallaci,“ řekl jsem mu.

Světlo znovu naskočilo, ale už jen slabé, skomírající.

„Radiálky prostě  nepíchneš,“ poučoval mě bratr hlasem, kterým mluvil k větším shromážděním, v tomto případě ke mně a Wallaceovi mladšímu. „A i kdyby se ti to náhodou povedlo,  prostě na ně prskneš trochu toho lepidla a jedeš dál. Tři dolary devadesát pět centů za plechovku.“

„Strýček Bobby dokáže vyměnit penumatiku sám,“ hájil mě Wallace mladší, asi z jisté věrnosti mé osobě.

„Pneumatiku,“ opravil jsem ho zpod auta. Kdyby to bylo jen na Wallaceovi, ten kluk by mluvil jako „barbar z hor“, jak říkávala maminka. Ale jezdil by na radiálkách.

„Zkus tou baterkou znovu zatřepat,“ požádal jsem ho. Baterie byla skoro vybitá. Odšrouboval jsem poklici a zatáhl za kolo. Pneumatika praskla na bočnici. „Tak tuhle už spravovat nebudu,“ okomentoval jsem její stav. Ne že by mi to zvlášť vadilo. Doma u kůlny mi jich stojí hromada stejně velká jako já.

Světlo znovu zhaslo, ale když jsem pasoval rezervu na šrouby, rozzářilo se silněji než kdykoli předtím. „To už je lepší,“ pochvaloval jsem si. Světlo bylo silné, tmavě oranžové a mihotavé. Ale když jsem sáhl pro matky ke šroubům, překvapilo mě, že chlapcova baterka definitivně zhasla. Světlo přicházelo ode dvou medvědů, kteří stáli na kraji lesa a drželi pochodně. Byli to pěkní cvalíci, každý tak kolem sto třiceti kil, a vzpřímení na zadních tlapách mohli měřit metr padesát. Wallace mladší i jeho otec je viděli a stáli teď naprosto nehnutě. Medvědy je lepší nedráždit.

Vylovil jsem matky z poklice a našrouboval je. Obvykle je radši trochu namáznu olejem, tentokrát jsem to ale nechal být. Sáhl jsem pod auto, spustil hever a vytáhl ho. S úlevou jsem zjistil, že v rezervě je dost vzduchu, abychom na ní mohli jet. Hever, francouzák a píchlou pneumatiku jsem uložil do kufru. Místo abych poklici přišrouboval, kam patří, strčil jsem ji za nimi. Za celou tu dobu se medvědi ani nepohnuli. Jenom drželi pochodně – jestli byli zvědaví nebo nám chtěli pomoct, to se nedalo říct. Vypadalo to, že ve stromoví za nimi by jich mohlo být daleko víc.

Dveře jsme otevřeli všichni naráz, zapadli do auta a odjeli. Wallace promluvil první. „Vypadá to, že medvědi objevili oheň.“

\bigskip

\noindent
Když jsme maminku před takřka čtyřmi lety (sedmačtyřiceti měsíci) odvezli do Domova, svěřila se Wallaceovi a mně, že se už chystá umřít. „Nemějte o mě chlapci starost,“ zašeptala a přitáhla nás k sobě, aby ji sestra neslyšela. „Najela jsem milión mil a teď už můžu přejet na druhý břeh. Moc dlouho si tu už nepobudu.“ Třicet devět let jezdila školním autobusem. Později, když Wallace odešel, mi pověděla o svém snu. V kroužku kolem ní seděla skupina doktorů a diskutovala o jejím zdravotním stavu. Jeden řekl: „Udělali jsme všechno, co je v lidských silách, chlapci, nechme ji odejít.“ Všichni pak zvedli ruce a usmáli se. Když pak na podzim neumřela, trochu ji to zklamalo, i když na jaře už na to zapomněla, jak to u starých lidí bývá.

Kromě toho, že beru Wallace staršího i mladšího na návštěvu v neděli večer, sám jezdím ještě v úterý a ve čtvrtek. Obvykle ji najdu před televizí, i když se na ni nedívá. Sestry ji ani nevypínají. Říkají, že staří lidé mají rádi zrnění. Prý je to uklidňuje.

„Tuhle jsem slyšela, že medvědi objevili oheň,“ řekla mi v pondělí.

„To je pravda,“ ujistil jsem ji a dál jí rozčesával dlouhé bílé vlasy hřebenem ze želvoviny, který jí koupil Wallace na Floridě. V pondělí se objevila zpráva v louisvillském  Kurýru  a v úterý reportáž ve večerních zprávách buď na NBC nebo CBS. Lidé vídali medvědy na celém území státu a ve Virginii taky. Ukončili zimní spánek a zjevně hodlali strávit zimu v zalesněných pásech mezi dálnicemi. Ve virginských horách byli medvědi odjakživa, ale do západní Kentucky nezabloudili už celých sto let. Posledního zastřelili, když byla matka ještě děvčátko.  Kurýr  přišel s teorií, že sešli po I-65 z lesů Michiganu a Kanady, ale jeden muž z Allenského okresu tvrdil v rozhovoru na celonárodním televizním kanálu, že v horách vždycky pár medvědů zůstalo a že ti se teď přidali k ostatním, když objevili oheň.

„Už se neukládají k zimnímu spánku,“ pokračoval jsem. „Rozdělají si oheň a udržují ho celou zimu.“

„To mě podrž,“ prohlásila matka. „Na co ještě nepřijdou!“ Sestra jí přišla sebrat tabák, čímž mi jako obvykle naznačila, že si má maminka jít lehnout.

\bigskip

\noindent
Každý říjen zůstává Wallace mladší u mě, zatímco jeho rodiče jedou do tábora. Vím, že to zní hrozně zpátečnicky, ale tak to prostě je. Můj bratr je sice kazatel (Reformovaná církev spravedlivého života), ale na živobytí si ze dvou třetin vydělává prodejem nemovitostí. Spolu s Elizabeth jezdí do Křesťanského centra úspěchu v Jižní Karolíně, kde se lidé z celé země učí, jak těm druhým prodat různé věci. Vím, jak to chodí – ne že by se mi to kdy obtěžovali říct, ale v noci jsem v televizi viděl inzerovat kampaň „Výměnou nemovitostí k úspěchu“.

Wallace mladší vystoupil ze školního autobusu ve středu, tedy v den, kdy jeho rodiče odjeli. Dokud bydlí u mě, rozhodně si nemusí balit moc věcí. Má tu vlastní pokoj. Jako prvorozenému mi připadl náš starý domek poblíž Smith Grove. Je už trochu sešlý, ale mně ani Wallaceovi mladšímu to nevadí. V Bowling Green má rovněž svůj pokoj, ale vzhledem k tomu, že se Wallace s Elizabeth každé tři měsíce stěhují (jedna z podmínek kampaně), nechává si svou malorážku ráže dvacet dva a komiksy, které hrají v jeho věku velkou roli, ve svém pokoji u mě. Je to stejný pokoj, ve kterém jsme bydleli já a jeho otec.

Wallaceovi mladšímu je dvanáct. Když jsem se vrátil z práce, našel jsem ho sedět na zadní verandě, ze které je hezký výhled na mezistátní dálnici. Já se živím pojišťováním úrody.

Když jsem se převlékl, ukázal jsem mu, jak se dá rozbít patka pláště u pneumatiky – buď ji rozmlátíte kladivem nebo přes ni přejedete autem. Opravovat pneumatiky ručně dnes umí už jen málokdo, tak jako vyrábět sirup z čiroku. Kluk ovšem chápal rychle. „Zítra ti ukážu, jak nasadit pneumatiku s pomocí kladiva a montpáky.“

„Hrozně rád bych viděl ty medvědy,“ svěřil se mi. Díval se přes pole na I-65, kde nám pruhy silnice na sever odřízly roh pole. Když je člověk v noci vzhůru, hluk aut mu někdy připomíná vodopád.

„Ve dne jejich oheň neuvidíme,“ odtušil jsem, „ale počkej si do večera.“ Téhož večera dávali na CBS nebo NBC (pořád si je pletu) speciální dokumentární pořad o medvědech, kteří se těšili málem celonárodnímu zájmu. Zahlédli je v Kentucky, Západní Virginii, Missouri, na jihu Illinois a samozřejmě ve Virginii. Ve Virginii byli medvědi odjakživa. Pár lidí nadhodilo, že by je mohli začít lovit. Jeden vědec tvrdil, že mají namířeno do států, kde ještě zbyl nějaký sníh, ale ne zas mnoho, a kde je v dálničních pruzích dost dřeva na podpal. Zašel tam s videokamerou, ale pořídil jen pár rozmazaných záběrů postav usazených kolem ohně. Další vědec prohlásil, že medvědy přitahují bobule nového keříku, který roste pouze v zalesněných pruzích mezistátních dálnic. Tvrdil, že ta bobule je první nový rostlinný druh v moderní historii, který vznikl zkřížením semen kolem dálnice. Pak si na kameru do jedné kousl, ušklíbl se a pojmenoval ji „novobobule“. Klimatický ekolog uvedl, že teplé zimy (v Nashvillu nespadl minulou zimu žádný sníh a v Louisvillu se přehnala jen jedna metelice) ovlivnily cyklus medvědího zimního spánku, a proto si teď huňáči pamatují zkušenosti z minulých let. „Medvědi možná oheň objevili před několika staletími, ale zase u nich upadl v zapomnění.“ Podle další teorie objevili (nebo znovuobjevili) oheň, když před několika lety hořel Yellowstonský národní park.

V televizi ukazovali víc chlápků, kteří mluvili o medvědech, než medvědů samotných, a tak nás to s Wallacem mladším přestalo zajímat. Když jsme umyli nádobí, vzal jsem kluka za dům k plotu. Přes dálnici jsme skrz stromy viděli, jak medvědům plápolá oheň. Wallace mladší si chtěl zajít pro svou dvaadvacítku a jednoho si zastřelit, ale já mu vysvětlil, proč by to nebylo dobré. „Krom toho,“ dodal jsem, „by malorážka medvědovi neublížila, jen by ho rozzuřila.“

„A navíc,“ uzavřel jsem, „je lovení v dálničních pruzích přísně zakázáno.“

\bigskip

Jakmile pneumatiku dostanete do ráfku, stačí ji na její místo jenom nadrncat. Dělá se to tak, že si ji postavíte svisle a sednete si na ni. Vzduch jde dovnitř a vy ji nadhazujete mezi nohama, dokud patka pláště nedosedne do ráfku a neozve se uspokojivé plesknutí. Ve čtvrtek jsem Wallaceovi mladšímu místo školy ukázal jak na to, a nedal jsem mu pokoj, dokud to nezvládl. Pak jsme přelezli plot a vydali se po poli k medvědímu teritoriu.

Podle pořadu  Dobré ráno, Ameriko  udržují medvědi v severní Virginii oheň po celý den. Tady v západní Kentucky bylo na pozdní říjen ještě pořád dost teplo, a tak medvědi zůstávali kolem ohně jen v noci. Kam šli a co dělali ve dne, o tom nemám nejmenší tušení. Možná pozorovali z novobobulových keříků, jak já s Wallacem mladším šplháme přes vládní plot a běžíme přes dálnici. Já měl s sebou sekeru a Wallace mladší dvaadvacítku – ne že by si chtěl zabít medvěda, to jen proto, že kluci rádi nosí nějakou zbraň. V pruhu se pod javory, platany a duby rozlézala změť popínavých rostlin a křovinatého podrostu. Ačkoli to bylo od domu jen nějakých sto metrů, nikdy jsem tam nebyl já ani nikdo z mých známých. Vypadalo to jako uměle vytvořená krajina. Uprostřed jsme našli stezku a vydali se po ní přes pomalý, krátký potok, který vytékal z jedné mříže a vléval se do druhé. Stopy v šedém blátě byly to první, co jsme z medvědů viděli. Ve vzduchu se vznášel zápach zatuchliny, který ovšem nebyl zas tak nepříjemný. Na mýtince pod dutým bukem jsme našli ohniště plné popela. Kolem ohně ležely ve víceméně pravidelném kruhu klády a zápach už byl silnější. Prohrábl jsem oheň a našel dost uhlíků, aby se z nich dal rozfoukat nový oheň, a tak jsem je zase zahrábl na jejich místo.

Nasekal jsem něco dříví na podpal a naskládal ho na hromádku, abych ukázal dobrou sousedskou vůli.

Možná že nás medvědi sledovali z keřů už tehdy – to už se nikdy nedovíme. Ochutnal jsem novobobuli a zase jsem ji vyplivl. Byla příšerně sladká a trpká zároveň, přesně to, co byste si představovali, že medvědovi půjde k duhu.

\bigskip

Toho večera jsem se po večeři zeptal Wallace, jestli se mnou nechce na návštěvu k babičce. Nepřekvapilo mě, když mi na to kývl. Děti mají daleko více ohledů, než by se dospělým zdálo. Našli jsme ji na betonové verandě Domova, jak sleduje provoz na I-65. Sestra nám řekla, že maminka byla celý den neposedná. To mě taky nepřekvapilo. Jakmile se na podzim začne barvit listí, je najednou neklidná, lépe řečeno – zase začne „doufat“. Zavedl jsem ji do společenské místnosti a učesal jí dlouhé bílé vlasy. „V televizi už nedávaj nic jinýho než ty medvědy,“ stěžovala si sestra a přepínala z jednoho kanálu na druhý. Když odešla, Wallace mladší se zmocnil dálkového ovládání, a tak jsme zhlédli mimořádnou reportáž CBS nebo NBC o několika myslivcích ve Virginii, kterým někdo podpálil jejich domky. Následoval rozhovor s lovcem a jeho manželkou, kterým v údolí Shenandoah shořel dům v hodnotě 117 500 dolarů. Manželka byla pevně přesvědčena o vině medvědů. Manžel medvědy z ničeho nevinil, ale žaloval stát o náhradu, protože měl platný lovecký lístek. Státní zmocněnec pro lov se nechal v televizi slyšet, že vlastnictví loveckého lístku nezbavuje  lovené  práva na účinnou sebeobranu.  Pomyslel jsem si, že na státního zmocněnce je to docela liberální pohled na věc. Měl samozřejmě nezcizitelný zájem na tom, aby nemusel platit. Nevím, já sám myslivec nejsem.

„A v neděli se už ani nemusíš obtěžovat,“ řekla matka Wallaceovi mladšímu a mrkla na něj. „Já najela milión mil a jednou rukou už klepu na bránu.“ Já jsem už na její řeči zvyklý, zvlášť na podzim mě nepřekvapují, ale bál jsem se, že to kluka trochu rozhodí. Opravdu na odchodu vypadal ustaraně, a tak jsem se ho zeptal, co ho trápí.

„Jak mohla najet milión majlí?“ zeptal se mě. Řekla mu, že to bylo čtyřicet osm mil každý den, po celých devětatřicet let, a on si na kalkulačce spočítal, že to dělá 336 960 mil.

„Mil,  žádných majlí,“ já na to. „A je to čtyřicet osm ráno a čtyřicet osm odpoledne. A vyjížďky s fotbalovým mužstvem. A staří lidé taky trochu přehánějí.“ Maminka byla první řidičkou školního autobusu v našem státě. Jezdila každý den a k tomu vychovala rodinu. Tatínek jenom farmařil.

\bigskip

Z dálnice obvykle odbočuju u Smith Grove, ale toho večera jsem jel celou cestu na sever až ke Koňské jeskyni a zpátky, abychom si s Wallacem mladším prohlédli medvědí ohně. Nebylo jich tolik, kolik by si člověk představoval z televize – jedno ohniště každých šest sedm mil, schované ve shluku stromů nebo za skalním výběžkem. Nejspíš hledají jak dřevo, tak vodu. Wallace mladší si je chtěl prohlédnout, ale na mezistátních dálnicích se stavět nesmí a já se bál, že by nás policie rychle vyhmátla.

V poštovní schránce jsem měl od Wallace pohled. On i Elizabeth si vedou dobře a mají se přímo báječně. Pro Wallace mladšího ani slovo, ale klukovi to zřejmě nevadilo. Jako většina jeho vrstevníků jezdí s rodiči jen nerad.

\bigskip

V sobotu odpoledne mi z Domova zavolali do kanceláře a nechali vzkaz, že matka odešla. Byl jsem právě na cestě. Pracuji o sobotách – je to jediný den, kdy zastihnete hodně farmářů na částečný úvazek doma. Když jsem si zavolal do kanceláře a dostal tu zprávu, srdce mi doslova vynechalo – ale jen jeden tep. Už jsem se na to připravoval dlouho. „Je to pro ni vykoupení,“ řekl jsem sestře, když jsem se dovolal do Domova.

„Vy mi nerozumíte,“ opáčila sestra. „Ona  neodešla  ve smyslu umřela. Ona nám  utekla.  Vaše matka je pryč.“ Maminka prošla dveřmi na konci chodby, když se nikdo nedíval, zaklínila je hřebenem a nádavkem si s sebou vzala ústavní deku. „A co její tabák?“ zeptal jsem se. Taky pryč. Takže jsme věděli najisto, že si chce venku pobýt. Byl jsem zrovna ve Franklinu a do Domova jsem se po I-65 dostal až za hodinu a pár minut. Sestra mi sdělila, že matka se poslední dobou chovala čím dál víc zmateně. Mohl jsem se vsadit, že mi tohle řeknou. Prohledali jsme areál Domova, který se rozkládá na pouhém půl akru mezi dálnicí a polem se sojovými boby a na kterém nerostou ani žádné stromy. Pak mě přiměli, abych zatelefonoval šerifovi do kanceláře. Za pobyt v Domově jí budu muset platit, dokud nebude oficiálně vedena jako pohřešovaná, tedy do pondělí.

Když jsem se dostal domů, byla už tma a Wallace mladší dělal večeři. Nic moc složitého – stačilo mu otevřít pár vybraných konzerv převázaných gumičkou. Řekl jsem mu, že babička odešla, a on mi na to přikývl a odtušil: „Vždyť říkala, že tam už dlouho nezůstane.“ Zavolal jsem na Floridu a nechal vzkaz. Víc se už dělat nedalo. Sedl jsem si k televizi, ale nic v ní nedávali. Pak jsem se otočil k zadním dveřím a zahlédl plamínky ohně mezi stromy za dálnicí. Najednou jsem měl pocit, že bych přece jenom mohl vědět, kde ji hledat.

\bigskip

Citelně se ochladilo, a tak jsem si vzal bundu. Řekl jsem klukovi, ať čeká u telefonu, kdyby nám náhodou zatelefonoval šerif, ale když jsem se uprostřed pole ohlédl, už spěchal za mnou. Bez bundy. Zpomalil jsem, aby mě mohl dohonit. Malorážku s sebou měl, ale já ho přesvědčil, aby ji nechal opřenou u plotu. Ve tmě a v mém věku mi dělalo daleko větší potíže přelézt plot než za světla. Však už mi je jednašedesát. Po dálnici spěchaly náklaďáky na sever a osobní auta zase na jih.

V příkopu jsem si urousal záložky kalhot o dlouhou trávu, kterou už teď tížila rosa. Vlastně to nebyla tráva, ale lipnice.

Prvních pár metrů mezi stromy jsme šli v naprosté tmě a kluk se mě chytil za ruku. Pak už bylo trochu vidět. Nejdřív jsem si myslel, že to svítí měsíc, ale ne, to kužely světla z náklaďáků, osvětlující koruny stromů skoro jako měsíc, nás nakonec navedly na cestu houštinami a její důvěrně známý medvědí pach.

K medvědům jsem se v noci přibližoval jen nerad. Kdybychom zůstali na stezce, mohli jsme na nějakého narazit potmě, ale kdybychom šli podrostem, mohli by nás zase považovat za vetřelce. Napadlo mě, jestli jsme si opravdu neměli vzít pušku.

Zůstali jsme na stezce. Vypadalo to, že světlo kape z baldachýnu stromů jako déšť. Šlo se nám lehce, zvlášť když jsme se nesnažili dívat dolů a nechali jsme nohy, ať si najdou cestu samy.

A pak jsem skrz stromy uviděl jejich oheň.

\bigskip

Živily ho povětšinou platanové a bukové větve, přesně ty, které vydávají málo tepla, ale spoustu kouře. Bylo vidět, že v kvalitě dřeva medvědi ještě nemají jasno. Ale udržovat oheň už docela uměli. Velký, skořicově hnědý medvěd, pravděpodobně ze severu, prohraboval oheň klackem a tu a tam přidal větev z hromádky vedle sebe. Ostatní seděli kolem na kládách ve volném kruhu. Většinou to byli baribalové nebo ještě menší medvědi kynkažu a jedna samice s medvíďaty. Pár medvědů jedlo bobule z poklice. Moje matka nejedla, jen se dívala do ohně, po ramena zabalená do přikrývky z Domova.

Jestli si nás medvědi všimli, nedali to najevo. Matka poklepala na volné místo hned vedle sebe a já si k ní přisedl. Jeden medvěd se zase odsunul, aby si mohl Wallace mladší sednout na druhou stranu.

Medvědí pach je pronikavý, ale není nepříjemný, jakmile si na něj zvyknete. Trochu se podobá smradu ze stodoly, ale silnějšímu. Naklonil jsem se k matce, chtěl jsem jí něco pošeptat do ucha, ale ona zavrtěla hlavou.  Bylo by nezdvořilé špitat si před tvory, kteří nemají dar řeči,  naznačila mi mlčky. Wallace mladší byl taky zticha. Matka se s námi podělila o přikrývku, a tak jsme se tam snad celé hodiny dívali do ohně.

Velký medvěd se staral o oheň, lámal suché větve tak jako lidé – přidržel si jeden konec v prackách, druhý opřel o zem a šlápl doprostřed. Uměl udržovat stejnoměrný plamen, to se musí nechat. Jiný medvěd oheň čas od času prohrábl, ale ostatní ho nechali na pokoji. Vypadalo to, že s ohněm umí zacházet jen pár medvědů a ostatní se vezou. Ale není to tak se vším? Jednou za čas k osvětlenému prostranství došel malý medvěd a na hromadu přihodil náruč plnou dřeva. To má kolem dálnic stříbřitý odstín, zrovna jako naplavené.

Wallace mladší není na rozdíl od mnoha kluků jeho věku neposedný. Seděl jsem, díval jsem se do plamene a bylo mi dobře. Vzal jsem si trochu matčina tabáku Red Man, i když normálně nežvýkám. Připomínalo mi to návštěvy v Domově, jen díky medvědům daleko zajímavější. Celkem jich tu bylo asi osm nebo deset. Oheň taky nebyl nezajímavý – uvnitř se odehrávala malá dramata, ohnivé sály vznikaly a zase zanikaly v praskotu oharků. Popustil jsem uzdu fantazii. Díval jsem se na medvědy kolem a představoval si, co asi vidí oni. Někteří měli zavřené oči. I když tu byli pospolu, jejich duch byl pořád ještě individualistický, jako kdyby každý medvěd seděl před vlastním, soukromým ohněm.

Když k nám poklice doputovala, všichni jsme si vzali pár novobobulí. Nevím jak matka, ale já jsem jenom předstíral, že je jím. Wallace mladší protáhl obličej a vyplivl je. Když usnul, obtočil jsem kolem nás tří přikrývku. Ochlazovalo se a my jsme na rozdíl od medvědů neměli pořádný kožich. Byl bych už šel, ale matka nechtěla. Rukou ukázala ke stromovému baldachýnu, odkud se šířilo světlo, a pak na sebe. Myslela si, že pro ni přicházejí andělé shůry? Byla to jenom světla nějakého náklaďáku na cestě na jih, ale ji to viditelně potěšilo. Držel jsem ji za ruku a cítil, jak chladne.

\bigskip

Wallace mladší mi poklepal na koleno a tím mě probral. Slunce už vyšlo a jeho babička zemřela zaklíněná mezi námi. Oheň doutnal pod vrstvou uhlíků, medvědi byli pryč a někdo se k nám hlučně probíjel lesem, v blažené nevědomosti o pohodlné stezce. Byl to Wallace se dvěma policisty. Na sobě měl bílou košili a já si náhle uvědomil, že je neděle ráno. Z matčiny smrti byl Wallace smutný, ale pod povrchem jsem vycítil rozmrzelost.

Vojáci čichali ve vzduchu a kývli na sebe. Medvědí pach byl pořád ještě silný. Zabalili jsme s Wallacem matku do přikrývky a vydali se s jejím tělem k dálnici. Policisté zůstali, rozkopali popel a rozházeli nanošené dříví na všechny strany. Mně to připadalo malicherné. Sami vypadali v těch svých uniformách jako medvědi.

U kraje silnice stál Wallaceův Oldsmobil 98 a jeho radiálky na trávě vypadaly trochu splaskle. Před ním stálo policejní auto, vedle něj strážník a za ním pohřební vůz, rovněž Oldsmobil 98.

„To je první hlášení, že obtěžovali staré lidi,“ prohodil první policista k Wallaceovi. „Nic takového se nestalo,“ ohradil jsem se, ale nikdo se mě po vysvětlení neptal. Mají svoje vlastní postupy. Dva muži v obleku vyšli z pohřebního vozu a otevřeli zadní dveře. To pro mě byl okamžik, kdy maminka odešla z tohoto světa. Když jsme ji uložili dovnitř, vzal jsem kluka za rameno. Klepal se, ačkoli nebyla zima. Smrt tohle dělá, zvláště za úsvitu, s policejními auty kolem dokola na mokré trávě, dokonce i když přijde jako přítelkyně.

Chvíli jsme stáli a dívali se, jak kolem nás projíždějí osobní auta a náklaďáky. „Je to pro ni vykoupení,“ řekl Wallace. Je zvláštní, kolik jezdí v 6.22 ráno po silnicích aut.

\bigskip

Ještě odpoledne jsem zašel do pruhu mezi dálnicemi a nasekal něco dřeva místo toho, které policisté rozházeli. Večer jsem skrz stromy znovu zahlédl oheň.

Za dva dny po pohřbu jsem se vrátil. Kolem ohně seděli stejní medvědi, tedy pokud jsem to mohl ve tmě posoudit. Na chvilku jsem si k nim sedl, ale protože znatelně znervózněli, zase jsem šel domů. Z poklice jsem si vzal hrst novobobulí a v neděli jsem s chlapcem zašel na hřbitov a naaranžoval jsem je na maminčině hrobu. Ještě jednou jsem je ochutnal, ale není to k ničemu, jíst se nedají.

Pokud ovšem nejste medvěd.

\podpis{přeložil Viktor Janiš}