\section{Bubeníkům je prý sto\ldots}

\noindent
Letos v létě jsem dostala neobvyklý dárek -- dopis od zdravotní sestry, která tři měsíce pečovala o mého bezmocného otce:

\medskip

\noindent
\textit{,,Dobrý den, paní Mertinová. Tak jsem o víkendu zjistila, že mám průšvih. Ten průšvih se jmenuje Grigorij Kanovič a je starý asi půl roku. Já jsem si u vás doma ,,vypůjčila`` knížku Kůzle za dva groše. Chodila jsem s dědou po bytě a všimla jsem si jí tam v ložnici v knihovně. Zaujala mě, a tak jsem se dědy zeptala, jestli si ji můžu vzít, že bych si ji přečetla a v brzké době zase vrátila na původní místo. Děda samozřejmě souhlasil, ale já vím, že jsem se spíš měla zeptat někoho z rodiny. V tu chvíli mi to nedošlo. Dejte mi prosím rozhřešení, a jestli můžete, nezlobte se na mě. Ale\ldots směla bych si to dočíst?``}

\medskip

\noindent
Jen pamětníci si připomenou, že první řádky překladu, od jehož knižního vydání uplynulo už neuvěřitelných šest let, měly českou premiéru ještě o pár pátků dřív: u ,,prvních`` Bubeníků v Myslíkově ulici. V té době se nevědělo, kdo bude ochoten tu knížku vydat ani jestli já budu mít dost času a sil tu práci dokončit. 

Možná mě ,,pošťouchl`` Ilja Racek, který tenkrát přijal od paní principálky ,,roli`` a přečetl nahlas úryvek. Možná právě onoho večera jsem se ujistila, že má smysl ještě pár let pokračovat.

Stejně mi pomohl loni v září Josef Somr, který neopakovatelným způsobem interpretoval kapitolku z mého těžce se rodícího překladu Huckleberryho Finna.

Díky bubenickým večerům jsem zažila moc hezké chvíle a nečekané pocity spřízněnosti s jinými lidmi. Často jsem žasla nad pílí a vynalézavostí kolegů překladatelů a obdivovala se tomu, jak psané slovo náhle ožívá v ústech herců. Při ohlédnutí zpět se nestačím divit, co všechno se za tu dobu stalo a stihlo. Neskrývám svůj obdiv k Hance Kofránkové, velké matce a porodní bábě, která u Bubeníků nechala zaznít spoustu textů. Přijala do své kritické, avšak láskyplné péče mnoho překladů v prenatálním stadiu, jimž umožnila další vývoj a zrání. 

Jsem trochu sentimentální, jak se při kulatém jubileu sluší a patří. A vám, Želvo a Orle, chci aspoň od srdce poděkovat za všechen čas a energii, kterou jste věnovali ku prospěchu i zábavě nás ostatních.

\podpis{Jana Mertinová}

\bigskip

\section{Ach ano, deset let.}

\noindent
Řítí se to s námi čím dál rychleji. Kňourat ale nemá smysl a taky není proč. Život se skládá ze zážitků, a když je 
těch příjemných víc než těch nepříjemných, má člověk být spokojen.

Bubeníčci mi posledních deset let mocně pomáhají zajišťovat na mém osobním běžném účtu kladné saldo. 

Byl jich bezpočet, těch krásných zážitků, ale pro stručnost si připomenu jen jeden. Pochází hned z prvního roku bubeníčkování, kdy jsme se ještě scházeli v tom podzemním sálku v zadním traktu hospody U Bubeníčků. Dokončoval jsem tehdy svůj nový český překlad Evžena Oněgina a Hanka Kofra, která o tom věděla, se mě ptala, jestli už mám hotové oba 
dopisy, že by se hodily do květnového večera, který má být o lásce. Taťánin dopis Oněginovi jsem už měl, Oněginův Taťáně ještě ne. 

Spěchalo to natolik, že jsem si svůj diktafonek bral s sebou i do parku, kam jsem chodil běhat, a opravdu mám ještě na kazetě ten původní pracovní překlad, ze kterého je slyšet, jak při některých verších popadám dech. 

Stihl jsem to. V květnovém večeru říkal pan Somr autorský text, paní Medvecká představovala svou jmenovkyni a pan Rösner Oněgina. Dosud mám v sobě ten pocit ohromení, jak krásně najednou z mistrovských úst znělo, co jsem já v potu tváře a někdy i s dechem přerývaným namáhavě lepil dohromady. Přistihl jsem se při myšlence: ,,Hergot, Dvořák, to snad ani není možný, že tohle jsi sesumíroval ty.``

Bylo to možné. A právě v tom je jedné z bubeníčkovských kouzel, že v tomhle společenství text ožívá, dostává hlas. Řeč je prvotní, nejspíše shůry seslaná, psaní je až mnohem pozdější nálezek lidský. To, že se černá písmena odlepí z bílého papíru a začnou se vznášet prostorem od dovedných úst k pozorným uším, od autora přes překladatele k publiku, je jeden ze zázraků, dělajících z té naší společnosti to, čím je.

Uplynulo zase deset let našeho stále vzácnějšího času, ale jsem rád, že ne všechny dny těch uplynuly naplano.

\podpis{Milan Dvořák}

\bigskip

\section{Bubeníci pod laskavým,}

\noindent
ale spolehlivě razantním vedením Želvy, s vydatnou podporou vzletného Orla, za pravidelného přispívání ,,mlaďochů`` a účinkování spanilého týmu herců, muzikantů, zpěváků včetně nejmúzičtějšího ministerského úředníka, jakého tato země má, připravili dosud 99 nádherných večerů, při nichž docházelo k setkávání napříč věky, jazyky, zeměmi, generacemi a dimenzemi. Po dlouhou dobu byli jednou z mála jistot, které jsem v tomhle životě měl. Prošel jsem všemi štacemi od Bubeníčků až po Prádlo, a kdykoli jsem zavítal, s otevřenými ústy jsem zíral na lidi, kteří sice skvěle a zdatně fungovali a fungují všude možně, ale kteří, jak se mi zdálo, nejvíc ve své kůži byli v té bubenčí. Moc jim za všechno děkuju. Stovkou to prý končí, ale kdepak. Jede se dál. Nejde o to, jak jinak, o čem, kde a kdy to bude. Dokud u toho budou tihle lidi, moje jistota zůstává. 

\podpis{Jiří Josek}

\bigskip

\section{KANKÁNOVÁ HARFO!}

\noindent
Psala jsem na jakousi prahorní pohlednici, když mě konečně napadla ta správná přesmyčka. Nejenže dokonale sedí, písmenek je tak právě do počtu a délky lícují, ale navíc skvěle vystihuje tu neuvěřitelnou bytost. Rozpětí mezi kankánem a harfou -- mezi rozkošnou pěnou krajkových spodniček a velebným zvukem nástroje andělského -- to je velká matka HANA KOFRÁNKOVÁ, a ať si to každý popřesmykuje. 

V rozpětí kankánu a harfy se odbývalo i těch požehnaných deset roků Bubeníčků: nespočtu, kolikrát jsem se smála až k slzám a kolikrát plakala doopravdy. Počítám-li aspoň přibližně správně, strávila jsem s Bubeníčky asi dvě stě čtyřicet hodin, což je deset dnů života. V jedné básni Jana Vladislava počítá starý vojevůdce, kolik dnů štěstí mu život dopřál. Ještě nejsem zralá na takové účtování, ale těch deset dnů s Bubeníčky tam jednou rozhodně započtu.

Díky všem za ten tok energie, kterou do toho museli vložit: Hance, Liborovi a dalším a ostatním a jméno po jménu všem. Škoda, že ta šňůra končí, ale ona už se jistě  plete nová. Takže zatím nashledanou. Srdečně zdraví HAD A OSEL VE FRANCII 

\podpis{Daniela Fischerová}
 
\bigskip

\section{Společenství vtipných,}

\noindent
milých a vzdělaných lidí v atmosféře spříznění, legrace, zaujetí a hledačství? Jmenuje se Bubeníčci a teď kotví v Divadle Na Prádle. 

\ldots je to téměř deset let, co mne Hanka K. pozvala do Myslikovy ulice do hospody U Bubeníčků na překladatelský večer. Společně jsme se tehdy právě účastnily Halasova Kunštátu a při té příležitosti došla řeč i na vzpomínané ,,bubeníčky ``. 

Mnohé bubeníčkovské večery se mi vybavují, ale asi nejzřetelněji ten, ve kterém posluchači vymýšleli autorská haiku. Ta byla tak zdařilá, že Erwin Kukuczka se nabídl, že je vydá tiskem a já nabídla dodat k textům vlastní ilustrace. Tak začala má několikaletá spolupráce s Erwinovým nakladatelstvím LOUČ.

Ačkoliv jsem poslední dobou nechodila na Bubeníčky často, návštěva večerů se stala součástí mého života. To, že se lidé scházejí, aby se zabývali  slovem, jeho významem, smyslem a krásou a hlavně, aby byli spolu, mně dává radost.

S některými z přítomných jsem se spřátelila, a všichni zúčastnění ve mně vzbuzují respekt, obdiv a sympatii.

\podpis{Alena Laurová}

\bigskip

\section{Proč jezdím na ,,Bubeníky``}

\noindent
Doma jsem v Jičíně -- v Českém ráji a mám to kousek na festival mluveného slova ,,Šrámkova Sobotka`` do Sobotky, měla jsem blízko na literárně hudební programy ve Starých Hradech na Libáňsku, ledacos si mohu vybrat i na festivalu ,,Jičín -- město pohádky``, ale kdybych nemohla jezdit mezi překladatele, moc by mi to chybělo. Těší mě setkávat se s těmi, které jsem v osmdesátých letech poznala na překladatelských seminářích a jsem v duchu pyšná na to, že se skoro všichni stali uznávanými překladateli, někteří i úspěšnými nakladateli. Po roce 1989 moc šancí na svolávání talentovaných zájemců o překlad beletrie nebylo, pro nás ,,přespolní`` to platilo dvojnásobně. 

Proto nápad Hany Kofránkové a Libora Dvořáka pořádat večery z prací přeložených, leč v šuplíku ,,tlejících``, považuji dodnes za úžasný a možná některými z návštěvníků za nedoceněný. Na ,,Bubeníky`` jezdím  proto, že si zpět odvážím nezapomenutelné zážitky z interpretace vybraných textů, inspiraci a chuť pustit se taky do nějakých odložených  a nedokončených překladů poezie. Proto, že se pak necítím v Jičíně tak opuštěná a nemotivovaná. Velké potěšení mi přináší skutečnost, že jsem pro některé své dávné známé Pražačky připravila překvapení tím, že jsem je přivedla, ony se nadchly pro tento typ pořadu (mezi bližšími nazývaný ,,kofránkovský patvar``) a už se pokaždé těší spolu se mnou na další. Netušily, že něco takového v Praze existuje a že je možné vytvořit takovýto zvláštní druh atmosféry a navíc mít možnost vstupovat do ní. Také ráda sleduji, jak se tvoří nové překladatelské podhoubí z  řad "splaváků`` a ,,plaváků``, kteří studují, pak vstupují do života, zamilovávají se a rozmilovávají, žení se a zakládají rodiny, ale stále jsou nějak v kontaktu s ,,bubeníčkovskou`` komunitou. (Mnohé z nich jsem potkala nejdřív na ,,Šrámkových Sobotkách``, kde začali čeřit poklidnou hladinu festivalu týden vydávaným deníkem SPLAV!). 

Ani 3 hodiny, strávené celkem v autobuse na stokilometrové trase Jičín -- Praha a v sobotu zpět, nemusí být ztrátou času, protože mi občas připraví nečekaná setkání a změnou zaběhlého rytmu všedního dne i inspirační okamžiky. A tady si nenechám ujít příležitost prozradit, že já mám za to putování vlastně jeden zvláštní bonus: nekonečné pozdně noční až ranní debaty s hlavní pachatelkou programů Hanou Kofránkovou, neboť nejvíc přespávacích azylů mi během těch 10 let poskytla ona, a to u nich doma.

Těším se na další večer, na tradiční i nové interprety přeložených textů a v duchu vzpomínám na ty překladatele, kteří už mezi nás nikdy nepřijdou. 

\medskip

\noindent
A tak s díky

\hspace*{1cm}za ,,Bubeníky``

\podpis{Draga Zlatníková}